% Commands

\newcommand{\authorinfo}{Arne Struck, Tronje Krabbe}
\newcommand{\titleinfo}{FGI 2 [HA], 11. 11. 2013}
\newcommand{\qed}{\ \square}

% ------------------------------------------------------

% Packages & Stuff

\documentclass[a4paper,11pt,fleqn]{scrartcl}
\usepackage[german,ngerman]{babel}
\usepackage[utf8]{inputenc}
\usepackage[T1]{fontenc}
\usepackage{lmodern}
\usepackage{amssymb}
\usepackage{amsmath}
\usepackage{enumerate}
\usepackage{fancyhdr}
\usepackage{pgfplots}
\usepackage{multicol}
\usepackage{pst-node}
\usetikzlibrary{calc}
\usetikzlibrary{patterns}
\usetikzlibrary{arrows,automata}

% ------------------------------------------------------

% Title & Pages

\title{\titleinfo}
\author{\authorinfo}

\pagestyle{fancy}
\fancyhf{}
\fancyhead[L]{\authorinfo}
\fancyhead[R]{\titleinfo}
\fancyfoot[C]{\thepage}

\begin{document}
\maketitle
        \begin{enumerate}
        \item[\textbf{4.3.}]
                \begin{enumerate}
                        \item[1.]\quad \\
                        	$L(TS_{Waschmaschine}) = n((fn)+((cw+wc)sg^*d))^*$ \\
                            $L^\omega(TS_{Waschmaschine}) = n(\{fn\}\cup\{wcs\}\{g\}^*\{d\}\cup\{cws\}\{g\}^*\{d\})^\omega$
                        \item[2.]\quad \\
                        	$SS(M)=1(2(((3 + 4)56^+)+1))^\omega$
                        \item[3.]\quad \\
                        	Wir sind uns nicht über die genaue Anforderung dieser Aufgabe im Klaren. Im Grunde ist $E_S(SS(M))$ nur der gleiche $\omega$-reguläre Ausdruck wie SS(M), nur werden die Zustandsnamen durch Etiketten substituiert. Da man aber laut Aufgabe einfach die Zustandsnamen nehmen soll, bleibt es doch gleich... \\
                            Siehe Definition 2.18 auf Seite 36 sowie die Anmerkung dazu und Beispiel 2.19 auf Seite 37 im Skript. \\
                        	$E_S(SS(M)) = 1(2(((3 + 4)56^+)+1))^\omega$
                        \item[4.]\quad \\
                        	$Sat(Start \land Error) = \{2,5\}$\\
                            $Sat(Heat) = \{4,7\}$\\
                            Sei $\pi = 13(125253)^\omega$\\
                           	Hier gilt $Start \land Error$, jedoch niemals $Heat$. Somit gilt die LTL-Formel nicht.
                        \item[5.]\quad \\
                        	Sei $\pi = (13)^\omega$\\
                            Somit ist $f$ nicht erfüllt.\\
                            Mit $\pi = 1(25)^\omega$ ist $f$ erfüllt.
               \end{enumerate}
        \item[\textbf{4.4.}]\quad \\
        \(
		\begin{array}{|l|c|c|}
		f & M_{ofen}, \pi \models f & M_{ofen}\models f \\ \hline
		\circ \neg (Start\land Heat) & \checkmark &\checkmark \\
		\square \neg Start &\times&\times \\
		\square (Start \Rightarrow Close)&\times&\times \\
		\square \lozenge (Heat \lor Error \lor \neg Start)&\checkmark&\checkmark \\
		\lozenge ((Start\land Close\land \neg Error)\textbf{U}\ Heat)&\times&\times \\
		\square ((Close\land \neg Heat\land Start)\Rightarrow \circ \circ \neg Heat)&\checkmark&\times \\		
		\end{array}
		\)
    \end{enumerate}
\end{document}