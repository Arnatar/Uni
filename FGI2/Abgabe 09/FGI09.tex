 % Commands
\newcommand{\authorinfo}{Arne Struck, Tronje Krabbe}
\newcommand{\titleinfo}{FGI 2 [HA], 02. 12. 2013}
\newcommand{\qed}{\ \square}
\newcommand{\todo}{\textcolor{red}{\textbf{TODO}}}

% ------------------------------------------------------

% Packages & Stuff

\documentclass[a4paper,11pt,fleqn]{scrartcl}
\usepackage[german,ngerman]{babel}
\usepackage[utf8]{inputenc}
\usepackage[T1]{fontenc}
\usepackage{lmodern}
\usepackage{amssymb}
\usepackage{amsmath}
\usepackage{enumerate}
\usepackage{fancyhdr}
\usepackage{pgfplots}
\usepackage{multicol}
\usepackage{pst-node}
\usetikzlibrary{calc}
\usetikzlibrary{patterns}
\usetikzlibrary{arrows,automata,positioning}

% ------------------------------------------------------

% Title & Pages

\title{\titleinfo}
\author{\authorinfo}

\pagestyle{fancy}
\fancyhf{}
\fancyhead[L]{\authorinfo}
\fancyhead[R]{\titleinfo}
\fancyfoot[C]{\thepage}

\begin{document}
	\maketitle
	\begin{enumerate}
		\item[\textbf{9.3.}]
		\begin{enumerate}
			\item[1.:]\quad \\
			\begin{tikzpicture}[>=stealth',shorten >=1pt,auto,node distance=1cm,every path/.style={->}]
				%nodes
				\node (A) {\((001)^T\)};
				\node (B) [below left = of A] {\((100)^T\)};
				\node (C) [below = of B] {\((010)^T\)};
				
				\node (H) [below right = of A] {\((200)^T\)};
				\node (I) [below = of H] {\((110)^T\)};
				\node (J) [below = of I] {\((020)^T\)};
				
				\node (D) [right =3cm of H] {\((2\omega 0)^T\)};
				\node (E) [below = of D] {\((1\omega 0)^T\)};
				\node (F) [below = of E] {\((0\omega 0)^T\)};
				
				\node (G) [left = of E] {\((0\omega 1)^T\)};
				
				%paths
				\path (A) edge [midway, above left] node{c} (B);
				\path (B) edge [midway, left] node{a} (C);
				\path (A) edge [midway, above right] node{d} (H);
				\path (D) edge [midway, left] node{a} (E);
				\path (E) edge [midway, left] node{a} (F);
				\path (D) edge [midway, above left] node{a'} (G);
				\path (G) edge [midway, above] node{c} (E);
				\path (G) edge [midway, below left] node{b} (F);
				\path (G) edge [midway, above left, bend left=50] node{d} (D);
				\path (H) edge [midway, left] node{a} (I);
				\path (I) edge [midway, left] node{a} (J);
				\path (H) edge [midway, above right] node{a'} (G);
			\end{tikzpicture} \\ \\
			\begin{tikzpicture}[>=stealth',shorten >=1pt,auto,node distance=1cm,every path/.style={->}]
				%nodes
				\node (A) {\((001)^T\)};
				\node (B) [below left = of A] {\((100)^T\)};
				\node (C) [below = of B] {\((010)^T\)};
				
				\node (D) [below right = of A] {\((200)^T\)};
				\node (E) [below = of D] {\((110)^T\)};
				\node (F) [below = of E] {\((020)^T\)};
				
				\node (G) [right = of E] {\((011)^T\)};
				
				\node (H) [below = of G] {\((000)^T\)};
				
				%paths
				\path (A) edge [midway, above left] node{c} (B);
				\path (B) edge [midway, left] node{a} (C);
				\path (A) edge [midway, above right] node{d} (D);
				\path (D) edge [midway, left] node{a} (E);
				\path (E) edge [midway, left] node{a} (F);
				\path (D) edge [midway, above right] node{a'} (G);
				\path (G) edge [midway, above right] node{c} (E);
				\path (G) edge [midway, right] node{b} (H);
			\end{tikzpicture} \\ \\
			
			\item[2.:]\quad \\
			\begin{enumerate}
				\item[a)] \(\{p_2\}\)
				\item[b)] \(\{\}\)
			\end{enumerate}
			\item[3.:]\quad \\ %Das ist doch der von 9.3.1.:N9.3b
			\begin{tikzpicture}[>=stealth',shorten >=1pt,auto,node distance=1cm,every path/.style={->}]
				%nodes
				\node (A) {\((001)^T\)};
				\node (B) [below left = of A] {\((100)^T\)};
				\node (C) [below = of B] {\((010)^T\)};
				
				\node (D) [below right = of A] {\((200)^T\)};
				\node (E) [below = of D] {\((110)^T\)};
				\node (F) [below = of E] {\((020)^T\)};
				
				\node (G) [right = of E] {\((011)^T\)};
				
				\node (H) [below = of G] {\((000)^T\)};
				
				%paths
				\path (A) edge [midway, above left] node{c} (B);
				\path (B) edge [midway, left] node{a} (C);
				\path (A) edge [midway, above right] node{d} (D);
				\path (D) edge [midway, left] node{a} (E);
				\path (E) edge [midway, left] node{a} (F);
				\path (D) edge [midway, above right] node{a'} (G);
				\path (G) edge [midway, above right] node{c} (E);
				\path (G) edge [midway, right] node{b} (H);
			\end{tikzpicture} \\ \\
			\item[4.:]\quad \\
				Da Inhibitornetze bei passend gewählten Inhibitoren dem Überdeckungsgraph zum
				Erreichbarkeitsgraphen umformen, besitzen sie die selben Eigenschaften.
		\end{enumerate}
		\item[\textbf{9.4.}]
		\begin{enumerate}
			\item[1.:]\quad \\
			\( \Delta_{N9.4a} =
			\begin{pmatrix}
				1 & -1 & 0 & 0 \\
				-1 & 1 & 0 & 0 \\
				0 & 1 & -1 & 0 \\
				0 & 0 & -1 & 1 \\
				0 & 0 & 1 & -1 \\
			\end{pmatrix}
			\)
			\item[2.:]\quad \\
			\(\begin{array}{rclcl}
				i_1 &=& i_2 \\
				i_1 &=& i_2 + i_3 &\Leftrightarrow & i_3 = 0 \\
				i_3 &=& -i_4 + i_5 \\
				i_4 &=& i_5
			\end{array}
			\) \\
			P-Invariantenvektoren: \(\Big\{(a\,a\,0\,b\,b)^T\Big\}\ a,b\in\mathbb{N}/\{0\}\)
			\item[3.:]\quad \\
				Nach Theorem 7.35 dass kein Element der P-Invariantenvektoren gleich 0 sein darf, damit ein Netz
				strukturell beschränkt ist. \\
				Somit ist \(N_{9.4a}\) nicht strukturell beschränkt.
			\item[4.:]\quad \\
			\begin{itemize}
				\item \quad \\
				\( \Delta_{N9.4b} =
			\begin{pmatrix}
				1 & -1 & 0 & 0 \\
				-1 & 1 & 0 & 0 \\
				0 & 1 & -1 & 0 \\
				0 & 0 & -1 & 1 \\
				0 & 0 & 1 & -1 \\
				0 & -1 & 1 & 0 \\
			\end{pmatrix}
			\) \\
			\item \quad \\
			\(\begin{array}{rclcl}
				i_1 &=& i_2 \\
				i_1 &=& i_2 + i_3 - i_6 &\Leftrightarrow & i_3 = i_6 \\
				i_3 &=& -i_4 + i_5 + i_6 \\
				i_4 &=& i_5
			\end{array}
			\) \\
			P-Invariantenvektoren: \(\Big\{(a\,a\,c\,b\,b\,c)^T\Big\}\ a,b,c\in\mathbb{N}/\{0\}\) \\ \\
			\item \(N_{9.4b}\) ist nach Theorem 7.3 strukturell beschränkt, da kein \(i(p_k) = 0\) existiert. \\
			
			\end{itemize}
			\item[5.:]\quad \\
			Die Firma erhält in \(N_{9.4a}\) keinen Ausgleich für Produktion, in \(N_{9.4b}\) ist dies der Fall.
			Da \(p_3\) das Lager und der Rechte Teil des Netzes den Konsum darstellt, könnte \(c\) dem Verkauf 
			darstellen, womit \(p_6\) die Bezahlung repräsentiert.
			\item[6.:]\quad \\
			\(
				i_1^{tr} = (2,2,5,1,1,5) \\
				m_0 = (1,1,0,3,0,1)^{tr} \\
				\\
			\)
			Somit lautet die Invariantengleichung: \\
			\\
			\(
			\begin{array}{crcr}
				& i_1^{tr} \cdot m &=& i_1^{tr} \cdot m_0 \\
				\Leftrightarrow & (2,2,5,1,1,5) \cdot m &=& 12
			\end{array}
			\)
		\end{enumerate}
	\end{enumerate}
\end{document}
