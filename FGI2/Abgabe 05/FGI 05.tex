% Commands
\newcommand{\authorinfo}{Arne Struck, Tronje Krabbe}
\newcommand{\titleinfo}{FGI 2 [HA], 18. 11. 2013}
\newcommand{\qed}{\ \square}
\newcommand{\todo}{\textcolor{red}{\textbf{TODO}}}

% ------------------------------------------------------

% Packages & Stuff

\documentclass[a4paper,11pt,fleqn]{scrartcl}
\usepackage[german,ngerman]{babel}
\usepackage[utf8]{inputenc}
\usepackage[T1]{fontenc}
\usepackage{lmodern}
\usepackage{amssymb}
\usepackage{amsmath}
\usepackage{enumerate}
\usepackage{fancyhdr}
\usepackage{pgfplots}
\usepackage{multicol}
\usepackage{pst-node}
\usetikzlibrary{calc}
\usetikzlibrary{patterns}
\usetikzlibrary{arrows,automata}

% ------------------------------------------------------

% Title & Pages

\title{\titleinfo}
\author{\authorinfo}

\pagestyle{fancy}
\fancyhf{}
\fancyhead[L]{\authorinfo}
\fancyhead[R]{\titleinfo}
\fancyfoot[C]{\thepage}

\begin{document}
	\maketitle
	\begin{enumerate}
%		ALTER, 1 sind mal eben 11 Aufgaben, wie wollen die denn da 6 Punkte verstecken?
%		1/4 pro teilaufgabe oder was?!
		\item[\textbf{5.3}]
		\begin{enumerate}
			\item[1.]\quad \\
			\begin{tikzpicture}[>=stealth',shorten >=1pt,
									every node/.style={draw,circle,inner sep=2pt},
                	    			level 1/.style={sibling distance=50mm},
                	    			level 2/.style={sibling distance=40mm},
                	    			level 3/.style={sibling distance=30mm},
                	    			level 4/.style={sibling distance=20mm},
                				    every path/.style={->}]
                				    
				%Tree:
				\node[label=above : {$s_0$}]{$\{b\}$}
					child{node[label=left : {$s_0$}]{$\{b\}$}
						child{node[label=left : {$s_0$}]{$\{b\}$}
							child{node[label=left : {$s_0$}]{$\{b\}$}
								child{node[label=left : {$s_0$}]{$\{b\}$}
									child{node[draw=none]{}}
								}
							child{node[label=right : {$s_1$}]{$\emptyset$}
								child{node[draw=none]{}}
								}	
							}
							child{node[label=right : {$s_1$}]{$\emptyset$}
								child{node[label=right : {$s_2$}]{$\{g\}$}
									child{node[draw=none]{}}
								}
							}
						}
						child{node[label=right : {$s_1$}]{$\emptyset$}
							child{node[label=right : {$s_2$}]{$\{g\}$}
								child{node[label=right : {$s_2$}]{$\{g\}$}
									child{node[draw=none]{}}
								}
							}
						}
					}
					child{node[label=right : {$s_1$}]{$\emptyset$}
						child{node[label=right : {$s_2$}]{$\{g\}$}
							child{node[label=right : {$s_2$}]{$\{g\}$}
								child{node[label=right : {$s_2$}]{$\{g\}$}
										child{node[draw=none]{}}
								}
							}
						}
					}
				;           				
			\end{tikzpicture}
			\item[2.]
			\begin{enumerate}
				\item[a)]\quad \\ 
					\(Sat(\alpha_1) = \{s_0\} \quad | \alpha_1 = \textbf{EX}b\) 
				\item[b)]\quad \\ 
					\(Sat(\textbf{AG}\alpha_1) = \emptyset\)
				\item[c)]\quad \\
					\(Sat(\alpha_2) = \{s_1,s_2\}\quad | \alpha_2 = \textbf{AG}\lnot b\)
				\item[d)]\quad \\
					\(Sat(\textbf{EX}\alpha_2) = \{s_0,s_1,s_2\}\)
			\end{enumerate}
			\newpage
			\item[3.]
			\begin{enumerate}
				\item[a)]\quad \\
					\(\beta_1 = \textbf{AGEX}b\) gilt nicht, da das Ergebnis von 2b) $\emptyset$ ist.
				\item[b)]\quad \\
					\(\beta_2 = \textbf{EXAG}\lnot b\) gilt, da \(s_0\) Element der Ergebnismenge von 2d) 
					ist.
			\end{enumerate}
			\item[4.]
			\begin{enumerate}
				\item[a)]\quad \\
					\(\textbf{AXAG}a\) bedeutet, dass für alle Pfade im nächsten Zustand gelten muss, dass
					für alle folgenden Pfade der Folge $a$ gilt, also in allen Zuständen (außer dem Root)
					gilt $a$. \\
					\(\textbf{AGAX}a\) bedeutet, dass für alle folgenden Pfade der Folge in allen nächsten
					Zuständen $a$ gelten muss. Also gilt $a$ auch hier immer, außer im Root. \\
					Die beiden Ausdrücke sind also äquivalent.
				\item[b)]\quad \\
					\((\lnot b\land \lnot g)\) beschreibt den Zustand $s_1$ aus dem ersten Teil.
					$\textbf{EXEG}(\lnot b\land \lnot g)$ heißt, dass in einem der nächsten Zustände ein Pfad 
					existiert auf dem $(\lnot b\land \lnot g)$ gilt. Dies ist im $M_{AKW}$ kein einziges mal 
					der Fall, da nach $s_1$ zwangsläufig $s_2$ gilt. \\
					$\textbf{EGEX}(\lnot b\land \lnot g)$ heißt, dass ein Pfad existiert auf dem im folgenden
					Element $(\lnot b\land \lnot g)$ der Fall ist, also ein Pfad der als 2. Zustand $s_1$
					eintrifft, dies ist möglich (siehe 1). \\
					Damit sind die Ausdrücke nicht äquivalent.					
			\end{enumerate}
			\item[5.]
			\begin{enumerate}
				\item[a)]\quad \\
					\(\textbf{AGAX}b\) siehe 4a). \\
					\(\textbf{GX}b\) bedeutet, dass für allgemein im nächsten Zustand $b$ gelten mussdamit 
					gilt für alle Zustände außerhalb des Roots (rekursiver Aufbau). \\
					Also gilt für beide Ausdrücke, dass in jedem Zustand $b$ gilt (außer im Root).
					Damit sind sie äquivalent. 
				\item[b)]\quad \\
					$\textbf{EG}b$ gilt in $M_{AKW}$, da vom Root ein Pfad aus existiert in dem $b$ gilt 
					(siehe 1). $\textbf{G}b$ gilt allerdings nicht, da auch Pfade existieren, auf denen nicht 
					immer b gilt.
			\end{enumerate}
		\end{enumerate}
		
		\item[\textbf{5.4}]
		\begin{enumerate}
			\item[1.]\quad \\ \todo
			\item[2.]\quad \\ \todo
			\item[3.]\quad \\ \todo
			\item[4.]\quad \\ \todo
		\end{enumerate}
	\end{enumerate}
\end{document}
