 % Commands
\newcommand{\authorinfo}{Arne Struck, Tronje Krabbe}
\newcommand{\titleinfo}{FGI 2 [HA], 06. 1. 2014}
\newcommand{\qed}{\ \square}
\newcommand{\todo}{\textcolor{red}{\textbf{TODO}}}

% ------------------------------------------------------

% Packages & Stuff

\documentclass[a4paper,11pt,fleqn]{scrartcl}
\usepackage[german,ngerman]{babel}
\usepackage[utf8]{inputenc}
\usepackage[T1]{fontenc}
\usepackage{lmodern}
\usepackage{amssymb}
\usepackage{amsmath}
\usepackage{enumerate}
\usepackage{fancyhdr}
\usepackage{pgfplots}
\usepackage{multicol}
\usepackage{pst-node}
\usetikzlibrary{calc}
\usetikzlibrary{patterns}
\usetikzlibrary{arrows,automata,positioning}

% ------------------------------------------------------

% Title & Pages

\title{\titleinfo}
\author{\authorinfo}

\pagestyle{fancy}
\fancyhf{}
\fancyhead[L]{\authorinfo}
\fancyhead[R]{\titleinfo}
\fancyfoot[C]{\thepage}

\begin{document}
	\maketitle
	\begin{enumerate}
		\item[\textbf{10.3.}]
		\begin{enumerate}
			\item[1] \quad \\
				Menge aller T-Invarianten:
				\[
        		\bordermatrix{
       					   & a  & b  & c  & d  \cr
				        p1 & 1  & -2 & 2  & -1 \cr
        				p2 & -1 & 3  & -3 & 1  \cr
				        p3 & 0  & -1 & 1  & 0  \cr
				        p4 & 0  & 1  & -1 & 0  \cr}
				\Rightarrow
				\left\{ a
				\begin{pmatrix}
					1 \\ 0 \\ 0 \\ 1
				\end{pmatrix} + b
				\begin{pmatrix}
					0 \\ 1 \\ 1 \\ 0
				\end{pmatrix}
				\right\}
				\]
			\item[2] \quad \\
				\[
        		\begin{pmatrix}
        			3 \\ 3 \\ 1 \\ 1
        		\end{pmatrix} \overset{a}{\rightarrow}
        		\begin{pmatrix}
        			4 \\ 2 \\ 1 \\ 1
        		\end{pmatrix} \overset{d}{\rightarrow}
        		\begin{pmatrix}
        			3 \\ 3 \\ 1 \\ 1
        		\end{pmatrix} \overset{b}{\rightarrow}
        		\begin{pmatrix}
        			1 \\ 6 \\ 0 \\ 2
        		\end{pmatrix} \overset{c}{\rightarrow}
        		\begin{pmatrix}
        			3 \\ 3 \\ 1 \\ 1
        		\end{pmatrix} \overset{b}{\rightarrow}
        		\begin{pmatrix}
        			1 \\ 6 \\ 0 \\ 2
        		\end{pmatrix} \overset{c}{\rightarrow}
        		\begin{pmatrix}
        			3 \\ 3 \\ 1 \\ 1
        		\end{pmatrix}
				\] \\
				\[
				\Rightarrow adbcbc \rightarrow
				\begin{pmatrix}
					1 \\ 2 \\ 2 \\ 1
				\end{pmatrix}
				\]
		\end{enumerate}
		\item[\textbf{10.4.}]
		\begin{enumerate}
			\item[1] \quad \\
			\todo
			\item[2] \quad \\
			\todo
			\item[3] \quad \\
			\todo
			\item[4] \quad \\
			\todo
		\end{enumerate}
		\item[\textbf{10.5.}] \quad \\
		\todo
	\end{enumerate}
\end{document}
