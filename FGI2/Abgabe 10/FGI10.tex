 % Commands
\newcommand{\authorinfo}{Arne Struck, Tronje Krabbe}
\newcommand{\titleinfo}{FGI 2 [HA], 06. 1. 2014}
\newcommand{\qed}{\ \square}
\newcommand{\todo}{\textcolor{red}{\textbf{TODO}}}

% ------------------------------------------------------

% Packages & Stuff

\documentclass[a4paper,11pt,fleqn]{scrartcl}
\usepackage[german,ngerman]{babel}
\usepackage[utf8]{inputenc}
\usepackage[T1]{fontenc}
\usepackage{lmodern}
\usepackage{amssymb}
\usepackage{amsmath}
\usepackage{enumerate}
\usepackage{fancyhdr}
\usepackage{pgfplots}
\usepackage{multicol}
\usepackage{pst-node}
\usetikzlibrary{calc}
\usetikzlibrary{patterns}
\usetikzlibrary{arrows,automata,positioning}

% ------------------------------------------------------

% Title & Pages

\title{\titleinfo}
\author{\authorinfo}

\pagestyle{fancy}
\fancyhf{}
\fancyhead[L]{\authorinfo}
\fancyhead[R]{\titleinfo}
\fancyfoot[C]{\thepage}

\begin{document}
	\maketitle
	\begin{enumerate}
		\item[\textbf{10.3.}]
		\begin{enumerate}
			\item[1] \quad \\
				Menge aller T-Invarianten:
				\[
        		\bordermatrix{
       					   & a  & b  & c  & d  \cr
				        p1 & 1  & -2 & 2  & -1 \cr
        				p2 & -1 & 3  & -3 & 1  \cr
				        p3 & 0  & -1 & 1  & 0  \cr
				        p4 & 0  & 1  & -1 & 0  \cr}
				\Rightarrow
				\left\{ a
				\begin{pmatrix}
					1 \\ 0 \\ 0 \\ 1
				\end{pmatrix} + b
				\begin{pmatrix}
					0 \\ 1 \\ 1 \\ 0
				\end{pmatrix}
				\right\}
				\]
			\item[2] \quad \\
				\[
        		\begin{pmatrix}
        			3 \\ 3 \\ 1 \\ 1
        		\end{pmatrix} \overset{a}{\rightarrow}
        		\begin{pmatrix}
        			4 \\ 2 \\ 1 \\ 1
        		\end{pmatrix} \overset{d}{\rightarrow}
        		\begin{pmatrix}
        			3 \\ 3 \\ 1 \\ 1
        		\end{pmatrix} \overset{b}{\rightarrow}
        		\begin{pmatrix}
        			1 \\ 6 \\ 0 \\ 2
        		\end{pmatrix} \overset{c}{\rightarrow}
        		\begin{pmatrix}
        			3 \\ 3 \\ 1 \\ 1
        		\end{pmatrix} \overset{b}{\rightarrow}
        		\begin{pmatrix}
        			1 \\ 6 \\ 0 \\ 2
        		\end{pmatrix} \overset{c}{\rightarrow}
        		\begin{pmatrix}
        			3 \\ 3 \\ 1 \\ 1
        		\end{pmatrix}
				\] \\
				\[
				\Rightarrow adbcbc \rightarrow
				\begin{pmatrix}
					1 \\ 2 \\ 2 \\ 1
				\end{pmatrix}
				\]
		\end{enumerate}
		\item[\textbf{10.4.}]
		\begin{enumerate}
			\item[1] \quad \\
    			Da für eine Falle gilt, dass alle Transitionen, die Marken aus ihr entfernen, auch wieder Marken in ihr generieren,
                wird es immer mindestens eine Marke in einer Stelle der Falle geben, sobald es irgendwann mal eine gab.
                Das bedeutet, dass eine Falle immer markiert ist, wenn sie in $m_0$ markiert war.
			\item[2] \quad \\
                Eine Transition innerhalb von A ist nur aktiviert bzw. aktivierbar, wenn eine Marke in A liegt. Da der Falle von außen keine Marken gefüttert werden können, muss sie in $m_0$ bereits markiert sein, damit alle Transitionen aktivierbar sind.
			\item[3] \quad \\
                $\{p_5\}$ \\
                $\{p_3,p_4\}$ \\
                $\{p_1,p_2,p_4\}$ \\
                $\{p_1,p_3,p_4\}$ \\
                $\{p_2,p_4,p_5\}$ \\
                $\{p_1,p_2,p_3,p_4\}$ \\
                $\{p_1,p_2,p_4,p_5\}$ \\
                $\{p_1,p_3,p_4,p_5\}$ \\
                $\{p_2,p_3,p_4,p_5\}$ \\
                $\{p_1,p_2,p_3,p_4,p_5\}$
			\item[4] \quad \\
                    $m_0 := \{10011\}^T$ \\
                    a ist bereits aktiviert \\
                    a' ist bereits aktiviert \\
                    $b: m_0 \overset{a}{\rightarrow}$ \\
                    $c: m_0 \overset{a}{\rightarrow}$ \\
                    $c': m_0 \overset{ab}{\rightarrow}$
		\end{enumerate}
		\item[\textbf{10.5.}] \quad \\
		\begin{center}
			\begin{tikzpicture}[>=stealth',shorten >=1pt,auto,node distance=2.5cm,every path/.style={->}]
				%nodes
				\node [circle, draw, minimum size = 1.2cm] (p1) {\(p_1\ p_2\)};
				\node [rectangle, draw, minimum size = 0.7cm] (t1) [right of = p1] {};
				\node [circle, draw, minimum size = 1.2cm] (p2) [right of = t1] {};
				\node [rectangle, draw, minimum size = 0.7cm] (t2) [right of = p2] {};
				\node [circle, draw, minimum size = 1.2cm] (p3) [right of = t2] {};
				\node [rectangle, draw, minimum size = 0.7cm] (t3) [right of = p3] {};
				\node [circle, draw, minimum size = 1.2cm] (p4) [above left = 3cm and 0.15cm of t2] 
					{\(g_1\ g_2\)};
					
				%paths
				\path (p4) edge node[midway, above left] {\(g_1;g_2\)} (t1);
				\path (p4) edge node[midway, left] {\(g_1;g_2\)} (t2);
				\path (t3) edge node[midway, above right] {\(g_1;g_2\)} (p4);
				
				\path (p1) edge node[midway, above] {\(p_1;p_2\)} (t1);
				\path (t1) edge node[midway, above] {\(p_1;p_2\)} (p2);
				\path (p2) edge node[midway, above] {\(p_1;p_2\)} (t2);
				\path (t2) edge node[midway, above] {\(p_1;p_2\)} (p3);
				\path (p3) edge node[midway, above] {\(p_1;p_2\)} (t3);
				
				\path (t3) edge[bend left=30] node[midway, below] {\(p_1;p_2\)} (p1);
			\end{tikzpicture}
		\end{center}
	\end{enumerate}
\end{document}
