 % Commands
\newcommand{\authorinfo}{Arne Struck, Tronje Krabbe}
\newcommand{\titleinfo}{FGI 2 [HA], 27. 1. 2014}
\newcommand{\qed}{\ \square}
\newcommand{\todo}{\textcolor{red}{\textbf{TODO}}}

% ------------------------------------------------------

% Packages & Stuff

\documentclass[a4paper,11pt,fleqn]{scrartcl}
\usepackage[german,ngerman]{babel}
\usepackage[utf8]{inputenc}
\usepackage[T1]{fontenc}
\usepackage{lmodern}
\usepackage{amssymb}
\usepackage{amsmath}
\usepackage{enumerate}
\usepackage{fancyhdr}
\usepackage{pgfplots}
\usepackage{multicol}
\usepackage{pst-node}
\usetikzlibrary{calc}
\usetikzlibrary{patterns}
\usetikzlibrary{arrows,automata,positioning}

% ------------------------------------------------------

% Title & Pages

\title{\titleinfo}
\author{\authorinfo}

\pagestyle{fancy}
\fancyhf{}
\fancyhead[L]{\authorinfo}
\fancyhead[R]{\titleinfo}
\fancyfoot[C]{\thepage}

\begin{document}
	\maketitle
	\begin{enumerate}
		\item[\textbf{13.4.}]
		\begin{enumerate}
			\item[1.] \quad \\
			\begin{tabular}{ccc}
				\begin{tikzpicture}[>=stealth',
									shorten >=1pt,
									auto,
									node distance=1cm,
									every path/.style={->}]
									
					%nodes
					\node (1) {\(\partial_H(((c||d)+p)q \cdot r)\)};
					\node (2) [below =of 1] {\(\partial_H(q \cdot r)\)};	
					\node (3) [below =of 2] {\(\partial_H(r)\)};	
					\node (4) [below =of 3] {\(\surd\)};				
				
					%paths
					\path (1) edge [bend left, midway, right] node{m} (2);
					\path (1) edge [bend right, midway, left] node{p} (2);
					\path (2) edge [midway, left] node{q} (3);
					\path (3) edge [midway, left] node{r} (4);
				\end{tikzpicture}
				& $\quad \quad \quad $&
				\begin{tikzpicture}[>=stealth',
									shorten >=1pt,
									auto,
									node distance=1cm,
									every path/.style={->}]
									
					%nodes
					\node (1) {\(p \cdot q \cdot (r + r) + m\ \mathbb{L}\ q \cdot r \)};
					\node (2) [below =of 1] {\(q \cdot (r + r) + q \cdot r\)};	
					\node (3) [below =of 2] {\(r\)};	
					\node (4) [below =of 3] {\(\surd\)};				
				
					%paths
					\path (1) edge [bend left, midway, right] node{m} (2);
					\path (1) edge [bend right, midway, left] node{p} (2);
					\path (2) edge [midway, left] node{q} (3);
					\path (3) edge [midway, left] node{r} (4);
				\end{tikzpicture}
			\end{tabular} \\ \\
			Bisimulationsrelation: \\ \\
			\(
			\begin{array}{rcl}
				\partial_H(((c||d)+p)q \cdot r) &=& p \cdot q \cdot (r + r) + m\ \mathbb{L}\ q \cdot r \\
				\partial_H(q \cdot r) &=& q \cdot (r + r) + q \cdot r \\
				\partial_H(r) &=& r
			\end{array}
			\)
			\item[2.] \quad \\
			\(
				\partial_H(((c||d)+p)q \cdot r) \overset {\gamma}{=} \partial_H((m+p)q \cdot r) \overset {m}{\rightarrow} \partial_H(q \cdot r)
			\)
			\\ \\
			\(
				p \cdot q \cdot (r + r) + m\ \mathbb{L}\ q \cdot r \overset {m}{\rightarrow} q \cdot r
			\)\\ \\
			Zu der zweiten Transition ist nicht viel zu sagen. Wenn $m$ geschaltet wird, fällt sallopp gesagt $p \cdot q \cdot (r + r)$ weg, und es bleibt $q \cdot r$.
			\item[3.] \quad \\
			\(
			\begin{array}{rcl}
				\partial_H(((c||d)+p)q \cdot r) &\overset {\gamma}{=}& \partial_H((m + p) q \cdot r) \\
				&=& \partial_H(m \cdot q \cdot r + p \cdot q \cdot r) \\
				&=& \partial_H(p \cdot q \cdot r + m \cdot q \cdot r) \\
				&=& p \cdot q \cdot r + m \cdot q \cdot r \\
				&=& p \cdot q \cdot r + m\ \mathbb{L}\ q \cdot r \\
				&=& p \cdot q \cdot (r + r) + m\ \mathbb{L}\ q \cdot r \ \qed \\
			\end{array}
			\)
		\end{enumerate}
	\end{enumerate}
\end{document}
