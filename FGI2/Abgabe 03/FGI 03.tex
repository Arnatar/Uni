% Commands

\newcommand{\authorinfo}{Arne Struck, Tronje Krabbe}
\newcommand{\titleinfo}{FGI 2 [HA], 4. 11. 2013}
\newcommand{\qed}{\ \square}

% ------------------------------------------------------

% Packages & Stuff

\documentclass[a4paper,11pt,fleqn]{scrartcl}
\usepackage[german,ngerman]{babel}
\usepackage[utf8]{inputenc}
\usepackage[T1]{fontenc}
\usepackage{lmodern}
\usepackage{amssymb}
\usepackage{amsmath}
\usepackage{enumerate}
\usepackage{fancyhdr}
\usepackage{pgfplots}
\usepackage{multicol}
\usepackage{pst-node}
\usetikzlibrary{calc}
\usetikzlibrary{patterns}
\usetikzlibrary{arrows,automata}

% ------------------------------------------------------

% Title & Pages

\title{\titleinfo}
\author{\authorinfo}

\pagestyle{fancy}
\fancyhf{}
\fancyhead[L]{\authorinfo}
\fancyhead[R]{\titleinfo}
\fancyfoot[C]{\thepage}

\begin{document}
\maketitle
	\begin{enumerate}
	\item[\textbf{3.3.}]
		\begin{enumerate}
		\item[1.]\quad \\
        	\(
        	\begin{array}{lcl}
        		L(A_1) &=& (bca^* d)^*a^*\\
                L(A_2) &=& (a+b)((a+c)(a+d)(a+b))^*\\
        		L^\omega(A_1) &=& (bca^* d)^{\omega} + (bca^* d)^* a^{\omega}\\
        		L^\omega(A_2) &=& ((a+b)(a+c)(a+d))^{\omega}
	        \end{array}
        	\)
        \item[2.]\quad \\
    	    \begin{tikzpicture}[>=stealth',shorten >=1pt,auto,node distance=2.6cm]
					%nodes
					\node[initial,state] 	(QT)    		   		{$q,t$};
					\node[state,accepting]  (PU) [right of = QT]	{$p,u$};
					\node[state]			(PT) [right of = PU]	{$p,t$};	
					\node[state]			(PV) [right of = PT]	{$p,v$};
						
					\node[state]			(RU) [below of = QT]	{$r,u$};
					\node[state]			(SV) [right of = RU]	{$s,v$};
					\node[state]			(ST) [right of = SV]	{$s,t$};
					\node[state]			(SU) [right of = ST]	{$s,u$};
					
					
					%initial graph
					\path[->]				(QT) edge 				node{a}				(PU);
					\path[->]				(QT) edge 				node{b}				(RU);
					\path[->]				(PU) edge [bend left]  	node{a}				(PV);
					\path[->]				(RU) edge 				node{c}				(SV);
					\path[->]				(SV) edge 				node{a}				(ST);
					\path[->]				(SV) edge 				node{d}				(QT);
					\path[->]				(ST) edge 				node{a}				(SU);
					\path[->]				(PV) edge 				node{a}				(PT);
					\path[->]				(PT) edge 				node{a}				(PU);
					\path[->]				(SU) edge [bend left]	node{a}				(SV);
					
			\end{tikzpicture} \\
            
        \item[3.]\quad \\
        	\(
			\begin{array}{lclcl}
				L(A_3)&=&(bc(aaa)^*d)^*a(aaa)^* \\
   		        L^\omega(A_3)&=&(bc(aaa)^*d)^*a(aaa)^\omega \\ 
	            L(A_1)\cap L(A_2) &=& (bc(aaa)^*d)^* a(aaa)^* &=& L(A_3) \\
                L^\omega(A_1)\cap L^\omega(A_2) &=& (bc(aaa)^*d)^\omega + (bc(aaa)^*d)^* a(a)^\omega
			\end{array}			        	
        	\) \\
            $L(A_3)$ sowie der Schnitt von $L(A_1)$ und $L(A_2)$ sind identisch, was zu erwarten war, da $A_3$ extra konstruiert wurde, um den Schnitt er beiden ursprünglichen Automaten zu akzeptieren. \\
            $L^\omega(A_3)$ ist eine Teilmenge von $L^\omega(A_1)\cap L^\omega(A_2)$.
            \newpage
        \item[4.]\quad \\
        \begin{tikzpicture}[>=stealth',shorten >=1pt,auto,node distance=3cm]
					%nodes
					\node[initial,state, accepting](QT1)  	   		{$q,t,1$};
					\node[state]  			(PU2) [right of = QT1]	{$p,u,2$};
					\node[state]			(PV2) [right of = PU2]	{$p,v,2$};
					\node[state]			(PT2) [right of = PV2]	{$p,t,2$};
                    \node[state,accepting]  (PU1) [below of = PT2]	{$p,u,1$};
					                    
                    \node[state]			(RU2) [below of = QT1]	{$r,u,2$};
                    \node[state]			(SV1) [below of = RU2]	{$s,v,1$};
                    
                    \node[state]			(ST1) [right of = SV1]	{$s,t,1$};
                    \node[state]			(SU1) [right of = ST1]	{$s,u,1$};


					%initial graph
					\path[->]		(QT1)edge		node{a} (PU2);
                    \path[->]		(PU2)edge		node{a} (PV2);
                    \path[->]		(PV2)edge		node{a} (PT2);
                    \path[->]		(PT2)edge		node{a} (PU1);
                    \path[->]		(PU1)edge		node{a} (PV2);
                             
                    \path[->]		(QT1)edge		node{b} (RU2);
                    \path[->]		(RU2)edge		node{c} (SV1);
                    \path[->]		(SV1)edge		node{a} (ST1);
                    \path[->]		(ST1)edge		node{a} (SU1);
                    \path[->]		(SU1)edge [bend left] node{a} (SV1);
                    \path[->]		(SV1)edge [bend left] node{d} (QT1);

					
			\end{tikzpicture} \\
        \item[5.]\quad \\
        	\(
        	\begin{array}{lcl}
            	L(A_4) &=& ((bc(aaa)^*d)^* aaaa(aaa)^*)? \\
                L^\omega(A_4) &=& (bc(aaa)^*d)^\omega + (bc(aaa)^*d)^* a(a)^\omega \\
                L(A_1)\cap L(A_2) &=& (bc(aaa)^*d)^* a(aaa)^*
            \end{array}
            \) \\ \\
            \(L(A_1)\cap L(A_2)\) kann kene Teilmenge von $L(A_4)$ sein, da es ein Wort (a) bilden kann, 
            welches kein Element von $L(A_4)$ ist. umgekehrt kann $L(A_4)$ auch keine Teilmenge von
            \(L(A_1)\cap L(A_2)\), da $L(A_4)$ das leere Wort bilden kann, \(L(A_1)\cap L(A_2)\) aber nicht. 
            \\ \\
            Per Definition akzeptiert der Produktautomat $A_4$, konstruiert nach Satz 1.21, genau den Schnitt
            der Sprachen der Automaten $A_1$ und $A_2$. Daraus folgt: \\
            $L^\omega(A_4) = (bc(aaa)^*d)^\omega + (bc(aaa)^*d)^* a(a)^\omega = L^\omega(A_1)\cap
            L^\omega(A_2)$
        \end{enumerate}
     \item[\textbf{3.4.}]\quad \\
		\textbf{Bisimulationsrelation:} \\ \\
		\(TS_1\underline{\leftrightarrow}TS_3\) und \(TS_3\underline{\leftrightarrow}TS_1:\) \\
    	\(
    		\big\{(z_0,q_0),(z_1,q_2),(z_2,q_1),(z_4,q_0),(z_5,q_2),(z_6,q_1),(z_7,q_2),(z_8,q_0),(z_9,q_2),
    		(z_{10},q_1),(z_{11},q_2),...\big\}
    	\) \\ \\
    	
    	\(TS_2\underline{\leftrightarrow}TS_4\) und \(TS_4\underline{\leftrightarrow}TS_2:\) \\
	    \(
	    \big\{(p_0,r_0),(p_1,r_2),(p_2,r_1),(p_4,r_0),(p_5,r_2),(p_6,r_1),(p_7,r_2),(p_8,r_0),(p_9,r_2),
    		(p_{10},r_1),(p_{11},r_2),...\big\}
	    \) \\ \newpage
	    \textbf{Definitionsnachweis \(TS_1\underline{\leftrightarrow}TS_3\):}
	    \begin{enumerate}
	    	\item[a)]\quad \\
	    		Die Startzustände sind in der ersten Relation als Tupel zu finden.
	    	\item[b)]\quad \\
            \begin{enumerate}
				\item[Beh.:]\quad \\
                \(\forall (z_n,q_m)\in \mathcal{B} |\in \mathbb{N}, m\in[0,1,2] \)\\ \\
					\(z_{4n}\overset{c}{\underset{TS1}{\rightarrow}}z_{4n+1}\ \exists\ q_0 \in 
					TS3:q_0\overset{c}{\underset{TS3}{\rightarrow}}q_2 \land (z_{4n+1},q_2)\in \mathcal{B}\) \\
					
					\(q_0\overset{c}{\underset{TS3}{\rightarrow}}q_2\ \exists\ z_{4n} \in 
					TS1:z_{4n}\overset{c}{\underset{TS1}{\rightarrow}}z_{4n+1} \land (q_2,z_{4n+1})\in
					\mathcal{B}\) \\
					
					\(z_{4n-2}\overset{d}{\underset{TS1}{\rightarrow}}z_{4n-1}\ \exists\ q_1 \in 
					TS3:q_1\overset{d}{\underset{TS3}{\rightarrow}}q_2 \land (z_{4n-1},q_2)\in \mathcal{B}\) \\
					
					\(q_1\overset{d}{\underset{TS3}{\rightarrow}}q_2\ \exists\ z_{4n-2} \in 
					TS1:z_{4n-2}\overset{d}{\underset{TS1}{\rightarrow}}z_{4n-1} \land (q_2,z_{4n-1})\in
					\mathcal{B}\) \\
					
					\(z_{4n-2}\overset{b}{\underset{TS1}{\rightarrow}}z_{4n}\ \exists\ q_0 \in 
					TS3:q_0\overset{b}{\underset{TS3}{\rightarrow}}q_1 \land (z_{4n},q_1)\in \mathcal{B}\) \\
					
					\(q_0\overset{b}{\underset{TS3}{\rightarrow}}q_1\ \exists\ z_{4n-2} \in 
					TS1:z_{4n-2}\overset{b}{\underset{TS1}{\rightarrow}}z_{4n} \land (q_1,z_{4n})\in
					\mathcal{B}\) \\
					
					\(z_{4n}\overset{a}{\underset{TS1}{\rightarrow}}z_{4(n+1)-1}\ \exists\ q_1 \in 
					TS3:q_1\overset{a}{\underset{TS3}{\rightarrow}}q_0 \land (z_{4n},q_0)\in \mathcal{B}\) \\
					
					\(q_1\overset{a}{\underset{TS3}{\rightarrow}}q_0\ \exists\ z_{4n} \in 
					TS1:z_{4n}\overset{a}{\underset{TS1}{\rightarrow}}z_{4(n+1)-1} \land (q_0,z_{4(n+1)-1})\in
					\mathcal{B}\) \\
					
				\item[I.Anf.:]\quad \(n\in [0,1]\) \\
					\(z_0\overset{c}{\rightarrow}z_1\ \exists\ q_0\overset{c}{\rightarrow}q_2\land(z_1,q_2)\) \\
					\(q_0\overset{c}{\rightarrow}q_2\ \exists\ z_0\overset{c}{\rightarrow}z_1\land(z_1,q_2)\) \\
					\(z_2\overset{d}{\rightarrow}z_3\ \exists\ q_1\overset{d}{\rightarrow}q_2\land(z_3,q_2)\) \\
					\(q_1\overset{d}{\rightarrow}q_2\ \exists\ z_2\overset{d}{\rightarrow}z_3\land(z_3,q_2)\)
\\
					\(z_2\overset{b}{\rightarrow}z_4\ \exists\ q_1\overset{b}{\rightarrow}q_0\land(z_4,q_0)\)
\\
					\(q_1\overset{b}{\rightarrow}q_0\ \exists\ z_2\overset{b}{\rightarrow}z_4\land(z_4,q_0)\)
\\
					\(z_0\overset{a}{\rightarrow}z_2\ \exists\ q_0\overset{a}{\rightarrow}q_1\land(z_2,q_1)\)
\\
					\(q_0\overset{a}{\rightarrow}q_1\ \exists\ z_0\overset{a}{\rightarrow}z_2\land(z_2,q_1)\)
\\
				\item[I.A.:]
					Die Behauptung gilt für ein bestimmtes, aber frei wählbares \(n\in \mathbb{N}\).
				\newpage
				\item[I.S.:] (zu zeigen: Da die Bisimulation zyklisch ist, muss ein 4er-Zyklus (I.A.)
				nachgewiesen werden) \\ \\
					Schritt 1:
					\(z_{4(n+1)}\overset{c}{\underset{TS1}{\rightarrow}}z_{4(n+1)+1}\ \exists\ q_0 \in 
					TS3:q_0\overset{c}{\underset{TS3}{\rightarrow}}q_2 \land (z_{4(n+1)+1},q_2)\in
					\mathcal{B}\) \\
					
					\(q_0\overset{c}{\underset{TS3}{\rightarrow}}q_2\ \exists\ z_{4(n+1)} \in 
					TS1:z_{4(n+1)}\overset{c}{\underset{TS1}{\rightarrow}}z_{4(n+1)+1} \land
					(q_2,z_{4(n+1)+1})\in
					\mathcal{B}\) \\
					
					\(z_{4(n+1)-2}\overset{d}{\underset{TS1}{\rightarrow}}z_{4(n+1)-1}\ \exists\ q_1 \in 
					TS3:q_1\overset{d}{\underset{TS3}{\rightarrow}}q_2 \land (z_{4(n+1)-1},q_2)\in
					 \mathcal{B}\) \\
					
					\(q_1\overset{d}{\underset{TS3}{\rightarrow}}q_2\ \exists\ z_{4(n+1)-2} \in 
					TS1:z_{4(n+1)-2}\overset{d}{\underset{TS1}{\rightarrow}}z_{4(n+1)-1} \land 
					(q_2,z_{4(n+1)-1})\in \mathcal{B}\) \\
					
					\(z_{4(n+1)-2}\overset{b}{\underset{TS1}{\rightarrow}}z_{4(n+1)}\ \exists\ q_0 \in 
					TS3:q_0\overset{b}{\underset{TS3}{\rightarrow}}q_1 \land (z_{4(n+1)},q_1)\in \mathcal{B}\) \\
					
					\(q_0\overset{b}{\underset{TS3}{\rightarrow}}q_1\ \exists\ z_{4(n+1)-2} \in 
					TS1:z_{4(n+1)-2}\overset{b}{\underset{TS1}{\rightarrow}}z_{4(n+1)} \land 
					(q_1,z_{4(n+1)})\in \mathcal{B}\) \\
					
					\(z_{4n}\overset{a}{\underset{TS1}{\rightarrow}}z_{4((n+1)+1)-1}\ \exists\ q_1 \in 
					TS3:q_1\overset{a}{\underset{TS3}{\rightarrow}}q_0 \land (z_{4(n+1)},q_0)\in \mathcal{B}\) \\
					
					\(q_1\overset{a}{\underset{TS3}{\rightarrow}}q_0\ \exists\ z_{(4n+1)} \in 
					TS1:z_{4(n+1)}\overset{a}{\underset{TS1}{\rightarrow}}z_{4((n+1)+1)-1} \land 
					(q_0,z_{4((n+1)+1)-1})\in \mathcal{B}\) \\				
				
				Schritt 2:
				
					\(z_{4n+4}\overset{c}{\underset{TS1}{\rightarrow}}z_{4n+1+4}\ \exists\ q_0 \in 
					TS3:q_0\overset{c}{\underset{TS3}{\rightarrow}}q_2 \land (z_{4n+1+4},q_2)\in \mathcal{B}\) \\
					
					\(q_0\overset{c}{\underset{TS3}{\rightarrow}}q_2\ \exists\ z_{4n+4} \in 
					TS1:z_{4n+4}\overset{c}{\underset{TS1}{\rightarrow}}z_{4n+1+4} \land (q_2,z_{4n+1+4})\in
					\mathcal{B}\) \\
					
					\(z_{4n-2+4}\overset{d}{\underset{TS1}{\rightarrow}}z_{4n-1+4}\ \exists\ q_1 \in 
					TS3:q_1\overset{d}{\underset{TS3}{\rightarrow}}q_2 \land (z_{4n-1+4},q_2)\in \mathcal{B}\) \\
					
					\(q_1\overset{d}{\underset{TS3}{\rightarrow}}q_2\ \exists\ z_{4n-2+4} \in 
					TS1:z_{4n-2+4}\overset{d}{\underset{TS1}{\rightarrow}}z_{4n-1+4} \land (q_2,z_{4n-1+4})\in
					\mathcal{B}\) \\
					
					\(z_{4n-2+4}\overset{b}{\underset{TS1}{\rightarrow}}z_{4n+4}\ \exists\ q_0 \in 
					TS3:q_0\overset{b}{\underset{TS3}{\rightarrow}}q_1 \land (z_{4n+4},q_1)\in \mathcal{B}\) \\
					
					\(q_0\overset{b}{\underset{TS3}{\rightarrow}}q_1\ \exists\ z_{4n-2+4} \in 
					TS1:z_{4n-2+4}\overset{b}{\underset{TS1}{\rightarrow}}z_{4n+4} \land (q_1,z_{4n+4})\in
					\mathcal{B}\) \\
					
					\(z_{4n}\overset{a}{\underset{TS1}{\rightarrow}}z_{4(n+1)-1+4}\ \exists\ q_1 \in 
					TS3:q_1\overset{a}{\underset{TS3}{\rightarrow}}q_0 \land (z_{4n+4},q_0)\in \mathcal{B}\) \\
					
					\(q_1\overset{a}{\underset{TS3}{\rightarrow}}q_0\ \exists\ z_{4n+4} \in 
					TS1:z_{4n+4}\overset{a}{\underset{TS1}{\rightarrow}}z_{4(n+1)-1+4} \land 
					(q_0,z_{4(n+1)-1+4})\in
					\mathcal{B}\) \\ \\
					Da jedes z um 4 Stellen verschoben wird, wird der Viererzyklus kompletiert dadurch ist 
					nachgewiesen, \(TS1\underline{\leftrightarrow}TS3\) gilt.
			\end{enumerate}
	    	
	    	\item[c)]\quad \\ 
	    	Da alle \(z_{4n-1}\) und \(z_{4n+1}\) Auf \(q_2\) abgebildet werden und alles
	    	Endzustände sind geht der c)-Teil aus dem b)-Teil hervor. \\
	    	\\
	    	Anmerkung: War das ernsthaft mit "weisen sie für eine davon nach..." gemeint? Wenn nicht in Zukunft 
	    	bitte eindeutiger ausdrücken, da ein Nachweis einem formalen Beweis gleichzusetzen ist. \\
	    \end{enumerate}
	    
	    \textbf{Nichtbisimilaritätsnachweis:} \\
	        TS2 und TS3 sind nicht bisimilar. Versuchte man, die Bisimulationsrelation aufzustellen, würde man
	        auf einen Fehler stoßen: \\
                $TS_2\underline{\leftrightarrow}TS_3: \mathcal{B} = \{(P0,Q0),(P4,Q1)\}$ \\
                Bereits hier ist die Relation Fehlerhaft. Man erreicht P4 von P0 und Q1 von Q0 aus mit einem a,
                jedoch kann nun von P4 aus ein Endzustand mithilfe eines c erreicht werden, von Q1 aus jedoch
                nicht. $\qed$ \\ \\
                TS3 und TS4 sind ebenfalls nicht bisimilar. Wieder wird versucht, die Bisimulationsrelation 
                aufzustellen: \\
                $TS_3\underline{\leftrightarrow}TS_4: \mathcal{B} = \{(Q0, R0),(Q1,R0)\}$ \\
                Von Q0 kann Q1 mit einem a erreicht werden, und von R0 kann R0 ebenfalls mit einem a erreicht
                werden. Jedoch können Q1 und R0 gar nicht bisimilar sein, da R0 ein Startzustand ist und Q1
                nicht. \\
                Unabhängig davon ist es leicht ersichtlich, dass TS3 und TS4 nicht bisimilar sind, da ihr
                Aufbau sehr ähnlich ist. Der einzige Unterschied sind die beiden Schleifen an R0 bzw. R1.
                Hieran erkennt man bereits trivialerweise, dass TS3 und TS4 nicht bisimilar sein können. $\qed$
    \end{enumerate}
\end{document}