% Commands

\newcommand{\authorinfo}{Arne Struck, Tronje Krabbe}
\newcommand{\titleinfo}{FGI 2 [HA], 21. 10. 2013}
\newcommand{\qed}{\ \square}

% ------------------------------------------------------

% Packages & Stuff

\documentclass[a4paper,11pt,fleqn]{scrartcl}
\usepackage[german,ngerman]{babel}
\usepackage[utf8]{inputenc}
\usepackage[T1]{fontenc}
\usepackage{lmodern}
\usepackage{amssymb}
\usepackage{amsmath}
\usepackage{enumerate}
\usepackage{fancyhdr}
\usepackage{pgfplots}
\usepackage{multicol}
\usepackage{pst-node}
\usetikzlibrary{calc}
\usetikzlibrary{patterns}
\usetikzlibrary{arrows,automata}

% ------------------------------------------------------

% Title & Pages

\title{\titleinfo}
\author{\authorinfo}

\pagestyle{fancy}
\fancyhf{}
\fancyhead[L]{\authorinfo}
\fancyhead[R]{\titleinfo}
\fancyfoot[C]{\thepage}


\begin{document}
\maketitle
	\begin{enumerate}
		\item[\textbf{1.3}]
		\begin{enumerate}
			\item[1.]\quad \\
			$L(A_n) = c| acb|aacbb|...|a^ncb^n|a^nc$ \\
			Die Ausdrücke $acb|aacbb$ sind Teil von $a^ncb^n|a^nc$, allerdings die einzige 
			Möglichkeit die Auslassungspunkte zu verwenden und einen richtigen RegEx zu erzeugen
			(welche bekanntermaßen nicht zählen können, was bedeutet, dass diese Sprache nicht in 
			einen RegEx überführbar ist).
			
			\item[2.]\quad \\
			$L(A_n) = \{a^ncb^n\} \cup \{a^nc\}$
			
			\item[3.]\quad \\
			Der Automat erzeugt klar ein Wort, welches eine gewisse Anzahl (inklusive 0) an a's ein c 
			und genau die gleiche Anzahl b's oder $n a's$ und ein $c$ enthält.
			Die gegebenen Antworten spiegeln dies wieder.
			
			\item[4.]\quad \\
			Vorausgesetzt, dass $L(A_n)$ als (echter) regulärer Ausdruck darstellbar ist, muss $A_n$ 
			regulär sein (siehe Definition regulärer Ausdruck).
			
			\item[5.]\quad \\
			$A$ ist regulär wenn, da die Vereinigung mehrerer regulärer Mengen regulär ist.
			Da man reguläre Mengen durch eine Grammatik darstellen kann, ist A auch durch eine 
			Grammatik darstellbar. \\		
		\end{enumerate}
		
		\item[\textbf{1.4}]
		\begin{enumerate}
			\item[1.]\quad \\
				Es müsste die Zustandsmenge $(Q)$ um eine Menge an Zuständen, deren Mächtigkeit der 
				Anzahl der Übergangsfunktionen $(|\delta|)$ entspricht, erweitert werden. Jede 
				Übergangsfunktion wird nun so modifiziert, dass sie zu einem der neuen Zustände führt
				(dies geschieht bijektiv, also also ist jedem neuen Zustand eine Übergangsfunktion 
				zugeordnet). Nun muss die Menge der Übergangsfunktionen so erweitert werden, dass
				von jedem der neuen Zustände eine neue Übergangsfunktion zu dem ursprünglichen Ziel 
				der Übergangsfunktion, welche nun den neuen Zustand als Ziel hat, führt.
			
			\item[3.]\quad \\
				$b^*a\big(a|(ba)\big)^*bb(a|b)^*$
				
			\item[2.]\quad \\
				An dem in 3. gefundenen regulären Ausdruck (dieser ist ablesbar) ist ersichtlich, 
				dass auf jeden Fall 2 aufeinander folgende b's in jedem akzeptierten Wort vorkommen
				müssen. Der Teilausdruck $a\big(a|(ba)\big)^*$, welcher das Zeichen vor den doppelten  
				b bestimmt, zeigt uns, dass auf jeden Fall ein a vor den beiden b's existiert. \\
				Nun ist noch zu klären, ob vor oder nach dem $abb$ effektiv $(a|b)^*$ gilt.
				durch die Schleife an $p_3$ ist dies für nach $abb$ geklärt.
				Vor $abb$ gilt $(a|b)^*$ nur, wenn man die Kombination $abb^+$ ausschließt, durch die 
				Schleife an $p_0$, der Schleife an $p_1$ und der Rückkante von $p_2$ zu $p_1$ sind 
				jegliche Kombinationen von a's und b's mögliche, die nicht $abb$ enthalten.
			
			\item[3.] (Zusatz)\\
				Aus 2. kann man nun auch $(a|b)^*abb(a|b)^*$ ableiten.
				
			\item[4.]\quad \\
				Der Ursprungsautomat: \\ \\
				\begin{tikzpicture}[>=stealth',shorten >=1pt,auto,node distance=2cm]
					\node[initial,state] 	(S)    			   {$p_0$};					
					\node[state]			(A) [right of = S] {$p_1$};					
					\node[state]			(B) [right of = A] {$p_2$};					
					\node[state,accepting] 	(C) [right of = B] {$p_3$};
					
					\path[->]				(S) edge				node {a}	(A);
					\path[->]				(A) edge [bend left]	node {b}	(B);
					\path[->]				(B) edge				node {b}	(C);
					\path[->]				(B) edge [bend left]	node {a}	(A);
					\path[->]				(S) edge [loop above]	node {b}	(S);
					\path[->]				(A) edge [loop above]	node {a}	(A);
					\path[->]				(C) edge [loop above]	node {a,b} 	(C);
				\end{tikzpicture}
				\\
				Die Zusatzzustände werden hinzugefügt: \\ \\
				\begin{tikzpicture}[>=stealth',shorten >=1pt,auto,node distance=2cm]
					\node[initial,state] 	(S)    								{$p_0$};
					\node[state]		 	(SS) [above of = S]  				{$p_{00}$};
					\node[state]			(SA) [right of = S, above of = S]	{$p_{01}$};
					\node[state]			(A)  [right of = SA, below of = SA] {$p_1$};
					\node[state]			(AA) [above of = A]  				{$p_{11}$};
					\node[state]			(AB) [right of = A, above of = A]	{$p_{12}$};
					\node[state]			(BA) [right of = A, below of = A]	{$p_{21}$};
					\node[state]			(B)  [right of = AB, below of = AB] {$p_2$};
					\node[state]			(BC) [right of = B, below of = B]	{$p_{23}$};
					\node[state,accepting] 	(C)  [below of = BA, right of = BA] {$p_3$};
					\node[state]			(CC) [left of = C]					{$p_{331}$};
					\node[state]			(CCC)[right of = C]					{$p_{332}$};
					
					\path[->]				(S) edge				node {a} 	(A);
					\path[->]				(A) edge [bend left]	node {b} 	(B);
					\path[->]				(B) edge				node {b} 	(C);
					\path[->]				(B) edge [bend left]	node {a} 	(A);
					\path[->]				(S) edge [loop above]	node {b} 	(S);
					\path[->]				(A) edge [loop above]	node {a} 	(A);
					\path[->]				(C) edge [loop below]	node {a,b}	(C);
				\end{tikzpicture}				
				\\ 
				Die ursprünglichen Übergangsfunktionen werden modifiziert: \\ \\
				\begin{tikzpicture}[>=stealth',shorten >=1pt,auto,node distance=2cm]
					\node[initial,state] 	(S)    								{$p_0$};
					\node[state]		 	(SS) [above of = S]  				{$p_{00}$};
					\node[state]			(SA) [right of = S]					{$p_{01}$};
					\node[state]			(A)  [right of = SA] 				{$p_1$};
					\node[state]			(AA) [above of = A]  				{$p_{11}$};
					\node[state]			(AB) [right of = A]					{$p_{12}$};
					\node[state]			(BA) [right of = A, below of = A]	{$p_{21}$};
					\node[state]			(B)  [right of = AB] 				{$p_2$};
					\node[state]			(BC) [below of = B]					{$p_{23}$};
					\node[state,accepting] 	(C)  [below of = BA, right of = BA] {$p_3$};
					\node[state]			(CC) [left of = C]  				{$p_{331}$};
					\node[state]			(CCC)[right of = C]					{$p_{332}$};
					
					\path[->]				(S) edge				node {a} 	(SA);
					\path[->]				(A) edge				node {b} 	(AB);
					\path[->]				(B) edge				node {b} 	(BC);
					\path[->]				(B) edge 				node {a} 	(BA);
					\path[->]				(S) edge [bend left]	node {b} 	(SS);
					\path[->]				(A) edge [bend left]	node {a} 	(AA);
					\path[->]				(C) edge [bend left]	node {a}	(CC);
					\path[->]				(C) edge [bend left]	node {b}	(CCC);
				\end{tikzpicture}				
				\\
				Die neuen Übergangfunktionen komplettieren $A'$: \\ \\
				\begin{tikzpicture}[>=stealth',shorten >=1pt,auto,node distance=2cm]
					\node[initial,state] 	(S)    								{$p_0$};
					\node[state]		 	(SS) [above of = S]  				{$p_{00}$};
					\node[state]			(SA) [right of = S]					{$p_{01}$};
					\node[state]			(A)  [right of = SA] 				{$p_1$};
					\node[state]			(AA) [above of = A]  				{$p_{11}$};
					\node[state]			(AB) [right of = A]					{$p_{12}$};
					\node[state]			(BA) [right of = A, below of = A]	{$p_{21}$};
					\node[state]			(B)  [right of = AB] 				{$p_2$};
					\node[state]			(BC) [below of = B]					{$p_{23}$};
					\node[state,accepting] 	(C)  [below of = BA, right of = BA] {$p_3$};
					\node[state]			(CC) [left of = C]  				{$p_{331}$};
					\node[state]			(CCC)[right of = C]					{$p_{332}$};
					
					\path[->]				(S) edge				node {a} 	(SA);
					\path[->]				(A) edge 				node {b} 	(AB);
					\path[->]				(B) edge				node {b} 	(BC);
					\path[->]				(B) edge 				node {a} 	(BA);
					\path[->]				(S) edge [bend left]	node {b} 	(SS);
					\path[->]				(A) edge [bend left]	node {a} 	(AA);
					\path[->]				(C) edge [bend left]	node {a}	(CC);
					\path[->]				(C) edge [bend left]	node {b}	(CCC);
					
					\path[->]				(SA) edge				node {a} 	(A);
					\path[->]				(SS) edge [bend left]	node {b} 	(S);
					\path[->] 				(AA) edge [bend left] 	node {a}	(A);
					\path[->]				(AB) edge				node {b}	(B);
					\path[->]				(BA) edge				node {a}	(A);
					\path[->]				(BC) edge				node {b}	(C);
					\path[->]				(CC) edge [bend left]   node {a}	(C);
					\path[->]				(CCC)edge [bend left]   node {b}	(C);
				\end{tikzpicture}
				

			\item[5.]\quad \\
				$\forall\ \delta : Q\times\Sigma\rightarrow Q$ gilt die Modifikation: \\
				$ \delta_n : p_n\in Q\times x\in\Sigma\rightarrow p_{n+m}\in Q\times x\rightarrow 
				p_n$ jedes n, n+m sei einmalig. \\
				Damit ist die Spezifikation zwangsläufig erfüllt.
			
		\end{enumerate}
	\end{enumerate}

\end{document}