% Commands

\newcommand{\authorinfo}{Arne Struck, Tronje Krabbe}
\newcommand{\titleinfo}{FGI 2 [HA], 28. 10. 2013}
\newcommand{\qed}{\ \square}

% ------------------------------------------------------

% Packages & Stuff

\documentclass[a4paper,11pt,fleqn]{scrartcl}
\usepackage[german,ngerman]{babel}
\usepackage[utf8]{inputenc}
\usepackage[T1]{fontenc}
\usepackage{lmodern}
\usepackage{amssymb}
\usepackage{amsmath}
\usepackage{enumerate}
\usepackage{fancyhdr}
\usepackage{pgfplots}
\usepackage{multicol}
\usepackage{pst-node}
\usetikzlibrary{calc}
\usetikzlibrary{patterns}
\usetikzlibrary{arrows,automata}

% ------------------------------------------------------

% Title & Pages

\title{\titleinfo}
\author{\authorinfo}

\pagestyle{fancy}
\fancyhf{}
\fancyhead[L]{\authorinfo}
\fancyhead[R]{\titleinfo}
\fancyfoot[C]{\thepage}

\begin{document}
\maketitle
	\begin{enumerate}
	\item[\textbf{2.3.}]
		\begin{enumerate}
		\item[1.]\quad \\
        	\(
        	\begin{array}{lcl}
        		L(A_{2.3}) &=& (a^+ + b(cd)^* (c+e))\\
        		L^\omega(A_{2.3}) &=& (a)^{\omega} + b(cd)^{\omega}\\
        		\big(L(A_{2.3})\big)^\omega &=& (a^+ + b(cd)^* (c+e))^{\omega}
	        \end{array}
        	\)
        \item[2.]\quad \\
        	$L^\omega(A_{2.3})$ bezieht sich auf auf den Automaten $A_{2.3}$ und verändert die Akzeptierte 
        	Sprache, während $\big(L(A_{2.3})\big)^\omega$ die akzeptierte Sprache in eine neue 
        	$\omega$-Sprache verwandelt. \\ \\
 			\(
 			\begin{array}{lcl}
 				L^\omega(A_{2.3}) &:& (a)^\omega\\
 					&&b(cd)^\omega \\
	 			\big(L(A_{2.3})\big)^\omega &:& (be)^\omega \\
	 				&& (bc)^\omega
 			\end{array}
 			\)       	
        \item[3.]\quad \\
        \begin{tikzpicture}[>=stealth',shorten >=1pt,auto,node distance=2cm]
					%nodes
					\node[initial,state] 	(S)    			   {$q_1$};					
					\node[state,accepting]	(A) [below of = S] {$q_0$};
					\node[state]			(B) [right of = S] {$q_2$};
					\node[state,accepting]	(C) [above of = B] {$q_3$};
					\node[state,accepting]	(D) [below of = B] {$q_4$};
					
					%initial graph
					\path[->]				(S) edge [bend right]	node{a}				(A);
					\path[->]				(A) edge [loop left]	node{a}				(A);
					\path[->]				(S) edge 				node{b}				(B);
					\path[->]				(B) edge [bend right]	node{c}				(C);
					\path[->]				(C) edge [bend right]	node{d}				(B);
					\path[->]				(B) edge 				node{e}				(D);
					
					%epsilon edges
					\path[->]				(A) edge [bend right]	node{$\epsilon$}	(S);
					\path[->]				(D) edge 				node{$\epsilon$}	(S);
					\path[->]				(C) edge 				node{$\epsilon$}	(S);
				\end{tikzpicture} \\
	Die Korrektheit des Automaten wird in Aufgabe 2.4 beschrieben, denn er wurde mit dem dort beschriebenen Verfahren konstruiert.
        \end{enumerate}
    \item[\textbf{2.4.}]
	    \begin{enumerate}
    	\item[1.]\quad \\
        	Wenn U eine reguläre Menge ist, dann ist $U^\omega$ die Menge aller abzählbar unendlichen Konkatenationen von Worten aus U. \\
        	Es soll ein Verfahren gefunden werden, das aus einem beliebigen NFA, der U akzeptiert, einen Büchi-Automaten erstellt, der $U^\omega$ akzeptiert.
        \newpage
    	\item[2.]\quad \\
        	Wenn der NFA, der U akzeptiert, mehrere Startzustände hat, mache diese zu normalen Zuständen, und füge einen neuen Startzustand hinzu, der mit $\epsilon$-Kanten zu jedem der originalen Startzuständen führt. \\
        	An jeden Endzustand des NFA wird nun eine $\epsilon$-Kante zurück zu dem Startzustand hinzugefügt.
    	\item[3.]\quad \\
        	Mit dem Verfahren aus 2. kann mit einem $\omega$-Wort, das aus einer unendlichen Konkatenation aus Worten aus U besteht, mindestens ein Endzustand unendlich oft durchlaufen werden, was die Akzeptanzbedingung eines Büchi-Automaten ist.
    	\item[4.]\quad \\
        	Aufgabe 2.3.3. wurde mit dem hier beschriebenen Verfahren gelöst.
   		\end{enumerate}
    \end{enumerate}
\end{document}