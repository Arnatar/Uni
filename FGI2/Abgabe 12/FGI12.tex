 % Commands
\newcommand{\authorinfo}{Arne Struck, Tronje Krabbe}
\newcommand{\titleinfo}{FGI 2 [HA], 20. 1. 2014}
\newcommand{\qed}{\ \square}
\newcommand{\todo}{\textcolor{red}{\textbf{TODO}}}

% ------------------------------------------------------

% Packages & Stuff

\documentclass[a4paper,11pt,fleqn]{scrartcl}
\usepackage[german,ngerman]{babel}
\usepackage[utf8]{inputenc}
\usepackage[T1]{fontenc}
\usepackage{lmodern}
\usepackage{amssymb}
\usepackage{amsmath}
\usepackage{enumerate}
\usepackage{fancyhdr}
\usepackage{pgfplots}
\usepackage{multicol}
\usepackage{pst-node}
\usetikzlibrary{calc}
\usetikzlibrary{patterns}
\usetikzlibrary{arrows,automata,positioning}

% ------------------------------------------------------

% Title & Pages

\title{\titleinfo}
\author{\authorinfo}

\pagestyle{fancy}
\fancyhf{}
\fancyhead[L]{\authorinfo}
\fancyhead[R]{\titleinfo}
\fancyfoot[C]{\thepage}

\begin{document}
	\maketitle
	\begin{enumerate}
		\item[\textbf{12.3.}]
		\begin{enumerate}
			\item[1.] \quad \\
			\begin{tabular}{ccc}
				\begin{tikzpicture}[>=stealth',
									shorten >=1pt,
									auto,
									node distance=2.5cm,
									every path/.style={->},
									every state/.style={circle, draw, minimum size = 0.75cm}]
				
					%nodes
					\node[state] (1) {1};
					\node[state] (2) [below left of =1] {2};
					\node[state] (3) [below right of = 1] {3};
					\node[state] (4) [below of = 1] {4};
					\node[state] (5) [below left of = 4] {5};
					\node[state] (6) [below of = 5] {6};
					\node[state] (7) [right =3cm of 6] {7};
					
					%paths
					\path (1) edge [midway, above left]  node{abb} (2);
					\path (1) edge [midway, above right]  node{b} (3);
					\path (1) edge [midway, left]  node{b} (4);
					\path (4) edge [bend left = 10, midway, right]  node{b} (7);
					\path (5) edge [bend left, midway, above left]  node{b} (4);
					\path (4) edge [bend left, midway, below right]  node{b} (5);
					\path (5) edge [midway, left]  node{b} (6);
				\end{tikzpicture} &\quad\quad\quad&
				\begin{tikzpicture}[>=stealth',
									shorten >=1pt,
									auto,
									node distance=2.5cm,
									every path/.style={->},
									every state/.style={circle, draw, minimum size = 0.75cm}]
				
					%nodes
					\node[state] (d) {d};
					\node[state] (g) [below right of = d] {g};
					\node[state] (h) [right of = g] {h};
					\node[state] (e) [below left of = d] {e};
					\node[state] (f) [below of = d] {f};
					\node[state] (i) [below of = f] {i};
					\node[state] (j) [below of = h] {j};
					\node[state] (k) [below of = i] {k};
					
					%paths
					\path (d) edge [midway, above left]  node{b} (e);
					\path (d) edge [bend right = 20, midway, left]  node{a} (f);
					\path (d) edge [bend left = 20, midway, right]  node{b} (f);
					\path (d) edge [midway, above right]  node{b} (g);
					\path (d) edge [bend left = 20, midway, above right]  node{b} (h);
					\path (h) edge [bend right = 20, midway, left]  node{b} (j);
					\path (h) edge [bend left = 20, midway, right]  node{bb} (j);
					\path (g) edge [bend left = 20, midway, right] node{b} (k);
					\path (f) edge [midway, right] node{b} (i);
					\path (i) edge [midway, right] node{b} (k);
					\path (f) edge [bend right = 30, midway, left] node{b} (k);
					
				\end{tikzpicture}				
			\end{tabular} \\ \\
			\item[2.] \quad \\
			\begin{tabular}{ccccc}
				\(
				\begin{array}{ccc}
					\text{d} & \overset{\wedge}{=} & 1 \\
					\text{e} & \overset{\wedge}{=} & 3 \\
					\text{f} & \overset{\wedge}{=} & 4
				\end{array}
				\)
				&\quad \quad \quad &
				\(
				\begin{array}{ccc}
					\text{g} & \overset{\wedge}{=} & 5 \\
					\text{h} & \overset{\wedge}{=} & 4 \\
					\text{i} & \overset{\wedge}{=} & 5 
				\end{array}
				\)
				&\quad \quad \quad &
				\(
				\begin{array}{ccc}
					\text{j} & \overset{\wedge}{=} & 7 \\
					\text{k} & \overset{\wedge}{=} & 6 \\
					\text{k} & \overset{\wedge}{=} & 2 
				\end{array}
				\)
			\end{tabular}
			\item[3.] \quad \\
				Da alle Knoten eine Entsprechung aufweisen sind die Graphen bisimilar.
			
		\end{enumerate}
		\item[\textbf{12.4.}] \quad \\
		Siehe ungetexter Part. \\
		\item[\textbf{12.5.}]
		\begin{enumerate}
			\item[1.] \quad \\
			\(
			\begin{array}{cl}
				& (x + y) + (z + z) \\
				\overset{A3}{=} & (x + y) + z \\
				\overset{A1}{=} & z + (x + y) \\
				\overset{A1}{=} & z + (y + x) \\
				\overset{A2}{=} & (z + y) + x \\
				\overset{A1}{=} & (y + z) + x \\
				\overset{A2}{=} & y + (z + x) \\				
				\overset{A3}{=} & (y + y) + (z + x)
			\end{array}
			\)
			
			\item[3.] \quad \\
			Unter der Voraussetzung, dass 12.5.2 gelöst wurde. \\
			\(
			\begin{array}{cl}
				& (x + y) + z  \\
				\overset{A3}{=} & (x + x) + (y + y) + (z + z) \\
				\overset{A1}{=} & (x + x) + (z + z) + (y + y) \\
				\overset{A6}{=} & (x + x) + (z + z) + (y + z) \\
				\overset{A1}{=} & (x + x) + (y + z) + (z + z) \\
				\overset{A3}{=} & x + (y + z) \\
			\end{array}
			\)

		\end{enumerate}
	\end{enumerate}
\end{document}
