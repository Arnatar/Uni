\documentclass[a4paper,11pt,fleqn]{scrartcl}
\usepackage[german,ngerman]{babel}
\usepackage[utf8]{inputenc}
\usepackage[T1]{fontenc}
\usepackage[top=1.3in, bottom=1.2in, left=0.9in, right=0.9in]{geometry}
\usepackage{lmodern}
\usepackage{amssymb}
\usepackage{amsmath}
\usepackage{enumerate}
\usepackage{fancyhdr}
\usepackage{color}
\usepackage{url}

% ------------------------------------------------------

% Commands

\newcommand{\todo}{\textcolor{red}{\textbf{TODO}}}
\newcommand{\authorinfo}{Arne Struck, Knut Götz}
\newcommand{\titleinfo}{GWV-Abgabe zum 31.10.2014}
\newcommand{\qed}{\ \square}


% ------------------------------------------------------

% Title & Pages

\title{\titleinfo}
\author{\authorinfo}

\pagestyle{fancy}
\fancyhf{}
\fancyhead[L]{\authorinfo}
\fancyhead[R]{\titleinfo}
\fancyfoot[C]{\thepage}

\begin{document}
\maketitle

\section*{1.1}
\subsection*{1.}
Als Heuristik für A* bietet sich die Manhattan-Distanz von einem gegebenen State zu dem Ziel an. Sie wird bestimmt durch die Summe der absoluten Differenzen der Koordinaten und ist gut für Abschätzungen in einer rasterartigen Umgebung.
Zusätzlich zur Manhattan-Distanz des States müssen auch noch die bereits vom Start zurückgelegte Distanz vom Startknoten zum State berücksichtigt werden.
Diese wird dann zur Manhattan-Distanz addiert, um so aus Sackgassen zu gelangen, die eine kleine Manhattan Distanz zum Ziel haben, aber in denen der direkte Weg zum Zielknoten blockiert ist.\\
Eine typische Ausgabe unseres Algorithmus ('p' markiert den Pfad):\\
\begin{verbatim}
Starte A*:
Pfad gefunden:
xxxxxxxxxxxxxxxxxxxx
x      pppppppp    x
x      pxxx   pp   x
x      px xxxxxp   x
x   sppp  x pppp   x
x       x x pxxxxxxx
x  xx xxxxx ppp    x
x      x      g    x
x      x           x
xxxxxxxxxxxxxxxxxxxx
\end{verbatim}

\subsection*{2.}
Bei unserer Implementation des A*-Algorithmus ist eine Terminierung sichergestellt, da alle bereits abgearbeiteten States in einer Menge gespeichert werden und neue explorte States, die beriets 
abgearbeitet (in der Menge enthalten sind) wurden, nicht erneut betrachtet werden. Wenn alle möglichen States betrachtet wurden, bricht der Algorithmus ab und gibt aus, dass er keinen Pfad finden konnte.
Ausgabe dann:
\begin{verbatim}
Field:
xxxxxxxxxxxxxxxxxxxx
x                  x
x       xxx        x
x       x xxxxx    x
x   s     x        x
x       x x  xxxxxxx
x  xx xxxxx  x     x
x      x     xg    x
x      x     x     x
xxxxxxxxxxxxxxxxxxxx

Starte A*:
Kein Pfad auffindbar
\end{verbatim}

\subsection*{3.}

\subsection*{4.}

\end{document}