\documentclass{beamer}
\usepackage[utf8]{inputenc}
\usepackage{lmodern}
\usepackage[ngerman]{babel}
\usepackage{listings}
\usepackage{hyperref}
\usepackage{color}
\usepackage{todonotes}


\definecolor{darkred}{rgb}{0.75,0,0.3}
\usetheme{Ilmenau}
\usecolortheme{beaver}
\setbeamercovered{invisible}
\beamertemplatenavigationsymbolsempty % macht die Navigationsleiste weg
\setbeamercolor{block body}{bg=darkred!7.5}
\setbeamercolor{block title}{bg=darkred}
\setbeamercolor*{item}{fg=darkred!90}

\lstset{
  basicstyle=\footnotesize
}

\title{Ist die Disruption der Demokratie noch aufzuhalten?}
\subtitle{}
\author{Arne Struck \& Kim Möller}
\institute{Universität Hamburg, Fachschaft Informatik, Des Googles Kern}
\date{\today}

\begin{document}
\begin{frame}
\maketitle
\end{frame}

\begin{frame}{}
\tableofcontents
\end{frame}

\section{Grundlage des Datenschutzes}
\subsection{Datenschutzrichtlinien}
\begin{frame}{Die Entwicklung des Datenschutzes}
\begin{figure}[h]
\begin{center}
	\includegraphics[scale=0.28]{pics/datenschutz.png}
\end{center}
\caption{Entwicklung des Datenschutzes (Quelle: \cite{europData})}
\label{pic:datenschutz}
\end{figure}
\end{frame}

\begin{frame}{Richtlinie 95/46/EG}
\begin{itemize}
	\item Richtlinie der Europäischen Gemeinschaft
	\item 1995 erlassen
	\item Schutz der Privatsphäre von natürlichen Personen bei der Verarbeitung von personenbezogenen Daten
	\item Drittstaatenregelung
\end{itemize}
\end{frame}



\begin{frame}{Weltweiter Stand des Datenschutzes}
\begin{figure}[h]
\begin{center}
	\includegraphics[scale=0.28]{pics/datenschutzstand.png}
\end{center}
\caption{Einschätzung des weltweiten Standes zum Thema Datenschutz}
\label{pic:worldmap}
\end{figure}
\end{frame}

\subsection{Safe harbor Abkommen}
\begin{frame}{Safe harbor Abkommen}
\underline{Safe harbor privacy principles}
\begin{itemize}
	\item Informationspflicht
	\item Wahlmöglichkeit
	\item Weitergabe
	\item Sicherheit
	\item Datenintegrität
	\item Auskunftsrecht
	\item Durchsetzung
\end{itemize}
\end{frame}

\section{Chancen des Datenschutzes}
\subsection{Datenschutz-Grundverordnung}

\begin{frame}{Datenschutz-Grundverordnung}
\begin{itemize}
\item Teil der beabsichtigten Datenschutzreform der Europäischen Kommission
\item Soll die Richtlinien von 1995 ersetzten
\item Zweckbindung, Transparenz, Nutzerprofile, Strafen/Sanktionen ?
\item US-amerikanische Lobbyarbeit
\item Wann wird die Verordnung verabschiedet?
\end{itemize}
\end{frame}


\section{Netzpolitik}
\subsection{Netzneutralität}
\begin{frame}{Netzneutralität}
\begin{block}{Definition}
	Gleichberechtigte Transport aller Daten in Datennetzen
\end{block}
\uncover<2>{
\quad \\
\underline{Status:}
\begin{itemize}
	\item EU: Parlament (meist) pro, Rat contra
	\item USA: Obama spricht für gesetzlich festgelegte Netzneutralität
	\item Deutschland: Regierung (durch Merkel) spricht sich gegen Netzneutralität aus
	\item Google: Pro (lange Zeit still)
\end{itemize}
}
\end{frame}

\begin{frame}{Die Spitzenpolitik zu Netzneutralität}
\begin{center}
\href{./oettinger.mp4}{
	\centering
	\includegraphics[scale=0.15]{pics/Play-button.png}
}
\end{center}
\end{frame}

\begin{frame}{Pro Contra}
\begin{columns}[t]
    \begin{column}{.5\textwidth}      	
		\begin{block}{Contra}
       	\begin{itemize}
			\item Netzneutralität tötet
			\item stark steigende Datenvolumen \(\Rightarrow\) Flussregulierung
			\item Breitbandausbau muss finanziert werden
			\item Entscheidungsfreiheit der Netzeigentümer
			\item keine Quersubventionierung von Big Usern
       	\end{itemize}
		\end{block}
    \end{column}
	\begin{column}{.5\textwidth}
		\uncover<2> {
      	\begin{block}{Pro}
		\begin{itemize}
			\item unerlässlich für Startups \(\Rightarrow\) Konkurrenz für Monopole
			\item diskriminierungsfreies Netz Demokratievoraussetzung
			\item Netzneutralität nutzt Bevölkerungsmehrheit
			\item nicht kommerzielle Projekte
			\item kommunikative Chancengleichheit (Recht)
		\end{itemize}
		\end{block}		
		}
	\end{column}
\end{columns}
\end{frame}


\section{Konzerne übernehmen das Internet}
\subsection{Internet.org}
\begin{frame}{Was ist Internet.org}
\begin{itemize}
	\item Non Profit Organisation
	\item Kooperation mehrerer namhafter Unternehmen, initiiert von Facebook
	\item Ziele: 
	\begin{itemize}%[<+->]
		\item kostenloses (Grund-)Internet für die Welt
		\item Effiziente Lösung
		\item Kooperation als Geschäftsmodell
	\end{itemize}
\end{itemize}
\end{frame}

\begin{frame}{Probleme}
\begin{itemize}
	\item Kein echter Internetzugang, da nur von Facebook akzeptierte Dienstleistungen zugelassen (bspw: zero rating Klausel)
	\item erhobene Nutzungsdaten gehören Facebook
	\item einseitige nachträgliche Vertragsänderungen seitens Facebook möglich
	\item alle ''inkompatiblen'' Seiten nicht über Internet.org erreichbar
	\item in Drittweltländern mögliche Konkurrenz zu Grundbedürfnissen (finanziell)
	\item unsicher (bspw. kein TLS/SSL/HTTPS)
\end{itemize}
\end{frame}

\begin{frame}{Rezeption des Internets}
\begin{figure}[h]
\begin{center}
	\includegraphics[scale=0.35]{pics/internetvsfacebook.png}
\end{center}
\caption{\begin{footnotesize}
Internet- und Facebooknutzer in Prozent der Bevölkerung (Quelle: \cite{quartz:inetvfb})
\end{footnotesize}
}
\label{pic:inetvfb}
\end{figure}
\end{frame}

\section*{Quellen}
\setbeamertemplate{bibliography item}{\insertbiblabel}
\setbeamercolor{bibliography item}{parent=palette primary}
\setbeamercolor*{bibliography entry title}{parent=palette primary}
\begin{frame}[shrink=10]{Quellen}
\bibliographystyle{alpha}
\bibliography{literature}
\end{frame}

\end{document}
