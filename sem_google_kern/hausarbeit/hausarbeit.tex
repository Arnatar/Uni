\documentclass[
	12pt,
	a4paper,
	BCOR10mm,
	%chapterprefix,
	DIV14,
	listof=totoc,
	bibliography=totoc,
	headsepline
]{scrreprt}

\usepackage[T1]{fontenc}
\usepackage[utf8]{inputenc}
\usepackage[ngerman]{babel}

\usepackage{lmodern}

\usepackage[footnote]{acronym}
\usepackage[page,toc]{appendix}
\usepackage{fancyhdr}
\usepackage{float}
\usepackage{graphicx}
\usepackage[hyphens]{url}
\usepackage[pdfborder={0 0 0}]{hyperref}
\usepackage{listings}
\usepackage{lscape}
\usepackage{microtype}
\usepackage{nicefrac}
\usepackage{subfig}
\usepackage{textcomp}
\usepackage[subfigure,titles]{tocloft}
\usepackage{units}
\usepackage{pgf}
\usepackage{amsmath}
\usepackage{placeins}
\usepackage{titletoc}
\usepackage{todonotes}

\lstset{
	basicstyle=\ttfamily,
	frame=single,
	numbers=left,
	language=C,
	breaklines=true,
	breakatwhitespace=true,
	postbreak=\hbox{$\hookrightarrow$ },
	showstringspaces=false,
	tabsize=4
}

\renewcommand*{\lstlistlistingname}{Listing catalog}

\renewcommand*{\appendixname}{Appendix}
\renewcommand*{\appendixtocname}{Appendices}
\renewcommand*{\appendixpagename}{Appendices}

\newboolean{todos}
\setboolean{todos}{false}

\begin{document}
\begin{titlepage}
	\begin{center}
		{\titlefont\huge Handelt die Politik netzpolitisch im Sinne der Bevölkerung? \par}

		\bigskip
		\bigskip

		{\titlefont\Large --- Hausarbeit ---\par}

		\bigskip
		\bigskip

		{\large Seminar: Des Googles Kern\\
		Fachbereich Informatik\\
		Fakultät für Mathematik, Informatik und Naturwissenschaften\\
		Universität Hamburg\par}
	\end{center}

	\vfill

	{\large \begin{tabular}{ll}
		Vorgelegt von: & Arne Struck \\
		E-Mail-Adresse: 
			& \href{mailto:1struck@informatik.uni-hamburg.de}{1struck@informatik.uni-hamburg.de} \\ 
		%Matrikelnummer: & 1234567 \\
		Studiengang: & Bsc. Informatik \\
		\\
		Veranstalter: & Prof. Dr. Arno Rolf\\
		\\
		Hamburg, den \today
	\end{tabular}\par}
\end{titlepage}

\thispagestyle{empty}

\newpage\null\thispagestyle{empty}\newpage
\tableofcontents
\newpage\null\thispagestyle{empty}\newpage

\chapter{Einleitung}
\label{intro}

\section{Themenumriss Netzpolitik}
\label{themeintro}
Nach nun ca. 25 Jahren Netzpolitik hat diese eine große Veränderung hinter sich.
Was um 1990, als das Internet anfing durch die Entwicklung der Grundlagen des World Wide Webs sich als Massenmedium zu etablieren, noch obskure Gedanken einiger weniger war, hat sich heutzutage als immer wichtiger werdender Teil der Bürgerrechte etabliert.
Zumindest legen die in letzter Zeit immer stärker in die Öffentlichkeit tretenden Diskussionen zwischen Bürgerrechtlern, Aktivisten auf der einen Seite und den verschiedensten Interessenvertretern auf der anderen Seite diesen Schluss nahe. 

Medien sind für eine lebendige Demokratie im westlichen Sinne eine Grundvoraussetzung, unter anderem da der Wähler eine informierte Entscheidung treffen muss.
Nun hat sich das Internet zu einer der Hauptinformationsquellen für die Generationen unter 50 entwickelt, wenn es nicht gar die wichtigste für diese ist.
Somit ist es nicht verwunderlich, dass Aktivistengruppen das Internet trotz all seiner Schwächen von äußeren Einflüssen beschützen wollen.
Aus solchen Beweggründen resultieren Grundsätze wie beispielsweise: Ein freies Netz ist eine wichtige Grundlage für unsere Gesellschaft und daher zu bewahren. 

Da das Internet als Teil der modernen Informationstechnologie relativ schnellen, massiven Veränderungen unterliegt, ist es sowohl für Politik als auch für alle Interessenverbände schwierig Handlungsweisen zu entwickeln, welche ihren Interessen entsprechen.
Ein weiteres Problem in der Netzpolitik stellen selbstredend die verschiedenen Interessenlagen dar, die sich unter Umständen diametral gegenüberstehen.
Hier können sich große Probleme entwickeln, wenn technische Notwendigkeiten oder Unternehmensinteressen den Grundsätzen gegenüberstehen.

\ifthenelse{\boolean{todos}}{
\todo[inline,caption={}]{
\begin{itemize}
\item ca 25 Jahre Netzpolitik
\item von obskur bis zum Großteil aktuell
\item Internet = wichtig für Demokratie (Quelle)
\item daher Grundsatz: Freies Netz wichtig
\item Probleme: dauerhafte veränderung blabla
\end{itemize}}
}
{}

\section{Ziele der Arbeit}
\label{aims}
Das Ziel dieser Hausarbeit ist, die vergangene, aber vor allem die momentane Netzpolitik zu analysieren im Hinblick auf ihre Verträglichkeit mit den Interessen der Bürger.
Hierzu müssen ebenjene Interessen definiert werden.
Des weiteren soll untersucht werden, ob sich die Beschlüsse der Politik von den Interessen der Bürger unterscheidet und wenn ja wie.
Zum Ende wird eine Bilanz gezogen, ob die Politik ihrem Auftrag als Volksvertreter nachkommt oder nicht.

\ifthenelse{\boolean{todos}}{
\todo[inline,caption={}]{
\begin{itemize}
\item funktioniert Netzpolitik im Sinne der Bürger?
\item definieren
\item Beratung
\item Einschränkungen von Freiheiten für Sicherheit zeigen
\item anders als google? (ja, da google = return of data investment, staat = nichts)
\item Staat Aufgabe verfehlt (Besseres zusammenleben der Bürger gewährleisten)
\end{itemize}}
}{}


\section{Aufbau}
\label{structure}
Im Hauptteil wird zuerst beschrieben, warum ein von äußeren Einflüssen bewahrtes Internet wichtig für die Bevölkerung ist.
Im Anschluss steht ein kleiner Einblick in die Entscheidungsfindungsprozesse der Entscheidungsträger und woher sie ihre Informationen beziehen.
Darauf soll eine Beschreibung einiger aktueller Themen im Bereich Netzpolitik folgen.
Hierbei sind der Verlauf, die verschiedenen beteiligten Parteien und ihre Positionen zu beachten.
Daraufhin soll eine vorläufige Bewertung gezogen werden, welche die aktuelle Lage in Bezug auf die wahrscheinlichsten Interessen der Bevölkerung bewertet. 

Im finalen Teil sollen zuerst die einzelnen Teilbefunde zusammengefasst werden, worauf eine finale Bewertung erstellt wird.
Am Ende soll ein Ausblick stehen, welcher aufzeigt, ob und wenn ja was verändert werden müsste, um in Zukunft zumindest den heutigen Stand erhalten können.

\ifthenelse{\boolean{todos}}{
\todo[inline,caption={}]{
\begin{itemize}
\item Hauptteil:
\begin{itemize}
\item erst: Wozu brauchen wir Netzpolitik Definition
\item Beispiele für aktuelle Netzpolitik mit jeweiligem Verlauf
\item dabei Vergleich zu google zeigen
\end{itemize}
\item Endteil:
\begin{itemize}
\item Zusammenfassung
\item Bewertung
\item wie kann es weitergehen?
\end{itemize}
\end{itemize}
}
}{}

\chapter{Hauptteil}
\label{main}
\section{Die netzpolitischen Interessen der Bevölkerung}
Das Bedürfnis der Bürger, der Wille der Bevölkerung an sich kann nicht existieren, da ein Volk eine Ansammlung von Individuen darstellt und diese als solche unterschiedliche Interessen haben.
Allerdings kann angenommen werden, dass eine stabile, informierte Demokratie im Interesse der meisten Bürger liegt. 

Ein primäres Interesse der Bevölkerung dürfte daher die Wahrung der Grund- und Menschenrechte auch in technischen Bereichen darstellen.

Das heutige Internet ist in der ersten Welt sowohl das am meisten genutzte Massenmedium, als auch die größte Anwendung von Datennetzen.
Die bisherigen Massenmedien haben einen Auftritt im Internet und sind somit ein Teil des Internets.
Dieses beschränkt sich allerdings nicht nur auf die bisherigen Medien, sondern stellt weitere Möglichkeiten zur Informationsverbreitung zur Verfügung.
Zwar kann im Internet theoretisch jeder auch falsche Informationen verbreiten, allerdings kann dies durch eine Medienkompetenz in der Bevölkerung ausgeglichen werden.
Ebenjene Tatsache stellt einen enormen Mehrwert sowohl für die Meinungsvielfalt, als auch für die Wissensverbreitung dar, welche wiederum Grundlagen für die informierte Demokratie sind.
Um dies zu erreichen dürfen nach dem Grundsatz der kommunikativen Chancengleichheit die Informationen allerdings nicht durch technische Regulierungen ungleich behandelt werden.


\ifthenelse{\boolean{todos}}{
\todo[inline]{vllt mehr}
\todo[inline,caption={}]{
\begin{itemize}
\item immer mehr nr 1 Infoquelle (Quelle?)
\item (Überwachung führt zu ''Feigheit'' (Quelle))
\end{itemize}}
}{}


\section{Informationsgewinnung der Politik}
\label{lobby}
Zur Informationsgewinnung über gewisse Sachverhalte, als auch über verschiedene Interessenlagen ist es in einer so komplexen Welt wie der heutigen für Politiker obligatorisch, sich mit verschiedenen Interessengruppen zu treffen und sich über deren Position und deren Begründungen informieren zu lassen.
Dies geschieht, damit möglichst fachkundige und informierte Entscheidungen in Bezug auf zukünftige Gesetzgebungen getroffen werden können. 

Die Primärbeschäftigung von Lobbyorganisationen jeglicher Richtung ist, Überzeugungsarbeit für ihre jeweilige Position zu leisten.
Hier sind Politiker rein zeitlich nicht in der Lage, mit einer ausgearbeiteten Strategie von Seiten der Lobbygruppen auf gleichem Informationsniveau zu sein, da sie nicht über jedes spezielles Teilgebiet ihres Aufgabenbereichs informiert sein können.
Sofern die Beratungen mit gleicher Verteilung zwischen den verschiedenen Interessenlagen gegeben ist, kann man von einer Gleichgewichtung der verschiedenen Argumentationen reden.
Es ergibt sich allerdings ein Problem, wenn diese Beratungen einseitig stattfinden. 

Wie aus dem jüngsten Bericht der Organisation Transparency International \cite{lobby} hervorgeht, ist leider genau dies auf europäischer Ebene der Fall.
Der Bericht beruft sich auf eine Analyse der von der europäischen Kommission angegebenen Lobbymeetings im letzten halben Jahr.
So werden 63\% der Lobbyisten durch Unternehmen und Beratungsfirmen gestellt und 75\% der Treffen durch sie veranstaltet, die restlichen Treffen verteilen sich auf Think Tanks, NGOs und Gemeinden.
Der Bereich der Digitalwirtschaft, welcher einen großen Teil der Netzpolitik darstellt, ist einer der Bereiche mit den meisten Treffen.
In diesem Bereich ist die Verteilung zwischen NGO- und Unternehmenslobbyismus noch einseitiger, 89\% der Treffen wird von Unternehmen organisiert. 

Die Datenlage legt nahe, dass dies auch in anderen staatlichen Bereichen, abseits der EU, ähnlich ist. 
Dies ist um so Besorgnis erregender, wenn man sich vor Augen führt, dass einige wichtige Entscheidungsträger trotz einer mehr als 10-jährigen Verbreitung des Internets als Massenmedium selbiges immer noch als Neuland wahrnehmen. \cite{internetneuland}

\ifthenelse{\boolean{todos}}{
\todo[inline,caption={}]{
\begin{itemize}
\item Öttinger 98\% Beratungen von Lobbyisten der Netzbetreiber
\item internet für alle neuland
\item \(\Rightarrow\) Politiker Ahnungslos \(\Rightarrow\) Entscheidungsträger?!
\end{itemize}
}
}{}

\section{Vorratsdatenspeicherung in EU und Deutschland}
\label{survailance}
In regelmäßigen Abständen zieht ein terroristischer Vorfall seit den Anschlägen von 2001 eine Diskussion über die Vorratsdatenspeicherung nach sich.
Die Vorratsdatenspeicherung hat verschiedene Ausprägungen, wobei die meisten eine zeitweise Speicherung von Kommunikationsmetadaten gemein haben.
Damit diese auch Personen beziehungsweise deren Anschlüssen zugeordnet werden können, müssen Bestandsdaten gespeichert werden.
Zuletzt hat das Thema durch die Anschläge von Paris unter anderem auf die Satire-Zeitschrift Charlie Hebdo auch in Deutschland wieder einmal Prominenz erlangt.
Seitdem ist ein Gesetz zur Vorratsdatenspeicherung im Gesetzgebungsprozess, wobei beide Regierungsparteien ihre Zustimmung signalisiert haben.
Es scheint nur noch eine Frage der Zeit zu sein, bis dieses Gesetz beschlossen ist. 
Bis 2014 existierte eine EU-Richtlinie zur Vorratsdatenspeicherung.
Diese wurde 2007 in deutsches Recht umgesetzt.
2010 wurde das Gesetz in seiner bisherigen Form vom Bundesverfassungsgericht als unvereinbar mit dem Grundrecht auf die Unverletzlichkeit des Fernmeldegeheimnisses eingestuft und somit für nichtig erklärt \cite{bvg:no}.
Die EU-Richtlinie wurde 2014 vom EUGH mit dem Hinweis auf ihre Unverhältnismäßigkeit im Bezug auf die Grundrechte für ungültig erklärt \cite{eugh:no}.

Im Fall der Vorratsdatenspeicherung existieren 3 Interessenparteien.
Auf der Seite der Befürworter stehen diejenigen mit einem großen Interesse an Sicherheit, sie sind meistens in der Politik und bei Sicherheitsbehörden zu lokalisieren und die Inhalteanbieter, welche ihnen zugefügten Schaden entdecken und wieder Kompensation erreichen wollen.
Die Seite der Kritiker besteht aus denjenigen, welche sich um die Grundrechte sorgen.

Die Befürworter der Vorratsdatenspeicherung sehen das Internet in seiner jetzigen Form nicht als einen rechtlosen, aber einen kontrolllosen Raum an.
Dieser kann und wird nach ihrer Meinung durch den internationalen Terrorismus als ein Mittel genutzt, um sich zu koordinieren und somit die Grundrechte in der westlichen Welt anzugreifen.
Schwerverbrechen sei eine weitere Bedrohung, die durch ausgeweitete Kontrolle bekämpft werden könne.
Je nach Lager der Befürworter sind dieses unterschiedliche Verbrechen.
Während Sicherheitsbefürworter sich eher den traditionellen Schwerverbrechen zuwenden, sehen Inhalteanbieter auch Verstöße gegen ihre Rechte als solche Verbrechen an.

Kritiker der Vorratsdatenspeicherung sehen durch sie das Volk unter einen Generalverdacht gestellt, welcher dem Rechtsgrundsatz \textit{unschuldig bis zum Beweis des Gegenteils} entgegensteht und somit unseren Rechtsstaat aushöhlt.
Als ultimativ bedrohlichster Fall wird ein kaskadierender Verfall der Grundrechte durch schrittweise Ausweitung der möglichen Abfragefälle gesehen.
Weiterhin werden Missbrauchsmöglichkeiten durch einzelne Beamte angeführt, wobei wenige bis keine Maßnahmen gegen potentielle Missbräuche öffentlich gemacht wurden \cite{ct:vorratsdaten}.
Da sogar die Befürworter des Gesetzes, wie man am IT-Sicherheitsgesetz sehen kann, auch den Internet Service Providern, welche die Speicherung dezentral vornehmen sollen, keine expliziten Sicherheitskompetenzen zubilligen, scheint eine technisch sichere Umsetzung extrem schwierig.
Große Informationssammlungen stellen wiederum ein lohnendes Angriffsziel dar.
Somit wird die technische Sicherheit der gesammelten Daten durch mögliche Angriffe stark herausgefordert werden.
Bei dem neuen Gesetzesentwurf wird weiterhin die Problematik der Zugriffsbeschränkung angeführt.
Zwar soll eine richterliche Zustimmung bei einem vollen Datenzugriff von Nöten sein, aber nicht bei Teilabfragen, womit diese Zugriffsbeschränkung ad absurdum geführt wird.

Da die Bewahrung der Grundrechte eine der obersten Prioritäten für die Bevölkerung sein sollte, können solche Grundrechte nicht zu ihrer Bewahrung aufgegeben werden.
Somit läuft das Verhalten der Politik, vielleicht durch guten Willen angeregt, dennoch dem Bevölkerungsinteresse zuwider.

\ifthenelse{\boolean{todos}}{
\todo[inline,caption={}]{
\begin{itemize}
\item 2.-3. Versuch für Vorratsdatenspeicherung
\item pro contra
\item ganz viele Probleme damit (Quellen)
\begin{itemize}
\item Generalverdacht
\item Missbrauchsmöglichkeiten
\item technisch sichere Umsetzung
\item Zugriffsrechte
\end{itemize}
\end{itemize}}
}{}

%\section{Leistungsschutzrecht}
%\label{LSR}

%\ifthenelse{\boolean{todos}}{
%\todo[inline,caption={}]{
%\begin{itemize}
%\item Verlauf
%\item Erpressung?
%\item weiteres Lobbyismusproblem
%\item durch Ahnungslosigkeit der Politiker google gestärkt
%\end{itemize}}
%}{}

\section{De-Mail und E-Government}
\label{demail}
Da die E-Mail als Lösung für rechtssicheren Nachrichten- und Dokumentaustausch einige Probleme mit sich bringt, wurde das De-Mail-Gesetz 2011 auf den Weg gebracht.
Zu den Problemen der normalen E-Mail gehören unter anderem die Unmöglichkeit eine sichere Personenzuordnung zu einer Person zu generieren, die Sicherstellung einer Empfangsbestätigung, der Mangel an verschlüsseltem Datenverkehr und die Möglichkeit von Virenangriffen per Mail.
Die De-Mail wird von einigen wenigen, akkreditierten Anbietern zur Verfügung gestellt.
Das Problem der Personenzuordnung wird durch ein Antragssystem gelöst, bei dem per Lichtbildausweis die Identität bestätigt wird. 
Hierdurch und durch ein anschließendes hybrides Zugriffsmodell für den De-Mail-Account wird die Zuordnung aufrecht erhalten.
Die Möglichkeit als Absender eine Empfangsbestätigung zu verlangen, ist Teil des De-Mail-Standards.
Die Verschlüsselungsproblematik soll durch das hybride Verschlüsselungssystem TSL gelöst werden, wobei die Ende-zu-Ende Verschlüsselung eine Möglichkeit, aber keine Verpflichtung darstellt.
Den E-Mail-Viren soll durch Scans auf der Anbieterseite vorgebeugt werden.

Die Sicherheitsaspekte sahen sich relativ heftiger Kritik ausgesetzt, beispielsweise von Seiten des Computer Chaos Clubs.
Eine Ende-zu-Ende-Verschlüsselung schließt einen Virenscan auf Anbieterseite aus, während eine Entschlüsselung auf dem Transportweg das Konzept einer Verschlüsselung als vertrauliche Informationsübertragung ad absurdum führt.
Dies lässt sich darin begründen, dass jeder Kommunikationspartner zu einem gewissen Übertragungszeitpunkt eine unverschlüsselte Nachricht vorliegen hat, denn zum Virenscan muss die Nachricht entschlüsselt werden.
Eine zusammenfassende Darstellung der Kritikersicht liefert der Vortrag \textit{Bullshit made in Germany} von Linus Neumann \cite{bsinger}.
Der Vortrag stellt deswegen hier im folgenden eine der Hauptquellen dar. \footnote{Weitere Quellen mit den gleichen Aussagen finden sich auf \url{http://logbuch-netzpolitik.de} und \url{http://netzpolitik.org}}

Seit 2013 wurde durch das E-Government-Gesetz die De-Mail mit Empfangsbestätigung als Ersatz für rechtlichen Schriftverkehr zugelassen \cite{egov}.
Damit die De-Mail die hierfür notwendigen Sicherheitsstandards erfüllt, musste allerdings ihre Technik juristisch aufgebessert werden:
\textit{\glqqÜbermitteln ...; das Senden von Sozialdaten durch eine De-Mail-Nachricht an die jeweiligen akkreditierten Diensteanbieter – zur kurzfristigen automatisierten Entschlüsselung zum Zweck der Überprüfung auf Schadsoftware und zum Zweck der Weiterleitung an den Adressaten der De-Mail-Nachricht – ist kein Übermitteln\grqq} \cite{sozgesb10}.
In Zusammenhang mit dem E-Government-Gesetz und dem restlichen Gesetzestext löst dieser Passus das durch die fehlende Verschlüsselung entstandene juristische Problem der De-Mail als rechtverbindlichen Übertragungsweg im Sinne der Gesetzeslage.
Allerdings lösen Formulierungszusätze keine technischen Problematiken. 
Laut den in die federführenden Ausschüsse geladenen Experten des CCCs wurde dies von den Verantwortungsträgern jedoch ignoriert \cite{bsinger}.
Durch die geringe Anzahl der Anbieter wird die Attraktivität für digitale Angriffe auf diese gesteigert, da durch einen Erfolg potentiell viele Daten in die Hände der Angreifer fallen könnten.
Sofern die De-Mail wie gedacht genutzt wird, wird also ein großer Teil des sensiblen Datenverkehrs durch die wenigen akkreditierten Anbieter und die auf der Anbieterseite stattfindende Entschlüsselung wider besseren Wissens auf technischer Ebene in Gefahr gebracht.

Diese Situation ist laut Linus Neumann auf Hintergrundlobbyismus und geheime Verträge zurückzuführen \cite{bsinger}.

\ifthenelse{\boolean{todos}}{
\todo[inline,caption={}]{
\begin{itemize}
\item bullshit made in germany, nothing more said
\end{itemize}}
}{}

\section{Netzneutralität}
\label{netneutr}
Die Netzneutralität ist ein weiteres sich aktuell in der Diskussion befindlichen Thema.
Momentan läuft ein Verfahren zur Bestimmung einer Richtlinie die Netzneutralität und das Roaming betreffend in der EU.
Netzneutralität bezeichnet den gleichberechtigten Datentransport aller Daten in einem (Hardware) Netz.
Damit ist gemeint, dass alle Datenpakete, die durch ein physisches Netz weitergeleitet werden, keine Priorisierungen für ihre Weiterleitung bekommen.
Diese Priorisierung wird erst durch die Deep Package Inspection Technologie möglich, da es erlaubt Datenpakete auf ihren Inhalt hin zu analysieren.
In der Diskussion um Netzneutralität existieren derzeit drei Interessengruppen.
Auf der Seite der Befürworter finden sich zum einen Bürger(rechtler), welche die Informationsfreiheit in Gefahr sehen und zum anderen ein Teil der Contentprovider, welche ihre Geschäftsmodelle in Gefahr sehen.
Auf der Seite der Netzneutralitätsgegner befinden sich die Internet Service Providers (ISPs).

Die ISPs führen an, dass Netzneutralität schädlich für die Gesundheit einiger Menschen sein könnte, wenn spezielle Dienste nicht bevorzugt weitergeleitet werden dürfen, beispielsweise bei einer Remote-OP.
Die Befürworter entgegnen diesem Punkt zumeist, dass eine solch kritische Infrastruktur eine Standleitung benötigen würde und man nicht so verantwortungslos sein sollte, dies über das Internet zu lösen.
Insofern bestünde keine Veranlassung deswegen die Netzneutralität einzuschränken.
Des weiteren führen die ISPs an, dass wegen steigender Nutzung an Datenvolumen durch die Bevölkerung, welche exponentiell anzusteigen scheinen\cite{nnIrrtum}, Flusssteuerung von Nöten sei. 
Flusssteuerung widerspricht allerdings dem Prinzip der Netzneutralität.
Flusssteuerung ist das Umleiten von Datenpaketen an ausgelasteten Netzabschnitten vorbei.
Dies ist laut der Befürworter allerdings nur eine kurzfristige Lösung eines langfristigen Problems, um einen Netzausbau kommt man nicht herum.
Die ISPs sehen sich als Eigentümer des Hardwarebestandteils des Netzes und leiten daraus die Forderung nach der Entscheidungsfreiheit über ihr Eigentum ab.
In diesem Punkt werden aber von den Befürwortern der Netzneutralität Infrastrukturen, die das Internet, Festnetztelefonie und Fernsehanschlüsse umfassen, als so zentral für die heutige Gesellschaft angesehen, dass den ISPs ihre Rechte zugunsten der Bevölkerungsmehrheit abgesprochen werden sollten.
Ein weiterer häufig angeführter Contra-Punkt findet sich in der Quersubventionierung der Big User durch Normal User im Flatratebereich. 
Quersubventionierung bedeutet hier, dass wenn Nutzer den gleichen Betrag für einen Anschluss zahlen, einige allerdings wesentlich höhere Datenvolumen nutzen als die meisten anderen, der für wenige günstige Tarif von der Mehrheit der Nutzer subventioniert wird.
Dem wird entgegengehalten, dass die Big User von heute nur den Trend angeben und die Datenvolumen der Mehrheit sich in ein paar Jahren auf dem Niveau der heutigen Big User befinden wird.
Es existiert eine Analogie bei den Contentprovidern, hier müssten Anbieter wie google unter einer Abschaffung der Netzneutralität leiden.
Weiterhin nutzt die Netzneutralität den sich möglicherweise aufbauenden Konkurrenten der heutigen Quasimonopole, da es durch die Netzneutralität nicht möglich ist, die konkurrierenden Anbieter durch spezielle Deals zwischen Contentprovider und ISPs unterschiedlich zu behandeln. 
Solche Verträge können auch unter Zwang durch Eingriff in die Pakete eines bestimmten Contentproviders zustande kommen. 
Einen solchen Fall stellten die Verhandlungen zwischen netflix und Commcast in den USA dar \cite{nfvcc} unter welchen nicht nur netflix sondern auch eine Großzahl seiner Kunden zu leiden hatte.

Am 30. Juni 2015 nun haben sich die EU Kommission, der EU Rat und das EU Parlament in einem Trilog darauf geeinigt die Roaming Kosten zu drosseln.
Die Netzneutralität scheint beim ersten Lesen der Stellungnahmen gesichert zu sein \cite{ecwelcomennn}.
Sie ist allerdings bei genauerem Hinsehen abgeschafft \cite{nepo:endnn}, da die umstrittenen Special Services unter Einschränkungen erlaubt sein sollen und Drosselungen bestimmter Pakete bei überlasteten Leitungen im Rahmen der vorliegenden Formulierung in den Bereich des Möglichen rücken.
Allerdings ist diese vorliegende Formulierung nicht final, strotzt nur vor Mehrdeutigkeiten und harrt weiterer Definitionen und Erklärungen \cite{nndealblurry}.

Trotz ihrer milden Form handelt es sich bei der Einigung um eine Quasiabschaffung der Netzneutralität.
Die vielen Formulierungsungenauigkeiten lassen weitere Einschränkungen befürchten.
Somit hat auch hier die Politik nicht im besten Interesse der Bürger gehandelt.


\ifthenelse{\boolean{todos}}{
\todo[inline,caption={}]{
\begin{itemize}
\item was ist das?
\item warum wichtig?
\item wiederhole pro und contra punkte
\item vs roaming :(
\item irrtümer der Netzneutralität c't
\end{itemize}}
}{}

\chapter{Schluss}
\label{end}
\section{Fazit}
\label{conclusion}
Wie im Hauptteil gezeigt, hat sich die Politik im Bereich der Netzpolitik aus verschiedenen Gründen eher gegen die Interessen der Bevölkerung gestellt.
Während -siehe Abschnitt \ref{survailance}- bei der Vorratsdatenspeicherung die Sicherheitsinteressen über die Interessen zum Erhalt der Rechte der Bevölkerung gestellt werden, wird -wie in Abschnitt \ref{demail} gezeigt- versucht technischen Sachverstand durch juristischen zu ersetzen, und somit sogar noch mehr Unsicherheit geschaffen.
Im Abschnitt \ref{netneutr} konnte gezeigt werden, dass die Politik netzpolitisch nicht vollständig auf der Seite der Bürger steht.
Wie stark der Zusammenhang zwischen den Beispielen in den Abschnitten \ref{survailance}, \ref{demail} und \ref{netneutr} und den im Abschnitt \ref{lobby} aufgezeigten Lobbypraktiken ist, ist unklar, allerdings ist er wohl gegeben.
Honi soit qui mal y pense (Beschämt sei, wer schlecht darüber denkt).

Nun könnte der Einwand auftauchen, dass viele Unternehmen ein schlimmere Position gegenüber den Bürgerinteressen einnehmen.
Google, Facebook und co. sind weitaus größere Datensammler, als mit der Vorratsdatenspeicherung je möglich ist, Sicherheitslücken existieren in rauen Mengen und nicht nur bei der De-Mail werden Sicherheitsstandards gebeugt und schließlich hat auch beispielsweise Facebook eine eher eigenwillige Definition von Netzneutralität. 
Dies ist allerdings auch nicht der Auftrag der Privatwirtschaft.
Der Auftrag der Politik in einer Demokratie ist allerdings, das Beste für den Souverän, die Bevölkerung, zu tun.
Manche Politiker mögen uninformiert sein, manche mögen sich einseitige Informationsquellen gesucht haben, allerdings muss geschlussfolgert werden, dass sie im Bereich der Netzpolitik ihren Auftrag verfehlt haben.

\ifthenelse{\boolean{todos}}{
\todo[inline,caption={}]{
\begin{itemize}
\item nochmal punkte zusammenfassen
\item Politik glänzt durch Inkompetenz und oder einseitiger Beeinflussung
\item Tun ähnliches bis schlimmeres, als Google und co
\item Aber: anderer Auftrag => Verfehlt, mäh
\end{itemize}
}
}{}

\section{Ausblick}
\label{lookout}
Wenn die Bevölkerung in Zukunft auch noch die Vorteile eines freien Netzes nutzen wollen, muss Druck auf die Politik ausgeübt werden.
Politiker müssen anscheinend, wie beispielsweise bei den ACTA-Demonstrationen, in eine bestimmte Richtung gestoßen werden, sie müssen auf den Willen der Bevölkerung aufmerksam gemacht werden.
Auch muss für ein ausgeglichene Repräsentation der Interessengruppen in der Beratung der Politiker gesorgt werden, damit sich der Politik ein breiteres Blickfeld auf ihre Entscheidungsfindung eröffnet.

\ifthenelse{\boolean{todos}}{
\todo[inline]{
naja werden ich dann sehen, sowas wie: wenn wir uns nicht ändern, dann problem
}
}{}

\nocite{*}
\bibliographystyle{alpha}
\bibliography{literature}

%\listoftables

%\listoffigures


\end{document}