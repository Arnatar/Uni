\documentclass[
	12pt,
	a4paper,
	BCOR10mm,
	%chapterprefix,
	DIV14,
	listof=totoc,
	bibliography=totoc,
	headsepline
]{scrreprt}

\usepackage[T1]{fontenc}
\usepackage[utf8]{inputenc}
\usepackage[ngerman]{babel}

\usepackage{lmodern}

\usepackage[footnote]{acronym}
\usepackage[page,toc]{appendix}
\usepackage{fancyhdr}
\usepackage{float}
\usepackage{graphicx}
\usepackage[pdfborder={0 0 0}]{hyperref}
\usepackage[htt]{hyphenat}
\usepackage{listings}
\usepackage{lscape}
\usepackage{microtype}
\usepackage{nicefrac}
\usepackage{subfig}
\usepackage{textcomp}
\usepackage[subfigure,titles]{tocloft}
\usepackage{units}
\usepackage{pgf}
\usepackage{amsmath}
\usepackage{placeins}
\usepackage{titletoc}
\usepackage{todonotes}

\lstset{
	basicstyle=\ttfamily,
	frame=single,
	numbers=left,
	language=C,
	breaklines=true,
	breakatwhitespace=true,
	postbreak=\hbox{$\hookrightarrow$ },
	showstringspaces=false,
	tabsize=4
}

\renewcommand*{\lstlistlistingname}{Listing catalog}

\renewcommand*{\appendixname}{Appendix}
\renewcommand*{\appendixtocname}{Appendices}
\renewcommand*{\appendixpagename}{Appendices}


\begin{document}
\begin{titlepage}
	\begin{center}
		{\titlefont\huge Handelt die Politik netzpolitisch im Sinne der Bevölkerung? \par}

		\bigskip
		\bigskip

		{\titlefont\Large --- Hausarbeit ---\par}

		\bigskip
		\bigskip

		{\large Seminar: Des Googles Kern\\
		Fachbereich Informatik\\
		Fakultät für Mathematik, Informatik und Naturwissenschaften\\
		Universität Hamburg\par}
	\end{center}

	\vfill

	{\large \begin{tabular}{ll}
		Vorgelegt von: & Arne Struck \\
		E-Mail-Adresse: 
			& \href{mailto:1struck@informatik.uni-hamburg.de}{1struck@informatik.uni-hamburg.de} \\ 
		%Matrikelnummer: & 1234567 \\
		Studiengang: & Bsc. Informatik \\
		\\
		Veranstalter: & Prof. Arno Rolf\\
		\\
		Hamburg, den \today
	\end{tabular}\par}
\end{titlepage}

\thispagestyle{empty}

\newpage\null\thispagestyle{empty}\newpage
\tableofcontents
\newpage\null\thispagestyle{empty}\newpage

\chapter{Einleitung}
\label{intro}

\section{Themenumriss}
\label{themeintro}
\todo[inline,caption={}]{
\begin{itemize}
\item ca 25 Jahre Netzpolitik
\item von obskur bis zum Großteil aktuell
\item Internet = wichtig für Demokratie (Quelle)
\item daher Grundsatz: Freies Netz wichtig
\item Probleme: dauerhafte veränderung blabla
\end{itemize}}

\section{Ziele der Arbeit}
\label{aims}
\todo[inline,caption={}]{
\begin{itemize}
\item funktioniert Netzpolitik im Sinne der Bürger?
\item definieren
\item Beratung
\item Einschränkungen von Freiheiten für Sicherheit zeigen
\item anders als google? (ja, da google = return of data investment, staat = nichts)
\item Staat Aufgabe verfehlt (Besseres zusammenleben der Bürger gewährleisten)
\end{itemize}}

\section{Aufbau}
\label{structure}
\todo[inline,caption={}]{
\begin{itemize}
\item Hauptteil:
\begin{itemize}
\item erst: Wozu brauchen wir Netzpolitik Definition
\item Beispiele für aktuelle Netzpolitik mit jeweiligem Verlauf
\item dabei Vergleich zu google zeigen
\end{itemize}
\item Endteil:
\begin{itemize}
\item Zusammenfassung
\item Bewertung
\item wie kann es weitergehen?
\end{itemize}
\end{itemize}
}

\chapter{Hauptteil}
\label{main}
\section{Warum brauchen wir ein freies Netz?}
\todo[inline,caption={}]{
\begin{itemize}
\item immer mehr nr 1 Infoquelle (Quelle?)
\item Grundsatz informationelle Selbstbestimmung
\item Überwachung führt zu ''Feigheit'' (Quelle)
\end{itemize}}

\section{Beeinlussung der Politik}
\label{lobby}
\todo[inline,caption={}]{
\begin{itemize}
\item Öttinger 98\% Beratungen von Lobbyisten der Netzbetreiber
\item safetheinternet
\item internet für alle neuland
\item \(\Rightarrow\) Politiker Ahnungslos \(\Rightarrow\) Entscheidungsträger?!
\end{itemize}
}

\section{Überwachung}
\label{survailance}
\todo[inline,caption={}]{
\begin{itemize}
\item 2.-3. Versuch für Vorratsdatenspeicherung
\item Antiterrordatei
\item pro contra
\item ganz viele Probleme damit (Quellen)
\begin{itemize}
\item Generalverdacht
\item Missbrauchsmöglichkeiten
\item technisch sichere Umsetzung
\item Zugriffsrechte
\item Politikerlügen
\end{itemize}
\item Bundestrojaner
\end{itemize}}

\section{Leistungsschutzrecht}
\label{LSR}
\todo[inline,caption={}]{
\begin{itemize}
\item Verlauf
\item Erpressung?
\item weiteres Lobbyismusproblem
\item durch Ahnungslosigkeit der Politiker google gestärkt
\end{itemize}}

\section{De-Mail}
\label{demail}
\todo[inline,caption={}]{
\begin{itemize}
\item bullshit made in germany, nothing more said
\end{itemize}}

\section{Netzneutralität}
\label{netneutr}
\todo[inline,caption={}]{
\begin{itemize}
\item was ist das?
\item warum wichtig?
\item wiederhole pro und contra punkte
\item vs roaming :(
\item irrtümer der Netzneutralität c't
\end{itemize}}


\chapter{Schluss}
\label{end}
\section{Fazit}
\label{conclusion}
\todo[inline,caption={}]{
\begin{itemize}
\item nochmal punkte zusammenfassen
\item Politik glänzt durch Inkompetenz und oder einseitiger Beeinflussung
\item Tun ähnliches bis schlimmeres, als Google und co
\item Aber: anderer Auftrag => Verfehlt, mäh
\end{itemize}
}

\section{Ausblick}
\label{lookout}
\todo[inline]{
naja werden ich dann sehen, sowas wie: wenn wir uns nicht ändern, dann problem
}

\nocite{*}
\bibliographystyle{alpha}
\bibliography{literature}

\listoftables

\listoffigures


\end{document}