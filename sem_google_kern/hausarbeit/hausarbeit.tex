\documentclass[
	12pt,
	a4paper,
	BCOR10mm,
	%chapterprefix,
	DIV14,
	listof=totoc,
	bibliography=totoc,
	headsepline
]{scrreprt}

\usepackage[T1]{fontenc}
\usepackage[utf8]{inputenc}
\usepackage[ngerman]{babel}

\usepackage{lmodern}

\usepackage[footnote]{acronym}
\usepackage[page,toc]{appendix}
\usepackage{fancyhdr}
\usepackage{float}
\usepackage{graphicx}
\usepackage[pdfborder={0 0 0}]{hyperref}
\usepackage{listings}
\usepackage{lscape}
\usepackage{microtype}
\usepackage{nicefrac}
\usepackage{subfig}
\usepackage{textcomp}
\usepackage[subfigure,titles]{tocloft}
\usepackage{units}
\usepackage{pgf}
\usepackage{amsmath}
\usepackage{placeins}
\usepackage{titletoc}
\usepackage{todonotes}

\lstset{
	basicstyle=\ttfamily,
	frame=single,
	numbers=left,
	language=C,
	breaklines=true,
	breakatwhitespace=true,
	postbreak=\hbox{$\hookrightarrow$ },
	showstringspaces=false,
	tabsize=4
}

\renewcommand*{\lstlistlistingname}{Listing catalog}

\renewcommand*{\appendixname}{Appendix}
\renewcommand*{\appendixtocname}{Appendices}
\renewcommand*{\appendixpagename}{Appendices}

\newboolean{todos}
\setboolean{todos}{false}

\begin{document}
\begin{titlepage}
	\begin{center}
		{\titlefont\huge Handelt die Politik netzpolitisch im Sinne der Bevölkerung? \par}

		\bigskip
		\bigskip

		{\titlefont\Large --- Hausarbeit ---\par}

		\bigskip
		\bigskip

		{\large Seminar: Des Googles Kern\\
		Fachbereich Informatik\\
		Fakultät für Mathematik, Informatik und Naturwissenschaften\\
		Universität Hamburg\par}
	\end{center}

	\vfill

	{\large \begin{tabular}{ll}
		Vorgelegt von: & Arne Struck \\
		E-Mail-Adresse: 
			& \href{mailto:1struck@informatik.uni-hamburg.de}{1struck@informatik.uni-hamburg.de} \\ 
		%Matrikelnummer: & 1234567 \\
		Studiengang: & Bsc. Informatik \\
		\\
		Veranstalter: & Prof. Arno Rolf\\
		\\
		Hamburg, den \today
	\end{tabular}\par}
\end{titlepage}

\thispagestyle{empty}

\newpage\null\thispagestyle{empty}\newpage
\tableofcontents
\newpage\null\thispagestyle{empty}\newpage

\chapter{Einleitung}
\label{intro}

\section{Themenumriss Netzpolitik}
\label{themeintro}
Nach nun ca. 25 Jahren Netzpolitik hat diese eine große Veränderung hinter sich.
Was um 1990, als das Internet anfing durch die Entwicklung der Grundlagen des World Wide Webs sich als Massenmedium zu etablieren, noch obskure Gedanken einiger weniger war, hat sich heutzutage als immer wichtiger werdender Teil der Bürgerrechte etabliert.
Zumindest legen die in letzter Zeit immer stärker in die Öffentlichkeit tretenden Diskussionen zwischen Bürgerrechtlern, Aktivisten auf der einen Seite und den verschiedensten Interessenvertretern auf der anderen Seite diesen Schluss nahe. 

Medien sind für eine lebendige Demokratie in den meisten ihrer Formen eine Grundvoraussetzung, unter anderem da der Wähler eine informierte Entscheidung treffen muss.
Nun hat sich das Internet zu einer der Hauptinformationsquellen für die Generationen unter 50 entwickelt, wenn es nicht gar die wichtigste für diese ist.
Somit ist es nicht verwunderlich, dass Aktivistengruppen das Internet trotz all seiner Schwächen von äußeren Einflüssen beschützen wollen.
Aus solchen Beweggründen resultieren Grundsätze wie: ein freies Netz ist eine wichtige Grundlage für unsere Gesellschaft und daher zu bewahren. 

Da das Internet als Teil der modernen Informationstechnologie relativ schnellen, massiven Veränderungen unterliegt ist es sowohl für Politik, als auch für alle Interessenverbände schwierig Handlungsweisen zu entwickeln, welche ihren Interessen entspricht.
Ein weiteres Problem in der Netzpolitik stellen selbstredend die verschiedenen Interessenlagen dar, welche sich unter Umständen diametral gegenüberstehen.
Hier können sich große Probleme entwickeln, wenn technische Notwendigkeiten oder Unternehmensinteressen den Grundsätzen gegenüberstehen.

\ifthenelse{\boolean{todos}}{
\todo[inline,caption={}]{
\begin{itemize}
\item ca 25 Jahre Netzpolitik
\item von obskur bis zum Großteil aktuell
\item Internet = wichtig für Demokratie (Quelle)
\item daher Grundsatz: Freies Netz wichtig
\item Probleme: dauerhafte veränderung blabla
\end{itemize}}
}
{}

\section{Ziele der Arbeit}
\label{aims}
Das Ziel dieser Hausarbeit soll die vergangene, aber vor allem die momentane Netzpolitik zu analysieren in Hinblick auf ihre Verträglichkeit mit den Interessen der Bürger.
Hierzu müssen ebenjene Interessen definiert werden.
Des weiteren soll untersucht werden, ob und wenn ja in wie fern sich die Vorgehensweise der Politik gegenüber den Bürgern unterscheidet.
Zum Ende wird eine Bilanz gezogen werden, ob die Politik ihrem Auftrag als Volksvertreter nachkommt oder nicht.

\ifthenelse{\boolean{todos}}{
\todo[inline,caption={}]{
\begin{itemize}
\item funktioniert Netzpolitik im Sinne der Bürger?
\item definieren
\item Beratung
\item Einschränkungen von Freiheiten für Sicherheit zeigen
\item anders als google? (ja, da google = return of data investment, staat = nichts)
\item Staat Aufgabe verfehlt (Besseres zusammenleben der Bürger gewährleisten)
\end{itemize}}
}{}


\section{Aufbau}
\label{structure}
Im Hauptteil wird zu erst beschrieben, warum ein von äußeren Einflüssen bewahrtes Internet wichtig für die Bevölkerung ist.
Im Anschluss steht ein kleiner Einblick in die Entscheidungsfindungsprozesse der Entscheidungsträger und woher sie ihre Informationen beziehen.
Darauf soll eine Beschreibung einiger aktuellen Themen im Bereich Netzpolitik folgen.
Hierbei sind der Verlauf, die verschiedenen beteiligten Parteien und ihre Positionen zu beachten, darauf soll eine vorläufige Bewertung gezogen werden, welche die aktuelle Lage in Bezug auf die wahrscheinlichsten Interessen der Bevölkerung bewertet. 

Im finalen Teil sollen zu erst die einzelnen Teilbefunde zusammengefasst werden, worauf eine finale Bewertung erstellt wird.
Am Ende soll ein Ausblick stehen, welcher aufzeigt ob und wenn ja was verändert werden könnte, um in Zukunft auf dem selben Level dazustehen, wie heute.

\ifthenelse{\boolean{todos}}{
\todo[inline,caption={}]{
\begin{itemize}
\item Hauptteil:
\begin{itemize}
\item erst: Wozu brauchen wir Netzpolitik Definition
\item Beispiele für aktuelle Netzpolitik mit jeweiligem Verlauf
\item dabei Vergleich zu google zeigen
\end{itemize}
\item Endteil:
\begin{itemize}
\item Zusammenfassung
\item Bewertung
\item wie kann es weitergehen?
\end{itemize}
\end{itemize}
}
}{}

\chapter{Hauptteil}
\label{main}
\section{Warum braucht die Bevölkerung ein freies Netz?}
Das Bedürfnis der Bürger, der Wille der Bevölkerung an sich kann zwar nicht existieren, da ein Volk eine Ansammlung von Individuen darstellt.
Allerdings kann angenommen werden, dass eine stabile, informierte Demokratie im Interesse der meisten Bürger liegt. 
Das heutige Internet ist in der ersten Welt das am meisten genutzte Massenmedium.
Nicht nur haben die bisherigen Massenmedien einen Auftritt und somit durchs Internet zu erreichen, sondern stellen weitere Möglichkeiten zu Informationsverbreitung dar.
Zwar kann im Internet theoretisch jeder auch falsche Informationen verbreiten, allerdings kann dies durch eine medienkompetente Bevölkerung ausgeglichen werden.
Des weiteren stellt ebenjene Tatsache einen Mehrwert für die Meinungsvielfalt dar. 

Nur ein von starken Regulierungen freies Netz kann solche Eigenschaften mit sich bringen.

\ifthenelse{\boolean{todos}}{
\todo[inline]{vllt mehr}
\todo[inline,caption={}]{
\begin{itemize}
\item immer mehr nr 1 Infoquelle (Quelle?)
\item (Überwachung führt zu ''Feigheit'' (Quelle))
\end{itemize}}
}{}


\section{Beratung der Politik}
\label{lobby}
Zur Beratung und Informationsgewinnung ist es in einer so komplexen Welt, wie der unsrigen für Politiker obligatorisch sich mit verschiedenen Interessengruppen zu treffen und sich über deren Position und deren Begründungen informieren zu lassen.
Dies geschieht, damit möglichst fachkundige und informierte Entscheidungen in Bezug auf zukünftige Gesetzgebungen getroffen werden können. 

Die Primärbeschäftigung von Lobbyorganisationen jeglicher Richtung ist Überzeugungsarbeit für ihre jeweilige Position zu leisten.
Hier sind Politiker rein zeitlich nicht in der Lage mit einer ausgearbeiteten Strategie von Seiten der Lobbygruppen auf gleichem Informationsniveau zu sein, da sie sich nicht über ein Teilgebiet informieren müssen.
Sofern die Beratungen mit gleicher Verteilung zwischen den verschiedenen Interessenlagen gegeben ist, kann man von einer Gleichgewichtung der verschiedenen Argumentationen reden.
Es ergibt sich allerdings ein Problem, wenn diese Beratungen einseitig stattfinden. 

Wie aus dem jüngsten Bericht der Organisation Transparency International \cite{lobby} hervorgeht, ist leider genau dies auf europäischer Ebene der Fall.
Der Bericht beruft sich auf eine Analyse der von der europäischen Kommission angegebenen Lobbymeetings im letzten halben Jahr.
So werden 63\% der Lobbyisten und 75\% der Treffen durch Unternehmen und Beratungsfirmen veranstaltet, die restlichen Treffen verteilen sich auf Think Tanks, NGOs und Gemeinden.
Der Bereich der Digitalwirtschaft, welcher einen großen Teil der Netzpolitik darstellt, ist einer der Bereiche mit den meisten Treffen.
In diesem Bereich ist die Verteilung zwischen noch einseitiger, 89\% der Treffen wird von Unternehmen organisiert. 

Die Datenlage legt nahe, dass dies auch in anderen staatlichen Bereichen, abseits der EU ähnlich ist. 
Dies ist um so Besorgnis erregender, wenn man sich vor Augen führt, dass einige wichtige Entscheidungsträger trotz einer mehr als 10 jährigen Verbreitung als Massenmedium des Internets selbiges als Neuland wahrgenommen wird. \cite{internetneuland}

\ifthenelse{\boolean{todos}}{
\todo[inline,caption={}]{
\begin{itemize}
\item Öttinger 98\% Beratungen von Lobbyisten der Netzbetreiber
\item internet für alle neuland
\item \(\Rightarrow\) Politiker Ahnungslos \(\Rightarrow\) Entscheidungsträger?!
\end{itemize}
}
}{}

\section{Vorratsdatenspeicherung in EU und Deutschland}
\label{survailance}
In regelmäßigen Abständen zieht ein terroristischer Vorfall seit den Anschlägen von 2001 eine Diskussion über die Vorratsdatenspeicherung nach sich.
Die Vorratsdatenspeicherung hat verschiedene Ausprägungen, wobei die meisten eine zeitweise Speicherung von Kommunikationsmetadaten gemein haben.
Damit diese auch Personen beziehungsweise deren Anschlüssen zugeordnet werden können, werden Bestandsdaten gespeichert werden.
Das letzte mal hat das Thema durch die Anschläge von Paris unter anderem auf die Satire-Zeitschrift Charlie Hebdo auch in Deutschland an Prominenz erlangt.
Seit dem ist ein Gesetz zur Vorratsdatenspeicherung im Gesetzgebungsprozess, wobei beide Regierungsparteien ihre Zustimmung signalisiert haben, wodurch es nur noch eine Frage der Zeit zu sein scheint, bis dieses Gesetz beschlossen ist. 
Bis 2014 existierte eine EU-Richtlinie zur Vorratsdatenspeicherung.
Diese wurde 2007 in deutsches Recht umgesetzt.
2010 wurde das Gesetz in seiner bisherigen Form vom Bundesverfassungsgericht als unvereinbar mit dem Grundrecht auf die Unverletzlichkeit des Fernmeldegeheimnisses eingestuft und somit für nichtig erklärt \cite{bvg:no}.
Die EU-Richtlinie wurde 2014 vom EUGH mit dem Hinweis auf ihre Unverhältnismäßigkeit im Bezug auf die Grundrechte für ungültig erklärt \cite{eugh:no}. 

Im Fall der Vorratsdatenspeicherung existieren 3 Interessenparteien zum einen diejenigen mit einem großen Interesse an Sicherheit, meistens in der Politik und bei Sicherheitsbehörden zu lokalisieren, die Inhalteanbieter, welche ihnen zugefügten Schaden entdecken und wieder gut machen wollen und diejenigen, die sich um die Grundrechte sorgen.

Die Befürworter sehen das Internet in seiner jetzigen Form nicht als einen rechtlosen, aber einen kontrolllosen Raum an.
Dieser kann und wird nach ihnen durch den internationalen Terrorismus als ein Mittel genutzt, um sich zu koordinieren und somit die Grundrechte in der westlichen Welt zu entkräften.
Schwerverbrechen ist eine weitere Bedrohung, die durch ausgeweitete Kontrolle bekämpft werden könnte.
Je nach Lager der Befürworter sind dieses andere Verbrechen.
Währen Sicherheitsbefürworter sich eher die traditionellen Schwerverbrechen zuwenden, sehen Inhalteanbieter auch Verstöße gegen ihre Rechte als solche Verbrechen.

Kritiker der Vorratsdatenspeicherung sehen durch ebenjene das Volk unter einen Generalverdacht gestellt, welcher dem Rechtsgrundsatz \textit{unschuldig bis zum Beweis des Gegenteils} entgegensteht und somit unseren Rechtsstaat aushöhlt.
Als ultimativ schlechtester Fall wird ein kaskadierender Verfall der Grundrechte durch schrittweise Ausweitung der möglichen Abfragemöglichkeiten gesehen.
Weiterhin werden Missbrauchsmöglichkeiten durch einzelne Beamte angeführt, wobei wenige bis keine Maßnahmen gegen potentielle Missbräuche öffentlich gemacht wurden \cite{ct:vorratsdaten}.
Da sogar die Befürworter des Gesetzes, wie man am IT-Sicherheitsgesetz sehen kann, auch den Internet Service Providern, welche die Speicherung dezentral vornehmen sollen, keine expliziten Sicherheitskompetenzen zubilligt, scheint eine technisch sichere Umsetzung extrem schwierig.
Große Informationsmengen sind wiederum ein lohnendes Ziel, um die technische Sicherheit auf einen Prüfstand zu stellen.
Bei dem neuen Gesetzesentwurf wird weiterhin die Problematik der Zugriffsbeschränkung angeführt.
Zwar soll eine richterliche Zustimmung bei einem vollen Datenzugriff von Nöten sein, aber nicht bei Teilabfragen, womit diese Zugriffsbeschränkung ad absurdum geführt wird.

Da die Bewahrung der Grundrechte eine der obersten Prioritäten für die Bevölkerung sein sollte, können solche Grundrechte nicht zu ihrer Bewahrung aufgegeben werden.
Somit läuft das Verhalten der Politik, vielleicht durch guten Willen angeregt, dennoch dem Bevölkerungsinteresse zuwider.

\ifthenelse{\boolean{todos}}{
\todo[inline,caption={}]{
\begin{itemize}
\item 2.-3. Versuch für Vorratsdatenspeicherung
\item pro contra
\item ganz viele Probleme damit (Quellen)
\begin{itemize}
\item Generalverdacht
\item Missbrauchsmöglichkeiten
\item technisch sichere Umsetzung
\item Zugriffsrechte
\end{itemize}
\end{itemize}}
}{}

%\section{Leistungsschutzrecht}
%\label{LSR}

%\ifthenelse{\boolean{todos}}{
%\todo[inline,caption={}]{
%\begin{itemize}
%\item Verlauf
%\item Erpressung?
%\item weiteres Lobbyismusproblem
%\item durch Ahnungslosigkeit der Politiker google gestärkt
%\end{itemize}}
%}{}

\section{De-Mail und E-Government}
\label{demail}
Da die E-Mail als Lösung für rechts sicheren Nachrichten- und Dokumentaustausch einige Probleme mit sich bringt, wurde das De-Mail-Gesetz 2011 auf den Weg gebracht.
Zu den Problemen der normalen E-Mail gehören unter anderem die Unmöglichkeit eine sichere Zuordnung zu einer Person zu generieren, die Sicherstellung einer Empfangsbestätigung, der Mangel an Verschlüsseltem Datenverkehr und die Möglichkeit von Virenangriffen per Mail.
Die De-Mail soll von einigen wenigen, akkreditierten Anbietern zur Verfügung gestellt.
Das Problem der Personenzuordnung wurde durch ein Antragssystem gelöst, bei dem per Lichtbildausweis die Identität bestätigt wird und damit durch ein hybrides Zugriffsmodell die Zuordnung aufrecht erhalten wird.
Die Möglichkeit als Absender eine Empfangsbestätigung zu verlangen, ist Teil des De-Mail-Standards.
Die Verschlüsselungsproblematik soll durch das hybride Verschlüsselungssystem TSL gelöst werden, wobei die Ende-zu-Ende Verschlüsselung eine Möglichkeit, aber keine Verpflichtung darstellt.
Den E-Mail-Viren soll durch Scans auf der Anbieterseite vorgebeugt werden.

Die Sicherheitsaspekte sahen sich relativ heftiger Kritik ausgesetzt, beispielsweise von Seiten des CCCs ausgesetzt.
Eine Ende-zu-Ende-Verschlüsselung schließt einen Virenscan auf Anbieterseite aus, während eine Entschlüsselung auf dem Transportweg das Konzept einer Verschlüsselung als vertrauliche Informationsübertragung ad absurdum führt.
Dies lässt sich darin begründen, dass jeder Kommunikationspartner zu einem gewissen Übertragungszeitpunkt eine unverschlüsselte Nachricht vorliegen hat (zum Virenscan muss die Nachricht entschlüsselt werden).
Eine zusammenfassende Darstellung der Kritikersicht liefert der Vortrag \textit{Bullshit made in Germany} von Linus Neumann \cite{bsinger}, welcher hier aufgrund dieser Eigenschaft im folgenden eine der Hauptquellen darstellt. \footnote{Weitere Quellen mit den gleichen Aussagen finden sich auf \url{http://logbuch-netzpolitik.de} und \url{http://netzpolitik.org}}

Seit 2013 wurde durch das E-Government-Gesetz die De-Mail mit Empfangsbestätigung als Ersatz für rechtlichen Schriftverkehr zugelassen \cite{egov}.
Damit die De-Mail die hierfür notwendigen Sicherheitsstandards erfüllt, musste allerdings ihre Technik juristisch aufgebessert werden:
\textit{\glqqÜbermitteln ...; das Senden von Sozialdaten durch eine De-Mail-Nachricht an die jeweiligen akkreditierten Diensteanbieter – zur kurzfristigen automatisierten Entschlüsselung zum Zweck der Überprüfung auf Schadsoftware und zum Zweck der Weiterleitung an den Adressaten der De-Mail-Nachricht – ist kein Übermitteln\grqq} \cite{sozgesb10}.
In Zusammenhang mit dem E-Government-Gesetz und dem restlichen Gesetzestext löst dies das durch die fehlende Verschlüsselung entstandene juristische Problem der De-Mail als rechtverbindlichen Übertragungsweg, im Sinne der Gesetzeslage.
Allerdings lösen Formulierungszusätze keine technischen Problematiken. 
Laut den in die federführenden Ausschüsse geladenen Experten des CCCs wurde dies von den Verantwortungsträgern jedoch ignoriert \cite{bsinger}.
Durch die geringe Anzahl der Anbieter wird die Attraktivität für digitale Angriffe auf diese gesteigert, da durch einen Erfolg potentiell viele Daten in die Hände der Angreifer Fallen könnten.
Sofern die De-Mail wie gedacht genutzt wird, wird also ein großer Teil des sensiblen Datenverkehrs durch die wenigen akkreditierten Anbieter und die auf der Anbieterseite stattfindende Entschlüsselung wider besseren Wissens auf technischer Ebene in Gefahr gebracht.

Dies ist laut Linus Neumann auf Hintergrundlobbyismus und geheime Verträge zurückzuführen \cite{bsinger}.



\ifthenelse{\boolean{todos}}{
\todo[inline,caption={}]{
\begin{itemize}
\item bullshit made in germany, nothing more said
\end{itemize}}
}{}

\section{Netzneutralität}
\label{netneutr}
Die Netzneutralität ist ein weiteres sich aktuell in der Diskussion befindlichen Thema.
Momentan läuft ein Verfahren zur Bestimmung einer Richtlinie die Netzneutralität und das Roaming betreffend in der EU.
Netzneutralität bezeichnet den gleichberechtigten Datentransport aller Daten in einem (Hardware) Netz.
Dies bedeutet, dass alle Datenpakete, die durch ein physisches Netz weitergeleitet wird keine Priorisierungen für ihre Weiterleitung bekommen.
Diese Priorisierung wird erst durch die Deep Package Inspection Technik möglich, da diese es erlaubt Datenpakete auf ihren Inhalt zu analysieren.
In der Diskussion um Netzneutralität existieren derzeit drei Interessengruppen, auf der Seite der Befürworter finden sich zum einen Bürger(rechtler), welche die Informationsfreiheit in Gefahr sehen, zum anderen ein Teil der Contentprovider, welche ihre Geschäftsmodelle in Gefahr sehen.
Auf der Seite der Netzneutralitätsgegner befinden sich die Internet Service Providers (ISPs).

Die ISPs führen an, dass Netzneutralität schädlich für die Gesundheit einiger Menschen sein könnte, wenn spezielle Dienste nicht bevorzugt weitergeleitet werden dürfen, beispielsweise bei einer Remote-OP.
Die Befürworter entgegnen diesem Punkt zumeist, dass eine solch kritische Infrastruktur eine Standleitung benötigen würde und man nicht so verantwortungslos sein sollte, dies über das Internet zu lösen.
In so fern bestünde keine Veranlassung deswegen die Netzneutralität einzuschränken.
Des weiteren führen die ISPs an, dass wegen der steigender Nutzung an Datenvolumen der Bevölkerung (diese scheinen exponentiell anzusteigen \cite{nnIrrtum}) Flusssteuerung von Nöten, welche der Netzneutralität widerspricht.
Mit Flusssteuerung ist das Umleiten von Datenpaketen an ausgelasteten Netzabschnitten vorbei gemeint.
Dies ist laut der Befürworter allerdings nur eine kurzfristige Lösung eines langfristigen Problems, um einen Netzausbau kommt man nicht herum.
Die ISPs sehen sich als Eigentümer des Hardwarebestandteil des Netzes und leiten daraus die Forderung nach der Entscheidungsfreiheit über ihr Eigentum ab.
Hier wird von den Befürwortern der Netzneutralität Infrastrukturen, die das Internet, Festnetztelefonie und Fernsehanschlüsse als so zentral für die heutige Gesellschaft angesehen, dass den ISPs ihre Rechte zugunsten der Bevölkerungsmehrheit abgesprochen werden sollten.
Ein weiterer häufig angeführter Contra-Punkt findet sich in der Quersubventionierung der Big User durch Normal User im Flatratebereich. 
Dies bedeutet, dass Nutzer den gleichen Betrag für einen Anschluss zahlen, einige allerdings wesentlich höhere Datenvolumen nutzen, als die meisten anderen, somit wird der für die wenigen günstige Tarif von der Mehrheit der Nutzer subventioniert.
Dem wird entgegen gehalten, dass die Big User von heute nur den Trend angeben und die Datenvolumen der Mehrheit sich in ein paar Jahren auf dem Niveau der heutigen Big User befinden wird.
Es existiert eine Analogie bei den Contentprovidern, hier müssten Anbieter wie google unter einer Abschaffung der Netzneutralität leiden.
Weiterhin nutzt die Netzneutralität den sich aufbauenden möglichen Konkurrenten der heutigen Quasimonopole, da es durch die Netzneutralität nicht möglich ist die konkurrierenden auf Anbieterseite durch spezielle Deals zwischen Contentprovider und ISPs unterschiedlich zu behandeln. 
Solche Verträge können auch unter Zwang durch Eingriff in die Pakete eines bestimmten Contentproviders zustande kommen. 
Ein solcher Fall waren die Verhandlungen zwischen netflix und Commcast in den USA \cite{nfvcc} unter welchen nicht nur netflix, sondern auch eine Großzahl seiner Kunden zu leiden hatte.

Am 30. Juni 2015 nun haben sich die EU Kommission, der EU Rat und das EU Parlament in einem Trilog darauf geeinigt die Roaming Kosten zu drosseln.
Den Punkt der Netzneutralität scheint beim ersten Lesen der Stellungnahmen gesichert zu sein \cite{ecwelcomennn}, ist allerdings bei genauerem Hinsehen abgeschafft \cite{nepo:endnn}, da die umstrittenen Special Services unter Einschränkungen erlaubt sein sollen und Drosselungen bestimmter Pakete bei überlasteten Leitungen im Rahmen der vorliegenden Formulierung in den Bereich des Möglichen Rücken.
Allerdings ist diese vorliegende Formulierung keine finale und strotzt nur vor Mehrdeutigkeiten und harrt weiterer Definitionen und Erklärungen \cite{nndealblurry}.

Trotz der milden Form, handelt es sich bei der Einigung um eine Quasiabschaffung der Netzneutralität und die vielen Formulierungsungenauigkeiten lassen weitere Befürchtungen zu, somit hat auch hier die Politik nicht im besten Interesse der Bürger gehandelt.


\ifthenelse{\boolean{todos}}{
\todo[inline,caption={}]{
\begin{itemize}
\item was ist das?
\item warum wichtig?
\item wiederhole pro und contra punkte
\item vs roaming :(
\item irrtümer der Netzneutralität c't
\end{itemize}}
}{}

\chapter{Schluss}
\label{end}
\section{Fazit}
\label{conclusion}
Wie im Hauptteil gezeigt hat sich die Politik im Bereich der Netzpolitik aus verschiedenen Gründen eher gegen die Interessen der Bevölkerung gestellt.
Während wie im Abschnitt \ref{survailance} bei der Vorratsdatenspeicherung die Sicherheitsinteressen über die Interessen zum Erhalt der Rechte der Bevölkerung gestellt wird, wird wie in Abschnitt \ref{demail} gezeigt versucht technischen Sachverstand durch juristischen zu ersetzen und somit sogar noch mehr Unsicherheit geschaffen.
Auch wie im Abschnitt \ref{netneutr} konnte gezeigt werden, dass die Politik netzpolitisch nicht vollständig auf der Seite der Bürger steht.
Wie stark der Zusammenhang zwischen den Beispielen und den anfangs aufgezeigten Lobbypraktiken steht ist unklar, allerdings ist er wohl gegeben.
Honi soit qui mal y pense (Beschämt sei, wer schlecht darüber denkt).

Nun könnte der Einwand auftauchen, dass viele Unternehmen viel schlimmere Vorgehensweisen gegenüber den Bürgerinteressen aufzeigen.
Google, Facebook und co. sind weitaus größere Datensammler, als mit der Vorratsdatenspeicherung je möglich ist, Sicherheitslücken existieren in rauen Mengen und nicht nur bei der De-Mail werden Sicherheitsstandards gebeugt und schließlich hat auch beispielsweise Facebook eine eher eigenwillige Definition von Netzneutralität. 
Dies ist allerdings auch nicht der Auftrag der Privatwirtschaft.
Der Auftrag der Politik in einer Demokratie ist allerdings das beste für den Souverän, die Bevölkerung zu tun.
Manche mögen uninformiert sein, manche mögen sich die falsche Informationen gesucht haben, allerdings haben sie im Bereich der Netzpolitik ihren Auftrag verfehlt.

\ifthenelse{\boolean{todos}}{
\todo[inline,caption={}]{
\begin{itemize}
\item nochmal punkte zusammenfassen
\item Politik glänzt durch Inkompetenz und oder einseitiger Beeinflussung
\item Tun ähnliches bis schlimmeres, als Google und co
\item Aber: anderer Auftrag => Verfehlt, mäh
\end{itemize}
}
}{}

\section{Ausblick}
\label{lookout}
Wenn wir als Bevölkerung in Zukunft auch noch die Vorteile eines freien Netzes nutzen wollen, muss Druck auf die Politik ausgeübt werden.
Politiker müssen scheinbar, wie beispielsweise bei den ACTA-Demonstrationen in eine bestimmte Richtung gestoßen werden, sie muss auf den Willen der Bevölkerung aufmerksam gemacht werden.
Gleichzeitig muss für einen Beratungsausgleich gesorgt werden, damit der Politik sich ein breiteres Sichtfeld auf ihre Entscheidungsfindung eröffnet.

\ifthenelse{\boolean{todos}}{
\todo[inline]{
naja werden ich dann sehen, sowas wie: wenn wir uns nicht ändern, dann problem
}
}{}

\nocite{*}
\bibliographystyle{alpha}
\bibliography{literature}

%\listoftables

%\listoffigures


\end{document}