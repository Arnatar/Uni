% Packages & Stuff

\documentclass[a4paper,11pt]{scrartcl}
\usepackage[german,ngerman]{babel}
\usepackage[utf8]{inputenc}
\usepackage[T1]{fontenc}
\usepackage[top=1.3in, bottom=1.2in, left=0.9in, right=0.9in]{geometry}
\usepackage{lmodern}
\usepackage{amssymb}
\usepackage{amsmath}
\usepackage{enumerate}
\usepackage{fancyhdr}
\usepackage{pgfplots}
\usepackage{multicol}

% ------------------------------------------------------

% Commands

\newcommand{\authorinfo}{Arne Struck, Alissa Dammmer, Sven-Hendrik Haase}
\newcommand{\titleinfo}{GSS-Übungsblatt 1 zum 16.04.2014}
\newcommand{\qed}{\ \square}
\newcommand{\todo}{\textcolor{red}{\textbf{TODO}}}
\newcommand{\opt}{\textcolor{blue}{\textbf{Optional}}}

% ------------------------------------------------------

% Title & Pages

\title{\titleinfo}
\author{\authorinfo}

\pagestyle{fancy}
\fancyhf{}
\fancyhead[L]{\authorinfo}
\fancyhead[R]{\titleinfo}
\fancyfoot[C]{\thepage}

\begin{document}
\maketitle
\section*{Aufgabe 1}
	\subsection*{1.:}
	\opt
	\subsection*{2.:}
	\opt
	\subsection*{3.:}
		\subsubsection*{a)}
		\opt
		\subsubsection*{b)}
		\opt
		
\section*{Aufgabe 2}
	\subsection*{1.:}
	\opt
	\subsection*{2.:}
	\opt
	\subsection*{3.:}
	\opt

\section*{Aufgabe 3}
	\subsection*{1.:}
	Ein Angreifermodell ist ein Modell, welches den stärksten Angreifer auf ein System (mit einem in dem 
	Modell spezifizierten Angriffsvektor), den der Systemschutz noch gerade abwehren kann. Ein solches Modell 
	wird aufgestellt, um aufzuzeigen wie gut bzw. schlecht ein System ungefähr geschützt ist. \\
	Es werden die folgenden Kriterien berücksichtigt: \\
	\begin{itemize}
		\item \textbf{Rollen:} 
		Unterteilt in In- und Outsider, enthält das Rollen-Kriterium die Position des Angreifers in Relation zum
		System. \\
		\item \textbf{Verbreitung:} 
		Beschreibt die Verbreitung des Angriffs, beispielsweise lokal oder netzweit. \\ 
		%(bspw. jeder OpenSSL-Nutzer) gnarf
		\item \textbf{Verhalten:} 
		Unterteilt in aktiv und passiv, beschreibt das Verhaltens-Kriterium, welche Eingriffe vorgenommen werden.\\
		\item \textbf{Rechenkapazitäten:} 
		Beschreibt die ungefähre Rechenkapazität des Angreifenden (Ausprägungen sind beispielsweise beschränkt 
		und unbeschränkt)
	\end{itemize}		
	
   	\subsection*{2.:}
	\opt


\section*{Aufgabe 4}
	\subsection*{1.:}
	\opt
	\subsection*{2.:}
	\opt
	\subsection*{3.:}
	\opt
	\subsection*{4.:}
	\opt
	\subsection*{5.:}
	\opt
\end{document}