\documentclass{beamer}
\usepackage[utf8]{inputenc}
\usepackage{lmodern}
\usepackage[ngerman]{babel}
\usepackage{graphicx}
\usepackage{listings}
\usepackage{hyperref}
\usepackage{color}


\definecolor{darkred}{rgb}{0.75,0,0.3}
\usetheme{Ilmenau}
\usecolortheme{beaver}
\setbeamercovered{invisible}
\beamertemplatenavigationsymbolsempty % macht die Navigationsleiste weg
\setbeamercolor{block body}{bg=darkred!7.5}
\setbeamercolor{block title}{bg=darkred}
\setbeamercolor*{item}{fg=darkred!90}

\lstset{
  basicstyle=\footnotesize
}

\title{Wochenbericht 3}
\subtitle{Analyse und Nachverarbeitung großer wissenschaftlicher Datenmengen mit Big-Data-Tools}
\author{Arne Struck}
\institute{Universität Hamburg, Fachschaft Informatik, Abschlussarbeitenseminar}
\date{\today}

\begin{document}
\begin{frame}
\maketitle
\end{frame}

\begin{frame}{Was ist passiert?}
\begin{columns}[t]
    \begin{column}{.5\textwidth}      	
      	\begin{block}{Plan:}
        	\begin{itemize}
				\item Thema wählen
	   			\item Exposé nochmal anfangen
		        \item mit WR Ausarbeitung des Themas
				\item Literatursuche
		    	\item Einarbeitung in das Thema
			\end{itemize}
    	\end{block}
    \end{column}
	\begin{column}{.5\textwidth}
		\uncover<2> {
		\begin{block}{umgesetzt:}
        	\begin{itemize}
        		\item Thema gewählt (sort of)
	        	\item Einarbeitung in Hadoop
	        	\begin{itemize}
	        		\item Aufsetzen eines 1 Node Clusters
	        		\item Einrichtung von Pig
	        		\item Einrichtung von Hive 
	        	\end{itemize}
	        	\item Ein wenig damit gearbeitet
    	    	\item Exposé angefangen
    	    	\item Literatursuche (leider nicht all zu erfolgreich)
    	    	\item SciDB angesehen
        	\end{itemize}
		\end{block}
		}
	\end{column}
\end{columns}
\end{frame}

\begin{frame}{Apache Hadoop}
	\begin{block}{}
		Freies Java-Framework zur Berechnung von parallelisierbaren Problemen auf verteilten Systemen. 
		Die Verteilung erfolgt halbautomatisiert.
	\end{block}		
	\quad \\
	\uncover<2-3> {
	Hauptkomponenten:
	\begin{itemize}
		\item HDFS (Hadoop distributed file system)
		\item MapReduce Implementation
	\end{itemize} 
	}

	\uncover<3> {
	Einige Erweiterungen (+Hauptfunktion):
	\begin{itemize}
		\item Hive (SQL-artige Abfragesprache HiveQL)
		\item Pig (Skriptsprache, um MapReduce overhead zu verringern)
	\end{itemize}
	}
\end{frame}

\begin{frame}[fragile]{HiveQl select statement}
\begin{lstlisting}  
SELECT [ALL | DISTINCT] select_expr, select_expr, ...
FROM table_reference
[WHERE where_condition]
[GROUP BY col_list]
[CLUSTER BY col_list
  | [DISTRIBUTE BY col_list] [SORT BY col_list]
]
[LIMIT number]
\end{lstlisting}
\end{frame}

\begin{frame}{Pig (Latin)}
Pig:
\begin{itemize}
	\item Pig environment
	\item Pig Latin
\end{itemize}

Pig Latin:
\begin{itemize}
	\item Load \(\Rightarrow\) Transform \(\Rightarrow\) Dump or Store
	\item Ausführung per Konsole, Interpreter oder eingebettet in Java Programme
	\item Was passiert: Script wird per MapReduce auf den Cluster verteilt und ausgeführt 
	\item (Angeblich) komplett, aber UDFs (User defined functions) in Java, Javascript, Python, Ruby möglich
\end{itemize}
\end{frame}

\begin{frame}{Plan für die nächsten Wochen:}
	\begin{block}{Plan}
		\begin{itemize}
			\item Konflikt beseitigen
			\item Thema/Titel konkretisieren
			\item Kenntnisse erweitern
			\item Exposé endlich fertig stellen
			\item Anmelden
			\item Literatursuche
		\end{itemize}
	\end{block}
\end{frame}

\setbeamertemplate{bibliography item}{\insertbiblabel}
\setbeamercolor{bibliography item}{parent=palette primary}
\setbeamercolor*{bibliography entry title}{parent=palette primary}
\begin{frame}[shrink=10]{References}
\nocite{*}
\bibliographystyle{alpha}
\bibliography{literature} 
\end{frame}

\end{document}
