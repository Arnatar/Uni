\documentclass[a4paper]{scrartcl}
\usepackage[ngerman]{babel}
\usepackage[utf8]{inputenc}
\usepackage[T1]{fontenc}
\usepackage{lmodern}
\usepackage{amssymb}
\usepackage{amsmath}
\usepackage{enumerate}
\usepackage{pgfplots}
\usepackage{scrpage2}\pagestyle{scrheadings}
\usepackage{tikz}
\usetikzlibrary{patterns}

\newcommand{\titleinfo}{Hausaufgaben zum 9. 11. 2012}
\title{\titleinfo}
\author{Tronje Krabbe 6435002, The-Vinh Jackie Huynh 6388888,\\Arne Struck 6326505}
\date{\today}
\chead{\titleinfo}
\ohead{\today}
\setheadsepline{1pt}
\setcounter{secnumdepth}{0}
\setlength{\textheight}{23cm}
\newcommand{\qed}{\ \square}

\begin{document}
\maketitle
\notag
\section{1.}
	\subsection{a)}
		\[
		\begin{array}{cl}
			&1385-0532 \\
			\Leftrightarrow &1385+K_ {10}(0532)_{10} \\
			=&1385+9468+1 \\ 
			=&[1]\underline{0854} \\
		\end{array}\\
		\]
		

	\subsection{b)}
		\[
		\begin{array}{cl}
			&0372-0687 \\
			\Leftrightarrow & 0372 + K_{10}(0687)_{10} \\
			=&0372+9312+1 \\
			=&9685 \\
			\overset{*}{\Leftrightarrow} & K_{10}(9685)_{10} \\
			=& (-)315
		\end{array}
		\]


	\subsection{c)}
	\[
	\begin{array}{cl}
		 &010101101001_2-001000010100_2 \\
		 \Leftrightarrow & 010101101001_2 + K_2(001000010100)_2 \\
		=&010101101001_2+110111101100_2 \\
		=&[1]\underline{001101010101_2} = \underline{853_{10}}\\
	\end{array}
	\]

	
	\subsection{d)}
	\[
	\begin{array}{cl}
		 &000101110100_2-001010101111_2 \\
		 \Leftrightarrow & 000101110100_2 + K_2(001010101111)_2 \\
		=&000101110100_2+110101010001_2 \\
		=&111011000101_2 \\	
		\overset{*}{\Leftrightarrow} & K_2(111011000101)_2 \\
		=& \underline{000100111010_2} = \underline{(-)315_{10}}\\
	\end{array}
	\] \\
	* Da eine negative Zahl erwartet wird (Das Ergebnis ist größer als \(4999_{10}\)), muss noch das 
	entsprechende Komplement gebildet werden.
	
\section{2.}
	\subsection{a)}
		Im Zehnersystem:
		\begin{align}
			 &(69,242\mid 3)_{10} \\
			\Rightarrow & 69,242\cdot 10^3\\
			=&6,9242\cdot 10^4		
		\end{align}
		
	\subsection{b)}
		Im Binärsystem:
		\begin{align}
			&(-11001,01\mid -110)_2 \\
			\Rightarrow & -11001,01\cdot 10^{-110} \\
			= & -1,100101\cdot 10^{-010}
		\end{align}
	\subsection{c)}	
		Im Hexadezimalsystem:
		\begin{align}
			&(-0,002\text{D4A}\mid \text{E}) \\
			\Rightarrow & -0,002\text{D4A} \cdot 10^\text{E} \\
			=&-0,2\text{D4A}\cdot 10^\text{B}
		\end{align}
		

\section{3.}	
	\subsection{a)}
		\(1011011\) in Gleitkommadarstellung: \\ \\
		\(0\quad 1000\ 0101\quad 0110\ 1100\ 0000\ 0000\ 0000\ 000\)		
		
	\subsection{b)}
		\(-1010 1000, 101\) in Gleitkommadarstellung: \\ \\
		\(1\quad 1000\ 0110\quad 0101\ 0001\ 0100\ 0000\ 0000\ 000\)
\newpage
\section{4.}
	\subsection{a)}
		\[
		\begin{array}{rrcl}
			&8,626 \cdot 10^5  + 9,9442 \cdot 10^7 &=& 0,08626 \cdot 10^7 + 9,9442 \cdot 10^7 \\
			&&=&(0,08626 + 9,9442) \cdot 10^7
		\end{array}
		\] \\	
		\[
		\begin{array}{cl}
				 	 &\ \, 0,08626  \\
					+&\ \, 9,9442   \\ \hline
			\text{Ü} & 1 1,1 1\ \ \\ \hline\hline
			&10,03046
		\end{array}
		\] \\
		\[
		\begin{array}{rcl}
			10,03046\cdot 10^7 = 1,003046 \cdot 10^8 \approx \underline{1,0030\cdot 10^8}
		\end{array}
		\]
		
	\subsection{b)}
		\[
		\begin{array}{rrcl}
			&8,626 \cdot 10^5  + 9,9442 \cdot 10^7 &=& 0,0863 \cdot 10^7 + 9,9442 \cdot 10^7 \\
			&&=&(0,0863 + 9,9442) \cdot 10^7
		\end{array}
		\] \\	
		\[
		\begin{array}{cl}
				 	 &\ \, 0,0863  \\
					+&\ \, 9,9442   \\ \hline
			\text{Ü} & 1 1,1 1\ \ \\ \hline\hline
			&10,0305
		\end{array}
		\] \\
		\[
		\begin{array}{rcl}
			10,0305\cdot 10^7 \approx \underline{1,0031 \cdot 10^8}
		\end{array}
		\]

	\subsection{c)}
		Das in a) genutzte Rundungsverfahren ist das genauere und damit vorzuziehende.
		In b) wird eine doppelte Rundung durchgeführt welche dazu führt, dass \(0,46\) zu \(1\) wird. 
		und somit ein weitaus ungenaueres Ergebnis liefert.

\section{5.}
	\[
	\begin{array}{rcll}
			5,6538 \cdot 10^7 \cdot 3,1415 \cdot 10^4 &=& (5,6538 \cdot 3,1415) \cdot 10^{7+4}&\mid \text{Vorzeichen: } 0\oplus 0 = 0 \\
			&=&17,7614127 \cdot 10^{11} \\
			&\approx & 1,7761 \cdot 10^{12}
	\end{array}
	\]


\end{document}