\documentclass[a4paper]{scrartcl}
\usepackage[ngerman]{babel}
\usepackage[utf8]{inputenc}
\usepackage[T1]{fontenc}
\usepackage{lmodern}
\usepackage{amssymb}
\usepackage{amsmath}
\usepackage{enumerate}
\usepackage{pgfplots}
\usepackage{scrpage2}\pagestyle{scrheadings}
\usepackage{tikz}
\usepackage{listings}
\usetikzlibrary{patterns}

\newcommand{\titleinfo}{Hausaufgaben zum 18. 1. 2013}
\title{\titleinfo}
\author{Tronje Krabbe 6435002, The-Vinh Jackie Huynh 6388888,\\Arne Struck 6326505}
\date{\today}
\chead{\titleinfo}
\ohead{\today}
\setheadsepline{1pt}
\setcounter{secnumdepth}{0}
\lstset{language=Java}
\newcommand{\qed}{\ \square}

\begin{document}
\maketitle

\section{11.1}
	\begin{itemize}
		\item a) 30
		\item b) 40
		\item c) 50
		\item d) 30
		\item e) 40
	\end{itemize}

\section{11.2)}
	1. Befehlsformat: \\
	\framebox[2cm]{3 bit}\framebox[3cm]{5 bit}\framebox[10cm]{24 bit}
	Opcode \hspace{1,3cm} Register \hspace{4,9cm} Adresse
\\
\\
2. Befehlsformat: \\
\framebox[2cm]{3 bit}\framebox[3.5cm]{7 bit}\framebox[3cm]{5 bit}\framebox[3cm]{5 bit}\framebox[3.5cm]{12 bit}
Codepräfix \hspace{0,9cm} Opcode 2 \hspace{1,4cm} Register 1 \hspace{1cm} Register 2 \hspace{1,2cm} Adresse
\\
\\
3. Befehlsformat: \\
\framebox[2cm]{3 bit}\framebox[3.5cm]{7 bit}\framebox[2,5cm]{4 bit}\framebox[7cm]{18 bit}
Codepräfix \hspace{0.7cm} Codepräfix 2 \hspace{0,8cm} Opcode 3\hspace{3cm} ungenutzt 

\newpage
\section{11.3}
	\subsection{a)} 0000\ 1011\ 1001
	
	\subsection{b)} Nicht darstellbar, der Abstand vom ersten 1-bit bis zum letzten 1-bit im
	 Ergebnis ist 9 Stellen, dies kann nicht in imm8 dargestellt werden.

	\subsection{c)} Da eine Rotation um 29 Stellen nötig wäre nicht darstellbar.

	\subsection{d)} 1110\ 0110\ 0011

	\subsection{e)} 0010\ 0000\ 1001

\section{Aufgabe 4)}
	\subsection{a)}
		\begin{center}
			\begin{tabular}{|c|c|c|c|}
				\hline
				0-Adress: & 0-Adress: & 2-Adress: & 3-Adress:\\ \hline
				PUSH F & LOAD D		& MUL B,C & MOV M,B \\
				PUSH E & MUL E		& SUB B,A & MUL M,C \\
				PUSH D & ADD F  	& MUL D,E & SUB M,A \\
				MUL	   & STORE M	& ADD D,F & MOV N,D \\
				ADD	   & LOAD B		& DIV A,D & MUL N,E \\
				PUSH C & MUL C		& MOV M,A & ADD N,F \\
				PUSH B & SUB A		&	 * 	  &	DIV M,N \\
				MUL	   & DIV M		&	 *    &     *   \\
				SUB	   &     *		&	 *    &     *   \\						
				DIV	   &     *		&	 *    &     *   \\						
				POP R  &     *		&	 *    &     *   \\ \hline
			\end{tabular}
		\end{center}				 
		
	\subsection{b)}
		\begin{center}
			\begin{tabular}{c|c|c|c|c}
				Maschine & Opcodes & Registernummer & Speicheradressen & Größe \\ \hline
				0-Adress & \(12\times 8\) bit& \(0\times 4\) bit& \(7\time 16\) bit& 208 bit\\
				1-Adress & \(11\times 8\) bit& \(0\times 4\) bit& \(11\time 16\) bit& 264 bit\\
				2-Adress & \(6 \times 8\) bit& \(0\times 4\) bit& \(12\time 16\) bit& 240 bit\\
				3-Adress & \(12\times 8\) bit& \(22\times 4\) bit& \(7\time 16\) bit& 296 bit\\
			\end{tabular}
		\end{center}
	
\end{document}