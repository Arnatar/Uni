\documentclass[a4paper]{scrartcl}
\usepackage[ngerman]{babel}
\usepackage[utf8]{inputenc}
\usepackage[T1]{fontenc}
\usepackage{lmodern}
\usepackage{amssymb}
\usepackage{amsmath}
\usepackage{enumerate}
\usepackage{pgfplots}
\usepackage{scrpage2}\pagestyle{scrheadings}
\usepackage{tikz}
\usepackage{listings}
\usetikzlibrary{patterns, calc}
\input {kvmacros}


\newcommand{\titleinfo}{Hausaufgaben zum 21. 12. 2012}
\title{\titleinfo}
\author{Tronje Krabbe 6435002, The-Vinh Jackie Huynh 6388888,\\Arne Struck 6326505}
\date{\today}
\chead{\titleinfo}
\ohead{\today}
\setheadsepline{1pt}
\setcounter{secnumdepth}{0}
\lstset{language=Java}
\newcommand{\qed}{\ \square}

\begin{document}
\maketitle
\notag
\tikzstyle{kvloop}=[draw, opacity=1, line width=0.4mm, rounded corners=2mm]
\section{9.1}
\begin{tikzpicture}[x=0.5cm,y=0.5cm]
        \foreach \y/\l in {0/C,-2/D,-4/Q (D-Latch),-6/Q (D-FF)} {
            \draw[step=1,draw=gray] (0, \y) grid (18,{1+\y});
            \draw[draw=black,thin] (0, \y) to (0,{1.4 + \y})
                (-0.4, \y) to (18.4, \y)
                (-0.4, {1 + \y}) -> (0, {1 + \y});

            \node[label=left:\tiny0] at (0, \y) {};
            \node[label=left:\tiny1] at (0, {\y + 1}) {};

            \node[label=left:\small\l] at (-1, {\y + 0.5}) {};
        }

        % C input
        \draw[draw=black,very thick] (0, 0) to (2, 0) to (2, 1) to (3, 1) to (3, 0) to (5, 0) to
        (5, 1) to (6, 1) to (6, 0) to (8, 0) to (8, 1) to (10, 1) to (10, 0) to (12, 0) to (12, 1)
        to (15, 1) to (15, 0) to (16, 0) to (16, 1) to (18, 1);

        % D input
        \draw[draw=black,very thick] (0, -2) to (1, -2) to (1, -1) to (4, -1) to (4, -2) to (7, -2) 
        to (7, -1) to (9, -1) to (9, -2) to (11, -2) to (11, -1) to (12, -1) to (12, -2) to (14, -2)
        to (14, -1) to (17, -1) to (17, -2) to (18, -2);

        % D-Latch
        \draw[draw=black,very thick] (0, -4) to (3, -4) to (3, -3) to (6, -3) to (6, -4) to (9, -4)
            to (9, -3) to (10, -3) to (10, -4) to (17, -4) to (17, -3) to (18, -3);

        % D-Flipflip
        \draw[draw=black,very thick] (0, -6) to (3, -6) to (3, -5) to (6, -5) to (6, -6) to (9, -6)
        to (9, -5) to (13, -5) to (13,-6) to (17,-6) to (17,-5) to (18, -5);


    \end{tikzpicture}
	
\section{9.2}
    \subsection{a)}
    \begin{center}
	    \begin{tabular}{ccc}
  			1. Flipflop mit Multiplexer: &\quad\quad & 2. Flipflop mit Taktausblendung: \\ \\
	        \begin{tabular}{|c|c|c||c|}\hline
	            D & E & CLK & $Q^+$ \\ \hline
	            0 & 0 & 0 & Q \\
	            0 & 0 & $\uparrow$ & Q \\
	            0 & 1 & 0 & Q \\
	            0 & 1 & $\uparrow$ & 0 \\
	            1 & 0 & 0 & Q \\
	            1 & 0 & $\uparrow$ & Q \\
	            1 & 1 & 0 & Q \\
	            1 & 1 & $\uparrow$ & 1 \\ \hline
	        \end{tabular}
	       	&\quad\quad &
	        \begin{tabular}{|c|c|c||c|}\hline
	            D & E & CLK & $Q^+$ \\ \hline
	            0 & 0 & 0 & Q \\
	            0 & 0 & 1 & Q \\
	            0 & 1 & 0 & Q \\
	            0 & 1 & 1 & 0 \\
	            1 & 0 & 0 & Q \\
	            1 & 0 & 1 & Q \\
	            1 & 1 & 0 & Q \\
	            1 & 1 & 1 & 1 \\ \hline
	        \end{tabular}  
    	\end{tabular}
    \end{center}
    
	\newpage
    \subsection{b)}
        Flipflops werden zur Speicherung von Daten verwendet. Die Schaltungen in diesem Fall
        garantieren, dass ein Datum (D) nur gespeichert wird, wenn eine Enable Signal (E) gegeben,
        und gerade getaktet (CLK) wird.

    \subsection{c)}
        Nachteil beim Ersten: Da D zuerst durch einen Multiplexer läuft, bevor es das Flipflop
        erreicht, kommt es einen Takt später als das Clock-Signal. Dies kann dazu führen, dass ein
        falsches D gespeichert wird. Nachteil beim Flipflop mit Taktausblendung: Dadurch, dass E und
        CLK durch ein AND-Gatter verknüpft sind, kommt das Taktsignal immer einen Takt später als D,
        was dazu führen kann, dass ein falsches D eingelesen und gespeichert wird. 

	
\section{9.3}
	\subsection{a)}
		\begin{center}
			\begin{tikzpicture}[x=2em,y=-1.5em,every node=radius\=3]
		        	\tikzstyle{node}=[draw,circle,radius=3cm,font=\small,inner sep=0,minimum 
		                       size=2.3em]
	                	\node [node] (z0) at (1,6) {$z_0$};
						\node [node] (z1) at (2.5,2.75) {$z_1$};
						\node [node] (z2) at (6,1) {$z_2$};
						\node [node] (z3) at (9.5,2.75) {$z_3$};
						\node [node] (z4) at (11,6) {$z_4$};
						\node [node] (z5) at (9.5,9.5) {$z_5$};
						\node [node] (z6) at (6,11) {$z_6$};
						\node [node] (z7) at (2.5,9.5) {$z_7$};
						
						\draw[thick, ->] (z0) [bend left = 15 pt] to (z1);
						\draw[thick, ->] (z1) [bend left = 15 pt] to (z2);
						\draw[thick, ->] (z2) [bend left = 15 pt] to (z3);
						\node[label={right:i=1}] at ($(z2)+(1.5,-0.2)$) {};
						\draw[thick, ->] (z3) [bend left = 15 pt] to (z4);
						\draw[thick, ->] (z4) [bend left = 15 pt] to (z5);
						\draw[thick, ->] (z5) [bend left = 15 pt] to (z6);
						\draw[thick, ->] (z6) [bend left = 15 pt] to (z7);
						\draw[thick, ->] (z7) [bend left = 15 pt] to (z0);
						\draw[thick, ->] (z2) [loop] to (z2);
						\node[label={right:i=0}] at ($(z2)+(0.9,-2)$) {};
	
		    \end{tikzpicture}
		\end{center}
		  
	\subsection{b)}
		\begin{center}
			\begin{tabular}{|c| c c c || c c c | c c c | c c c |}\hline
				i & $z_2$ & $z_1$ & $z_0$ & $z_{2}^+$ & $z_{1}^+$ & $z_{0}^+$ & $rt_H$ & $ge_H$ & 
				$gr_H$ & $rt_N$ & $ge_N$ & $gr_N$ \\ \hline

				* & 0 & 0 & 0 & 0 & 0 & 1 &	1 & 0 & 0 & 1 & 0 & 0\\
				* & 0 & 0 & 1 & 0 & 1 & 0 & 1 & 1 & 0 & 1 & 0 & 0\\
				0 & 0 & 1 & 0 & 0 & 1 & 0 & 0 & 0 & 1 & 1 & 0 & 0\\
				1 & 0 & 1 & 0 & 0 & 1 & 1 & 0 & 0 & 1 & 1 & 0 & 0\\
				* & 0 & 1 & 1 & 1 & 0 & 0 & 0 & 1 & 0 & 1 & 0 & 0\\
				* & 1 & 0 & 0 & 1 & 0 & 1 & 1 & 0 & 0 & 1 & 0 & 0\\
				* & 1 & 0 & 1 & 1 & 1 & 0 & 1 & 0 & 0 & 1 & 1 & 0\\
				* & 1 & 1 & 0 & 1 & 1 & 1 & 1 & 0 & 0 & 0 & 0 & 1\\
				* & 1 & 1 & 1 & 0 & 0 & 0 & 1 & 0 & 0 & 0 & 1 & 0\\ \hline
			\end{tabular}
		\end{center}
		
	\subsection{c)}
		\kvnoindex
		\begin{center}
			\begin{tabular}{lll}
				\karnaughmap{3}{$rt_H$}{{$z_1$}{$z_2$}{$z_0$}}{11110011}{}
				{
					\put(-2,0.5){\oval(3.9,0.8)[]}
					\put(-3,1){\oval(1.9,1.9)[]}
				}&\quad\quad &
				\karnaughmap{3}{$ge_H$}{{$z_1$}{$z_2$}{$z_0$}}{010001000}{}
				{
					\put(-2,1.5){\oval(1.9,0.9)[]}
				} \\
				DNF: $z_0\land\overline{z_2}$ &\quad\quad &
				DNF: $z_2\vee \overline{z_1}$ \\ \\ \\

				\karnaughmap{3}{$gr_H$}{{$z_1$}{$z_2$}{$z_0$}}{000010000}{}
				{
					\put(-0.5,1.5){\oval(0.9,0.9)[]}
				}&\quad\quad &
				\karnaughmap{3}{$rt_N$}{{$z_1$}{$z_2$}{$z_0$}}{11111100}{}
				{
					\put(-3,1){\oval(1.9,1.9)[]}
					\put(-2,1.5){\oval(3.9,0.8)[]}
				} \\
				DNF: $\overline{z_2}\land\overline{z_1}\land\overline{z_0}$ &\quad\quad &
				DNF: $\overline{z_2}\vee \overline{z_1}$ \\ \\ \\
				
				\karnaughmap{3}{$ge_N$}{{$z_1$}{$z_2$}{$z_0$}}{00010001}{}
				{
					\put(-2,0.5){\oval(1.9,0.8)[]}
				}&\quad\quad &
				\karnaughmap{3}{$gr_N$}{{$z_1$}{$z_2$}{$z_0$}}{00000010}{}
				{
					\put(-0.5,0.5){\oval(0.9,0.9)[]}
				} \\
				DNF: $z_2\land z_0$ &\quad\quad &
				DNF: $z_2\land z_1\land\overline{z_0}$ 
			\end{tabular}
		\end{center}
		\newpage
		\begin{center}
			\begin{tabular}{lll}
				\karnaughmap{3}{$z^+_2$}{{$z_1$}{$z_2$}{$z_0$}}{00110110}{}
				{
					\put(-3,0.5){\oval(1.9,0.8)[]}
					\put(0,0.5){\oval(1.9,0.9)[l]}
					\put(-4,0.5){\oval(1.9,0.9)[r]}
				}&\quad\quad &
				\karnaughmap{3}{$z^+_1$}{{$z_1$}{$z_2$}{$z_0$}}{01011010}{}
				{
					\put(-0.5,1){\oval(0.9,1.9)[]}
					\put(-2.5,1){\oval(0.9,1.9)[]}
				} \\				
				DNF: $(z_2\land\overline{z_1})\vee (z_2\land \overline{z_0})\vee(z_0\land 
				z_1\land \overline{z_2})$ &\quad\quad &
				DNF: $(z_0\land \overline{z_1})\vee (z_1\land \overline{z_0})$ \\ \\ \\

				\karnaughmap{3}{$z^+_0$}{{$z_1$}{$z_2$}{$z_0$}}{00101010}{}
				{
					\put(0,0.5){\oval(1.9,0.8)[l]}
					\put(-4,0.5){\oval(1.9,0.8)[r]}
					\put(-0.5,1){\oval(0.9,1.9){}}
				} \\
				DNF: $(z_1\land \overline{z_0})\vee (z_2\land \overline{z_0})$
			\end{tabular}
		\end{center}


	

\end{document}