\documentclass[a4paper]{scrartcl}
\usepackage[ngerman]{babel}
\usepackage[utf8]{inputenc}
\usepackage[T1]{fontenc}
\usepackage{lmodern}
\usepackage{amssymb}
\usepackage{amsmath}
\usepackage{enumerate}
\usepackage{pgfplots}
\usepackage{scrpage2}\pagestyle{scrheadings}
\usepackage{tikz}
\usetikzlibrary{patterns}

\newcommand{\titleinfo}{Hausaufgaben zum 29. 10. 2012}
\title{\titleinfo}
\author{Tronje Krabbe 6435002, The-Vinh Jackie Huynh 6388888,\\Arne Struck 6326505}
\date{\today}
\chead{\titleinfo}
\ohead{\today}
\setheadsepline{1pt}
\setcounter{secnumdepth}{0}
\setlength{\textheight}{22cm}
\newcommand{\qed}{\ \square}

\begin{document}
\maketitle
\notag
\section{1.}
	\subsection{a)}
			Ein Interpreter übersetzt Befehle von höheren Ebenen in Befehle niederer Ebenen. Dies 
			erfolgt zur Laufzeit des interpretierten Programmes. 
	\subsection{b)}
			Ein Compiler übersetzt, wie ein Interpreter Befehle von höheren Ebenen auf niedrigere 
			Ebenen (meistens die Systemebene). Allerdings erzeugt der Compiler zu diesem Zweck eine 
			auf der Zielebene ausführbare Datei, die Analyse des Codes erfolgt also vor der Laufzeit 
			der compilierten Programmes. 
	\subsection{c)}
			Eine virtuelle Maschine ist die Simulation einer Hardwareebene auf einem Host-System.
			Für Programme einer höheren Ebene bildet sie die eine neue L0-Ebene, auch wenn in 
			Wahrheit das Programm von der VM weiter verarbeitet wird. 


\section{2.}
	\(T(i)\) sei die Dauer eines Befehls auf der Ebene \(Li\).
	\[
	\begin{array}{rcl}
		T(0) &=& k \\
		T(1) &=& n\cdot k \\
		T(2) &=& n^2\cdot k \\
		T(3) &=& n^3\cdot k \\
		T(i) &=& n^i\cdot k
	\end{array}
	\]
	Um von der Ebene 0 zur Ebene 1  zu gelangen werden \(n\cdot k\) Nanosekunden benötigt, da 
	für diesen Sprung \(n\) Befehle benötigt werden und diese Befehle auf der Ebene 0 pro Befehl 
	\(k\) Nanosekunden zur Ausführung benötigen.
	Um zur Ebene \(2\ (=0+1+1)\) zu gelangen werden wieder \(n\) Befehle benötigt, also \(n\cdot 
	n\cdot k\). Nach diesem Muster erklärt sich die finale Formel für die Ebene \(i\).
	
	
\section{3.}
	\subsection{Vorteile:}
		Das von-Neumann-Konzept ist sehr wichtig für das Compiler-Prinzip, denn so kann ein 
		Compiler Maschinencode im Speicher erzeugen.\\
		Des weiteren kann sich ein Programm selber optimieren, indem es sich manipulieren.\\
		Durch das Prinzip kann der Hardwareaufwand verringert werden. \\
		Aufgrund des sequentiellen Ablaufs erfolgt ein einfacher Programmablauf.
		
	\subsection{Nachteile:}	
		Selbstmodifikation kann zu Laufzeitfehlern führen. Dies kann auch gezielt von Schadsoftware 
		genutzt werden.\\
		Es existiert nur eine Verbindung zwischen Speicher und Prozessor, weshalb nur eine 
		Aktion auf einmal bearbeitet werden kann.


\section{4.}
	\subsection{a)}
		\subsubsection{Nach der Formel:}
			\underline{Multiplikationen:}
			\begin{align}
				x^n &\Rightarrow n-1 \text{ Multiplikationen} \\
    			\\
    			a\cdot x^6 &\Rightarrow (5+1)\cdot 6\, ns = 36\, ns \\
    			b\cdot x^5 &\Rightarrow (4+1)\cdot 6\, ns = 30\, ns \\
   		 		c\cdot x^4 &\Rightarrow (3+1)\cdot 6\, ns = 24\, ns \\
   		 		d\cdot x^3 &\Rightarrow (2+1)\cdot 6\, ns = 18\, ns \\
   		 		e\cdot x^2 &\Rightarrow (1+1)\cdot 6\, ns = 12\, ns \\
   	 			f\cdot x^1 &\Rightarrow (0+1)\cdot 6\, ns = 6\, ns 
			\end{align}
			\[36\, ns+30\, ns+24\, ns+18\, ns+12\, ns+6\, ns=126\, ns\]\\
   		 	\underline{Additionen:}\\
    			Innerhalb des Polynoms wird 6 mal Addiert, also müssen zu den \(126\, ns\ 6\, ns\) 
    			addiert werden.\\
				\\    	
	    	Also dauert die Berechnung nach der Formel insgesamt \(132\, ns\)\\
 
    	\subsubsection{Horner-Schema:}
			\begin{align}
				f(x)&= a\cdot x^6+b\cdot x^5+ c\cdot x^4+d \cdot x^3+e\cdot x^2+ f\cdot x + g \\
				f(x)&= (a\cdot x^5+b\cdot x^4+ c\cdot x^3+d \cdot x^2+e\cdot x+ f) \cdot x + g \\
				f(x)&= ((a\cdot x^4+b\cdot x^3+ c\cdot x^2+d \cdot x+e)\cdot x+ f) \cdot x + g \\
				f(x)&= (((a\cdot x^3+b\cdot x^2+ c\cdot x+d) \cdot x+e)\cdot x+ f) \cdot x + g \\
				f(x)&= ((((a\cdot x^2+b\cdot x+ c)\cdot x+d) \cdot x+e)\cdot x+ f) \cdot x + g \\
				f(x)&= (((((a\cdot x+b)\cdot x+ c)\cdot x+d) \cdot x+e)\cdot x+ f) \cdot x + g 
			\end{align}			    
			Nun lässt sich sehen, dass die Menge der Rechenoperationen durch das Horner-Schema auf 
			\(6\) Additionen und \(6\) Multiplikationen reduziert wurde. Daraus folgt:
			\[6\cdot 1\, ns+6\cdot 6\, ns=42\, ns\]
			Also beträgt die durch das Horner-Schema benötigte Zeit nur \(42\, ns\).
				
	\subsection{b)}
		\(		
		\begin{array}{rcccl}
			a&=&x+1	\\
			b&=&a\cdot a &=& (x+1)^2 \\
			c&=&b\cdot b &=& (x+1)^4 \\
			d&=&c\cdot c &=& (x+1)^8 \\
			e&=&d\cdot d &=& (x+1)^{16} \\
			y&=&e\cdot c\cdot b \cdot a &=& (x+1)^{23} 
		\end{array}
		\) \\ \\
		Es wird eine Addition für  \(a\) benötigt. Des weiteren werden \(b\) eine, für \(c\) eine 
		weitere und für \(e\) 2 weitere Multiplikationen benötigt. Um \(y\) zu erhalten, werden noch 
		3 weitere Multiplikationen genutzt. Also braucht \((x+1)^{23}\) insgesamt 7 Multiplikationen.
		Daraus folgt, dass die CPU insgesamt \((7\cdot 6+1)\, ns=43\, ns\) benötigt.
		
		
		
\section{5.}	
	\subsection{a)}
		Im folgenden rechnen wir damit, dass ein Jahr genau 365,25 Tage besitzt und nehmen den 
		Fehler durch die Ungenauigkeit der Schaltjahre in kauf.\\ \\
		\(
		\begin{array}{llclcl}
			\text{Pro Tag:} & 5\,\frac{\text{MiB}}{\text{s}}\cdot (60\cdot 60\cdot 24)\,\text{s} &=& 
				432000\,\text{MiB} &=& 421,875\,\text{GiB} \\
			\text{Pro Jahr:} & 421,875\,\text{GiB}\cdot 365,25 &\approx & 154089,844\,\text{GiB} 	
				&\approx & 150,5\,\text{TiB} \\
			\text{Pro Leben:} & 150,5\,\text{TiB}\cdot 80 &\approx & 12040\,\text{TiB}
		\end{array}
		\)

	\subsection{b)}
		\(x\) sei die Anzahl der Jahre, die noch zum Wachstum der durchschnittlichen Festplattengröße 
		benötigt werden. 
		\begin{align}
			12040 &= 2\cdot 1,45^x \\
			6020 &= 1,45^x \\
			x &= \log_{1,45}(6020) \\
			x &\approx 23,4222 
		\end{align}
		Also ist nach 24 Jahren die benötigte Festplattengröße erreicht, das wäre im Jahr 2036.
		24 Jahre, da nach genau 23 Jahren die Festplatten noch nicht genügend Speicher haben, dies 
		geschieht erst im Verlauf des Jahres 2035.

	\subsection{c)}
		\(x\) sei die Anzahl der Jahre, die noch zum Wachstum der durchschnittlichen Festplattengröße 
		benötigt werden. 
		\begin{align}
			12040\,\text{TiB} &= \frac{32}{1024}\,\text{TiB} \cdot 1,52^x \\
			385280 &= 1,52^x \\
			x &= \log_{1,52}(385280) \\
			x &\approx 30,7175
		\end{align}
		Also ist nach 31 Jahren die benötigte Festplattengröße erreicht, das wäre im Jahr 2043.
		31 Jahre, da nach genau 30 Jahren die Festplatten noch nicht genügend Speicher haben, dies 
		geschieht erst im Verlauf des Jahres 2042.
		
	\subsection{d)}
		\[
		\begin{array}{lclcl}
			140\,\text{MiB}&=& \frac{140}{1024}\,\text{GiB}&=& \frac{140}{1024\cdot 1024}
				\,\text{TiB}\\\\
			\frac{12040\,\text{TiB}}{\frac{140}{1024\cdot 1024}\,\text{TiB}} &=& \frac{12624855040}
				{140} &=& 90177536\\
		\end{array}
		\]\\
		Also werden ungefähr (da mit einem gerundeten Ausgangswert gerechnet wurde) 90177536 
		Magnetbänder für die Aufzeichnung benötigt.		

\end{document}