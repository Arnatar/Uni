% Commands

\newcommand{\authorinfo}{Arne Struck, Lars Thoms}
\newcommand{\titleinfo}{AD [HA] zum 20. 11. 2013}
\newcommand{\qed}{\ \square}
\newcommand{\todo}{\textcolor{red}{\textbf{TODO}}}

% ------------------------------------------------------

% Packages & Stuff

\documentclass[a4paper,11pt,fleqn]{scrartcl}
\usepackage[german,ngerman]{babel}
\usepackage[utf8]{inputenc}
\usepackage[T1]{fontenc}
\usepackage[top=1.3in, bottom=1.2in, left=0.9in, right=0.9in]{geometry}
\usepackage{lmodern}
\usepackage{amssymb}
\usepackage{amsmath}
\usepackage{enumerate}
\usepackage{fancyhdr}
\usepackage{pgfplots}
\usepackage{algorithm}
\usepackage{algorithmicx}
\usepackage{algpseudocode}
\usepackage{multicol}
\usepackage[parfill]{parskip}
\usetikzlibrary{calc}
\usetikzlibrary{patterns}


% ------------------------------------------------------

% Title & Pages

\title{\titleinfo}
\author{\authorinfo}

\pagestyle{fancy}
\fancyhf{}
\fancyhead[L]{\authorinfo}
\fancyhead[R]{\titleinfo}
\fancyfoot[C]{\thepage}

\begin{document}
	\maketitle
	\begin{enumerate}
		\item[\textbf{1.}]
		\begin{enumerate}
			\item[(a)]\quad \\
				\(11 \cdot \mathbb{N} + 10\). Da für jeden Elferzyklus die 11. Stelle gesucht ist müssen nach dem Zyklus noch 10 
				addiert werden.
			\item[(b)]\quad \\
				\(11 \cdot \mathbb{N} + 5\). Das Ergebnis entsteht aus der Tatsache, dass nun \(2k \mod 11\) gilt, daher muss
				der Ausdruck aus (a) auch durch 2 geteilt werden, leider führt das dazu, dass jedes 2. Element aus der 
				Ergebnismenge gestrichen wird (\(\frac{11}{2}\) ist kein ganzes Vielfaches von 11). Somit muss \(\frac{11}{2}\) 
				durch \(11\) substituiert werden.
			\item[(c)]\quad \\
				\(11 \cdot \mathbb{N}\). Da die Operation 10 zu addieren (um an die 11. Stelle des Zyklus zu gelangen) schon 
				erfolgt ist, muss nur noch ein Vielfaches von 11 übergeben werden.
			\item[(d)]\quad \\
				Wenn Elemente existieren für die \(3^k - 1 mod 11 = 10\) gilt, müssen Elemente existieren für die gilt: 
				\(3^k \mod 11 = 0\). \\
				\(
				\begin{array}{rclclcl}
					3^0 &=& 1 \\
					3^1 &=& 3 \\
					3^2 &=& 9 \\
					3^3 &=& 27 \mod 11 &=& 5 \\
					3^4 &=& 5 \cdot 3 &=& 15 \mod 11 &=& 4 \\
					3^5 &=& 3 \cdot 4 &=& 12 \mod 11 &=& 1 \\
				\end{array}
				\) \\
				Hiermit kommt man in einen Zyklus, das bedeutet es existiert kein Element auf der 11. Stelle mit der Funktion
				\(3^k - 1 \mod 11\). Um etwas der leeren Menge gleichbedeutendes zu formulieren folgt \(\frac{\mathbb{N}}{0}\).
		\end{enumerate}
		\item[\textbf{2.}]\quad \\
		\todo
		\item[\textbf{3.}]
		\begin{enumerate}
			\item[(a)]\quad \\
			\todo
			\item[(b)]\quad \\
			\todo
			\item[(c)]\quad \\		
			\todo
		\end{enumerate}
		\item[\textbf{4.}]
		\begin{enumerate}
			\item[(a)]\quad \\
			\todo
			\item[(b)]\quad \\
			\todo
			\item[(c)]\quad \\		
			\todo
		\end{enumerate}
		\item[\textbf{5.}]
		\begin{enumerate}
			\item[(a)]\quad \\
			\todo
			\item[(b)]\quad \\
			\todo
		\end{enumerate}
		\item[\textbf{6.}]\quad \\
		\todo
	\end{enumerate}
\end{document}
