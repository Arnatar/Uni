% Commands

\newcommand{\authorinfo}{Arne Struck, Lars Thoms}
\newcommand{\titleinfo}{AD [HA] zum 18. 12. 2013}
\newcommand{\qed}{\ \square}
\newcommand{\limn}{\lim\limits_{n\to\infty}}
\newcommand{\todo}{\textcolor{red}{\textbf{TODO}}}
\newcommand{\Forall}{\textbf{for each}}

% ------------------------------------------------------

% Packages & Stuff

\documentclass[a4paper,11pt,fleqn]{scrartcl}
\usepackage[german,ngerman]{babel}
\usepackage[utf8]{inputenc}
\usepackage[T1]{fontenc}
\usepackage[top=1.3in, bottom=1.2in, left=0.9in, right=0.9in]{geometry}
\usepackage{lmodern}
\usepackage{amssymb}
\usepackage{amsmath}
\usepackage{enumerate}
\usepackage{fancyhdr}
\usepackage{pgfplots}
\usepackage{algorithm}
\usepackage{algorithmicx}
\usepackage{algpseudocode}
\usepackage{multicol}
\usepackage[parfill]{parskip}
\usetikzlibrary{automata,calc,patterns,shapes}


% ------------------------------------------------------

% Title & Pages

\title{\titleinfo}
\author{\authorinfo}

\pagestyle{fancy}
\fancyhf{}
\fancyhead[L]{\authorinfo}
\fancyhead[R]{\titleinfo}
\fancyfoot[C]{\thepage}

\begin{document}
	\maketitle
	\begin{enumerate}
		\item[\textbf{1.:}] \quad \\
			\begin{verbatim}
				BellmanFord_modified(G,s):
				    InitializeSingleSource(G,s)
				    for i = 1, ..., |V| - 1
				        nochanges = false
				        for all edges(u, v) in E
				            distTmp = v.dist
				            Relax(u,v)
				            if v.dist < distTmp
				                nochanges = true
				        if nochanges = true
				            return true
			\end{verbatim}
		
			Die Anpassung durch nochanges bewirkt eine Terminierung einen Durchlauf nachdem alle kürzesten
			Kantenpfade gefunden sind. Dies geschieht, da nach m Durchläufen alle kürzesten Kantenpfade entdeckt
			sind. Die Schleife wird noch einmal durchlaufen und hier wird festgestellt, dass insgesamt keine
			Änderungen an einem der Pfadgewichte vorgenommen wurde. Also wird der Algorithmus darauf Terminieren.
			Weitere Terminierungen sollten nicht sinnvoll sein, da sie durch den spezifizierten Input (kein
			negativen Zyklen) obsolet geworden sind. Der endgültige return kann auch weggelassen werden, da ein 	
			anderer return auf jeden Fall erreicht wird.
		\item[\textbf{2.:}] \quad \\
			\begin{verbatim}
			DAG-Shortest-Path(G,s):
			    sort G.V topologically
			    InitializeSingleSource(G,s)
			    for each u in G.V % now in topological order
			        for each v in Adj(u)
			            Relax(u,v)
			\end{verbatim}				
		% Bin mir bei der Begründung nicht sicher
		%	Die innere Schleife wird \(|E|\) mal durchlaufen, da jede Kante adressiert wird. \\
		%	Die äußere Schleife wird \(|V|\) mal durchlaufen (jeder Knoten ein mal). \\
		%	Durch die vorherige topologische Sortierung ist garantiert, dass der Algorithmus nicht mehrmals
		%	durchlaufen werden muss (da die Knoten jetzt in einer linearen Richtung vorliegen). \\
		%	Damit ist gezeigt, dass die Laufzeit \(\mathcal{O}(|E|+|V|)\) beträgt.
			
		\item[\textbf{3.:}] \quad \\
			Das Problem mit dem Dijkstra-Algorithmus mit negativem Kantengewicht ist, dass sie eventuell nicht
			berücksichtigt werden, da der Algorithmus nicht ''in die Zukunft sehen'' kann. \\
			Da wenn einer der Nachfolgeknoten von S auf dem kürzesten Pfad liegt, muss nur noch gezeigt werden, 
			dass Dijkstra den Knoten korrekt findet. \\
			Angefangen wird immer mit der kleinsten Kantengewichtung zu einem der Nachfolgeknoten von S (im 
			folgenden A). Sollte kein Pfad von A zu einem der anderen Nachfolger von B existieren, wird auch hier
			die negative Kante genommen. Sollte nun ein Pfad von A zu einem der anderen Nachfolger existieren, 
			wird Dijkstra korrekt vergleichen, welcher der beiden Pfade kürzer ist. \\
			Damit ist gezeigt, dass der Algorithmus funktioniert.
		\item[\textbf{4.:}]
		\begin{enumerate}
			\item[a)] \quad \\
				\todo
			\item[b)] \quad \\
				\todo
		\end{enumerate}
		\item[\textbf{5.:}]
		\begin{enumerate}
			\item[a)] \quad \\
				\todo
			\item[b)] \quad \\
				\todo
		\end{enumerate}
		\item[\textbf{6.:}] \quad \\
			\todo
	\end{enumerate}
\end{document}
