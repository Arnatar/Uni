% Commands

\newcommand{\authorinfo}{Arne Struck, Lars Thoms}
\newcommand{\titleinfo}{AD [HA] zum 6. 11. 2013}
\newcommand{\qed}{\ \square}

% ------------------------------------------------------

% Packages & Stuff

\documentclass[a4paper,11pt,fleqn]{scrartcl}
\usepackage[german,ngerman]{babel}
\usepackage[utf8]{inputenc}
\usepackage[T1]{fontenc}
\usepackage[top=1.3in, bottom=1.2in, left=0.9in, right=0.9in]{geometry}
\usepackage{lmodern}
\usepackage{amssymb}
\usepackage{amsmath}
\usepackage{enumerate}
\usepackage{fancyhdr}
\usepackage{pgfplots}
\usepackage{multicol}
\usepackage[parfill]{parskip}
\usetikzlibrary{calc}
\usetikzlibrary{patterns}


% ------------------------------------------------------

% Title & Pages

\title{\titleinfo}
\author{\authorinfo}

\pagestyle{fancy}
\fancyhf{}
\fancyhead[L]{\authorinfo}
\fancyhead[R]{\titleinfo}
\fancyfoot[C]{\thepage}

\begin{document}
	\maketitle
	\begin{enumerate}
		\item[\textbf{1.}]
		\begin{enumerate}
			\item[a)]\quad \\
				Es liegen \(k^l\) Blätter maximal in der l. Ebene. Leicht ersichtlich aus dem Folgenden: \\
				0. Ebene (root):\(1 = k^0\), 1. Ebene: \(k = k^1\), 2. Ebene: \(k\cdot k = k^2\),\\
				3. Ebene: \(k\cdot k \cdot k = k^3\) ... l.Ebene: \(k^l\)
			\item[b)]\quad \\
				Der volle Baum hat \(\sum\limits_{i=0}^lk^i\) Blätter, die Summe aller Ebenen (eine volle Ebene 
				bemisst sich, wie in a) dargestellt auf \(k^l\)).
			\item[c)]\quad \\
				Der vollständige Baum hat \(\sum\limits_{i=0}^{l-1}k^i +c\ |c\in\mathbb{N}:1\leq c\leq k^l\)
				Blätter. Der vollständige Baum ist bis zu seiner vorletzten Ebene maximal gefüllt, deswegen die 
				Summe bis $l-1$, c repräsentiert die Anzahl der Blätter in der letzten Ebene, welche zwischen 
				einem (sonst wäre der Baum voll und hätte l-1 Ebenen) und $k^l$ (ein voller Baum ist 
				vollständig) Blättern.
			\item[d)]\quad \\
				Der Baum hat n-1 Kanten, da jeder Knoten (bis auf den Wurzelknoten) eine Kante besitzt durch 
				die er mit seinem Elternknoten verbunden ist.
		\end{enumerate}
		
		\item[\textbf{2.}]
		\begin{enumerate}
			\item[a)]\quad \\
				\textcolor{red}{TODO}
			\item[b)]\quad \\
				\textcolor{red}{TODO}
			\newpage
			\item[c)]\quad \\
				\begin{tikzpicture}[every node/.style={draw,circle,inner sep=1.2pt},
                	    			level 1/.style={sibling distance=50mm},
			        	            level 2/.style={sibling distance=25mm},
            				        level 3/.style={sibling distance=12.5mm},
                				    level 4/.style={sibling distance=6.25mm}]
                				    \node{N}
                				    	child{node{A}
	                							child{node{0}
	                								child{node{E}
	                									child{node{I}}
	                								}
	                								child{node{F}}
	                							}
	                							child{node{M}
	                								child{node{R}}
	                								child{node{L}}
	                							}
	                					}
	                					child{node{U}
	                						child{node{S}
	                							child{node{G}}
	                							child{node{A}}
	                						}
	                						child{node{R}
	                							child{node{T}}
	                							child{node{H}}
	                						}
	                					};
					
				\end{tikzpicture} \\ \\
				Order1: NAOEIFMRLUSGARTH \\
				Order2: IEOFARMLNGSAUTRH \\
				Order3: IEFORLMAGASTHRUN \\
			\item[d)]\quad \\
				Der LOVELYTREE nach Order 2: \\ \\
				\begin{tikzpicture}[every node/.style={draw,circle,inner sep=1.2pt},
                	    			level 1/.style={sibling distance=50mm},
			        	            level 2/.style={sibling distance=25mm},
            				        level 3/.style={sibling distance=12.5mm},
                				    level 4/.style={sibling distance=6.25mm}]
                				    \node{T}
                				    	child{node{E}
                				    		child{node{O}
                				    			child{node{L}}
                				    			child{node{V}}
                				    		}
                				    		child{node{Y}
												child{node{L}}
                				    		}
                				    	}
                				    	child{node{E}
											child{node{R}}
											child{node{E}}                				    	
                				    	}
                				    ;					
				\end{tikzpicture} \\ \\
				Nach Level-Order: TEEOYRELVL \\
				\newpage
			\item[e)]\quad \\
				Ternärer Baum mit vorgegebener Befehlsreihenfolge: \\ \\
				\begin{tikzpicture}[every node/.style={draw,circle,inner sep=1.2pt},
                	    			level 1/.style={sibling distance=40mm},
			        	            level 2/.style={sibling distance=12.5mm},
            				        level 3/.style={sibling distance=10mm}]
                				    \node{N}
                				    	child{node{A}
                				    		child{node{M}
												child{node{T}}                				    			
												child{node{H}}
												child{node{I}}
                				    		}  
                				    		child{node{S}}
                				    		child{node{R}}
                				    	}
                				    	child{node{U}
        	        				    	child{node{E}}
    	            				    	child{node{F}}
	                				    	child{node{R}}
                				    	}
                				    	child{node{o}
        	        				    	child{node{L}}
    	            				    	child{node{G}}
	                				    	child{node{A}}
                				    	}
                				    ;					
				\end{tikzpicture} \\ \\
				Ausgabe: ALGORITHMSAREFUN
		\end{enumerate}
		\item[\textbf{3.}]
		\begin{enumerate}
			\item[a)]\quad \\
				\textcolor{red}{TODO}
			\item[b)]\quad \\
				\textcolor{red}{TODO}
			\item[c)]\quad \\
				\textcolor{red}{TODO}
			\item[d)]\quad \\
				\textcolor{red}{TODO}
			\item[e)]\quad \\
				\textcolor{red}{TODO}
			\item[f)]\quad \\
				\textcolor{red}{TODO}
		\end{enumerate}
		\item[\textbf{4.}]
		\begin{enumerate}
			\item[a)]\quad \\
				\textcolor{red}{TODO}
			\item[b)]\quad \\
				\textcolor{red}{TODO}
			\item[c)]\quad \\
				\textcolor{red}{TODO}
		\end{enumerate}
		\item[\textbf{5.}]
		\begin{enumerate}
			\item[a)]\quad \\
				\textcolor{red}{TODO}
			\item[b)]\quad \\
				\textcolor{red}{TODO}
		\end{enumerate}
	\end{enumerate}
\end{document}
