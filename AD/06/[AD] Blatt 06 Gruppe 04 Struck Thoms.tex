% Commands

\newcommand{\authorinfo}{Arne Struck, Lars Thoms}
\newcommand{\titleinfo}{AD [HA] zum 14. 01. 2014}
\newcommand{\qed}{\ \square}
\newcommand{\limn}{\lim\limits_{n\to\infty}}
\newcommand{\todo}{\textcolor{red}{\textbf{TODO}}}
\newcommand{\Forall}{\textbf{for each}}

% ------------------------------------------------------

% Packages & Stuff

\documentclass[a4paper,11pt,fleqn]{scrartcl}
\usepackage[german,ngerman]{babel}
\usepackage[utf8]{inputenc}
\usepackage[T1]{fontenc}
\usepackage[top=1.3in, bottom=1.2in, left=0.9in, right=0.9in]{geometry}
\usepackage{lmodern}
\usepackage{amssymb}
\usepackage{amsmath}
\usepackage{enumerate}
\usepackage{fancyhdr}
\usepackage{pgfplots}
\usepackage{algorithm}
\usepackage{algorithmicx}
\usepackage{algpseudocode}
\usepackage{multicol}
\usepackage[parfill]{parskip}
\usetikzlibrary{automata,calc,patterns,shapes}


% ------------------------------------------------------

% Title & Pages

\title{\titleinfo}
\author{\authorinfo}

\pagestyle{fancy}
\fancyhf{}
\fancyhead[L]{\authorinfo}
\fancyhead[R]{\titleinfo}
\fancyfoot[C]{\thepage}

\begin{document}
	\maketitle
	\begin{enumerate}
		\item[\textbf{1.:}]
		\begin{enumerate}
			\item[a)] \quad \\
			\begin{tabular}{c||c|c|c|c|c|c|c|c}
                Extract & 1 & 2 & 3 & 4 & 5 & 6 & 7 & \\ \hline\hline
                & 0 & $\infty$ & $\infty$ & $\infty$& $\infty$& $\infty$& $\infty$& $v.dist$ \\ \hline
                & - & - & - & - & - & - & - & v.$\pi$ \\ \hline
                1 & 0& 4& $\infty$& $\infty$& $\infty$& 5 & $\infty$ & v.dist \\ \hline
                & -& 1 & -& -& - & 1& - & v.$\pi$ \\ \hline
                2 & 0& 4& 14& $\infty$& $\infty$& 5 & 7 & v.dist \\ \hline
                & -& 1& 2& -& -& 1 & 2 & v.$\pi$ \\ \hline
                6 & 0& 4& 14& $\infty$& 14 & 5 & 7 & v.dist \\ \hline
                & -& 1& 2& -& 6 & 1 & 2 & v.$\pi$ \\ \hline
                7 & 0& 4& 13& $\infty$& 10 & 5 & 7 & v.dist \\ \hline
                & -& 1& 7& -& 7 & 1 & 2 & v.$\pi$ \\ \hline
                5 &0 &4 &12 & 15& 10 & 5 & 7 & v.dist \\ \hline
                & -& 1& 5& 5& 7 & 1 & 2 & v.$\pi$ \\ \hline
                3 &0 &4 &12 & 14& 10 & 5 & 7 & v.dist \\ \hline
                & -& 1& 5& 3& 7 & 1 & 2 & v.$\pi$ \\
			\end{tabular} \\ \\
			Der kürzeste Pfad von 1 nach 4 ist \(1 \to 2 \to 7 \to 5 \to 3 \to 4\), der Pfad kostet 14 Einheiten. \\ \\ \\
			\begin{tabular}{c||c|c|c|c|c|c|c|c}
                Ext & 3 & 1 & 2 & 4 & 5 & \\ \hline \hline
				& 0 & $\infty$ & $\infty$ & $\infty$ & $\infty$ & $v.dist$ \\ \hline
                & - & - & - & - & - & $v.\pi$ \\ \hline
				3 & 0 & $\infty$ & 9 & 4 & $\infty$ & $v.dist$ \\ \hline
				& - & - & 3 & 3 & - & $v.\pi$ \\ \hline
				4 & 0 & 5 & 8 & 4 & 6 & $v.dist$ \\ \hline
				& - & 4 & 4 & 3 & 4 & $v.\pi$ \\ \hline
				1 & 0 & 5 & 8 & 4 & 6 & $v.dist$ \\ \hline
				& - & 4 & 4 & 3 & 4 & $v.\pi$ \\ \hline
				5 & 0 & 1 & 8 & 4 & 6 & $v.dist$ \\ \hline
				& - & 5 & 4 & 3 & 4 & $v.\pi$ \\
			\end{tabular} \\ \\
			Augenscheinlich lässt sich kein kürzester Pfad von 1 nach 4 ablesen.
			\item[b)] \quad \\
			Wenn man versucht den kosteneffizienteste Weg von 3 nach 2 finden will, wird Dijkstra den Pfad 
			\(3 \to 4 \to 2\) mit den Kosten 8 finden. Allerdings ist der kosteneffizienteste Weg durch
			\(G_2\) \(3 \to 4 \to 5 \to 1 \to 2\) mit den Kosten 7. Damit ist die Behauptung bewiesen.
		\end{enumerate}
		\item[\textbf{2.:}] \quad \\
		Es wird anstatt der gesamten Distanz in der Relax-Funktion nun ausschließlich die kleinste längste
		Kante als Berechnungsgrundlage genommen, aus dieser Veränderung resultiert beim Dijkstra (Skript) 
		das gewünschte Verhalten.
		\begin{verbatim}
			Relax(u,v)
			    if v.dist > max(u.dist, w (u, v ))
			        v.dist = max(u.dist, w (u, v ))
			        v.pi = u
		\end{verbatim}
		\item[\textbf{3.:}]
		\begin{enumerate}
			\item[a)] \quad \\
			\todo
			\item[b)] \quad \\
			\todo
			\item[c)] \quad \\
			\todo
		\end{enumerate}
		\item[\textbf{4.:}]
		\begin{enumerate}
			\item[a)] \quad \\
			\todo
			\item[b)] \quad \\
			Bei W* werden kürzeste Pfade zwischen u und v ermittelt (im Gegensatz zu W). Jedes Mal, wenn ein kürzester Pfad ermittelt wurde, erhöht sich das Kantengewicht aller betroffener Kanten um 1, da nun Last auf diesem Pfad liegt. Dadurch wird der nächste kürzeste Pfad evt. eine andere Route ermitteln und dadurch die Last evt. verringern.

D.h. wenn es die Netztopologie zulässt, ist es möglich, dass die optimale, maximale Kantenlast geringer ist, als die maximale Kantenlast. Sie könnte aber auch gleich sein, wenn man beispielsweise zwei Knoten hat, welche verbunden sind.
		\end{enumerate}
	\end{enumerate}
\end{document}
