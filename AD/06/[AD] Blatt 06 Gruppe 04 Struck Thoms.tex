% Commands

\newcommand{\authorinfo}{Arne Struck, Lars Thoms}
\newcommand{\titleinfo}{AD [HA] zum 14. 01. 2014}
\newcommand{\qed}{\ \square}
\newcommand{\limn}{\lim\limits_{n\to\infty}}
\newcommand{\todo}{\textcolor{red}{\textbf{TODO}}}
\newcommand{\Forall}{\textbf{for each}}

% ------------------------------------------------------

% Packages & Stuff

\documentclass[a4paper,11pt,fleqn]{scrartcl}
\usepackage[german,ngerman]{babel}
\usepackage[utf8]{inputenc}
\usepackage[T1]{fontenc}
\usepackage[top=1.3in, bottom=1.2in, left=0.9in, right=0.9in]{geometry}
\usepackage{lmodern}
\usepackage{amssymb}
\usepackage{amsmath}
\usepackage{enumerate}
\usepackage{fancyhdr}
\usepackage{pgfplots}
\usepackage{algorithm}
\usepackage{algorithmicx}
\usepackage{algpseudocode}
\usepackage{multicol}
\usepackage[parfill]{parskip}
\usetikzlibrary{automata,calc,patterns,shapes}


% ------------------------------------------------------

% Title & Pages

\title{\titleinfo}
\author{\authorinfo}

\pagestyle{fancy}
\fancyhf{}
\fancyhead[L]{\authorinfo}
\fancyhead[R]{\titleinfo}
\fancyfoot[C]{\thepage}

\begin{document}
	\maketitle
	\begin{enumerate}
		\item[\textbf{1.:}]
		\begin{enumerate}
			\item[a)] \quad \\
			\todo
			\item[b)] \quad \\
			\todo
		\end{enumerate}
		\item[\textbf{2.:}] \quad \\
		\todo
		\item[\textbf{3.:}]
		\begin{enumerate}
			\item[a)] \quad \\
			\todo
			\item[b)] \quad \\
			\todo
			\item[c)] \quad \\
			\todo
		\end{enumerate}
		\item[\textbf{4.:}]
		\begin{enumerate}
			\item[a)] \quad \\
			\todo
			\item[b)] \quad \\
			Bei W* werden kürzeste Pfade zwischen u und v ermittelt (im Gegensatz zu W). Jedes Mal, wenn ein kürzester Pfad ermittelt wurde, erhöht sich das Kantengewicht aller betroffener Kanten um 1, da nun Last auf diesem Pfad liegt. Dadurch wird der nächste kürzeste Pfad evt. eine andere Route ermitteln und dadurch die Last evt. verringern.

D.h. wenn es die Netztopologie zulässt, ist es möglich, dass die optimale, maximale Kantenlast geringer ist, als die maximale Kantenlast. Sie könnte aber auch gleich sein, wenn man beispielsweise zwei Knoten hat, welche verbunden sind.
		\end{enumerate}
	\end{enumerate}
\end{document}
