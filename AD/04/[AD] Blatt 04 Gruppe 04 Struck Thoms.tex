% Commands

\newcommand{\authorinfo}{Arne Struck, Lars Thoms}
\newcommand{\titleinfo}{AD [HA] zum 20. 11. 2013}
\newcommand{\qed}{\ \square}
\newcommand{\limn}{\lim\limits_{n\to\infty}}
\newcommand{\todo}{\textcolor{red}{\textbf{TODO}}}
\newcommand{\Forall}{\textbf{for each}}

% ------------------------------------------------------

% Packages & Stuff

\documentclass[a4paper,11pt,fleqn]{scrartcl}
\usepackage[german,ngerman]{babel}
\usepackage[utf8]{inputenc}
\usepackage[T1]{fontenc}
\usepackage[top=1.3in, bottom=1.2in, left=0.9in, right=0.9in]{geometry}
\usepackage{lmodern}
\usepackage{amssymb}
\usepackage{amsmath}
\usepackage{enumerate}
\usepackage{fancyhdr}
\usepackage{pgfplots}
\usepackage{algorithm}
\usepackage{algorithmicx}
\usepackage{algpseudocode}
\usepackage{multicol}
\usepackage[parfill]{parskip}
\usetikzlibrary{automata,calc,patterns,shapes}


% ------------------------------------------------------

% Title & Pages

\title{\titleinfo}
\author{\authorinfo}

\pagestyle{fancy}
\fancyhf{}
\fancyhead[L]{\authorinfo}
\fancyhead[R]{\titleinfo}
\fancyfoot[C]{\thepage}

\begin{document}
	\maketitle
	\begin{enumerate}
		\item[\textbf{1.}]
		\begin{enumerate}
			\item[a)]\quad \\
			\todo
			\item[b)]\quad \\
			\todo
			\item[c)]\quad \\
			\todo
			\item[d)]\quad \\
			\todo
		\end{enumerate}
		\item[\textbf{2.}]
		\begin{enumerate}
			\item[a)]
			\begin{enumerate}
				\item[(i)]\quad \\
				\todo
				\item[(ii)]\quad \\
				\todo
				\item[(iii)]\quad \\
				\todo
			\end{enumerate}
			\item[b)]
			\begin{enumerate}
				\item[(i)]\quad \\
				\todo
				\item[(ii)]\quad \\
				\todo
				\item[(iii)]\quad \\
				\todo
			\end{enumerate}
			\item[c)]
			\begin{enumerate}
				\item[(i)]\quad \\
				\todo
				\item[(ii)]\quad \\
				\todo
				\item[(iii)]\quad \\
				\todo
				\item[(iv)]\quad \\
				\todo
			\end{enumerate}
		\end{enumerate}
		\item[\textbf{3.}]
		\begin{enumerate}
			\item[a)]\quad
			\begin{description}
				\item[$G_1$] 1, 3, 4, 5, 2, 8, 6, 7
				\item[$G_2$] 1, 3, 5, 6, 4, 2, 7
			\end{description}
			\item[b)]\quad
			\begin{description}
				\item[$G_1$] 4, 3, 1, 7, 6, 8, 2, 5
				\item[$G_2$] 4, 6, 5, 3, 1, 2, 7
			\end{description}
			\item[c)]\quad
			\begin{description}
				\item[$G_1$] 1, 3, 5, 4, 2, 7, 8, 6
				\item[$G_2$] 1, 3, 4, 7, 5, 2, 6
			\end{description}
			\item[d)]\quad
			\begin{description}
				\item[$G_1$] Für diesen Graphen kann keine topologishce Sortierung existieren, da er sowohl einen reflexiven Knoten enthält (8, Verstoß gegen die Bedingung der Irreflexität) und mehrere Zyklen enthält ([1,5], [7,8], [1,5,2], ...)
				\item[$G_2$] 1, 7, 2, 3, 5, 6, 4
			\end{description}
			\todo
			\item[e)]\quad \\
			\todo
			\item[f)]\quad \\
			\todo
		\end{enumerate}
		\item[\textbf{4.}]
		\begin{enumerate}
			\item[a)]\quad \\
			\todo
			\item[b)]\quad \\
			\todo
			\item[c)]\quad \\
			\todo
		\end{enumerate}
	\end{enumerate}
\end{document}
