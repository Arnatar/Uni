% Commands

\newcommand{\authorinfo}{Tim Dobert, Kai Sonnenwald, Arne Struck}
\newcommand{\titleinfo}{GDB [HA] zum 14. 11. 2013}
\newcommand{\qed}{\ \square}
\newcommand{\todo}{\textcolor{red}{\textbf{TODO}}}
\newcommand{\bra}[1]{(#1)}

% ------------------------------------------------------

% Packages & Stuff
\documentclass[a4paper,11pt,fleqn]{scrartcl}
\usepackage[german,ngerman]{babel}
\usepackage[utf8]{inputenc}
\usepackage[T1]{fontenc}
\usepackage[top=1.3in, bottom=1.2in, left=0.9in, right=0.9in]{geometry}
\usepackage{lmodern}
\usepackage{amssymb}
\usepackage{amsmath}
\usepackage{enumerate}
\usepackage{tikz}
\usepackage{fancyhdr}
\usepackage{pgfplots}
\usepackage{multicol}
\usepackage{vsis-gdb}
\usepackage[parfill]{parskip}
\usetikzlibrary{calc,patterns,arrows,positioning,calc,fit,shapes}


% ------------------------------------------------------

% Title & Pages

\title{\titleinfo}
\author{\authorinfo}

\pagestyle{fancy}
\fancyhf{}
\fancyhead[L]{\authorinfo}
\fancyhead[R]{\titleinfo}
\fancyfoot[C]{\thepage}

\begin{document}
    \maketitle
	\begin{enumerate}
		\item[\textbf{1.:}]\quad \\
		\begin{tikzpicture}
			%Biomolekül
			\node[entity] (bm) {Biomolekül};
			\node[attribut] (bmID) [above =0.6cm of bm] {ID} edge(bm);
			\node[attribut] (bmBesch) [above right =0.6cm of bm] {Beschreibung} edge(bm);
			
			%besteht aus
			\node[relationship] (besteht) [right =3cm of bm] {besteht aus};
			
			%Organismus
			\node[entity] (org) [right =3cm of besteht] {Organismus};
			\node[attribut] (tID) [above left=0.6cm of org] {Taxonomie-ID} edge(org);			
			\node[attribut] (oName) [above =0.6cm of org] {Name} edge(org);
			\node[attribut] (oTName) [above right=0.6cm of org] {Trivialname} edge(org);
			
			%veröffentlicht in
			\node[relationship] (publ) [below =1.cm of besteht] {veröffentlicht in};
			
			%Artikel
			\node[entity] (arti) [below left =2cm and 0.25cm of publ] {Artikel};
			\node[attribut] (titel) [above left= 0.6cm of arti] {Titel} edge(arti);
			\node[attribut] (vdate) [below right =0.8cm and 0.4 of arti] {Veröffentlichungsdatum} edge(arti);

			%schrieb[]
			\node[relationship] (wby) [right =1.5cm of arti] {geschrieben};
			
			%Wissenschaftler
			\node[entity] (scien) [right =1.5cm of wby] {Wissenschaftler};
			\node[attribut] (sname) [below =0.6cm of scien] {Name} edge(scien);
			\node[multivalentattribut] (kinfo) [above =0.6cm of scien] {Kontaktinformationen} edge(scien);
			\node[attribut] (phone) [above left =0.6cm and 0.2cm of kinfo] {Telefonnummer} edge(kinfo);
			\node[attribut] (mail) [above right =0.6cm and 0.2cm of kinfo] {Mail-Adresse} edge(kinfo);
			
			%isa
			\node[relationship] (isa) [below =9cm of bm] {ist ein};
			
			%DNA
			\node[entity] (dna) [right =3cm of isa] {DNA-Molekül};
			\node[attribut] (dnseq) [above left =0.6cm of dna] {Nukleotidsequenz} edge(dna);
			\node[attribut] (sori) [above =1cm of dna] {Strang-Orientierung} edge(dna);
			\node[attribut] (cnum) [above right =0.6cm of dna] {Chromosomennummer} edge(dna);
			
			%RNA
			\node[entity] (mrna) [below = 3cm of isa] {mRNA-Molekül};
			\node[attribut] (rnseq) [below  =1cm of mrna] {Nukleotidsequenz} edge(mrna);
			\node[attribut] (vstri) [below right =0.6cm of mrna] {Vienna-String} edge(mrna);
			
			%transscript
			\node[relationship] (tran) [below right =1cm of isa] {Transskribiert};
			
			%gehört zu
			\node[relationship] (joins) [below right =1cm of tran] {gehört zu};
			
			%synth
			\node[relationship] (synth) [right =3cm of mrna] {Synthetisiert zu};
			
			%Protein
			\node[entity] (prot) [right = 2cm of synth] {Protein};
			\node[attribut] (as) [below left= 0.6cm of prot] {AS-Sequenz} edge(prot);
			\node[attribut] (cath) [below =1cm of prot] {CATH-Klassifikation} edge(prot);
			\node[attribut] (mweight) [below right =0.6cm of prot] {Molekulargewicht} edge(prot);
			
			%enthält
			\node[relationship] (cont) [above = 1cm of prot] {enthält};
			
			%Domäne
			\node[entity] (dom) [above =1cm of cont] {Domäne};	
			\node[attribut] (dID) [right =0.6cm of dom] {ID} edge(dom);
			\node[attribut] (hmm) [below right =0.6cm of dom] {HMM} edge(dom);
			\node[attribut] (func) [below left =0.6cm of dom] {Funktion} edge(dom);

			
			%-------paths-------
			%besteht
			\path (bm) edge node[midway, above] {$[1,*]$} (besteht);
			\path (besteht) edge node[midway, above] {$1$} (org);
			%veröffentlicht
			\path (bm) edge node[midway, above right] {$[1,*]$} (publ);
			\path (publ) edge node[midway, below right] {$1$} (arti);
			%geschrieben
			\path (wby) edge node[midway, above] {$[1,*]$} (scien);
			\path (wby) edge node[midway, below] {$1$} (arti);
			%isa
			\path (isa) edge[erbt] (bm);
			\path (dna) edge[erbt] (isa);
			\path (mrna) edge[erbt] (isa);
			%trans
			\path (tran) edge node[midway, right] {$[0,*]$} (mrna);
			\path (tran) edge node[midway, below] {$1$} (dna);
			%joins
			\path (joins) edge node[midway, above] {$1$} (mrna);
			\path (joins) edge node[midway, right] {$1$} (dna);
			%synth
			\path (synth) edge node[midway, above] {$1$} (mrna);
			\path (synth) edge node[midway, above] {$1$} (prot);
			%cont
			\path (cont) edge node[midway, right] {$[1,*]$} (dom);
			\path (prot) edge node[midway, right] {$1$} (cont);
		\end{tikzpicture}
		\newpage
		%Aufgabe 2
\item
    Für die Generalisierung wurde das Hausklassenmodell verwendet.
    \paragraph{Person}(\underline{Name}, DOB, Geschlecht)
    \paragraph{Regisseur}(\underline{Name}, DOB, Geschlecht)
    \paragraph{Schauspieler}(\underline{Name}, DOB, Geschlecht, Markenzeichen)
    \paragraph{Charakter}(\underline{CID}, Name, Charakterbeschreibung)
    \paragraph{Film}(\underline{Titel}, Zusammenfassung, 1. Drehtag, letzter Drehtag, Regisseur $\rightarrow$ Regisseur.Name,\\
    \phantom{Film(00} G1 $\rightarrow$ Genre.Name, G2 $\rightarrow$ Genre.Name, G3 $\rightarrow$ Genre.Name, G4 $\rightarrow$ Genre.Name)
    \paragraph{Genre}(\underline{Name})
    \paragraph{spielt}(\dashuline{CID $\rightarrow$ Charakter.CID, Titel $\rightarrow$ Film.Titel}, Drehbeginn, Drehende, Gage)\\
    
%----------
%Aufgabe 3
\item
    \begin{enumerate}
    \item %a
        \begin{enumerate}
            \item Gib die Nachnamen aller Rennfahrer, die auf dem Malaysia GP den 1. Platz erreicht haben.\\
                  Ergebnis: $\emptyset$
            \item Gib Vor- und Nachnamen aller Rennfahrer deren Rennstall ein Budget < 350 hat.\\
                  Ergebnis: \{Louis Hamilton, Jensen Button, Kimi Raikönen\}
            \item Gib die Namen aller Rennställe, deren Fahrer auf dem Australien GP eine Platzierung geholt haben.\\
                  Ergebnis: \{Sebastian Vettel, Fernando Alonso, Marc Webber, Lewis Hamilton, Jenson Button, Felipe Massa\} 
        \end{enumerate}
    \item %b
        \begin{enumerate}
            \item Ausdruck: $\pi_{\text{Name}}(\sigma_{\text{Geburt} > 1985-01-01}(\text{Rennfahrer} \bowtie_{\text{RSID = Rennstall }}\text{Rennstall}))$\\
                  Ergebnis: \{Sebastian Vettel, Louis Hamilton\}\\
            \item Ausdruck: $\pi_{\text{Vorname, Nachname, Geburt}}((\sigma_{\text{Name = Australien GP}}(Rennort)\\ 
            \phantom{Ausdruck:} \bowtie $ Platzierung$) \bowtie_{\text{RSID = Rennstall}}(\sigma_{\text{RSID = 31}} $ Rennfahrer$))$ \\
                  Ergebnis: \{Lewis Hamilton, Jenson Button\}\\
            \item Ausdruck: Rennfahrer $ - \pi_{\text{Vorname, Nachname, Geburt, Wohnort, Rennstall}}($Rennfahrer$ \bowtie $Platzierung$)) $\\
                  Ergebnis: \{Kimi Raikkönen \}\\
            \item Ausdruck: $\pi_{\text{Vorname, Nachname}}(\sigma_{\text{Rennstall = 31}}\text{Rennfahrer})$\\
                  Ergebnis: \{Lewis Hamilton\} \\
        \end{enumerate}    
    \item %c
    \begin{enumerate}
    \item
    \begin{verbatim}
    SELECT Rennfahrer.Vorname, Rennfahrer.Nachname, Rennfahrer.Geburt
    FROM Rennfahrer, Platzierung
    WHERE Rennfahrer.Rennstall = '31'
    AND Platzierung.RID = Rennfahrer.RID
    AND Platzierung.OID = '4';
    \end{verbatim}
    \item
    \begin{verbatim}
    SELECT Rennfahrer.Vorname, Rennfahrer.Nachname
    FROM Rennfahrer
    WHERE Rennfahrer.Rennstall IN(    SELECT Rennstall
    FROM Rennfahrer
    WHERE Nachname = 'Button');
    \end{verbatim}
    \end{enumerate}
    
    \end{enumerate}
		\newpage
		\item[\textbf{4.:}]
		\begin{enumerate}
			\item[a)]\quad \\
			\begin{tikzpicture}[style={sibling distance=4cm}]
            	\node{\(\pi_{\text{Wohnort}}(\sigma_{\text{Name=''ChinaGP''}}(\sigma_{\text{Platz}<11}
            				((\text{Rennfahrer $\bowtie$ Platzierung})\bowtie \text{Rennort})))\)}
            		child{node{\(\pi_{\text{Wohnort}}\)}
            			child{node{\(\sigma_{\text{Name = ''ChinaGP''}}\)}
            				child{node{\(\sigma_{\text{Platz}< 11}\)}
            					child{node{\(\text{Rennfahrer $\bowtie$ Platzierung}\)}
            						child{node{Rennfahrer}}
            						child{node{Platzierung}}
            					}
            					child{node{Rennort}
            					}
            				}
            			}
            		}
            	;
				
			\end{tikzpicture} \\
			\item[b)]\quad \\
			\begin{tikzpicture}[style = {level distance=1.5cm},
           				        level 3/.style={sibling distance=4cm, level distance=2cm},
           				        level 4/.style={level distance=1.5cm}]
            				        
				\node{\(\pi_{\text{Wohnort}}(\text{Rennfahrer}\bowtie(\sigma_\text{Platz<11}\text{Platzierung})\bowtie
				(\sigma_\text{Name = ''ChinaGP''}\text{Rennort}))\)}
            		child{node{\(\pi_{\text{Wohnort}}\)}
            			child{node{\(\text{Rennfahrer}\bowtie(\sigma_\text{Platz<11}\text{Platzierung})\bowtie
							(\sigma_\text{Name = ''ChinaGP''}\text{Rennort})\)}
							child{node{Rennfahrer}}
							child{node{\(\sigma_\text{Platz<11}\)}
								child{node{Platzierung}}
							}
							child{node{\(\sigma_\text{Name = ''ChinaGP''}\)}
								child{node{Rennort}}
							}
            			}
            		}
            	;
			\end{tikzpicture} \\ \\
			b) hat höheren Optimierungsgrad, da es in mehr Heuristiken umsetzt (I,III,VII) als a).
		\end{enumerate}
	\end{enumerate}
\end{document}