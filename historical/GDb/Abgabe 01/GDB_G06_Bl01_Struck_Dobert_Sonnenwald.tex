% Commands

\newcommand{\authorinfo}{Tim Dobert, Kai Sonnenwald, Arne Struck}
\newcommand{\titleinfo}{GDB [HA] zum 1. 11. 2013}
\newcommand{\qed}{\ \square}

% ------------------------------------------------------

% Packages & Stuff

\documentclass[a4paper,11pt,fleqn]{scrartcl}
\usepackage[german,ngerman]{babel}
\usepackage[utf8]{inputenc}
\usepackage[T1]{fontenc}
\usepackage[top=1.3in, bottom=1.2in, left=0.9in, right=0.9in]{geometry}
\usepackage{lmodern}
\usepackage{amssymb}
\usepackage{amsmath}
\usepackage{enumerate}
\usepackage{fancyhdr}
%\usepackage{pgfplots}
\usepackage{multicol}
\usepackage[parfill]{parskip}
%\usetikzlibrary{calc}
%\usetikzlibrary{patterns}



% ------------------------------------------------------

% Title & Pages

\title{\titleinfo}
\author{\authorinfo}

\pagestyle{fancy}
\fancyhf{}
\fancyhead[L]{\authorinfo}
\fancyhead[R]{\titleinfo}
\fancyfoot[C]{\thepage}

\begin{document}
    \maketitle
    \begin{enumerate}
        \item[\textbf{1.}]
        \begin{enumerate}
                \item Informationssysteme stellen einen so genanntes MAT-System (Mensch/Aufgabe/Technik) dar. Sie
                sind zentrale Bestandteile in rechnergestützten Prozessen.Informationssysteme müssen Daten schreiben,             
                manipulieren/modifizieren/nutzbar machen und löschen können.
                
                \item 
                    \begin{itemize}
                        \item Logische Datenunabhängikeit bezieht sich auf die strukturellen Änderungen einer Datenbank.
                            Eine solche Änderung würde also nichts an der grundlegenden Funktionsweise der Datenbank ändern.
                        \item Physische Datenunabhängikeit bezieht sich auf eine physische Umstrukturierung, bei einer solchen
                            hat der Eingriff keine Auswirkung auf die Organisation und Wirkungsweise der Datenbank.
                    \end{itemize}

                \item \begin{itemize}
                            \item Online Spiele müssen Accounts und zugehörige Attribute wie beispielsweise Spielerstatistiken,
                            Ausrüstung, Kämpfe etc. verwalten.  
                            \item Parteien/Vereine brauchen Informationssysteme für die Mitgliederverwaltung.
                            Es müssen beispielsweise Mitgliederlisten mit den dazugehörigen Informationen, Beitragszahlungen 
                            etc. verwaltet werden. 
                            \item Die Ergebnisse der NSA-Spionage, die ihre Ergebnisse einzelnen Personen/Profilen zuordnen
                             muss. Beispielsweise: Bewegungs-/Aufenthaltsdaten, Kommunikationsabfragen. \\
                       \end{itemize}
            \end{enumerate}  
        \item[\textbf{2.}]
        \begin{enumerate}
            \item[a)]\quad \\
            \begin{tabular}{c|c}
                Objekttyp & Eigenschaft \\ \hline\hline
                User & UserID, UserName \\ \hline 
                Gemeinschaft & GemeinschaftsID, List: UserId \\ \hline
                Mannschaft & Mannschaftsname\\ \hline
                Begegnung & BegegnungsID, Ergenbnis, List(2): Mannschaft \\ \hline
                Tipp & TippID, UserID, BegegnungsID, StandTipp\\ \hline 
                   Wettbewerb & WettbewerbsID, GemeinschaftsID, List: BegegnungsID, List: TippID\\ \hline
                PunkteBoard & WettbewerbsID, UserID, Punkte\\ \hline 
            \end{tabular}
            \newpage
            \item[b)]%\quad \\
            Bei der Anmeldung eines neuen Nutzers, wird der Tabelle \texttt{User} ein neuer Eintrag hinzugefügt. Das Erstellen einer
             Gemeinschaft führt zu einem neuen Eintrag in der \texttt{Gemeinschaft}-Tabelle. Die dazu hinzugefügten Benutzer werden in der 
              UserID-Liste gespeichert. Auf die Tabellen \texttt{Mannschaft} und \texttt{Begegnung} haben die Nutzer keinen Zugriff, sie werden von der
               Verwaltung aktualisiert. Das Hinzufügen eines Wettbewerbs zu einer Gemeinschaft, führt zu einem neuen Eintrag
                in der \texttt{Wettbewerb}-Tabelle mit ihrer ID. Dabei können vom  Gemeinschaftsgründer neue Elemente in die Begegnungsliste und
                 von Nutzern neue in die Tipp-Liste eingefügt werden.
                  \newline
            Vergleichend mit den Anforderungen aus dem Skript halten wir folgende Eigenschaften für besonders wichtig in 
            unserer Datenbank:
            \begin{enumerate}
                \item \textbf{Robustheit(Persistenz, Atomarität):}
                \begin{itemize}
                    \item Durchgeführte Aktionen sollen nicht verloren gehen.
                    \item Änderungen werden entweder ganz oder gar nicht ausgeführt, um zu gewährleisten, dass keine 
                    beschädigten Datensätze entstehen.
                \end{itemize}
                \item \textbf{Nutzerfreundlichkeit:}
                \begin{itemize}
                    \item Nutzbarkeit ohne Kenntnisse über Datenbanken muss als Spieler nutzbar sein.
                    \item Die Programmschnittstellen (für Entwickler) sollten leicht und verständlich sein.
                \end{itemize}
                \item \textbf{Portierbarkeit:}
                \begin{itemize}
                    \item Da heutzutage von vielen Geräten und Systemen auf Funktionen zugegriffen wird, sollte sie portierbar 
                    sein.
                    \item Unterschiedliche Anwendungsspezifikationen führen zu unterschiedlichen Zugriffsanwendungen (bspw.: 
                    Desktopanwendung für Verwaltende Ebene, Webanwendung für die User).
                \end{itemize}
                \item \textbf{Isolationseigenschaften:}
                \begin{itemize}
                    \item Mehrbenutzerbetrieb muss ohne Konflikte machbar sein.
                    \item Leistung muss auch bei Belastung stabil bleiben (viele Spieler könnten zu ähnlichen Zeiten 
                    zugreifen).
                \end{itemize}
                \item \textbf{Kontrolle der operationalen Daten durch das System:}
                \begin{itemize}
                    \item Verhinderung der Abgabe mehrerer Tipps einer Person auf das selbe Spiel.
                    \item Verteilung unterschiedlicher Rechte an unterschiedliche Nutzergruppen (mehr Rechte für die 
                    Verwalter).
                \end{itemize}
            \end{enumerate}
            
        \end{enumerate}
        \item[\textbf{3.}]\quad \\
        \textbf{Dateisystem:}
            \begin{enumerate}
                \item[A]
                \begin{enumerate}
                    \item Informationen noch nicht geschrieben: \\
                        Die Änderung im Arbeitsspeicher hatte noch keinen Einfluss auf die Daten auf der Platte. \\
                    \item Informationen bereits geschrieben (teilweise): \\
                        Wenn jedes Konto als eigene Datei gespeichert ist, wird Konto 5 der Betrag abgezogen, aber nicht auf 
                        Konto 7 addiert. das Geld ist also verloren. \\
                \end{enumerate}
                \item[B]
                \begin{enumerate}
                    \item Informationen noch nicht geschrieben: \\
                        Es wird ein Auszug für Konto 7 gedruckt, auf dem steht, dass eine Überweisung stattgefunden hat. Die 
                        Beträge haben sich aber nicht geändert.
                    \item Informationen bereits geschrieben (teilweise): \\
                        Die Überweisung hat richtig stattgefunden, es wird aber nur ein Auszug für Konto 7 gedruckt. \\
                \end{enumerate}
            \end{enumerate}
        \textbf{Datenbanksystem:}
        \begin{enumerate}
                \item[A]
                \begin{enumerate}
                    \item Informationen noch nicht geschrieben: \\
                        Die Änderung im Arbeitsspeicher hatte noch keinen Einfluss auf die Daten auf der Platte. \\
                    \item Informationen bereits geschrieben (teilweise): \\
                        Das System würde die Subtraktion rückgängig machen um zu verhindern, das Geld verloren geht. \\
                \end{enumerate}
                \item[B]
                \begin{enumerate}
                    \item Informationen noch nicht geschrieben: \\
                        Die Überweisung findet nicht statt, da nichts auf die Platte geschrieben wurde. \\
                    \item Informationen bereits geschrieben (teilweise): \\
                        Je nachdem ob das Drucken des Auszuges mit zu der atomaren Transaktion einer Überweisung gehört, wird 
                        sie entweder rückgängig gemacht oder es gibt keinen Ausdruck für Konto 5.\\
                \end{enumerate}
            \end{enumerate}
        \item[\textbf{4.}]
        \begin{enumerate}
            \item[a)]\quad \\
            Die erste Anfrage erstellt eine Tabelle namens \texttt{gdb\_gruppe041.user} mit den 
            Spalten \texttt{id, name} und \texttt{passwort}. Die zweite Anfrage fügt dieser 
            Tabelle eine Zeile hinzu. Dabei ist die id=1, der Name \texttt{gdbNutzer} und das 
            passwort \glqq geheim\grqq. 
            \item[b)]\quad \\
            Die erste Zeile wählt aus allen Datenbankeinträgen in der Tabelle 
            \texttt{gdb\_gruppe041.user} diejenigen aus, bei denen \texttt{name} \texttt{gdbNutzer} ist. 
            Es wird dann der Eintrag geliefert, der in a hinzugefügt wurde. 
            \item[c)]\quad \\
            Die Architektur von MySQL folgt der Drei-Schema Architektur. Sie besitzt ebenfalls 
            eine externe, konzeptuelle und interne Ebene. Die Dokumentation ist noch um einiges 
            detaillierter und zeigt an, welche Komponenten zu diesen drei Ebenen gehören.
        \end{enumerate}
    \end{enumerate}

\end{document}