 % Commands
\newcommand{\authorinfo}{Arne Struck, Tronje Krabbe}
\newcommand{\titleinfo}{FGI 2 [HA], 02. 12. 2013}
\newcommand{\qed}{\ \square}
\newcommand{\todo}{\textcolor{red}{\textbf{TODO}}}

% ------------------------------------------------------

% Packages & Stuff

\documentclass[a4paper,11pt,fleqn]{scrartcl}
\usepackage[german,ngerman]{babel}
\usepackage[utf8]{inputenc}
\usepackage[T1]{fontenc}
\usepackage{lmodern}
\usepackage{amssymb}
\usepackage{amsmath}
\usepackage{enumerate}
\usepackage{fancyhdr}
\usepackage{pgfplots}
\usepackage{multicol}
\usepackage{pst-node}
\usetikzlibrary{calc}
\usetikzlibrary{patterns}
\usetikzlibrary{arrows,automata,positioning}

% ------------------------------------------------------

% Title & Pages

\title{\titleinfo}
\author{\authorinfo}

\pagestyle{fancy}
\fancyhf{}
\fancyhead[L]{\authorinfo}
\fancyhead[R]{\titleinfo}
\fancyfoot[C]{\thepage}

\begin{document}
	\maketitle
	\begin{enumerate}
		\item[\textbf{7.4}]
		\begin{enumerate}
			\item[1.:]\quad \\
			\begin{tikzpicture}[>=stealth',shorten >=1pt, every path/.style={->}]
					%nodes
					\node (A1) {$\{20010\}^T$};
 					\node (A2) [below = 1cm of A1] {$\{11001\}^T$};
 					\node (A3) [below = 1cm of A2] {$\{01110\}^T$};
					\node (A6) [below = 0.3cm of A3] {$N_{7.4a}$};
					%paths
					\path (A1) edge node[left] {$t_1$} (A2);
					\path (A2) edge node[left] {$t_2$} (A3);
					\path (A3) edge[bend left = 60] node[left] {$t_3$} (A1);
				\end{tikzpicture}
				\begin{tikzpicture}[>=stealth',shorten >=1pt, every path/.style={->},
									level 1/.style={sibling distance=4cm, level distance=1.5cm},
			        	            level 2/.style={sibling distance=4cm},
            				        level 3/.style={sibling distance=3cm}]
					%nodes
					\node (A1) {$\{30010\}^T$}
						child{node{$\{21001\}^T$}
							child{node{$\{11120\}^T$}
								child{node (B1) {$\{02111\}^T$}
								edge from parent node[above left] {$t_1$}
								}
								child{node{$\{30020\}^T$}
									child{node (B2) {$\{21011\}^T$}
										child{node (C1) {$\{12002\}^T$}
										edge from parent node[above left] {$t_1$}
										}
										child{node{$\{11130\}^T$}
											child{node (C2) {$\{02121\}^T$}
											edge from parent node[above left] {$t_1$}
											}
											child{node{$\{30030\}^T$}
												child{node (C3) {$\{21021\}^T$}
													child{node (D1) {$\{12012\}^T$}
														child{node {$\{03003\}^T$}
														edge from parent node[left] {$t_1$}
														}
													edge from parent node[above left] {$t_1$}
													}
													child{node{$\{11140\}^T$}
														child{node (D2) {$\{02131\}^T$}
														edge from parent node[above left] {$t_1$}
														}
														child{node{$\{30040\}^T$}
														edge from parent node[above right] {$t_3$}
														}
													edge from parent node[right] {$t_2$}
													}
												edge from parent node[right] {$t_1$}
												}
											edge from parent node[above right] {$t_3$}
											}
										edge from parent node[above right] {$t_2$}
										}
									edge from parent node[right] {$t_1$}
									}
								edge from parent node[above right] {$t_3$}
								}
							edge from parent node[left] {$t_2$}
							}
						edge from parent node[left] {$t_1$}
						}
					;
					\path (B1) edge node[above right] {$t_3$} (B2);
					\path (C1) edge node[above right] {$t_2$} (C2);
					\path (C2) edge node[above right] {$t_3$} (C3);
					\path (D1) edge node[above right] {$t_2$} (D2);
					\node (AX) [below = 0.3cm of D2] {$N_{7.4b}$}; 
				\end{tikzpicture}
			\item[2.:]\quad \\
            	Das Netz $N_{7.4a}$ ist $k$-beschränkt  für $k=3$ (siehe Graph) und demzufolge beschränkt. 
			Es ist außerdem verklemmungsfrei (ablesbar), reversibel (es wird in den Anfangszustand zurückgesetzt)
			und strukturell lebendig.\\
			Das Netz $N_{7.4b}$ ist nicht beschränkt (p4 wird inkrementiert), daher auch nicht strukturell
			beschränkt oder $k$-beschränkt für irgendein $k$. Es ist nicht reversibel, da der Ausgangszustand 
			nicht wiederhergestellt werden kann (Inkrementor).
			\item[3.:]
			\begin{enumerate}
				\item[a)]\quad \\
                Wenn $t_0$ lebendig ist, muss \(\forall m \in \textbf{R}(N)\exists \sigma \in \textbf{T}^*:
					m\overset{\sigma t}{\rightarrow}m'\) gelten. Da $m_0$ in der Menge aller $m$ enthalten ist 
					und \(\sigma t\) durch die Wörter substituiert werden kann, gilt die Behauptung. \\
                    Man bedenke allerdings, dass Tronje lebendig ist, jedoch nicht fleißig. Welche Bedeutung dies für Formale Informatik hat, sei dahingestellt.
				\item[b)]\quad \\
                Da sich fleißig nur über die Startmarkierung definiert und eine Voraussetzung für lebendig 	
					ist, dass es für alle Markierungen schaltbar ist. Somit ist eine fleißige Transition nicht 
					lebendig.
			\end{enumerate}
		\end{enumerate}
	\end{enumerate}
\end{document}
\end{document}