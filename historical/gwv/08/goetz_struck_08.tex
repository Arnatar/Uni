\documentclass[a4paper,11pt,fleqn]{scrartcl}
\usepackage[german,ngerman]{babel}
\usepackage[utf8]{inputenc}
\usepackage[T1]{fontenc}
\usepackage[top=1.3in, bottom=1.2in, left=0.9in, right=0.9in]{geometry}
\usepackage{lmodern}
\usepackage{amssymb}
\usepackage{amsmath}
\usepackage{enumerate}
\usepackage{fancyhdr}
\usepackage{color}
\usepackage{url}
\usepackage{tikz}
\usetikzlibrary{automata,shapes,positioning}

% ------------------------------------------------------

% Commands

\newcommand{\todo}{\textcolor{red}{\textbf{TODO}}}
\newcommand{\authorinfo}{Arne Struck, Knut Götz}
\newcommand{\titleinfo}{GWV-Abgabe zum 05.02.14}
\newcommand{\qed}{\ \square}


% ------------------------------------------------------

% Title & Pages

\title{\titleinfo}
\author{\authorinfo}

\pagestyle{fancy}
\fancyhf{}
\fancyhead[L]{\authorinfo}
\fancyhead[R]{\titleinfo}
\fancyfoot[C]{\thepage}

\begin{document}
\maketitle
\notag

\section*{1}
\subsection*{1.1 Language Modelling}
Für das Programm siehe Files.\\
Die Sequenzen, die das Programm ausspuckt basieren auf der Wahrscheinlichkeit (in dem Fall abgeschätzt durch die relative Häufigkeit, die aus den Heise-Texten ermittlet wurde), dass auf ein gegebenes Wort ein bestimmtes nächstes Wort folgt. Die Sequenzen sind somit oftmals keine grammatikalisch korrekten Sätze. 
Allerdings sollte die relative Häufigkeit des Aufeinanderfolgends von Wörtern der in ``echten'' Texten entsprechen. Wählt man entsprechend der Worthäufigkeit in echten Texten Wörter aus und erstellt dann ausgehend von diesen Sequenzen, dann sollte auch die Worthäufigkeit der Wörter in den entstandenen Sequenzen der Worthäufigkeit in echten Texten entsprechen. 
\subsection*{1.2 Diagnosis (cont.) }
Die Menge der Zufallsvariablen $V$ sei
\begin{align*}
  \{X_{battery},X_{Ignition key}, X_{starter},X_{oil regulation},X_{engine}, X_{filter}, X_{fuel pump}, X_{fuel tank}\}
\end{align*}
wobei jeder dieser Variablen mir $Ber(0.9)$ verteilt ist, mit $P(X_i=1)$ beduetet, dass das Teil $X_i \in V$ funktioniert (unabhängig von den anderen Komponenten, von denen die tatsächliche  Lauffähihkeit abhängt) ist.\\
Da die Batterie auch dann funktioniert, wenn alle anderen Komponenten kaputt sind, gilt
\begin{align*}
P(X_{battery} = 1) = 0.9
\end{align*}
Dass der Starter läuft, hängt von der Funktionstüchtigkeit des Ignition Keys, welcher wiederum von, der Battery abhängt.
\begin{align*}
P(X_{battery} = 1, X_{Ignition key} = 1, X_{starter} = 1) = 0.9^3 = 0.729
\end{align*}
Dass die Engine läuft, hängt davon ab, der Starter und der Filter laufen. Das der Filter läuft hängt wiederum, davon ab, dass die Fuel Pump, die Oil Regulation laufen. Starter wie oben.
\begin{align*}
P(X_{battery} = 1, X_{Ignition key} = 1, X_{starter} = 1, X_{oil regulation} = 1, X_{fuel pump} = 1, X_{filter} = 1, X_{engine} = 1)  = 0.9^7 = 0.4782969
\end{align*} 
enn gegeben ist, dass die Fuelpump läuft, dann hängt das Funktioniern der Enginge nur noch von dem Filter dem Starter und dem Key ab (natürlich auch von dem Funktionieren der Engine).
\begin{align*}
P(X_{Ignition key} = 1, X_{starter} = 1, X_{filter} = 1, X_{engine} = 1 \vert X_{fuel pump} = 1 )  = 0.9^4 = 0.6561 
\end{align*} 
\subsection*{1.3 Baysien Probabilities}
\begin{tikzpicture}
[
	var/.style={ellipse, draw, text centered},
	auto,
	node distance = 3cm
	]

%-------- nodes --------
\node[var](Sniff) [] {Sniff};
\node[var] (Police) [right = 1.5cm of Sniff] {Police Controls};
\node[var](Smuggler) [below  = 3cm of Police] {Smuggler};
\node[var](Bark) [below  = 1.5cm of Sniff] {Bark};
\node[var](Fever) [right = 1.5cm of Smuggler] {Fever};
\node[var](Sweats) [below  = 1.5cm of Fever] {Sweats};

%-------- paths --------
\path (Police) edge [] (Smuggler);
\path (Smuggler) edge [] (Sweats);
\path (Fever) edge [] (Sweats);
\path (Sniff) edge [] (Bark);
\path (Bark) edge [] (Smuggler);
\end{tikzpicture}

Aus dem Text ergeben sich folgende Wahrscheinlichkeiten
\begin{align*}
P(Smuggler= t \vert Police=t) = 0.01\\
 P(Barks= t \vert Smuggler=t) = 0.8\\
 P(Smuggler = f \vert Sniff = t, Bark = t)  = 0.05\\
 P(Sweat = f \vert Smuggler = f, Fever = f) = 0\\
 P(Sweat = t \vert Smuggler = t, Fever = f) = 0.4\\
 P(Sweat = t \vert Smuggler = t, Fever = t) = 0.8\\
 P(Sweat = t \vert Smuggler = f, Fever = t) = 0.6\\
 P(Fever = t) = 0.013\\
\end{align*}

Explaining away kann bei diesem Netz beispielsweise bedeuten, dass das Schwitzen durch Fieber erklärt werden kann, obwohl es sich bei der Person auch um einen Schmuggler handelt. 
\end{document}
