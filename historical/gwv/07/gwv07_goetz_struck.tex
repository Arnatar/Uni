\documentclass[a4paper,11pt,fleqn]{scrartcl}
\usepackage[german,ngerman]{babel}
\usepackage[utf8]{inputenc}
\usepackage[T1]{fontenc}
\usepackage[top=1.3in, bottom=1.2in, left=0.9in, right=0.9in]{geometry}
\usepackage{lmodern}
\usepackage{amssymb}
\usepackage{amsmath}
\usepackage{enumerate}
\usepackage{fancyhdr}
\usepackage{color}
\usepackage{url}
\usepackage{tikz}
\usetikzlibrary{automata,shapes,positioning}

% ------------------------------------------------------

% Commands

\newcommand{\todo}{\textcolor{red}{\textbf{TODO}}}
\newcommand{\authorinfo}{Arne Struck, Knut Götz}
\newcommand{\titleinfo}{GWV-Abgabe zum 28.11.2014}
\newcommand{\qed}{\ \square}


% ------------------------------------------------------

% Title & Pages

\title{\titleinfo}
\author{\authorinfo}

\pagestyle{fancy}
\fancyhf{}
\fancyhead[L]{\authorinfo}
\fancyhead[R]{\titleinfo}
\fancyfoot[C]{\thepage}

\begin{document}
\maketitle
\notag

\section*{1}
\subsection*{1.1 CSI Stellingen}

Assumables 
\begin{align*}
gardener\_in\_garden, butler\_in\_garage
\end{align*}
Obersabvations
\begin{align*}
 gardener\_no\_dirt\\
 butler\_dirt
\end{align*}
Rules
\begin{align*}
 gardener\_dirt &\gets gardener\_in\_garden\\
 butler\_dirt &\gets butler\_in\_garage 
\end{align*}
Integrity Constraints
 \begin{align*}
  false &\gets gardener\_dirt \And gardener\_no\_dirt\\
  false &\gets butler\_dirt \And butler\_no\_dirt 
 \end{align*}
Anwendung des ConflictTD-Algorithmus\\
Hier wird immer die Menge $G$ angegeben.\\ 
\begin{align*}
&\{false\}\\ %choose false
&\{gardener\_dirt, gardener\_no\_dirt\}\\ %choose gardener_dirt
&\{gardener\_in\_garden, gardener\_no\_dirt\}\\ % choose gardener_no_dirt
&\{gardener\_in\_garden\}\\
\end{align*}
Dann stoppt der Algorithmus, da $G \subseteq Assumables$ . 
Der Konflikt ist minimal, da keine echte Teilmenge (hier nur die leere Menge) auch ein Konflikt ist.\\
\begin{align*}
&\{false\}\\
&\{butler\_dirt, butler\_no\_dirt\}\\
&\{butler\_in\_garage,butler\_no\_dirt\}\\
\end{align*}
Da bei anderem Verhalten des Algorithmus kein Konflikt ermittelt werden kann, ist \{gardener\_in\_garden\} eine minimale Diagnose.
Also war der Gärtner der Mörder.

\subsection*{1.2 Diagnosis}
Code: siehe diag.pl \\
Die Inputparameter entsprechen den 3 Geräuschen, der Output einer Liste mit potentiell defekten Teilen, wobei ein Element nicht zwangsläufig immer alle Defekte erklären kann. \\
Wir haben einige Annahmen über den genauen Zusammenhang getätigt, einige eher eindeutig, einige diskutabel. So sind beispielsweise die pipes als gerichtete Kanten in Richtung engine aufgefasst. \\
Eine weitere, diskutable Annahme ist dass wir die meisten Transitivitäten ausgeschlossen haben (ansonsten wäre in unserem Modell es Zwangsläufig das Geräusch 1 zu hören, wenn man Geräusch 2 hört, da die Pumpe ohne ein korrektes Anlassen nicht arbeiten würde). Die Transitivität der Batterien bleibt allerdings bestehen. \\
Ausgaben :
\begin{verbatim}
?- prob(1,0,0,R).
R = [regulation, tank, filter, engine, pump].

?- prob(0,1,0,R).
R = [key, filter, engine, starter].

?- prob(1,1,0,R).
R = [filter, engine].

?- prob(1,1,1,R).
R = [].
\end{verbatim}
\end{document}