% Packages & Stuff

\documentclass[a4paper,11pt,fleqn]{scrartcl}
\usepackage[german,ngerman]{babel}
\usepackage[utf8]{inputenc}
\usepackage[T1]{fontenc}
\usepackage[top=1.3in, bottom=1.2in, left=0.9in, right=0.9in]{geometry}
\usepackage{lmodern}
\usepackage{amssymb}
\usepackage{amsmath}
\usepackage{enumerate}
\usepackage{fancyhdr}
\usepackage{color}
\usepackage{url}

% ------------------------------------------------------

% Commands

\newcommand{\todo}{\textcolor{red}{\textbf{TODO}}}
\newcommand{\authorinfo}{Arne Struck, Knut Götz}
\newcommand{\titleinfo}{GWV-Abgabe zum 17.10.2014}
\newcommand{\qed}{\ \square}


% ------------------------------------------------------

% Title & Pages

\title{\titleinfo}
\author{\authorinfo}

\pagestyle{fancy}
\fancyhf{}
\fancyhead[L]{\authorinfo}
\fancyhead[R]{\titleinfo}
\fancyfoot[C]{\thepage}

\begin{document}
\maketitle
\section*{1.1}
\subsection*{Spiele-KI}
\begin{enumerate}
	\item Aufgabe der KI \\
	Die Aufgabe einer solchen KI ist es in einem beliebigen für das Verhalten natürlich bestimmenden Spiel dem Menschen einen künstlichen Mit- oder Gegenspieler zu bieten.
	\item Existiert eine solche KI bereits? \\
	Ja und nein, beispielsweise existiert(e) Deep Blue (ein Schachcomputer, welcher den Schach-Weltmeister Garry Kasparov schlug. Allerdings lässt sich hier, wie bei jeder KI die Intelligenz anzweifeln, da sie zwar ihr Ziel gut zu spielen verfolgen, aus Erfahrung lernen (Deep Blue setzte die Parameter zur Bewertung einzelner Positionen fest) und angemessene Entscheidungen treffen, sich allerdings nicht an neue Umgebungen (bspw. ein anderes Spiel) anpassen können.
	\url{http://en.wikipedia.org/wiki/Deep_Blue_%28chess_computer%29}
	\item Warum eine KI und welche Schwierigkeiten bringt dies mit sich? \\
	Da schon bei einem 8x8 Schachfeld die Anzahl an Möglichen Situationen die Berechenbarkeit in angemessenen Zeitrahmen übersteigt (und dies mit der Komplexität der Spiele zunehmen dürfte) stellt eine KI eine vernünftige Alternative mit einer Kombination aus ''intelligentem'' und nicht deterministischen, sich wiederholendem Verhalten dar. Natürlich existieren die üblichen Probleme von lernender Software, wie beispielsweise lokale Minima. Des weiteren ist die KI natürlich durch ihre Welt (ihr Spiel) in der Fähigkeit Intelligenz zu zeigen beschränkt. Auch die Bereitstellung von genügend Beispielen, aus denen gelernt werden kann, kann in einigen Fällen problematisch werden. 
\end{enumerate}

\subsection*{Erkennen der Sprache}
\begin{enumerate}
	\item Aufgabe der KI \\
	Das Erkennen der Sprache in der ein Text verfasst wurde. Input ist ein Text und der Output ist ein String der die Sprache bezeichnet, in der der Text verfasst wurde. 
	\item Existiert eine solche KI bereits? \\
	Ja, beispielsweise bei IBM's Watson (\url{http://www.ibm.com/smarterplanet/us/en/ibmwatson/developercloud/language-identification.html})
	\item Warum eine KI und welche Schwierigkeiten bringt dies mit sich? \\
	Ein solches System kann als intelleigent bezeichnet werden, da es verschiedene (ihm vorher unbekannte) Inputs korrekt bearbeitet. Außerdem ist es vorstellbar, dass ein solches System lernt. Hier beiten sich sicherlich neuronalte Netze an. Dann können in immer mehr Trainingsphasen die Netze lernen, auch noch weitere Sprachen erlernt werden. Problematisch für eine solchen KI könnte es sein, dass zum einen ein Text auch in mehreren Sprachen geschrieben sein kein und Sprachen sich verändern.     
\end{enumerate}

\subsection*{Raumanalyse und Orientierung}
\begin{enumerate}
	\item Aufgabe der KI \\
	Durch die Wahrnehmung, wahlweise Erkundung eines beliebigen Raumes und der Elemente in diesem eine Orientierung und gegebenenfalls Wegfindung zu schaffen.
	\item Existiert eine solche KI bereits? \\
	 Viele Beispiele aus dem Bereich der autonomen Fahrzeuge \url{http://en.wikipedia.org/wiki/Vehicular_automation}, wie die google-cars.
	\item Warum eine KI und welche Schwierigkeiten bringt dies mit sich? \\
	Zum einen müssen Anhand von Aufnahmen einer Kamera Objekte erkannt werden, die auf keinen Fall Überfahren werden sollten wie beispielsweise Menschen. Hier handelt es sich um zielgerichtetes Verhalten. Außerdem muss eine Art von Erwartungshaltung geben, welche Objekte sich bewegen und welche nicht.   
\end{enumerate}

\section*{1.2}
\subsection*{Wissen}
\begin{enumerate}
	\item Information \\
	Manche Menschen verfügen über die Information, welche Telefonnummer ihre Mutter hat. Die Information wäre in diesem Fall die Ziffernfolge. Ein Beispiel für eine digatales Datum wäre die MAC-Adresse der Netzwerkkarte.        
	\item Explizites Wissen \\
	Wissen über die Regeln des Schachspiels bei einem Schachspieler. Diese können in der Regel klar von einem Schachspieler formuliert und angewendet werden. Solches Wissen lässt sich formalisieren und findet sich also auch in den anderen beiden Kategorien (als Print- und als Digitalmedium). Neben (Schach)Regeln gehören auch mathematische Gesetze zu dieser Kategorie.
	\item Implizites Wissen\\
	Wie man ein Objekt, dass vor mir liegt greife. Man verfügt zwar über dieses Wissen, kann es anwenden aber nicht formulieren, da wohl niemand genau motorische und kognitive Abläufe beschreiben kann. Ein Beispiel für implizites Wissen, dass auf einem Smartphone vorhanden ist, wäre die gesamte Anzahl an Minuten, die der Nutzer in diesem Monat telefoniert hat. Zwar wird die Dauer der einzelnen Gespräche gespeichert nicht aber (das mag in manchen Fällen auch anders sein) die die aufsummierte Dauer.          
\end{enumerate}

\subsection*{Umgebungen}
\begin{enumerate}
	\item vollkommen wahrnehmbar \& partiell wahrnehmbar \\
	Ein Beispiel für eine vollkommen wahrnehmbare Umgebung ist die aktuelle Stellung eines Schachspiel für einen Schachcomputer. Die Position jeder relavanten Figur ist bekannt. Eine Umgebung die partiell wahrnehmbar ist, wäre die Umwelt eines Roboters, der nur über eine Kamera verfügt. Dieser verfügt zu einem bestimmten Zeitpunkt immer nur über visuelle Daten eines bestimmten Ausschnittes. Im Gegensatz zu vollkommenen wahrnehmbarer Umgebung müssen von der KI Schätzungen über andere nicht wahrnehmbare Objekte gemacht werden, um ein Problem zu lösen.          
	\item diskret \& kontinuierlich \\
    Während in einer diskreten Umgebung jedwede Information auf ganze Zahlen abgebildet werden kann (also die Grenzen zwischen 2 Informationseinheiten klar definiert sind) ist genau dies bei kontinuierlichen Umgebungen nicht der Fall.
	Der Unterschied zwischen diskreten und kontinuierlichen hat vor allem Auswirkungen auf die Implementation der Welt in Rechensystemen. Da Computer auf diskreten Modellen beruhen muss die Welt für die KI zwangsläufig diskretisiert werden. Somit kommt bei kontinuierlichen Welten eine weitere Dimension der Implementation hinzu.
	\item deterministisch \& stochastisch \\
Im Gegensatz zu deterministischen Umgebungen, in denen Folgezustände klar definiert sind, kann in einer stochastischen nur mit Wahrscheinlichkeiten von Folgezuständen operiert werden. Das bedeutet, dass Systeme die in einer stochastischen Umgebungen operieren, mit einer gewissen Unsicherheit umgehen können müssen, d.h. dass sie in der Regel robuster sein müssen. Dies zieht auch große Problematiken beim Auffinden von Softwarefehlern mit sich.
\end{enumerate}

\end{document}