\documentclass[a4paper]{scrartcl}
\usepackage[ngerman]{babel}
\usepackage[utf8]{inputenc}
\usepackage[T1]{fontenc}
\usepackage{lmodern}
\usepackage{amssymb}
\usepackage{amsmath}
\usepackage{enumerate}
\usepackage{pgfplots}
\usepackage{scrpage2}\pagestyle{scrheadings}
\usepackage{tikz}
\usepackage{listings}
\usetikzlibrary{patterns}

\newcommand{\titleinfo}{Hausaufgaben zum 23. 11. 2012}
\title{\titleinfo}
\author{Tronje Krabbe 6435002, The-Vinh Jackie Huynh 6388888,\\Arne Struck 6326505}
\date{\today}
\chead{\titleinfo}
\ohead{\today}
\setheadsepline{1pt}
\setcounter{secnumdepth}{0}
\lstset{language=Java}
\newcommand{\qed}{\ \square}

\begin{document}
\maketitle
\notag
\section{1.}
	\begin{lstlisting}
		...
		int b1 = (a1 << 2 | a2 >>> 4) % 256;
		int b2 = (a2 << 4 | a3 >>> 2) % 256;
		int b3 = (a3 << 6 | a4) % 256;
		...
	\end{lstlisting}
	Der Modulo 256 wird benötigt, damit nur die 8 letzten Bits übernommen werden, sofern die b's 
	nicht vorher so definiert wurden, dass sie nur die 8 letzten Bits annehmen.
	

\section{2.}
	\subsection{a)}
		\begin{align}
			\frac{360^{\circ}}{15^{\circ}} &= 24 \text{ Codewörter} 
		\end{align}
		
	\subsection{b)}
		\textbf{2 Codewörter:} \\ \\
		\(
		\begin{array}{ccc}
			0&\ & 1
		\end{array}
		\)\\ \\
		\textbf{4 Codewörter:} \\ \\
		\(
		\begin{array}{ccc}
			00&\ & 01 \\
			11&\ & 10 
		\end{array}
		\)\\ \\
		\textbf{8 Codewörter:} \\ \\
		\(
		\begin{array}{ccccccc}
			(000)&\ & 001 &\ & 011&\ & 010 \\
			110&\ & 111 &\ & 101&\ & (100) 
		\end{array}
		\)\\ \\
		\newpage
		\begin{flushleft}
			\textbf{12 Codewörter:} \\
		\end{flushleft}		
		(die Wörter in () wurden schon hier weggelassen, um in 2 Schritten 
		auf ein valides Ergebnis zu kommen.) \\ \\
		\(
		\begin{array}{ccccccccccccc}
			0001&\ & 0011&\	& 0010 &\ & 0110&\ &0111&\ & 0101 \\
			1101&\ & 1111&\ & 1110 &\ & 1010&\ &1011&\ & 1001
		\end{array}
		\)\\ \\
		\textbf{24 Codewörter:} \\ \\
		\(
		\begin{array}{cccccccccccccccccccccccccccccccccccccc}
			00001&\ &00011&\ &00010&\ &00110&\ &00111&\ &00101&\ &01101&\ &01111 \\
			01110&\ &01010&\ &01011&\ &01001 \\
			11001&\ &11011&\ &11010&\ &11110&\ &11111&\ &11101&\ &10101&\ &10111 \\
			10110&\ &10010&\ &10011&\ &10001 \\
		\end{array}
		\)
		
	
\section{3.}
	\subsection{a)}
		\begin{figure}[!h]
	        \centering
			\begin{tikzpicture}[x=2em,y=-1.5em,every node=radius\=3]
	        	\tikzstyle{n}=[draw,circle,radius=3cm,font=\small,inner sep=0,minimum 
	                       size=2.3em]
	
		            \node[n] (R) at (-1,0) {1};
	
	    	        \node[n] (R0) at (-3,3) {0,57};
	
	              	\node[n] (R00) at (-4,6) {0,27};
	        		\node[n,label=below:d] (R01) at (-2,6) {0,3};
	                \node[n] (R1) at (3,4.5) {0,43};
	
	                \node[n] (R000) at (-3,9) {0,15};
	                \node[n,label=below:k] (R001) at (-5,9) {0,12};
	                \node[n] (R10) at (1,9) {0,23};
	                \node[n] (R11) at (5,9) {0,2};
	
	                \node[n,label=below:l] (R0000) at (-2,12) {0,6};
	                \node[n] (R0001) at (-4,12) {0,9};
	                \node[n,label=below:a] (R100) at (0.25,12) {0,12};
	                \node[n] (R101) at (2,12) {0,11};
	                \node[n,label=below:j] (R110) at (4,12) {0,1};
	                \node[n,label=below:g] (R111) at (6,12) {0,1};
	
	                \node[n] (R00010) at (-3,15) {0,04};
	                \node[n,label=below:c] (R00011) at (-5,15) {0,05};
	                \node[n] (R1010) at (1,15) {0,06};
	                \node[n,label=below:f] (R1011) at (3,15) {0,05};
	
	                \node[n,label=below:e] (R000100) at (-4,18) {0,02};
	                \node[n,label=below:h] (R000101) at (-2,18) {0,02};
	                \node[n,label=below:b] (R10100) at (0,18) {0,03};
	                \node[n,label=below:i] (R10101) at (2,18) {0,03};
	
	                \draw
		                (R) to (R0)
	                    (R) to (R1)
	
	    	            (R0) to (R00)
	                    (R0) to (R01)
	                    (R1) to (R10)
	                    (R1) to (R11)
	
	                    (R00) to (R000)
	                    (R00) to (R001)
	                    (R10) to (R100)
	                    (R10) to (R101)
	                    (R11) to (R110)
	                    (R11) to (R111)
	
	                    (R000) to (R0000)
	                    (R000) to (R0001)
	                    (R101) to (R1010)
	                    (R101) to (R1011)
	
	            		(R0001) to (R00010)
	                    (R0001) to (R00011)
	                    (R1010) to (R10100)
	                    (R1010) to (R10101)
	
	                    (R00010) to (R000100)
	                    (R00010) to (R000101)
	                    ;
			\end{tikzpicture}
		\end{figure}
                \textbf{Codierung:} Linke Kante entspricht 0, rechte Kante entspricht 1. \\ \\
               \begin{center}
	               \begin{tabular}{|c|l|c|l|}
	               		\hline
	               		Symbol & Codierung & Symbol &Codierung \\ \hline
	               		a & 100 & g & 111 \\
	               		b & 10100 & h & 001011 \\
	               		c & 00100 & i & 10101 \\
	               		d & 01 & j & 110 \\
	             		e & 001010 & k & 000 \\
	             		f & 1011 & l & 0010 \\ \hline
	               \end{tabular}
               \end{center}
               
                
	\subsection{b)}
		\(
		\begin{array}{lcl}
			H &=& -2 \cdot 0,02 \cdot \log_2(0,02) -2 \cdot 0,03 \cdot \log_2(0,03)-2\cdot 0,05 
			\cdot\log_2(0,05) \\ 
			&& -2 \cdot 0,1 \cdot \log_2(0,1) -2 \cdot 0,12 \cdot \log_2(0,12)-0,3\cdot\log_2(0,3)
			-0,06\cdot\log_2(0,06) \\
			&\approx & 3,125
		\end{array}
		\)

	
\section{4.}	
	\subsection{a)}
		\begin{tabular}{l|c|c|c|c|c|c|c|c|c|c}
		Ziffer& 0&1&2&3&4&5&6&7&8&9\\ \hline
		Codierung&0001&0010&0011&0100&0101&0110&0111&1000&1001&1010
		\end{tabular}
		\begin{align}
			H_0&=10\cdot 0,1\cdot \log_2(2^4) = \log_2(16)=4\text{ Bit} \\
			H &= -10\cdot 0,1\cdot \log_2(0,1) \approx 3,219\text{ Bit} \\
			R &= H_0-H = 4+\log_2(0,1) \approx 0,678 \text{ Bit} 
		\end{align}
		
	\subsection{b)}
		Will man die 100 verschiedenen Paare binär codieren, braucht man mindesten sieben stellige 
		Binärworte, also eine Wortlänge von \(2^7\). Da aber jeder Ziffer eine Tetrade zugeordnet 
		werden soll, muss die Wortbreite \(2^8\) betragen. \\
		Eine mögliche Neucodierung wäre, analog zu a) die bisherige Codierung jeder Ziffer um eine 
		Stelle zu verschieben und diese dann zu gruppieren. \\
		\(\Rightarrow \Big\{00010001,00010010,...,10101010\Big\}\)
		\begin{align}
			H_0&=100\cdot 0,01\cdot \log_2(2^8) = \log_2(128)=8\text{ Bit} \\
			H &= -100\cdot 0,01\cdot \log_2(0,01) \approx 6,643\text{ Bit} \\
			R_{ges} &= H_0-H = 7+\log_2(0,01) \approx 1,356 \text{ Bit} \\ 
			R &= 0,01\cdot \log_2(2^7)- 0,01\cdot \log_2(0,01) \approx 0,1464
		\end{align}
		
	\subsection{c)}
		Da bei 1000 verschiedenen Paaren \(2^8=256\) offensichtlich nicht mehr ausreicht, muss die 
		neue Wortbreite \(2^{12} = 4096\) betragen. \\
		\begin{align}
			H_0&= 0,001\cdot\log_2(2^{12}) = 0,012
		\end{align}
		Dies gilt analog auch für 10000 Paare (4 Stellen), allerdings reicht hier auch nicht 
		\(2^{12}\), da auch hier wieder zu wenig Kombinationsmöglichkeiten vorhanden sind.
		also ist es \(2^{16}\) 
		\begin{align}
			H_0&= 0,0001\cdot\log_2(2^{16}) = 0,0016 
		\end{align}
			
	\subsection{d)}
		Bei variabler Codierungslängen ergibt sich folgende Aufteilung: \\ \\
		\begin{tabular}{c|c|c|c|c|c|c|c|c|c}
			0&1&2&3&4&5&6&7&8&9 \\ \hline
			0000&0001&001&010&011&100&101&110&1110&1111
		\end{tabular}
		\begin{align}
			H_0 &= \frac{1}{10} \cdot (6\cdot 3 + 4\cdot 4) \\
			&= 3,4 \\
			R &= 3,4 -\log_2(0,1) = H_0 - H \\
			&\approx 0,078 \\
		\end{align}
		Die Redundanz hat sich deutlich verringert. \\
		Da sich die Redundanz aber auch mit dem Kombinieren von Zahlen zu Gruppen verringert könnte 
		man beide Verfahren kombinieren, um die Redundanz noch weiter zu verringern.
		



\end{document}