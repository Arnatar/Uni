\documentclass[a4paper]{scrartcl}
\usepackage[ngerman]{babel}
\usepackage[utf8]{inputenc}
\usepackage[T1]{fontenc}
\usepackage[a4paper, left=2.5cm, right=2.5cm, top=4cm]{geometry}
\usepackage{lmodern}
\usepackage{amssymb}
\usepackage{amsmath}
\usepackage{enumerate}
\usepackage{pgfplots}
\usepackage{scrpage2}\pagestyle{scrheadings}
\usepackage{tikz}
\usepackage{listings}
\usetikzlibrary{patterns, arrows, automata}

\newcommand{\titleinfo}{Hausaufgaben zum 11. 1. 2013}
\title{\titleinfo}
\author{Tronje Krabbe 6435002, The-Vinh Jackie Huynh 6388888,\\Arne Struck 6326505}
\date{\today}
\chead{\titleinfo}
\ohead{\today}
\setheadsepline{1pt}
\setcounter{secnumdepth}{0}
\lstset{language=Java}
\newcommand{\qed}{\ \square}

\begin{document}
\maketitle
\notag
\section{10.1}
	
	\subsection{a)}
		Wir nehmen an, dass Zähler (dekrementierend) und Hauptautomat eigene Zustände besitzen. Bis 
		der Zähler nicht bei 0 angekommen ist, weist der aktuelle Zustand des Hauptautomates auf 
		sich selber. Sobald der Zähler auf 0 steht, wird der Zustand des Hauptautomates zum nächsten 
		Zustand	weiter geschaltet. \\
		Aus dem neuen Zustand im Hauptautomaten wird der korrekte Startpunkt für eine neue Zählung 
		gewählt (Ampelphasen sind erfahrungsgemäß unterschiedlich lang, woraus sich unterschiedlich 
		zu 	wartende Zeiten ergeben. Des weiteren hat die Taktung natürlich auch einen Einfluss auf 
		den zu 	wählenden Zählerstart). Nun beginnt der Zähler erneut und der Vorgang wiederholt 
		sich. \\
		Bei einem inkrementierenden Zähler müsste es folgende Leitungen geben: Zähler-Reset und 
		Zählerstand \\


	\subsection{b)}
		Hier liegen 2 verschiedene Zustände (die der beiden Automaten) vor, allerdings ist nicht 
		geklärt, welcher Zustand bei Wechsel als Eingabe für den anderen Automaten verwendet werden
		soll, das Verhalten ist also undefiniert. Eine kleine Verzögerung soll nun bewirken, dass
		immer der Vorzustand des anderen Automaten als Eingang verwendet wird (bei gleicher 
		Taktung). Bei dieser Variante sieht die Grünphase wie folgt aus:
		\begin{itemize}
			\item Ausgangszustände: Zähler auf 1, Ampel auf Rot/Gelb 
			\item Zähler springt auf 0, die Ampel bleibt auf Rot/Gelb (der Vorzustand des Zählers 
			ist 1).
		 	\item Die Ampel springt auf Grün. 
		 	\item Da die Ampel auf Rot/Gelb stand (Vorzustand), wechselt der Zähler auf den neuen 
		 	Ausgangswert. 
		 	\item Der Zähler fängt an zu zählen. 
		 	\item Der Zähler springt auf 0, die Ampel bleibt auf Grün. 
		 	\item Die Ampel schaltet auf Gelb...  
		\end{itemize}
		


	\subsection{c)}
		Im Gegensatz zu Teil b) werden die Zustände nicht gegenseitig abgefragt, während sich diese 
		ändern. So existieren nur feste Zustände: 
		\begin{itemize}
			\item Ausgangszustände: Zähler auf 1, Ampel auf Rot/Gelb. 
			\item Zähler springt auf 0, Ampel auf Rot/Gelb. 
			\item Ampel springt auf Grün. 
			\item Die Ampel steht auf Grün, deswegen wechselt der Zähler auf die für die Grünphase 
			vorgesehene	Zeit. 
			\item Ampel wartet auf Zählerzustand 0. 
			\item Zähler zählt abwärts bis 0. 
			\item Ampel wechselt auf Gelb. 
			\item Zähler wechselt auf die für Gelbphase vorgesehene Zeit.
		\end{itemize}
		

	\subsection{d)}
		Der Hauptautomat hat die vier Zustände $Z_0$ bis $Z_3$. Außerdem werden die Werte $z_0$ und 
		$z_1$ definiert, die den Zuständen der Zählerautomaten angeben, wann der durchgeführt werden
		soll (also führen sie vom Hauptautomaten zum Zähler). \\ 
		Des weiteren wird $e$ definiert, welches als Enablesignal für den Hauptautomaten dient ($e$ 
		führt also vom Zähler zum Hauptautomaten). 
		
		
		\begin{center}
			\begin{tabular}{ccc}
				\begin{tikzpicture}[->,>=stealth',shorten >=1pt,auto,node distance=2.8cm,
				semithick]
				\tikzstyle{every state}=[circle split, draw]
				
				\node[state] (A) {$Z_0$ \nodepart{lower} 
				$\overline{z_1} \overline{z_0} $  };
				\node[state] (B) [above right of=A] {$Z_1$ \nodepart{lower} 
				$\overline{z_1}z_0$ };
				\node[state] (C) [below right of=B] {$Z_2$ \nodepart{lower}
				$z_1\overline{z_0} $};
				\node[state] (D) [below right of=A]	{$Z_3$\nodepart{lower}
				$z_1z_0$};
				
				
				\path (A) edge node {e} (B)
				edge [loop left] node {$\overline{e}$} (A)
				(B) edge [loop above] node {$\overline{e}$} (B)
				edge node {e} (C)
				(C) edge node {e} (D)
				edge [loop right] node {$\overline{e}$} (C)
				(D) edge [loop below] node {$\overline{e}$} (D)
				edge node {e} (A);
				\end{tikzpicture}
				&\quad\quad &
				\begin{tabular}{ll}
					$\overline{z_1} \overline{z_0}$: Reset bei 30000 \\
					$\overline{z_1}z_0$: Reset bei 3000 \\
					$z_1\overline{z_0}$: Reset bei 35000 \\
					$z_1z_0$: Reset bei 5000 \\ \\
					$Z_0$: rot \\
					$Z_1$: rotgelb \\
					$Z_2$: grün \\
					$Z_3$: gelb			
				\end{tabular}
								
			\end{tabular}
		\end{center}


\section{10.2}
	\subsection{a)}
		\begin{center}
			\begin{tabular}{|c|c|c||c|c||c|c|c||c|c|}
	        	\hline
	            $Z$ & $z_1$ & $z_0$ & $x_1$ & $x_0$ & $Z^+$ & $z^+_1$ & $z^+_0$ & $y_1$ & $y_0$ \\
	            \hline
	            0 & 0 & 0 & * & 0 & 1 & 0 & 1 & 1 & 0 \\
	            0 & 0 & 0 & * & 1 & 0 & 0 & 0 & 1 & 0 \\
	            1 & 0 & 1 & * & 0 & 1 & 0 & 1 & 0 & 1 \\
	            1 & 0 & 1 & 0 & 1 & 3 & 1 & 1 & 0 & 1 \\
	            1 & 0 & 1 & 1 & 1 & 2 & 1 & 0 & 0 & 1 \\
	            2 & 1 & 0 & 0 & 0 & 1 & 0 & 1 & 0 & 1 \\
	            2 & 1 & 0 & 0 & 1 & 0 & 0 & 0 & 0 & 1 \\
	            2 & 1 & 0 & 1 & * & 2 & 1 & 0 & 0 & 1 \\
	            3 & 1 & 1 & * & 0 & 2 & 1 & 0 & 1 & 1 \\
	            3 & 1 & 1 & * & 1 & 3 & 1 & 1 & 1 & 1 \\ \hline
	     	\end{tabular}
		\end{center}
		\begin{align}
			\delta(z_1, z_0, x_1, x_0) &= (z^+_1(z_1, z_0, x_1, x_0), z^+_0(z_1, z_0, x_1, x_0)) \\
			z^+_1(z_1, z_0, x_1, x_0) &= z_1z_0 \vee z_1x_1 \vee z_0x_0 \\
			z^+_0(z_1, z_0, x_1, x_0) &= z_1z_0x_0 \vee \overline{z_1}\;\overline{x_0} \vee
			z_0\overline{x_1}x_0 \vee \overline{z_0}\;\overline{x_1}\;\overline{x_0}
		\end{align}
		
	\subsection{b)}
		\[\lambda(z_1, z_0) = (z_1 \oplus z_0 \oplus 1, z_1 \vee z_0)\]
		
	\subsection{c)}
		\begin{align}
			Z_0&: x_0\wedge \overline{x_0}=0 \\
			Z_1&: \overline{x_0} \wedge x_1x_0 \wedge \overline{x_1}x_0 =0  \\
			Z_2&: x_1\wedge\overline{x_1}x_0\wedge \overline{x_1}\overline{x_0}=0 \\
			Z_3&: x_0\wedge \overline{x_0}=0
		\end{align}
		\begin{align}
			Z_0&: x_0\vee \overline{x_0}=1 \\
			Z_1&: \overline{x_0} \vee x_1x_0 \vee \overline{x_1}x_0 =1  \\
			Z_2&: x_1\vee\overline{x_1}x_0\vee \overline{x_1}\overline{x_0}=1 \\
			Z_3&: x_0\vee \overline{x_0}=1
		\end{align}

\end{document}