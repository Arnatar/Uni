\documentclass[a4paper]{scrartcl}
\usepackage[ngerman]{babel}
\usepackage[utf8]{inputenc}
\usepackage[T1]{fontenc}
\usepackage{lmodern}
\usepackage{amssymb}
\usepackage{amsmath}
\usepackage{enumerate}
\usepackage{pgfplots}
\usepackage{scrpage2}\pagestyle{scrheadings}
\usepackage{tikz}
\usepackage{listings}
\usetikzlibrary{patterns}

\newcommand{\titleinfo}{Hausaufgaben zum 30. 11. 2012}
\title{\titleinfo}
\author{Tronje Krabbe 6435002, The-Vinh Jackie Huynh 6388888,\\Arne Struck 6326505}
\date{\today}
\chead{\titleinfo}
\ohead{\today}
\setheadsepline{1pt}
\setcounter{secnumdepth}{0}
\lstset{language=Java}
\newcommand{\qed}{\ \square}

\begin{document}
\maketitle
\notag
\section{1.}
	Es gibt keine zyklisch-einschrittigen Binärcodes die eine ungerade Zahl an Codewörtern besitzt, 
	denn falls ein Zeichen von einem Codewort verändert wird, muss dieses auch wieder auf
	seinen Ursprungszustand rückführbar sein (um einen zyklischen Binärcode zu erhalten), also wird
	dafür ein weiteres Codewort gebraucht.
	Analog funktioniert auch die Veränderung von mehr als einem Codewort, allerdings werden hierfür 
	natürlich mehr Schritte gebraucht. 
	
\section{2.}
	\subsection{a)}
		Die Minimaldistanz beträgt 4. Dies kann leicht verifiziert werden, indem ein Datenbit an 
		einer beliebigen Stelle verändert wird (innerhalb der Reichweite, also 7\(\times\)7). 
		Allerdings verändert dies die Paritätsbits der jeweils zugehörigen Spalte und der zugehörigen
		Zeile, was dazu führt, dass das Paritätsbit \(p_{7,7}\) auch angepasst werden muss.
	
	\subsection{b)}
		Ein Einbitfehler an einer beliebigen Position \((i,j\)) führt automatisch zur Verfälschung
		der	entsprechenden Paritätsbits der Zeile \(i\) und Spalte \(j\). Also wir der Fehler erkannt
		und	kann durch die Information von Zeile und Spalte auch lokalisiert werden und somit
		korrigiert werden. \\
		\\		
		Ein Zweibitfehler an den beliebigen Positionen \(((i_1,j_1),(i_2,j_2))\) führt zu 2 Fällen.
		Bei dem ersten Fall sind Spalte bzw. Zeile der beiden Fehler identisch
		\(((i_1 = i_2)\land(j_1\neq j_2))\) oder \(((i_1\neq i_2)\land(j_1 = j_2))\). Daraus
		resultiert,	durch die doppelte Fehlerbelegung entweder die Spalten oder Zeilenparität korrekt
		ist, allerdings kann das Vorliegen eines Fehlers dennoch an den jeweils anderen Paritäten 
		abgelesen werden. Da aber aufgrund der Fehlenden Informationen keine Möglichkeit gegeben ist
		den Fehler eindeutig zu lokalisieren, kann er auch nicht korrigiert werden. \\
		Im zweiten Fall sind die Positionen der Fehler weder in Spalte, noch in Zeile identisch
		\((i_1\neq i_2)\land(j_1\neq j_2)\). Hier wird der Fehler erkannt, kann jedoch auch nicht 
		lokalisiert werden, da die Fehler an 2 verschiedenen Positionen liegen könnte
		\(((i_1,j_1)\land(i_2,j_2))\) oder \(((i_1,j_2)\land(i_2,j_1))\). Also kann er nicht behoben 
		werden. \\
		\\
		Analog zur Zweibitfehlern gestaltet sich der Sachverhalt bei Dreibitfehlern. Hier existieren 
		weitaus mehr Fälle, als bei Zweibitfehlern. Allerdings ist auch hier in jeglicher Anordnung 
		mindestens eine Zeilen- oder Spaltenparität gestört, sodass ein Fehler erkannt wird. \\
		Auch dieser kann nicht korrigiert werden, da wie im Abschnitt über die Zweibitfehler 
		angedeutet die einzelnen Fehler nicht lokalisiert und somit auch nicht korrigiert werden 
		können.	
		
		
	\subsection{c)}
		Ein nicht erkennbarer Vierbitfehler wäre \(d_{0,0},\ d_{0,1},\ d_{1,0},\ d_{1,1}\) oder 
		jegliche andere Anordnung der Fehler, bei denen sich 2 Fehler in einer Spalte und 2 Fehler in 
		einer Zeile befinden sind nicht erkennbar, da sie die Paritätsfehler aufheben.
		
	
	\subsection{d)}
		Die Gesamtzahl der Vierbitfehler ist äquivalent zu dem aus der Kombinatorik bekanntem Ziehen 
		ohne Reihenfolge und ohne Zurücklegen:
			\(\begin{pmatrix}
				81\\4
			\end{pmatrix}=
			1663740\) Vierbitfehler existieren. \\ \\
		Die Anzahl der nichterkannten Vierbitfehler errechnet sich wie folgt: 
		\(\frac{81\cdot 8\cdot 8\cdot 1}{4}=1296\) \\
		Der Anteil der nichterkannten Fehler beträgt also \(\frac{1296}{1663740}\approx 0,0007897\)

	
\section{3.}
	\subsection{a)}
		Das 9. Codewort entspricht 1110000. Durch die Verfälschung von \(c_7\) ergibt sich 1110001.\\
		\begin{align}
			x_a &= 1\oplus 1\oplus 0\oplus 1 = 1 \\
			x_b &= 1\oplus 1\oplus 0\oplus 1 = 1\\
			x_c &= 0\oplus 0\oplus 0\oplus 1 = 1
		\end{align}
		Daraus folgt das Prüfwort 111 oder 7, womit der Fehler lokalisiert ist.
		
	
	\subsection{b)}
		Um die n-te Zeile der Generatormatrix G zu erhalten, wird aus dem Schema abgelesen, welche 	
		Datenbits den Wert des jeweiligen Codebits \((c_n)\) bestimmen. Das führt zu 2 Fällen, 
		entweder ist \(c_n\) ein Paritätsbit \((p_m)\) oder ein Datenbit \((d_o)\). \\
		Wenn \(c_n\) ein Paritätsbit ist, so wird in der n-ten Matrixzeile jedes Bit gesetzt, welches 
		(laut Zuordnungsvorschrift) zur Berechnung des entsprechenden Paritätsbits benötigt wird. \\
		Ist \(c_n\) hingegen ein Datenbit \((d_o)\), dann wird in der Matrix das m-te Bit gesetzt. \\
		Daraus resultiert eine Matrix, welche für jedes Datenbit eine Spalte und jedes Zeichenbit 
		eine Zeile besitzt. \\ 
		\\	
		Die Prüfmatrix H resultiert aus den Prüfgruppen (durch die farbigen Punkte über dem 
		Tabellenteil des Schemas dargestellt). Sie entsprechen zeilenweise gelesen den Zeilen der 
		Prüfmatrix in entgegengesetzter Reihenfolge (Reihenfolge der Spalten). Ein Punkt codiert in 
		diesem Fall eine 1, kein Punkt eine 0. \\
		\[\textbf{H}=\begin{pmatrix}
			1&0&1&0&1&0&1&... \\
			0&1&1&0&0&1&1&... \\
			0&0&0&1&1&1&1&... \\
			0&0&0&0&0&0&0&... \\
			0&0&0&0&0&0&0&... \\
		\end{pmatrix}\]



\end{document}