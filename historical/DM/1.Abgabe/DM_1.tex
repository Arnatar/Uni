\documentclass[a4paper]{scrartcl}
\usepackage[ngerman]{babel}
\usepackage[utf8]{inputenc}
\usepackage[T1]{fontenc}
\usepackage{lmodern}
\usepackage{amssymb}
\usepackage{amsmath}
\usepackage{enumerate}
\usepackage{pgfplots}
\usepackage{scrpage2}\pagestyle{scrheadings}
\usepackage{tikz}
\usetikzlibrary{patterns}

\newcommand{\titleinfo}{Hausaufgaben zum 25./26. 10. 2012}
\title{\titleinfo}
\author{Arne Struck}
\date{\today}
\chead{\titleinfo}
\ohead{\today}
\setheadsepline{1pt}
\setcounter{secnumdepth}{0}
\newcommand{\qed}{\ \square}
\def\firstcircle{(0,0) circle (1.5cm)}
\def\secondcircle{(0:2cm) circle (1.5cm)}

\begin{document}
\maketitle
\notag

\section{1.}
\subsection{a)}
	\begin{align}
		A={1,2,3}&\quad B={3,4}\\
		f: A \to B \quad g:& A\to B\quad h: A\to B
	\end{align}

\subsubsection{(i)}
	\begin{tabular}{c|c}
		A & B \\ \hline 
		1 & 3 \\
		2 & 4 \\
		3 & 4 
	\end{tabular}
	\makebox[\linewidth]{
	\centering
	\begin{minipage}[c]{0.5\textwidth}
    	\begin{tikzpicture}[]
            \node[label=$A$] (A) at (-2, 2) {};
        	\node[label=left:$1$, fill, circle, inner sep=2pt] (A1) at (-2, 1) {};
            \node[label=left:$2$, fill, circle, inner sep=2pt] (A2) at (-2, 0) {};
            \node[label=left:$3$, fill, circle, inner sep=2pt] (A3) at (-2,-1) {};
            \draw (-2.2, 0) ellipse (1 and 2) {}; 
			
			\node[label=$B$] (B) at (2, 2) {};
            \node[label=right:$3$, fill, circle, inner sep=2pt] (B3) at (2, 0.5) {};
            \node[label=right:$4$, fill, circle, inner sep=2pt] (B4) at (2,-0.5) {};
            \draw (2.2, 0) ellipse (1 and 2) {};

            \draw[->] (A1) to (B3);
            \draw[->] (A2) to (B4);
            \draw[->] (A3) to (B4);
	  	\end{tikzpicture}
	\end{minipage}
	}

\subsubsection{(ii)}
	\(g: A\to B\) kann nicht injektiv sein, da A mächtiger ist, als B. Das hat zur Folge, dass die 
	Bedingung für Injektivität nicht erfüllt werden kann.

\subsubsection{(iii)}
	\(h: A\to B\) kann nicht bijektiv sein, da \(g\) nicht injektiv ist.

\newpage
\subsection{b)}
	\begin{align}
		A={1,2,3}&\quad B={3,4,5}\\
		f: A \to B \quad g:& A\to B\quad h: A\to B
	\end{align}
	
\subsubsection{(i)}
	\(f\) ist nicht existent, da A und B die gleiche Mächtigkeit besitzen und Surjektivität sich in 	
	diesem Fall nicht ohne Injektivität erzeugen lässt.

\subsubsection{(ii)}
	\(g\) ist nicht existent, da A und B die gleiche Mächtigkeit besitzen und Injektivität sich in 	
	diesem Fall nicht ohne Surjektivität erzeugen lässt.

\subsubsection{(iii)}
	\begin{tabular}{c|c}
		A & B\\ \hline
		1 & 3\\ 
		2 & 4\\
		4 & 5
	\end{tabular}
\makebox[\linewidth]{
	\centering
	\begin{minipage}[c]{0.5\textwidth}
    	\begin{tikzpicture}[]
        	\node[label=left:$1$, fill, circle, inner sep=2pt] (A1) at (-2, 1) {};
            \node[label=left:$2$, fill, circle, inner sep=2pt] (A2) at (-2, 0) {};
            \node[label=left:$3$, fill, circle, inner sep=2pt] (A3) at (-2,-1) {};
            \draw (-2.2, 0) ellipse (1 and 2) {};
            \node[label=$A$] (A) at (-2, 2) {};
            \node[label=right:$3$, fill, circle, inner sep=2pt] (B3) at (2, 1) {};
            \node[label=right:$4$, fill, circle, inner sep=2pt] (B4) at (2, 0) {};
            \node[label=right:$5$, fill, circle, inner sep=2pt] (B5) at (2,-1) {};
            \draw (2.2, 0) ellipse (1 and 2) {};
            \node[label=$B$] (B) at (2, 2) {};

            \draw[->] (A1) to (B3);
            \draw[->] (A2) to (B4);
            \draw[->] (A3) to (B5);
	     \end{tikzpicture}
	  \end{minipage}
}

\subsection{c)}
\begin{align}
	A={1,2,3}&\quad B={3,4,5,6}\\
	f: A \to B \quad g:& A\to B\quad h: A\to B
\end{align}

\subsubsection{(i)}
	\(f\) ist nicht darstellbar, da B mächtiger ist, als A und bei dem Versuch Surjektivität 		
	herzustellen somit die Funktionsbedingung verletzt werden würde.

\subsubsection{(ii)}
	\begin{tabular}{c|c}
		A&B\\ \hline
		1&3\\
		2&4\\
		3&5
	\end{tabular}
\makebox[\linewidth]{
	\centering
	\begin{minipage}[c]{0.5\textwidth}
    	\begin{tikzpicture}[]
            \node[label=$A$] (A) at (-2, 2) {};
        	\node[label=left:$1$, fill, circle, inner sep=2pt] (A1) at (-2, 1) {};
            \node[label=left:$2$, fill, circle, inner sep=2pt] (A2) at (-2, 0) {};
            \node[label=left:$3$, fill, circle, inner sep=2pt] (A3) at (-2,-1) {};
            \draw (-2.2, 0) ellipse (1 and 2) {};

            \node[label=$B$] (B) at (2, 2) {};
            \node[label=right:$3$, fill, circle, inner sep=2pt] (B3) at (2, 1) {};
            \node[label=right:$4$, fill, circle, inner sep=2pt] (B4) at (2, 0.33) {};
            \node[label=right:$5$, fill, circle, inner sep=2pt] (B5) at (2, -0.33) {};
			\node[label=right:$6$, fill, circle, inner sep=2pt] (B6) at (2, -1) {};            
            \draw (2.2, 0) ellipse (1 and 2) {};

            \draw[->] (A1) to (B3);
            \draw[->] (A2) to (B4);
            \draw[->] (A3) to (B5);
		\end{tikzpicture}
	\end{minipage} 

}
                         
\subsubsection{(iii)}
	\(h\) ist nicht darstellbar, da A und B nicht die gleiche Mächtigkeit besitzen.
\\
\\

\section{2.}
	\begin{align}
		f(x) &= x^2-5 \\
		g(x) &= 5x-3 \\
		h(x) &= x+5
	\end{align}

\subsection{f(x):}
	Beh.: \(f(x)\) ist nicht injektiv.\\
	Annahme: \(f(x)\) ist injektiv,\\
	\(\forall x,y \in \mathbb{Z}:x\neq y : f(x) \neq f(y)\)\\
	\begin{align}
		f(-2)&= (-2)^2-5 = 4-5 = -1\\
		f(2) &= (2)^2-5 = 4-5 = -1 \\
		f(-2)&= f(2) \quad \text{\textbf{Widerspruch}}
	\end{align}
	\(\Rightarrow \ f(x)\) ist nicht injektiv \(\qed\) \\
	\newpage
	\begin{flushleft}
		Beh.: \(f(x)\) ist nicht surjektiv.\\
		Annahme: \(f(x)\) ist surjektiv\\
	\end{flushleft}
	\begin{align}
		\text{sei } y\in \mathbb{Z}\\
		f(x) =y &=x^2-5\\
		\Leftrightarrow x &= \sqrt{y+5}\\
		\text{sei } y_1 &= 1\\
		\Rightarrow x_1&=\sqrt{6}\\
		x_1\notin \mathbb{Z}\quad \text{\textbf{Widerspruch}}
	\end{align}
	\(\Rightarrow \ f(x)\) ist nicht surjektiv \(\qed\) \\
	Da \(f\) weder surjektiv noch injektiv ist, ist \(f\) auch nicht bijektiv.\\

\subsection{g(x):}
	Beh.: \(g\) ist injektiv.\\
	Annahme: \(g\) ist nicht injektiv,\\
	\(\exists\ x,y \in \mathbb{Z}:x\neq y:f(x)=f(y)\) 
	\begin{align}
		f(x) &= f(y)\\
		5x-3 &= 5y-3\\
		x&=y \quad \text{\textbf{Widerspruch}}
	\end{align}
	\(\Rightarrow \ g\) ist injektiv \(\qed\)\\

	Beh.: \(g\) ist nicht surjektiv.\\
	Annahme: \(g\) ist surjektiv.\\
	\begin{align}
		\text{sei } y\in\mathbb{Z}\\
		g(x)=y&=5x-3\\
		\frac{y+3}{5}&=x\\
		\frac{y+3}{5}\notin\mathbb{Z}\quad \text{\textbf{Widerspruch}}
	\end{align}
	\(\Rightarrow \ g\) ist nicht surjektiv \(\qed\)\\
	\(\Rightarrow \ g\) ist nicht bijektiv, da \(g\) nicht surjektiv ist.\\

\subsection{h(x):}
	Beh.: \(h\) ist injektiv.\\
	Annahme: \(h\) ist nicht injektiv,\\
	\(\exists\ x,y \in \mathbb{Z}:x\neq y:f(x)=f(y)\)
	\begin{align}
		f(x) &= f(y)\\
		x+5&=y+5\\
		x&=y \quad \text{\textbf{Widerspruch}}
	\end{align}
	\(\Rightarrow \ h\) injektiv \(\qed\)\\
	Beh.: \(h\) ist surjektiv.
	\begin{align}
		\text{sei } y\in\mathbb{Z}\\
		h(x)=y&=x+5\\
		\Leftrightarrow y-5&=x
	\end{align}
	\(\Rightarrow\ h\) ist surjektiv, da hier eine Subtraktion vorliegt. Hierdurch ist gezeigt, dass 
	jedes Element in der Bildmenge durch mindestens ein Element aus der Ursprungsmenge dargestellt 	
	werden kann \(\qed\)\\
	Da \(h\) sowohl injektiv, als auch surjektiv ist, ist \(h\) bijektiv \(\qed\)


\section{3.}
\subsection{a)}
	\(f: \mathbb{Z}\times\mathbb{Z}\rightarrow\mathbb{Z}\)\\
	\(f(n,m)=n-m\)
	\begin{flushleft}
		Beh.: \(f\) ist nicht injektiv.\\
		Annahme: \(f\) ist injektiv,\\
		\(\forall(n,m),(x,y)\in (\mathbb{Z},\mathbb{Z}): n\neq x,\ m\neq y: f(n,m)\neq f(x,y)\)
	\end{flushleft}	
	\begin{align}
		\text{sei }n&=3\\
		m&=4\\
		x&=2\\
		y&=3\\
		f(n,m)&=-1=f(x,y) \quad \text{\textbf{Widerspruch}}
	\end{align}
	\(\Rightarrow\ f\) ist nicht injektiv \(\qed\)\\ 
	\\
	\(f\) ist surjektiv, da jede Ganze Zahl sich durch die Subtraktion zweier ganzer Zahlen darstellen 
	lässt, somit ist die Surjektivität erfüllt.\\
	
\newpage
\subsection{b)}
	\(g: \mathbb{Z}\times\mathbb{Z}\rightarrow\mathbb{Z}\times\mathbb{Z}\)\\
	\(g(n,m)=(n+m,n-m)\)\\
	\begin{flushleft}
		Beh.: \(g\) ist injektiv\\
		Annahme: \(g\) ist nicht injektiv,\\
		\(\forall(n,m),(x,y)\in(\mathbb{Z},\mathbb{Z}):n\neq x,\ m\neq y :g(n,m)=g(x,y)\)
	\end{flushleft}
	\begin{align}
		f(n,m)&=(n+m, n-m)\\
		f(x,y)&=(x+y, x-y)\\
		f(n,m)&=f(x,y)
	\end{align}
	\begin{align}
		&I    &n+m=x+y\\
		&II   &n-m=x-y
	\end{align}
	\begin{center}
		\(I+II:\)	
	\end{center}
	\begin{align}
		2n&=2x\\
		\Leftrightarrow n&=x
	\end{align}
	\begin{center}
		\(I-II:\)
	\end{center}
	\begin{align}
		2m&=2y\\
		\Leftrightarrow m&=y
	\end{align}
	\begin{center}
		\textbf{Widerspruch}
	\end{center}
	\(\Rightarrow\ g\) ist injektiv.
	
	\begin{flushleft}
		Beh.:\(g\) ist nicht surjektiv.
		Annahme: \(g\) ist surjektiv.
	\end{flushleft}	
	\[f(n,m)=(2,9)\]
	I:\(=n+m\)\\
	II:	\(9=n-m\)\\	
	\\
	I+II:
	\begin{align}
		11&=2n\\
		\Leftrightarrow n&=\frac{11}{2} \notin \mathbb{Z} \text{\textbf{Widerspruch}}
	\end{align}
	\(\Rightarrow\ g\) ist nicht surjektiv \(\qed\)

\subsection{c)}	
	\(h:\mathbb{Z}\rightarrow\mathbb{Z}\times\mathbb{Z}\)\\
	\(h(n)=((n+1)^2,n^2+1)\)
	\begin{flushleft}
		Beh.: \(h\) ist injektiv.\\
		Annahme: \(h\) ist nicht injektiv,\\
		\(\forall n,x\in\mathbb{Z}:n\neq x: h(n)=h(x)\)
	\end{flushleft}
	\begin{align}
		h(n)&= (n+1)^2,n^2+1)\\
		h(x)&= (x+1)^2,x^2+1)\\
		h(x)&=h(n)
	\end{align}
	\begin{align}
		x^2+1&=n^2+1\\
		\Leftrightarrow x^2&=n^2
	\end{align}
	\begin{align}
		(x+1)^2&=(n+1)^2 \\
		\Leftrightarrow x^2+2x+1&=n^2+2n+1\\
		\Leftrightarrow x&=n \quad \text{\textbf{Widerspruch}}
	\end{align}
	\(\Rightarrow\ h\) ist injektiv \(\qed\)
	\begin{flushleft}
		Beh.: \(h\) ist nicht surjektiv.\\
		Annahme:\(h\) ist surjektiv. 
	\end{flushleft}
	\[\text{sei } h(n)=(1,3)\]:	
	\begin{align}
		3&=n^2+1 \\
		\Leftrightarrow 2&=n^2 \\
		\Leftrightarrow \sqrt{2}&=|n| \\
		\sqrt{2}\notin\mathbb{Z}\quad &\text{\textbf{Widerspruch}} 
	\end{align}
	\(\Rightarrow\ h\) ist nicht surjektiv, da jegliche \((x,3):x\in\mathbb{Z}\) nicht darstellbar sind.
	
	
	
\section{4.}
\subsection{a)}
	\begin{tabular}{c|c||c|c|c|c|c}
		$A$ & $B$ & $\overline{A}$ & $\overline{B}$ & $A\cap B$ & $\overline{A\cap B}$ & 
		$\overline{A} \cup \overline{B}$\\ \hline
		0 & 0 & 1 & 1 & 0 & 1 & 1 \\
		1 & 0 & 0 & 1 & 0 & 1 & 1 \\
		0 & 1 & 1 & 0 & 0 & 1 & 1 \\
		1 & 1 & 0 & 0 & 1 & 0 & 0
	\end{tabular}

\makebox[\linewidth]{
\begin{tikzpicture}
    \draw \firstcircle node[above] {$A$};
    \draw \secondcircle node [above] {$B$};    
    
    \begin{scope}
      	\clip \firstcircle;
      	\fill[gray] \secondcircle;
    \end{scope}  
    \draw (-2,-2) rectangle (4,2) node [text=black,above] {$H$}
    	(0,-2.5) circle (0.2) (0.5,-2.5) node [text=black, right] {$\overline{A\cap B}$} 
    	(0,-3)% circle (0.2) (0.5,-3) node [text=black, right] {$A\cap B$}
    	(0,-3.5) circle (0.25);
    \begin{scope}
    	\fill[gray] (0,-3) circle (0.2);
		\fill [white] (0,-3.5) circle (0.3);    
    \end{scope}	
\end{tikzpicture}
	
\begin{tikzpicture}
    \draw \firstcircle node[above] {$A$};
    \draw \secondcircle node [above] {$B$};    
    
    \begin{scope}
      	\clip \firstcircle;
      	\fill[gray] \secondcircle;
    \end{scope}  
    \draw (-2,-2) rectangle (4,2) node [text=black,above] {$H$};
	\begin{scope}
		\fill[orange, opacity=0.5](-2,-2) rectangle (4,2);
		\clip (-2,-2) rectangle (4,2);		
		\fill[red, opacity=0.5 ] \firstcircle;
      	\fill[gray, opacity=0.5 ] \secondcircle;
    \end{scope}    
    \draw (0,-2.5) circle (0.2) (0.25,-2.5) node [text=black, right] {$oder$} 
    	(-1.5,-2.5) circle (0.2) (-1.25,-2.5) node [text=black, right] {$oder$}
    	(1.5,-2.5) circle (0.2) (1.75,-2.5) node [text=black, right] {$:\overline{A}$}
    	(0,-3) circle (0.2) (0.25,-3) node [text=black, right] {$oder$}
    	(-1.5,-3) circle (0.2) (-1.25,-3) node [text=black, right] {$oder$}
    	(1.5,-3) circle (0.2) (1.75,-3) node [text=black, right] {$:\overline{B}$}
		(0,-3.5) circle (0.2) (0.25,-3.5) node [text=black, right] {$:\overline{A}\cup\overline{B}$}
    						  (-0.25,-3.5) node [text=black, left]{\text{alle außer:}};
    \begin{scope}
    	\fill[orange, opacity=0.5] (-1.5,-2.5) circle (0.2);
    	\fill[orange, opacity=0.5] (1.5,-2.5) circle (0.2);
    	\fill[orange, opacity=0.5] (-1.5,-3) circle (0.2);
    	\fill[orange, opacity=0.5] (1.5,-3) circle (0.2);
		
    	\fill[orange, opacity=0.5] (0,-2.5) circle (0.2);
    	\fill[red, opacity=0.5](1.5,-2.5) circle (0.2);
    	\fill[orange, opacity=0.5] (0,-3) circle (0.2);
    	\fill[red, opacity=0.5](-1.5,-3) circle (0.2);
    	\fill[red, opacity=0.5](1.5,-3) circle (0.2);    
		\fill[gray, opacity=0.5] (-1.5,-2.5) circle (0.2);
    	\fill[gray, opacity=0.5] (1.5,-2.5) circle (0.2);
		\fill[gray, opacity=0.5] (1.5,-3) circle (0.2);

		\fill[orange, opacity=0.5] (0,-3.5) circle (0.2);
		\fill[red, opacity=0.5] (0,-3.5) circle (0.2);
   		\fill[gray, opacity=0.5] (0,-3.5) circle (0.2);
    \end{scope}	
\end{tikzpicture}
}
	
\subsection{b)}
	\[M=\{a,b,c,d\}\]
	\begin{align}
		P(M)= \Big\{ &\emptyset , \{a\}, \{b\}, \{c\}, \{d\}, \{a,b\}, \{a,c\}, \\
			& \{a,d\}, \{b,c\}, \{b,d\}, \{c,d\}, \{a,b,c\}, \\
			& \{a,c,d\}, \{a,b,d\}, \{b,c,d\}, \{a,b,c,d\}\Big\}
	\end{align}

\subsection{c)}
	\[M=\{a\}\ P(M)=\Big\{\emptyset, \{a\} Big\}\]
	\begin{align}
		&\text{(i)\ \ \ falsch}\\
		&\text{(ii)\ \ falsch}\\
		&\text{(iii)\ richtig}\\
		&\text{(iv)\ \ falsch}\\
		&\text{(v)\ \ \ falsch}\\
		&\text{(vi)\ \ richtig}
	\end{align}
	
	
\end{document}