\documentclass[a4paper]{scrartcl}
\usepackage[ngerman]{babel}
\usepackage[utf8]{inputenc}
\usepackage[T1]{fontenc}
\usepackage{lmodern}
\usepackage{amssymb}
\usepackage{amsmath}
\usepackage{enumerate}
\usepackage{pgfplots}
\usepackage{scrpage2}\pagestyle{scrheadings}
\usepackage{tikz}
\usepackage{listings}
\usetikzlibrary{patterns}

\newcommand{\titleinfo}{Hausaufgaben zum 14. 12. 2011}
\title{\titleinfo}
\author{Arne Struck 6326505}
\date{\today}
\chead{\titleinfo}
\ohead{\today}
\setheadsepline{1pt}
\setcounter{secnumdepth}{0}
\lstset{language=Java}
\newcommand{\qed}{\ \square}
\newcommand{\com}{, }


\begin{document}
\maketitle
\notag
\section{1.}
	\subsection{a)}		
		\(G\) und \(G'\) sind nicht isomorph, da in dem Graphen \(G\) die Knoten 4. Grades nur mit 
		Knoten 3. Grades verbunden sind, dies in \(G'\) nicht der Fall ist. \\
		
	\subsection{b)}
		\begin{center}
			\begin{tabular}{ccc}	
				A & & B  \\
				\begin{tikzpicture}[every node/.style={fill, circle, inner sep=2pt}]		
					%Pentagon:
					\node[label=above:1] (1) at (0,3) {};	
					\node[label=right:2] (2) at (2,1.5) {};
					\node[label=below right:3] (3) at (1,-0.5) {};
					\node[label=below left:4] (4) at (-1,-0.5) {};
					\node[label=left:5] (5) at (-2,1.5) {};
					\draw (1) to (2) to (3) to (4) to (5) to (1);
		
					%Pentagramm:
					\node[label=above right:6] (6) at (0,2.1) {};
					\node[label=above:7] (7) at (1,1.5) {};
					\node[label=right:8] (8) at (0.7,0.3) {};
					\node[label=left:9] (9) at (-0.7,0.3) {};
					\node[label=above:10] (10) at (-1,1.5) {};			
					\draw (6) to (8) to (10) to (7) to (9) to (6);
					
					%Verbindung:
					\draw 
						(6) to (1)
						(7) to (2)
						(8) to (3)
						(9) to (4)
						(10) to (5);
				\end{tikzpicture}
				&\quad &			
				\begin{tikzpicture}[every node/.style={fill, circle, inner sep=2pt}]
					%Sextagon:
					\node[label=above:10] (10) at (-1,3) {};
					\node[label=above:7] (7) at (1,3) {};
					\node[label=right:2] (2) at (2,1.25) {};
					\node[label=below:3] (3) at (1,-0.5) {};
					\node[label=below:4] (4) at (-1,-0.5) {};
					\node[label=left:5] (5) at (-2,1.25) {};
					\draw (10) to (7) to (2) to (3) to (4) to (5) to (10);
					
					%Rest (T??):
					\node[label=below:1] (1) at (0,0.7) {};
					\node[label=left:9] (9) at (-0.7,1.6) {};
					\node[label=right:8] (8) at (0.7,1.6) {};
					\node[label=above:6] (6) at (0,1.4) {};
					
					\draw
						(5) to (1)
						(2) to (1)
						(4) to (9)
						(7) to (9)
						(3) to (8)
						(10) to (8)
						(1) to (6)
						(8) to (6)
						(9) to (6);
						
				\end{tikzpicture}
			\end{tabular} \\ 
			\begin{tabular}{cc}
				C& \\
				\begin{tikzpicture}[every node/.style={fill, circle, inner sep=2pt}]
					%Komisches Teil, zu dem ich kein Bock hab, seine nodes:
					\node[label=above right:6](6) at (0,1) {};
					\node[label=above right:8](8) at (2,2) {};
					\node[label=above left:9](9) at (-2,2) {};
					\node[label=below right:3] (3) at (2,0.75) {};					
					\node[label=below left:4] (4) at (-2,0.75) {};
					\node[label=above:7] (7) at (-1,3) {};
					\node[label=above:10] (10) at (1,3) {};
					\node[label=below:5] (5) at (-1,-0.25) {};
					\node[label=below:2] (2) at (1,-0.25) {};
					\node[label=below:1] (1) at (0,-0.5) {};

					
					\draw 
						(6) to (8)
						(6) to (9)
						(9) to (4)
						(8) to (3)
						(4) to (3)
						(9) to (7)
						(7) to (10)
						(10) to (8)
						(7) to (2)
						(10) to (5)
						(6) to (1)
						(5) to (1)
						(1) to(2)
						(4) to (5)
						(2) to (3);
						
				\end{tikzpicture}&
			\end{tabular}
		\end{center}
		Daraus folgen die Isomorphismen: \(f: A\rightarrow B\), \(g: B\rightarrow C\) und 
		\(h: A\rightarrow C\) \\ \\
		\[
		\begin{array}{c|c}
			x & f(x) \\ \hline
			1&1 \\
			2&2 \\
			3&3 \\
			4&4 \\
			5&5 \\
			6&6 \\
			7&7 \\
			8&8 \\
			9&9 \\
			10& 10
		\end{array}
		\quad
		\begin{array}{c|c}
			x & g(x) \\ \hline
			1&1 \\
			2&2 \\
			3&3 \\
			4&4 \\
			5&5 \\
			6&6 \\
			7&7 \\
			8&8 \\
			9&9 \\
			10& 10
		\end{array}
		\quad
		\begin{array}{c|c}
			x & h(x) \\ \hline
			1&1 \\
			2&2 \\
			3&3 \\
			4&4 \\
			5&5 \\
			6&6 \\
			7&7 \\
			8&8 \\
			9&9 \\
			10& 10
		\end{array}
		\]
		
	
\section{2.}
	\subsection{a)}
		Die Anzahl der Kanten bei einem vollständigen Graphen ist gleich der Anzahl der 
		Zweierkombinationen an Knoten.
		\begin{align}
			|E(G_n)| &=  \begin{pmatrix}
				n \\ 2
				\end{pmatrix} \\
			|E(G_{10})| &= \begin{pmatrix}
				10 \\ 2
				\end{pmatrix} 
			= 45 %sagt walpha
		\end{align}
		
	\subsection{b)}
		Die Anzahl an n-Kreisen bei einem vollständigen Graphen ist gleich der Anzahl der
		n-Kombinationen an Knoten.
		\begin{align}
			|K_n(G_k)|&= \begin{pmatrix}
				k \\ n
			\end{pmatrix} \quad | k > n \\
			 |K_3(G_{10})| &= \begin{pmatrix}
			 	10 \\ 3
			 \end{pmatrix} = 120
		\end{align}			
	
	\subsection{c)}
		\begin{align}
			|K_{4}(G_{10})|=\begin{pmatrix}
				10 \\ 4
			\end{pmatrix} = 210
		\end{align}
		

\section{3.}
	\subsection{a)}
		\begin{tabular}{ccc}
			n = 4& \quad & n = 6 \\ \\
			\begin{tikzpicture}	[every node/.style={fill, circle, inner sep=2pt}]
				%n=4
				%H_1:
				\node[label=above:$H_1$] (1) at (0,3) {};
				\node[] (2) at (0,2) {};
				\node[] (3) at (0,1) {};
				\node[] (4) at (0,0) {};
				\draw 
					(1) to (2) to (3) to (4) to (1) to[bend right = 30 pt] (3)
					(2) to[bend right = 30 pt] (4)
					(1) to[bend right = 30 pt] (4);
				%H_2:
				\node[label=above:$H_2$] (5) at (3,3) {};
				\node[] (6) at (3,2) {};
				\node[] (7) at (3,1) {};
				\node[] (8) at (3,0) {};
				\draw 
					(5) to (6) to (7) to (8) to (5) to[bend left = 30 pt] (7)
					(5) to[bend left = 30 pt] (8)
					(6) to[bend left = 30 pt] (8);
					
					
				%Verbinde den ganzen scheiß:
				\draw
					(1) to (5)
					(1) to (6)
					(1) to (7)
					(1) to (8)
					(2) to (5)
					(2) to (6)
					(2) to (7)
					(2) to (8)
					(3) to (5)
					(3) to (6)
					(3) to (7)
					(3) to (8)
					(4) to (5)
					(4) to (6)
					(4) to (7)
					(4) to (8);
				
			\end{tikzpicture}& \quad &
			\begin{tikzpicture}	[every node/.style={fill, circle, inner sep=2pt}]
				%n=6
				%H_1:
				\node[label=above:$H_1$] (1) at (0,2) {};
				\node[] (2) at (0,1) {};
				\node[] (3) at (0,0) {};
				\node[] (4) at (0,-1) {};
				\node[] (5) at (0,-2) {};
				\node[] (6) at (0,-3) {};
				\draw 
					(1) to (2) to (3) to (4) to (5) to (6)
					(1) to[bend right = 30 pt] (4)
					(2) to[bend right = 30 pt] (5)
					(3) to[bend right = 30 pt] (6)
					(1) to[bend right = 30 pt] (6);
					
				%H_2:
				\node[label=above:$H_2$] (7) at (3,2) {};
				\node[] (8) at (3,1) {};
				\node[] (9) at (3,0) {};
				\node[] (10) at (3,-1) {};
				\node[] (11) at (3,-2) {};
				\node[] (12) at (3,-3) {};
				\draw
					(7) to (8) to (9) to (10) to (11) to (12)
					(7) to[bend left = 50 pt] (9)
					(7) to[bend left = 50 pt] (10)
					(7) to[bend left = 50 pt] (11)
					(7) to[bend left = 50 pt] (12)
					
					(8) to[bend left = 50 pt] (10)
					(8) to[bend left = 50 pt] (11)
					(8) to[bend left = 50 pt] (12)
					
					(9) to[bend left = 50 pt] (11)
					(9) to[bend left = 50 pt] (12)
					
					(10) to[bend left = 50 pt] (12);

				%Verbinden:
				\draw
					(1) to (7)
					(1) to (8)
					(1) to (9)
					(1) to (10)
					(1) to (11)
					(1) to (12)
					(2) to (7)
					(2) to (8)
					(2) to (9)
					(2) to (10)
					(2) to (11)
					(2) to (12)
					(3) to (7)
					(3) to (8)
					(3) to (9)
					(3) to (10)
					(3) to (11)
					(3) to (12)
					(4) to (7)
					(4) to (8)
					(4) to (9)
					(4) to (10)
					(4) to (11)
					(4) to (12)
					(5) to (7)
					(5) to (8)
					(5) to (9)
					(5) to (10)
					(5) to (11)
					(5) to (12)
					(6) to (7)
					(6) to (8)
					(6) to (9)
					(6) to (10)
					(6) to (11)
					(6) to (12);	
			\end{tikzpicture}
		\end{tabular}
		
	\subsection{b)}
		\(H_1\) besitzt \(\frac{3}{2}n\) Kanten (jeder Knoten hat 3 Kanten, dies muss noch halbiert 
		werden, da eine Kante durch 2 Knoten definiert wird). \\
		Da \(H_2\) ein vollständiger Graph ist, kann die Formel aus 2 a) für die Anzahl der Kanten 
		angewandt werden:
		\[|E(G_n)|= \begin{pmatrix}n \\ 2 \end{pmatrix} = \frac{n!}{2(n-2)!} =
		\frac{n^{\underline{2}}}{2} = \frac{n(n-1)}{2}\] \\
		Das Zusammenfassen von \(H_1\) und \(H_2\) in \(H\) bringt ebenfalls noch \(n\cdot n\) Kanten 
		hinzu, da jeder Knoten aus \(H_1\) mit jedem aus \(H_2\) verbunden wird. \\
		Also folgt als Gesamtgleichung: 
		\[\frac{3}{2}n+\frac{n(n-1)}{2}+n^2\]
		\newpage
		\begin{flushleft}
			Nun muss nur noch umgeformt und zusammengefasst werden, um \(\frac{3}{2}n^2+n\) als 
			Gleichung für die Kantenzahl zu verifizieren:
		\end{flushleft}
		\begin{align}
			\frac{3}{2}n+\frac{n(n-1)}{2}+n^2 &= \frac{n^2-n}{2}+\frac{3n}{2}+\frac{2n^2}{2}\\
			&= \frac{3n^2+2n}{2} \\
			&= \frac{3}{2}n^2 + n
		\end{align}
		
	\subsection{c)}
		In den Abbildungen aus a) ist erkennbar, dass sich ein Hamiltonscher Kreis durch das 
		Durchlaufen sämtlicher Knoten von \(H_1\) oder \(H_2\) und danach den Wechsel zu dem 
		jeweiligen anderen Urgraphen, welcher auch durchlaufen wird und schlussendlich der 
		Rückwechsel zum Ausgangsknoten erreichen lässt. Hier eine Darstellung für einen Hamiltonschen 
		Kreis für n = 4: \\
		\begin{center}
			\begin{tikzpicture}	[every node/.style={fill, circle, inner sep=2pt}]
					\node[label=above:$H_1$] (1) at (0,3) {};
					\node[] (2) at (0,2) {};
					\node[] (3) at (0,1) {};
					\node[] (4) at (0,0) {};
								
					%H_2:
					\node[label=above:$H_2$] (5) at (3,3) {};
					\node[] (6) at (3,2) {};
					\node[] (7) at (3,1) {};
					\node[] (8) at (3,0) {};
									
					\draw 
						(1) to (4) to (8) to (5) to (1);
			\end{tikzpicture}
		\end{center}

		
	\subsection{d)}
		G besitzt keine Eulersche Linie, da der Grad der einzelnen Punkte immer ungerade bleibt, dies 
		widerspricht der Voraussetzung für eine Eulersche Linie.
		Dies resultiert aus dem Anfangsgrad von der Knoten von \(H_1\) (3), denn n muss 
		gerade sein, damit für jeden der der Knoten von \(H_1\)der Knotengrad 3 
		realisiert werden kann. Und eine gerade Zahl addiert mit einer ungeraden, ergibt (zumindest 
		für die Knoten von \(H_1\)) einen ungeraden Grad.
	\newpage
\section{4.}	
	\subsection{a)}
		\[M=\{a,b,c,d\}\]
		\begin{align}
			P(M)= \Big\{&\emptyset ,\{a\},\{b\},\{c\},\{d\},\\ &\{a,b\},\{a,c\},\{a,d\},\{b,c\},
			\{b,d\},\{c,d\},\\&\{a,b,c\},\{a,c,d\},\{a,b,d\},\{b,c,d\},\\ & \{a,b,c,d\}\Big\} 
		\end{align}
		Die Anzahl der Elemente beträgt \(2^4=16\).
				
	\subsection{b)}
		\begin{center}
			\begin{tikzpicture} [every node/.style={fill, circle, inner sep=2pt}]
				%Hassediagramm nodes:
				\node[label=above left: $\{a\com b\com c \com d\}$] (abcd) at (0,0){};
				
				\node[label=left: $\{a\com b\com c\}$] (abc) at (-3,-2){};
				\node[label=left: $\{a\com b\com d\}$] (abd) at (-1,-2){};
				\node[label=right: $\{a\com c\com d\}$] (acd) at (1,-2){};
				\node[label=right: $\{b\com c\com d\}$] (bcd) at (3,-2){};

				\node[label=left: $\{a\com b\}$] (ab) at (-5,-4){};
				\node[label=left: $\{a\com c\}$] (ac) at (-3,-4){};
				\node[label=left: $\{a\com d\}$] (ad) at (-1,-4){};
				\node[label=right: $\{b\com c\}$] (bc) at (1,-4){};
				\node[label=right: $\{b\com d\}$] (bd) at (3,-4){};				
				\node[label=right: $\{c\com d\}$] (cd) at (5,-4){};
				
				\node[label=left: $\{a\}$] (a) at (-3,-6){};
				\node[label=left: $\{b\}$] (b) at (-1,-6){};
				\node[label=right: $\{c\}$] (c) at (1,-6){};
				\node[label=right: $\{d\}$] (d) at (3,-6){};

				\node[label=below: $\emptyset$] (empty) at (0,-8){};
				
				
				%Verbinden:
				\draw
					(empty) to (a)
					(empty) to (b)
					(empty) to (c)
					(empty) to (d)
					
					(a) to (ab)
					(a) to (ac)
					(a) to (ad)					
					(b) to (ab)
					(b) to (bc)
					(b) to (bd)					
					(c) to (ac)
					(c) to (bc)
					(c) to (cd)					
					(d) to (ad)
					(d) to (bd)
					(d) to (cd)
					
					(ab) to (abc)
					(ab) to (abd)
					(ac) to (abc)
					(ac) to (acd)
					(ad) to (abd)
					(ad) to (acd)
					(bc) to (abc)
					(bc) to (bcd)
					(bd) to (abd)
					(bd) to (bcd)
					(cd) to (acd)
					(cd) to (bcd)
					
					(abc) to (abcd)
					(abd) to (abcd)
					(acd) to (abcd)
					(bcd) to (abcd);
			\end{tikzpicture}
		\end{center}
	\newpage
	\subsection{c)}
		Das Hassediagramm ist isomorph zu dem Hyperwürfel \(Q_4\). Dies kann man schon daran sehen, 
		dass beide 16 Knoten 4. Grades besitzen (trifft ansonsten auf keinen der Graphen zu).
		Ordnet man den Hyperwürfel, wie das Diagramm an, ergibt sich folgendes:

		\begin{center}
			\begin{tikzpicture} [every node/.style={fill, circle, inner sep=2pt}]
				%Hassediagramm nodes:
				\node[label=above left: 1111] (abcd) at (0,0){};
				
				\node[label=left: 1110] (abc) at (-3,-2){};
				\node[label=left: 1101] (abd) at (-1,-2){};
				\node[label=right: 1011] (acd) at (1,-2){};
				\node[label=right: 0111] (bcd) at (3,-2){};

				\node[label=left: 1100] (ab) at (-5,-4){};
				\node[label=left: 1010] (ac) at (-3,-4){};
				\node[label=left: 1001] (ad) at (-1,-4){};
				\node[label=right: 0110] (bc) at (1,-4){};
				\node[label=right: 0101] (bd) at (3,-4){};				
				\node[label=right: 0011] (cd) at (5,-4){};
				
				\node[label=left: 1000] (a) at (-3,-6){};
				\node[label=left: 0100] (b) at (-1,-6){};
				\node[label=right: 0010] (c) at (1,-6){};
				\node[label=right: 0001] (d) at (3,-6){};

				\node[label=below: 0000] (empty) at (0,-8){};
				
				
				%Verbinden:
				\draw
					(empty) to (a)
					(empty) to (b)
					(empty) to (c)
					(empty) to (d)
					
					(a) to (ab)
					(a) to (ac)
					(a) to (ad)					
					(b) to (ab)
					(b) to (bc)
					(b) to (bd)					
					(c) to (ac)
					(c) to (bc)
					(c) to (cd)					
					(d) to (ad)
					(d) to (bd)
					(d) to (cd)
					
					(ab) to (abc)
					(ab) to (abd)
					(ac) to (abc)
					(ac) to (acd)
					(ad) to (abd)
					(ad) to (acd)
					(bc) to (abc)
					(bc) to (bcd)
					(bd) to (abd)
					(bd) to (bcd)
					(cd) to (acd)
					(cd) to (bcd)
					
					(abc) to (abcd)
					(abd) to (abcd)
					(acd) to (abcd)
					(bcd) to (abcd);
			\end{tikzpicture}
		\end{center}		
		Dieser Graph folgt dem Prinzip, dass wenn ein Element der Urmenge ein Element der
		entsprechenden Teilmenge der Potenzmenge ist, dann ist eine 1 gesetzt. \(a\) entspricht der 
		Tausender-, b der Hunderter-, c der Zehnerstelle...\\
		Dies entspricht folgender Tabelle (Isomorphismus): \\ \\
		\[
		\begin{tabular}{l|c}
			\(\subseteq P(M)\)&\(Q_4\) \\ \hline
			\(\emptyset\) & 0000\\
			\{a\} & 1000\\
			\{b\} & 0100\\
			\{c\} & 0010\\
			\{d\} & 0001\\
			\{a,b\} & 1100\\
			\{a,c\} & 1010\\
			\{a,d\} & 1001\\
		\end{tabular} \quad
		\begin{tabular}{l|c}
			\(\subseteq P(M)\)&\(Q_4\) \\ \hline
			\{b,c\} & 0110\\
			\{b,d\} & 0101\\
			\{c,d\} & 0011\\
			\{a,b,c\} & 1110\\
			\{a,b,d\} & 1101\\
			\{a,c,d\} & 1011\\
			\{b,c,d\} & 0111\\
			\{a,b,c,d\} & 1111\\
		\end{tabular}
		\]
\end{document}