\documentclass[a4paper]{scrartcl}
\usepackage[ngerman]{babel}
\usepackage[utf8]{inputenc}
\usepackage[T1]{fontenc}
\usepackage{lmodern}
\usepackage{amssymb}
\usepackage{amsmath}
\usepackage{enumerate}
\usepackage{pgfplots}
\usepackage{scrpage2}\pagestyle{scrheadings}
\usepackage{tikz}
\usetikzlibrary{patterns}

\newcommand{\titleinfo}{Hausaufgaben zum 30. 11. 2012}
\title{\titleinfo}
\author{Arne Struck 6326505}
\date{28. November 2012}
\chead{\titleinfo}
\ohead{28. November 2012}
\setheadsepline{1pt}
\setcounter{secnumdepth}{0}
\newcommand{\qed}{\ \square}

\begin{document}
\maketitle
\notag
\section{1.}
	\subsection{a)}
		BA, AC und AA existieren nicht. \\ \\		
		\(
		\begin{array}{lllll}
			\text{AB}&=\begin{pmatrix}
				7&5&-2 \\2&1&-1 \\ 30&17&5 \\ 3&2&5 
			\end{pmatrix} \quad
			\text{AD}&=\begin{pmatrix}
				2 \\4 \\26 \\ -4
			\end{pmatrix}\quad
			\text{BB}&=\begin{pmatrix}
				10&5&1 \\ 5&4&-1 \\ 4&2&1
			\end{pmatrix}\quad
			\text{CD}&=\begin{pmatrix}
				12
			\end{pmatrix} \\  \\ \\
			\text{DC}&=\begin{pmatrix}
				2&4&-4 \\ 3&6&-6 \\ -2&-4&4
			\end{pmatrix} \\
		\end{array} \\ \\
		\)
		
		
	\subsection{b)}
		\(\text{(AB)}_{i=3,\ j=2}=15\) \\ \\
		\(\text{(AB)}_{i=4,\ j=1\dots 4}=\begin{pmatrix}
			...&13 \\
			...&8 \\
			...&3 \\
			...&23
		\end{pmatrix}\)
	
	
\section{2.}
	\subsection{a)}
		\begin{align}
			\text{A}(\text{B}_1+\text{B}_2)&=\begin{pmatrix}
				5&7 \\ 9&-1 \\ 8&2
			\end{pmatrix}\Bigg(\begin{pmatrix}
				1&2 \\ 3&6
			\end{pmatrix}+\begin{pmatrix}
				1&-2 \\ 3&2
			\end{pmatrix}\Bigg) \\
			&=\begin{pmatrix}
				5&7 \\ 9&-1 \\ 8&2
			\end{pmatrix}\begin{pmatrix}
				2&0 \\ 6&8
			\end{pmatrix}\\
			&= \begin{pmatrix}
				52&56 \\ 12&-8 \\ 28&16
			\end{pmatrix}
		\end{align}
		\begin{align}
			\text{AB}_1 &= \begin{pmatrix}
				5&7 \\ 9&-1 \\ 8&2
			\end{pmatrix}\begin{pmatrix}
				1&2 \\ 3&6
			\end{pmatrix} 
			=\begin{pmatrix}
				26&52 \\ 6&12 \\ 14&28
			\end{pmatrix} \\
			\text{AB}_2 &= \begin{pmatrix}
				5&7 \\ 9&-1 \\ 8&2
			\end{pmatrix} 
			\begin{pmatrix}
				1&-2 \\ 3&2
			\end{pmatrix}
			=\begin{pmatrix}
				26&4 \\ 6&-20 \\ 14&-12
			\end{pmatrix} \\
		\end{align}
		\[\text{AB}_1+\text{AB}_2=\begin{pmatrix}
			52&56 \\ 12&-8 \\ 24&16
		\end{pmatrix}=\text{A}(\text{B}_1+\text{B}_2)\]


	\subsection{b)}
		\begin{align}
			(\text{AB})^\text{T}&=\begin{pmatrix}
				11&5&17 \\
				22&10&34
			\end{pmatrix}^\text{T}
			=\begin{pmatrix}
				11&22\\5&10\\17&34
			\end{pmatrix}\\
			\text{B}^\text{T}\text{A}^\text{T}&=\begin{pmatrix}
				2&3\\-1&2\\5&4
			\end{pmatrix}\begin{pmatrix}
				1&2\\3&6
			\end{pmatrix}=\begin{pmatrix}
				11&22\\5&10\\17&34
			\end{pmatrix}
		\end{align}
		
	\subsection{c)}
		\(\text{A}^\text{T}\text{B}^\text{T}\) ist nicht definiert, da hier eine \(2\times 2\)-Matrix mit 
		einer \(3\times 2\)-Matrix multipliziert wird.
		
		
\section{3.}
	Beh.: Das Distributivgesetz gilt für Matrizen. \\
	Bew.: \\
	\begin{align}
		\text{Sei \ \,}A &= (a_{ij}) \quad |\ i= 1...m,\ j=1...n \\
		\text{Sei }B_1 &= (b_{jk}) \quad |\ j=1...n,\ k=1...p \\
		\text{Sei }B_2 &= (c_{jk}) \quad |\ j=1...n,\ k=1...p
	\end{align}
	\begin{align}
		B_1+B_2 &= (b_{jk}+c_{jk}) \\
		A\cdot(B_1+B_2) &= \Big(a_{ij}(b_{jk}+c_{jk})\Big)
	\end{align}
	\begin{align}
		A\cdot B_1 &= (a_{ij}\cdot b_{jk}) \\
		A\cdot B_2 &= (a_{ij}\cdot c_{jk}) \\
		A\cdot B_1+A\cdot B_2 &= (a_{ij}\cdot b_{jk} + a_{ij}\cdot c_{jk}) \\
		&\overset{*}{=} \Big(a_{ij}(b_{jk}+c_{jk})\Big) = A\cdot(B_1+B_2) \qed
	\end{align}
	\begin{small}
		 * Hier ist das Distribution erlaubt, da es sich um zwei normale Zahlen handelt und das 
		 Distributivgesetz für diese definiert ist.
	\end{small}	
	
\section{4.}	
	\subsection{a)}
		Beh.: für \(f:A\rightarrow B \land B'\subseteq B \text{ gilt } f(f^{-1}(B'))\subseteq B'\)\\
		Bew.: \\
		Sei \(b\in f(f^{-1}(B'))\) \\
		\(\Rightarrow \exists\ a\in f^{-1}(B')\ |\ f(a)=b\) Da das Bild ein Urbild braucht. \\ \\
		Da \(a\in f^{-1}(B')\), muss \(f(a)\) auf ein Element in \(B'\) verweisen. Dieses Element 
		ist \(b\), weil \(f(a)=b\) gilt. \\
		Die Behauptung gilt also unter den gegebenen Bedingungen, da das Element \(b\) beliebig aus 
		\(f(f^{-1}(B'))\) gewählt werden kann. \(\qed\)

	\subsection{b)}
		\(f:A\rightarrow B\quad\quad A=\{1\}\ B=\{v\}=B'\) \\
		\(f(1)\) sei nicht definiert \\
		\(f(f^{-1}(B'))=\{\}\) \\
		
	
\end{document}