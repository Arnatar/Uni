\documentclass[a4paper]{scrartcl}
\usepackage[ngerman]{babel}
\usepackage[utf8]{inputenc}
\usepackage[T1]{fontenc}
\usepackage{lmodern}
\usepackage{amssymb}
\usepackage{amsmath}
\usepackage{enumerate}
\usepackage{pgfplots}
\usepackage{scrpage2}\pagestyle{scrheadings}
\usepackage{tikz}
\usetikzlibrary{patterns}

\newcommand{\titleinfo}{Hausaufgaben zum 2. 11. 2012}
\title{\titleinfo}
\author{Arne Struck 6326505, René}
\date{\today}
\chead{\titleinfo}
\ohead{\today}
\setheadsepline{1pt}
\setcounter{secnumdepth}{0}
\setlength{\textheight}{22cm}
\newcommand{\qed}{\ \square}

\begin{document}
\maketitle
\notag
\section{1.}
	\subsection{a)}
		\begin{align}
			\sum^{n}_{i=1}\frac{1}{i\cdot(i+1)}
		\end{align}
		
	\subsection{b)}
		\begin{align}
			A(1)&=\sum^{1}_{i=1}\frac{1}{i\cdot(i+1)}=\frac{1}{2}\\
			A(1)&=1-\frac{1}{1+1}=\frac{1}{2}\\
			A(2)&=\sum^{2}_{i=1}\frac{1}{i\cdot(i+1)}=\frac{1}{2}+\frac{1}{2\cdot 3}=\frac{2}{3}\\
			A(2)&=1-\frac{1}{2+1}=\frac{2}{3}\\
			A(3)&=\sum^{3}_{i=1}\frac{1}{i\cdot(i+1)}=\frac{2}{3}+\frac{1}{3\cdot 4}=\frac{3}{4}\\
			A(3)&=1-\frac{1}{3+1}=\frac{3}{4}\\
			A(4)&=\sum^{4}_{i=1}\frac{1}{i\cdot(i+1)}=\frac{3}{4}+\frac{1}{4\cdot 5}=\frac{4}{5}\\
			A(4)&=1-\frac{1}{4+1}=\frac{4}{5}
		\end{align}
	
	\newpage
	\subsection{c)}
		\(		
		\begin{array}{llcl}
			\underline{\text{Beh.: }}& \\
				&\sum\limits_{i=1}^{n}\frac{1}{i\cdot(i+1)}&=&1-\frac{1}{n+1}\\
		\end{array}
		\)\\ \\ \\
		\(
		\begin{array}{llclcl}
			\underline{\text{Induktionsanfang:}}& \\
				&A(1)&=&\sum\limits_{i=1}^{n}\frac{1}{i\cdot(i+1)}&=&\frac{1}{2}\\
				&A(1)&=&1-\frac{1}{1+1}&=&\frac{1}{2}\\
		\end{array}
		\)\\ \\ \\
		\underline{Induktionsannahme (IA):}\\
		Die Beh. gilt für ein beliebiges, aber festes \(n\in\mathbb{N}\).\\ \\ \\
		\underline{\text{Induktionsschritt:}}\ \(n\rightarrow n+1\) \\
		\(
		\begin{array}{llclcl}
			\text{zu zeigen:} & \sum\limits_{i=1}^{n+1}\frac{1}{i\cdot(i+1)}&=&1-\frac{1}{n+1+1}\\
			& \sum\limits_{i=1}^{n+1}\frac{1}{i(i+1)}&=&\sum\limits_{i=1}^{n}\frac{1}
				{i(i+1)}+\frac{1}
				{(n+1)(n+1+1)}\\
			& &\overset{IA}{=}&1-\frac{1}{n+1}+\frac{1}{(n+1)(n+2)}\\
			& &=&1-\frac{n+2}{(n+1)(n+2)}+\frac{1}{(n+1)(n+2)}\\
			& &=&1-\frac{n+1}{(n+1)(n+2)}\\
			& &=&1-\frac{1}{n+2}\\
		\end{array}
		\)\\ \\ 
		\(
		\begin{array}{lcl}
			&\text{Damit ist gezeigt, dass die Behauptung gilt}\qed \\
		\end{array}
		\)
				
		

\section{2.}
	\subsection{a)}
		\(
		\begin{array}{lclcl}
			B(1)&=&1^2&=&1 \\
			B(1)&=&2\cdot 1-1&=&1 \\
			B(2)&=&2^2&=&4 \\
			B(2)&=&1+2\cdot 2-1&=&4 \\
			B(3)&=&3^2&=&9 \\
			B(3)&=&4+2\cdot 3-1&=&9 \\
			B(4)&=&4^2&=&16 \\
			B(4)&=&9+2\cdot 4-1&=&16 \\
		\end{array}
		\)		
		
	\subsection{b)}
		\[B(n)=(2\cdot 1-1)+(2\cdot 2-1)+(2\cdot 3-1)+... +(2\cdot n-1)\] \\
		\(B(n)\) ist die \(n\)-te Quadratzahl oder auch die Quadratzahl mit der Basis \(n\).
	\subsection{c)}
		\(		
		\begin{array}{llcl}
			\underline{\text{Beh.: }}& \\
				&\sum\limits_{i=1}^{n}(2i-1)&=&n^2\\
		\end{array}
		\)\\ \\ \\
		\(
		\begin{array}{llclcl}
			\underline{\text{Induktionsanfang:}}& \\
				&B(1)&=&1^2&=&1 \\
				&B(1)&=&2\cdot 1-1&=&1 \\

		\end{array}
		\)\\ \\ \\
		\underline{Induktionsannahme (IA):}\\
		Die Beh. gilt für ein beliebiges, aber festes \(n\in\mathbb{N}\).\\ \\ \\
		\underline{\text{Induktionsschritt:}}\ \(n\rightarrow n+1\) \\
		\(
		\begin{array}{llclcl}
			\text{zu zeigen:} & \sum\limits_{i=1}^{n+1}(2i-1)&=&(n+1)^2 \\ \\
			& \sum\limits_{i=1}^{n+1}(2i-1)&=&\sum\limits_{i=1}^{n}(2i-1)+(2(n+1)-1) \\
			& \sum\limits_{i=1}^{n+1}(2i-1)&=&\sum\limits_{i=1}^{n}(2i-1)+2n+1 \\
			& &\overset{IA}{=}&n^2+2n+1 \\
			& &=& (n+1)^2 \\
		\end{array}
		\)\\ \\ 
		\(
		\begin{array}{lcl}
			&\text{Damit ist gezeigt, dass die Behauptung gilt}\qed \\
		\end{array}
		\)


		
\section{3.}
	\subsection{a)}
		\(		
		\begin{array}{llcl}
			\underline{\text{Beh.: }}& \\
				&\forall n\geq 7:n\in\mathbb{N}\text{ gilt:} \\
				&13n<2^n \\
		\end{array}
		\)\\ \\ \\
		\(
		\begin{array}{llrclcl}
			\underline{\text{Induktionsanfang:}}& \\
				&&n&=&7 \\
				&&13\cdot 7&<&2^7 \\
				&\Leftrightarrow &91&<&128 \\

		\end{array}
		\)\\ \\ \\
		\underline{Induktionsannahme (IA):}\\
		Die Beh. gilt für ein beliebiges, aber festes \(n\in\mathbb{N}\).\\ \\ \\
		\newpage
		\begin{flushleft}
			\underline{\text{Induktionsschritt:}}\ \(n\rightarrow n+1\) \\
		\end{flushleft}
		\(
		\begin{array}{lrclclcl}
			\text{zu zeigen:} & 13\cdot (n+1)&<&2^{n+1} \\
			& 13n+13 &\overset{IA}{<}& 2^n+13 &\overset{\text{mit } n\geq 7}{<}& 2^n+2^n &=& 2^{n+1} \\
		\end{array}
		\)\\ \\ 
		\(
		\begin{array}{lcl}
			&\text{Damit ist gezeigt, dass die Behauptung gilt}\qed \\
		\end{array}
		\)
		
	\subsection{b)}
		\(		
		\begin{array}{lrcl}
			\underline{\text{Beh.: }}& \\
				&\forall n\geq 5:n\in\mathbb{N}\text{ gilt:} \\
				&n^2<2^n \\
		\end{array}
		\)\\ \\ \\
		\(
		\begin{array}{llrclcl}
			\underline{\text{Induktionsanfang:}}& \\
				&&n&=&5 \\
				&&5^2&<&2^5 \\
				&\Leftrightarrow &25&<&32 \\
		\end{array}
		\)\\ \\ \\
		\underline{Induktionsannahme (IA):}\\
		Die Beh. gilt für ein beliebiges, aber festes \(n\in\mathbb{N}\).\\ 
		\begin{flushleft}
			\underline{\text{Induktionsschritt:}}\ \(n\rightarrow n+1\) \\
		\end{flushleft}
		\(
		\begin{array}{llrclclcl}
			\text{zu zeigen:} && (n+1)^2&<&2^{n+1} \\ \\
		\end{array}			
		\)\\ \(
		\begin{array}{lrclclcl}
			& (n+1)^2&=& n^2+2n+1\\
			& n^2+2n+1&<&2\cdot 2^n\\
			\overset{IA}{\Rightarrow} & 2n+1&\overset{\text{mit }n\geq 5}{<}&2^n \\
			\Rightarrow & (n+1)^2&<&2^{n+1}
			\end{array}			
		\)\\ \\ 
		\(
		\begin{array}{lcl}
			&\text{Damit ist gezeigt, dass die Behauptung gilt}\qed \\
		\end{array}
		\)
	
\newpage
\section{4.}
	\(		
	\begin{array}{llcl}
		\underline{\text{Beh.: }}& \\
			&2^n<n!:\forall n\in\mathbb{N}:n\geq 4 \\
	\end{array}
	\)\\ \\ \\
	\(
	\begin{array}{llrclcl}
		\underline{\text{Induktionsanfang:}}& \\
			&&n&=&4 \\
			&&2^4&<&4! \\
			&\Leftrightarrow &16&<&24 \\
	\end{array}
	\)\\ \\ \\
	\underline{Induktionsannahme (IA):}\\
	Die Beh. gilt für ein beliebiges, aber festes \(n\in\mathbb{N}\).\\ 
	\begin{flushleft}
		\underline{\text{Induktionsschritt:}}\ \(n\rightarrow n+1\) \\
	\end{flushleft}
	\(
	\begin{array}{llrclclcl}
		\text{zu zeigen:} && 2^{n+1}<(n+1)!
	\end{array}\\ 
	\)\\ \(
	\begin{array}{lrclclcl}
		&(n+1)!&=&n!\cdot (n+1) \\
		& n!\cdot (n+1) &\overset{IA}{>}&2^n\cdot(n+1) \\
		&(n+1)\cdot 2^n &\overset{\text{mit }n\geq 4}{>}&2\cdot 2^n=2^{n+1} \\
		\Rightarrow & (n+1)!&>&2^{n+1}
		\end{array}			
	\)\\ \\ 
	\(
	\begin{array}{lcl}
		&\text{Damit ist gezeigt, dass die Behauptung gilt}\qed \\
	\end{array}
	\)


\end{document}