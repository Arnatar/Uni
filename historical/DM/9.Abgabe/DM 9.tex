\documentclass[a4paper]{scrartcl}
\usepackage[ngerman]{babel}
\usepackage[utf8]{inputenc}
\usepackage[T1]{fontenc}
\usepackage{lmodern}
\usepackage{amssymb}
\usepackage{amsmath}
\usepackage{enumerate}
\usepackage{pgfplots}
\usepackage{scrpage2}\pagestyle{scrheadings}
\usepackage{tikz}
\usepackage{listings}
\usetikzlibrary{patterns}

\newcommand{\titleinfo}{Hausaufgaben zum 21. 12. 2012}
\title{\titleinfo}
\author{Arne Struck 6326505}
\date{\today}
\chead{\titleinfo}
\ohead{\today}
\setheadsepline{1pt}
\setcounter{secnumdepth}{0}
\lstset{language=Java}
\newcommand{\qed}{\ \square}

\begin{document}
\maketitle
\notag
\section{1.}
	\subsection{a)}		
		Die Ordnung von \(M\) wird ist äquivalent zu \(|M|\). Da \(a\) und \(b\) sämtliche Werte in 
		\(\mathbb{Z}_7\) annehmen können, wobei \(a\neq 0\) gilt, muss \(|M|\) sämtliche 
		Kombinationen von \(a\) und \(b\) enthalten. Also ist \(|M| = 5\cdot 6= 30\). 
		
	\subsection{b)}
		Zur Bestimmung von \(A'\) gilt: \(AA'=\begin{pmatrix}1&0\\0&1\end{pmatrix}\) wir die Gauss-
		Jordan-Formel angewandt: \\
		\begin{align}
			\left(\begin{array}{cc|cc}
			2&1 & 1& 0 \\
			0&1 & 0& 1
			\end{array}\right)&=
			\left(\begin{array}{cc|cc}
			1&\frac{1}{2}&\frac{1}{2}&0 \\
			0&1&0&1
			\end{array}\right) \\
			&= \left(\begin{array}{cc|cc}
			1&0&\frac{1}{2}&-\frac{1}{2} \\
			0&1&0&1
			\end{array}\right)
		\end{align}
		Nun kann man das Inverse von \(A\) direkt aus der Formel ablesen, allerdings sind Einträge 
		von \(A'\notin\mathbb{Z}\), also ist \(\left(\begin{array}{cc} \frac{1}{2}&-\frac{1}{2}\\
		0&1\end{array}\right)\) zwar das Inverse von \(A\), allerdings nicht in \(\mathbb{Z}_7\).
		
	\subsection{c)}
		Die Berechneten Werte folgen aus dem im Skript vorgestellten Prinzip: \(Z^n = ZZ^{n-1}\).
		\begin{align}
			B&= \begin{pmatrix}
				4&1\\0&1
			\end{pmatrix} \\
			B^2&=\begin{pmatrix}
				2&5\\0&1
			\end{pmatrix} \\
			B^3&=\begin{pmatrix}
				8&21\\0&1
			\end{pmatrix}=\begin{pmatrix}
				1&0\\0&1
			\end{pmatrix} \Rightarrow \text{B hat die Ordnung 3.}
		\end{align}
		\begin{align}
			C^1 &=\begin{pmatrix}
				3&3 \\ 0&1
			\end{pmatrix} \\
			C^2 &=\begin{pmatrix}
				2&5\\ 0&1
			\end{pmatrix} \\
			C^3 &=\begin{pmatrix}
				6&4\\ 0&1
			\end{pmatrix} \\
			C^4 &=\begin{pmatrix}
				4&1\\ 0&1
			\end{pmatrix} \\
			C^5 &=\begin{pmatrix}
				5&6\\ 0&1
			\end{pmatrix} \\
			C^6 &=\begin{pmatrix}
				15&21\\ 0&1
			\end{pmatrix} =\begin{pmatrix}
				1&0\\0&1
			\end{pmatrix} \Rightarrow \text{C ist 6. Ordnung.}
		\end{align}
		\begin{align}
			D^1&=\begin{pmatrix}
				1&4\\ 0&1
			\end{pmatrix} \\
			D^2&=\begin{pmatrix}
				1&1\\ 0&1
			\end{pmatrix} \\
			D^3&=\begin{pmatrix}
				1&5\\ 0&1
			\end{pmatrix} \\
			D^4&=\begin{pmatrix}
				1&2\\ 0&1
			\end{pmatrix} \\
			D^5&=\begin{pmatrix}
				1&6\\ 0&1
			\end{pmatrix} \\
			D^6&=\begin{pmatrix}
				1&3\\ 0&1
			\end{pmatrix} \\
			D^7&=\begin{pmatrix}
				1&7\\ 0&1
			\end{pmatrix} =\begin{pmatrix}
				1&0 \\ 0&1
			\end{pmatrix} \Rightarrow \text{D ist 7. Ordnung.} 
		\end{align}
	\newpage
	\subsection{d)}
		Gesucht ist ein Element für das gilt: \(G^2=GG=\begin{pmatrix}1&0\\0&1\end{pmatrix}\). Wenn
		man aus \(GG\) ein LGS für die Einzelnen Elemente erstellt, dann erhält man folgendes:
		\[\begin{pmatrix}
			a&b\\0&1
		\end{pmatrix}
		\begin{pmatrix}
			a&b\\0&1
		\end{pmatrix}=\begin{pmatrix}1&0\\0&1\end{pmatrix}\]
		\[
		\begin{array}{lrclcl}
			I:&a^2+b\cdot 0 &=& 1 &\Leftrightarrow & a^2 = 1 \\
			II:&ab+b &=& 0 &\Leftrightarrow & b=0\\
			III:& 0\cdot a+0\cdot 1 &=& 0 \\
			IV:& 0b+1\cdot 1 &=&1		
		\end{array}
		\]
		Aus \(I\) ist ersichtlich, dass \(a^2=1\) und \(b=0\) gelten muss, um ein Element der Ordnung 
		2 zu finden. Dies erfüllen 2 Matrizen, die Einheitsmatrix (welche aber die Ordnung 1 besitzt) 
		und \(\begin{pmatrix}6&0\\0&1\end{pmatrix}\), da hier \(a=-1=6\Leftrightarrow a^2 = 1\) gilt.
		Also ist die gesuchte Matrix (die einzige der 2. Ordnung)
		\(\begin{pmatrix}6&0\\0&1\end{pmatrix}\).
			
\section{2.}
	\subsection{a)}
		Die Quadrate sind hier nur durch ihre Eckpunkte angedeutet:
		\begin{align}
			s*y&=\begin{matrix}
				C&B \\
				D&A
			\end{matrix} \\
			x*r&=\begin{matrix}
				C&B\\
				D&A
			\end{matrix} \\
			x*y&=\begin{matrix}
				B&C \\
				A&D
			\end{matrix}
		\end{align}
		\begin{tabular}{c|c|c|c|c|c|c|c|c}
				Element&i&r&s&t&w&x&y&z \\ \hline
				Inverses&i&t&s&r&w&x&y&z
		\end{tabular}
		\newpage
	\subsection{b)}
		\underline{Ordnung der einzelnen Elemente:} \\
		\(i\) ist das neutrale Element, hat also die Ordnung 1. \\
		\(r\) hat die Ordnung 4, da es sich um die \(90^{\circ}\) Drehung im Uhrzeigersinn handelt 
		und \(90\cdot 4\) eine volle Drehung ergibt. \\
		\(t\) hat ebenfalls die Ordnung 4, es handelt sich ebenfalls um eine \(90^{\circ}\) Drehung, 
		allerdings gegen den Uhrzeigersinn. \\
		Da die restlichen Elemente zu sich selbst invers sind, haben sie alle die Ordnung 2 inne. 
		\\\\
		\underline{Zyklisch? :} \\
		\(G\) kann nicht zyklisch sein, da (wie man aus dem vorherigen Teil entnehmen kann) kein 
		mögliches Erzeugerelement (Ein Element der Ordnung 8) existiert. \\ \\
		\underline{Kommutativ?:} \\ \\
		\begin{center}
			\begin{tabular}{c||c|c|c|c|c|c|c|c}
				 &i&r&s&t&w&x&y&z \\ \hline\hline
				i&i&r&s&t&w&x&y&z \\ \hline
				r&r&s&t&i&z&y&w&x \\ \hline
				s&s&t&i&r&x&w&z&y \\ \hline
				t&t&i&r&s&y&z&x&w \\ \hline
				w&w&z&x&y&i&s&r&t \\ \hline
				x&x&y&w&z&s&i&t&r \\ \hline
				y&y&w&z&x&r&t&i&s \\ \hline
				z&z&x&y&w&t&r&s&i \\ 
			\end{tabular} 
		\end{center}
		Wie man eindeutig an der Gruppentafel erkennen kann, ist die Gruppe kommutativ.
	\subsection{c)}
		\(H_1=\{i,r,s,t\}\): \\
		\(H_1\) ist zyklisch, da es das Element \(r\) der Ordnung 4 enthält, für welches gilt: \\
		\begin{align}
			r^0&= i \\
			r^1&= r \\
			r^2&= s \\
			r^3&= t \\
			r^4&= r^0 \\
		\end{align}
		\begin{center}
			\begin{tabular}{c||c|c|c|c}
				\(H_1\)&i&r&s&t \\ \hline\hline
				i &i&r&s&t\\ \hline
				r &r&s&t&i\\ \hline	
				s &s&t&i&r\\ \hline
				t &t&i&r&s\\ 
			\end{tabular} 
		\end{center}
		Wie man eindeutig erkennen kann ist \(H_1\) nicht isomorph zur Rechtecksgruppe 
		(beispielsweise sind die Hauptdiagonalen unterschiedlich). \\ \\
		\(H_2=\{i,w,x,s\}\):\\
		\(H_2\) ist nicht zyklisch, da ein mögliches Erzeugerelement fehlt.
		\begin{center}
			\begin{tabular}{c||c|c|c|c}
				\(H_2\)&i&w&x&s \\ \hline\hline
				i &i&w&x&s \\ \hline
				w &w&i&s&x \\ \hline	
				x &x&s&i&w  \\ \hline
				s &s&x&w&i \\ 
			\end{tabular} 
		\end{center}
		Diese Untergruppe ist isomorph zur Rechtecksgruppe. Der entsprechende Isomorphismus wäre:
		\begin{tabular}{c||c|c|c|c}
			Rechtecksgruppe &i&r&x&y\\ \hline
			\(H_2\) & i	& w & x & s
		\end{tabular}
	
\section{3.}
	\subsection{a)}
		\begin{align}
			a^{-1}(bd^{-1})^{-1}bc(b^{-1}cdc)^{-1}ab^{-1}&= 
			a^{-1}db^{-1} bcc^{-1}d^{-1}c^{-1}bab^{-1} \\
			&=a^{-1}db^{-1} bd^{-1}c^{-1}bab^{-1} \\
			&=a^{-1}dd^{-1}c^{-1}bab^{-1} \\
			&=a^{-1}c^{-1}bab^{-1} \\
			&=c^{-1}a^{-1}bb^{-1}a \\
			&=c^{-1}a^{-1}a \\
			&=c^{-1} \\
		\end{align}
		
	\subsection{b)}
		In zyklischen Gruppen lässt sich jedes Element durch den Erzeuger darstellen. Um 
		Kommutativität zu zeigen muss bei der beliebigen Gruppe \(G\) gezeigt werden, dass für 
		\(a,b\in G\) \(ab=ba\) gilt. \\
		\(r\) sei nun der Erzeuger von \(G\), \(a\) das \(n\)-te und \(b\) das \(m\)-te Element von 
		\(G\), wobei die Zählung mit dem 0. Element (Identität/neutrales Element) beginnt und mit dem 
		\(|G|-1\)-tem Element endet. \\
		Daraus folgt, dass \(a=r^n\) und \(b=r^m\) gilt. \\
		\begin{align}
			ab&= r^nr^m \\
			&= r^{n+m} \\
			&= r^{m+n} \\
			&= r^mr^n \\
			&= ba
		\end{align}
		Womit gezeigt wäre, dass die Annahme jede zyklische Gruppe sei abelsch gilt.
	
	\subsection{c)}
		\subsubsection{(i)}
			Da die unendliche zyklische Gruppe schon im Skript definiert wurde, brauchen nur noch die 
			Elemente der unendlichen zyklischen Gruppe \(\{1..a^{n-1}\}\) gewählt werden und mit
			ihnen die neue zyklische Gruppe der Ordnung \(n\) gebildet werden, wobei für das Element 
			\(a^n = 1\) gilt.
		\subsubsection{(ii)}
			Der Argumentation von (i) folgend dürfte es nur isomorphe zyklische Gruppen der Ordnung 
			\(n\) geben, da sich zwar der Erzeuger, aber nicht die gegenseitigen Beziehungen in der 
			Gruppentafel ändern.
			
\section{4.}
	Eine der beiden Gruppen der Ordnung 4 ist die zyklische Gruppe \(G=<a> =\{1,a,a^2,a^3\}\).
	Da isomorphe Gruppen als gleich angesehen werden, muss gezeigt werden, dass der Aufbau der 
	Gruppentafel für die nichtzyklische Gruppe \(H=\{1,a,b,c\}\) schon festgelegt ist. 
	Das neutrale Element hat die natürlich die Ordnung 1, die anderen Elemente müssen (aufgrund der 	
	Folgerung 1, Abschnitt 6.8) die Ordnung 2 aufweisen.\\ 
	Dies kommt zustande, indem die Ordnung der Elemente ein Teiler der Ordnung der Menge sein muss, 
	3 kommt also schon mal nicht in Frage (\(3\nmid 4\)). \(1|4\) gilt zwar, allerdings ist die 
	Ordnung 1 dem neutralen Element vorbehalten. Bleibt noch 2, welches natürlich ein Teiler von 4 
	ist. Höhere Ordnungen sind nicht zu betrachten, da diese durch \(mod\ 4\) (aufgrund der 
	Endlichkeit der Gruppe) auf die genannten Fälle abgebildet werden können. \\\newpage
	Daraus folgt folgende Gruppentafel für den Start: \\ \\

	\(
	\begin{array}{ccccc}	
		\begin{tabular}{c||c|c|c|c}
			\(H\) & 1 & a & b & c \\ \hline \hline
				1 & 1 & a & b & c \\ \hline
				a & a & 1 &   &   \\ \hline
				b & b &   & 1 &   \\ \hline
				c & c &   &   & 1
		\end{tabular}&\overset{*}{\Rightarrow}&
		\begin{tabular}{c||c|c|c|c}
			\(H\) & 1 & a & b & c \\ \hline \hline
				1 & 1 & a & b & c \\ \hline
				a & a & 1 & \(\overline{1}\land\overline{b}\land\overline{a}\)  &  \(\overline{1}\land\overline{c}\land\overline{a}\) \\ \hline
				b & b & \(\overline{1}\land\overline{b}\land\overline{a}\)  & 1 & \(\overline{1}\land\overline{c}\land\overline{b}\)  \\ \hline
				c & c & \(\overline{1}\land\overline{c}\land\overline{a}\) & \(\overline{1}\land\overline{c}\land\overline{b}\)  & 1
		\end{tabular}
	\end{array} 
	\)\\ \\
	\begin{small}
		* \(\overline{x}\) stehe hier für \(nicht\ x\)  \\
	\end{small}
	
	Daraus folgt die vollständige Tabelle: \\ \\
	\begin{tabular}{c||c|c|c|c}
			\(H\) & 1 & a & b & c \\ \hline \hline
				1 & 1 & a & b & c \\ \hline
				a & a & 1 & c & b \\ \hline
				b & b & c & 1 & a \\ \hline
				c & c & b & a & 1
		\end{tabular} \\ \\ \\
	In den Tabellen ist deutlich zu sehen, dass für jedes noch unbesetzte Feld nur eine Wahl bleibt, 
	also ist die Tabelle vorbestimmt und die Annahme bewiesen\(\qed\)\\
	Eine solche Gruppe wäre beispielsweise die Rechtecksgruppe aus den Präsenzaufgaben.
	
	
\end{document}