\documentclass[a4paper]{scrartcl}
\usepackage[ngerman]{babel}
\usepackage[utf8]{inputenc}
\usepackage[T1]{fontenc}
\usepackage{lmodern}
\usepackage{amssymb}
\usepackage{amsmath}
\usepackage{enumerate}
\usepackage{pgfplots}
\usepackage{scrpage2}\pagestyle{scrheadings}
\usepackage{tikz}
\usetikzlibrary{patterns}

\newcommand{\titleinfo}{Hausaufgaben zum 16. 11. 2012}
\title{\titleinfo}
\author{Arne Struck 6326505}
\date{\today}
\chead{\titleinfo}
\ohead{\today}
\setheadsepline{1pt}
\setcounter{secnumdepth}{0}
\setlength{\textheight}{22cm}
\newcommand{\qed}{\ \square}

\begin{document}
\maketitle
\notag
\section{1.}
	\subsection{a)}
		\(
		\begin{array}{rcl}
			X&=&\{1,2,3,4,5\} \\
			Y&=&\{1,2,3,4,5,6,7\} \\ \\
		\end{array}
		\)
		\\ \\
		\(
		\begin{array}{lrcl}
		\underline{g:X\rightarrow Y:}& \\ \\
		\text{gesamt:} & \\
		&7^5&=&16807 \\ \\
		\text{injektiv:} & \\ 
		&\frac{7!}{(7-5)!}&=&2520 \\ \\
		g(2)\neq g(3)\neq g(4): & \\ 
		&7^3\cdot 6\cdot 5 &=&10290 \\
		\end{array}
		\)
		
	\subsection{b)}
		\[
		\begin{pmatrix}
			49 \\
			6
		\end{pmatrix}
		=13983816 \] \\

	\subsection{c)}
		\[
		\begin{pmatrix}
			1000 \\
			997
		\end{pmatrix}=
		\begin{pmatrix}
			1000 \\
			3
		\end{pmatrix}
		=166167000
		\]

\newpage
\section{2.}
	\subsection{a)}
		\underline{Nach bin. Lehrsatz:} \\ \\
		\(\text{zu bestimmen: }a\text{ aus }ax^5y^{11}\text{ aus }(x+y)^{16}: \)\\ \\
		\[
		\begin{array}{rclclcl}
		 	a&=&
		 	\begin{pmatrix}
		 		16 \\
		 		5
		 	\end{pmatrix}
		 	&=&
		 	\begin{pmatrix}
		 		16 \\
		 		11
		 	\end{pmatrix}
		 	&=&4368
		\end{array}
		\]
		\\ \\ \\
		\(\text{zu bestimmen: }a\text{ aus }ax^3y^5z^2\text{ aus }(x+y+z)^{10}: \)\\ \\
		\[
		\begin{array}{rclclcl}
		a&=&
		\begin{pmatrix}
			10 \\
			3,5,2
		\end{pmatrix}
		&=&\frac{10!}{3!\cdot 5!\cdot 2!}
		&=&2520
		\end{array}
		\]
	
	\subsection{b)}
		\begin{center}
			CAPPUCCINO: \\
		\end{center}
		\[		
		\begin{array}{ccc}
			\text{C}&=& 3 \\
			\text{A}&=& 1 \\
			\text{P}&=& 2 \\
			\text{U}&=& 1 \\
			\text{I}&=& 1 \\
			\text{N}&=& 1 \\
			\text{O}&=& 1 \\
		\end{array}
		\quad\quad\Rightarrow \frac{10!}{3!\cdot 2!}=302400 \\
		\] \\ \\
		\begin{center}
			MANGOLASSI: \\
		\end{center}
		\[
		\begin{array}{ccc}
			\text{M}&=& 1 \\
			\text{A}&=& 2 \\
			\text{N}&=& 1 \\
			\text{G}&=& 1 \\
			\text{O}&=& 1 \\
			\text{L}&=& 1 \\
			\text{S}&=& 2 \\
			\text{I}&=& 1 \\
		\end{array}
		\quad\quad\Rightarrow \frac{10!}{2!\cdot 2!}=907200 \\
		\] \\ \\
		\begin{center}
			SELTERWASSER: \\
		\end{center}
		\[
		\begin{array}{ccc}
			\text{S}&=& 3 \\
			\text{E}&=& 3 \\
			\text{L}&=& 1 \\
			\text{T}&=& 1 \\
			\text{R}&=& 2 \\
			\text{W}&=& 1 \\
			\text{A}&=& 1 \\
		\end{array}
		\quad\quad\Rightarrow \frac{12!}{3!\cdot 3!\cdot 2!}=19958400 \\
		\]

	\subsection{c)}
		Es handelt sich hierbei um Ziehen mit zurücklegen(da mehr Flaschen pro Sorte vorhanden sind, 
		als gezogen werden) und ohne Berücksichtigung der Reihenfolge (da es egal ist, in welcher 
		Reihenfolge die Kisten befüllt werden). \\
		\[
		\begin{array}{ccll}
			n&=&10 &\text{(Sorten)} \\
			k&=&6 &\text{(Kistengröße)} \\
		\end{array} \\ 
		\] \\
		\[
		\begin{pmatrix}
			10-1+6 \\
			6
		\end{pmatrix}
		=\frac{(9+6)!}{6!\cdot (15-6)!}
		=5005
		\] \\
		

\section{3.}
	\underline{Beh.:}\quad \(\forall n\in\mathbb{N}:n\geq 3\) gilt:\quad 
	\(\sum\limits_{i=3}^n 
	\begin{pmatrix}
		i \\
		i-3
	\end{pmatrix}=
	\begin{pmatrix}
		n+1 \\
		4
	\end{pmatrix}
	\) \\ \\
	\underline{Induktionsanfang:}\quad \(n=3\) \\
	\[
	\begin{array}{rcccc}
		\sum\limits_{i=3}^3
		\begin{pmatrix}
			i \\
			i-3
		\end{pmatrix}
		&=&
		\begin{pmatrix}
			3 \\
			0
		\end{pmatrix}&=&1 \\ \\
		\begin{pmatrix}
			3+1 \\
			4
		\end{pmatrix}&=&
		\begin{pmatrix}
			4 \\
			4
		\end{pmatrix}&=&1
	\end{array}
	\] \\ \\
	\underline{Induktionsanfang:} Die Behauptung gilt für ein beliebiges, aber bestimmtes 	
	\(n\in\mathbb{N}\). \\ \\
	
	\newpage
	\begin{flushleft}
		\underline{Induktionsschritt:} \\
		zu zeigen: 
	\end{flushleft}
	\[\sum\limits_{i=3}^{n+1} 
		\begin{pmatrix}
			i \\
			i-3
		\end{pmatrix}=
		\begin{pmatrix}
			n+2 \\
			4
		\end{pmatrix}\] \\ \\
	\[
	\begin{array}{rcl}
		\sum\limits_{i=3}^{n+1}
		\begin{pmatrix}
			i \\
			i-3
		\end{pmatrix}&=&
		\sum\limits_{i=3}^{n}
		\begin{pmatrix}
			i \\
			i-3
		\end{pmatrix}+
		\begin{pmatrix}
			n+1 \\
			n+1-3
		\end{pmatrix} \\ \\
		&\overset{\text{IA}}{=}&
		\begin{pmatrix}
			n+1 \\
			4
		\end{pmatrix}+
		\begin{pmatrix}
			n+1 \\
			n-2
		\end{pmatrix} \\ \\
		&=&\frac{(n+1)!}{4!\cdot(n+1-4)!}+\frac{(n+1)!}{(n-2)!\cdot (n+1-(n-2))!} \\ \\
		&=&\frac{(n+1)!}{4!\cdot(n-3)!}+\frac{(n+1)!}{3!\cdot (n-2)!} \\ \\
		&=&\frac{(n+1)!}{3!\cdot 4\cdot(n-3)!}+\frac{(n+1)!}{3!(n-3)!\cdot (n-2)} \\ \\
		&=&\frac{(n+1)!\cdot (n-2)}{3!\cdot 4\cdot(n-3)!\cdot (n-2)}+\frac{(n+1)!\cdot 4}{3!\cdot 
			4\cdot (n-3)!\cdot(n-2)} \\ \\
		&=&\frac{(n+1)!\cdot (n-2)+(n+1)!\cdot 4}{4!\cdot (n-2)!} \\ \\
		&=&\frac{(n+1)!\cdot (n-2+4)}{4!\cdot (n-2)!} \\ \\
		&=&\frac{(n+1)!\cdot (n+2)}{4!\cdot (n-2)!} \\ \\
		&=&\frac{(n+2)!}{4!\cdot (n+2-4)!} \\ \\
		&=&
		\begin{pmatrix}
			n+2 \\
			4
		\end{pmatrix}\qed
	\end{array}
	\] \\
	

\newpage
\section{4.}
	\subsection{a)}
		\[
		\begin{array}{rcl}
		2000&-& (666+400+285) \\
		&+& (133+95+57) \\
		&-& 19 \\
		&&= \underline{915}	
		\end{array}
		\]	
	
	\subsection{b)}
		\[
		\begin{array}{rcl}
		1000&-& (333+200+142+90) \\
		&+& (66+47+30+28+18+12) \\
		&-& (9+6+4+2) \\
		&+& 1 \\
		&&=\underline{416}
		\end{array}
		\]

	
\end{document}