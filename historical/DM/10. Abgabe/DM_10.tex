\documentclass[a4paper]{scrartcl}
\usepackage[ngerman]{babel}
\usepackage[a4paper, left=2cm, right=2cm, top=4cm]{geometry}
\usepackage[utf8]{inputenc}
\usepackage[T1]{fontenc}
\usepackage{lmodern}
\usepackage{amssymb}
\usepackage{amsmath}
\usepackage{enumerate}
\usepackage{pgfplots}
\usepackage{scrpage2}\pagestyle{scrheadings}
\usepackage{tikz}
\usepackage{listings}
\usepackage{polynom}
\usetikzlibrary{patterns}

\newcommand{\titleinfo}{Hausaufgaben zum 29. 10. 2012}
\title{\titleinfo}
\author{Arne Struck 6326505}
\date{\today}
\chead{\titleinfo}
\ohead{\today}
\setheadsepline{1pt}
\setcounter{secnumdepth}{0}
\lstset{language=Java}
\newcommand{\qed}{\ \square}

\begin{document}
\maketitle
\notag
\section{1.}
	\subsection{a)}		
		\(H_1\) ist keine Untergruppe von \(G\), da kein neutrales Element (1) vorhanden ist. \\
		\(H_2\) ist keine Untergruppe von \(G\), da sie nicht abgeschlossen ist. So entspricht 
		\(4\cdot 8 = 32\) in \(\mathbb{Z}_{13}\) der 6, welche kein Element von \(G\) ist. \\
		\(H_3\) ist eine Untergruppe von \(G\), da alle Axiome erfüllt sind, wie folgende 
		Multiplikationstabelle für die Multiplikation in \(\mathbb{Z}_{13}\) zeigt: \\
		\begin{center}
			\begin{tabular}{c||c|c}
				&1&12 \\ \hline\hline
				1&1&12 \\ \hline
				12&12&1			
			\end{tabular}
		\end{center}
		
	\subsection{b)}
		\begin{align}
			H &= \{1,3,9\} \\
			H_2 &= \{2,6,5\} \\
			H_4 &= \{4,12,10\} \\
			H_7 &= \{7,8,11\}
		\end{align}
		
	
\section{2.}
	\subsection{a)}
	\subsection{b)}
	\subsection{c)}
	

\section{3.}
	\subsection{a)}
		\begin{align}
			(4x^2-x+2)+(2x^3+x^2-3x+2) &=2x^3+5x^2-4x+4
		\end{align}
		\begin{align}
			(4x^2-x+2)\cdot(2x^3+x^2-3x+2) =&\ \ (8x^5+4x^4-12x^3+4x^2)\\
				& +(-2x^4-x^3+3x^2-2x) \\
				& +(4x^3+2x^2-6x+4) \\
				=& 8x^5+2x^4-9x^3+9x^2-8x+4
		\end{align}
		
	\subsection{b)}
		Als erstes werden die einzelnen Produkte berechnet, welche \(a\cdot x^7\) als Ergebnis 
		haben.
		\begin{align}
			x^7\cdot -2 &= -2x^7 \\
			3x^6\cdot 2x &= 6x^7 \\
			6x^5\cdot -3x^2 &= -18x^7 \\
			3x^4\cdot 3x^3 &= 9x^7 \\
			7x^3 \cdot -x^4 &= -7x^7 \\
			8x^2 \cdot 5x^5 &= 40x^7 \\
			x\cdot 6x^6 &= 6x^7 \\
			2\cdot x^7 &= 2x^7
		\end{align}
		Nun wird aus den Teilergebnissen eine Summe gebildet anhand der der Koeffizient von \(x^7\) 
		im Produkt von \(a(x)\) und \(b(x)\) abgelesen werden kann.
		\begin{align}
			-2x^7+6x^7-18x^7+9x^7-7x^7+40x^7+6x^7+2x^7 &= 36x^7
		\end{align}
		
	\subsection{c)}
		\begin{align}
			a(x)\cdot b(x)=& (4x^3+2x^2+3x+2)\cdot (3x^4+x^2+3) \\
			=&\ \ (2x^7+4x^5+2x^3) \\
			&+(x^6+2x^4+x^2) \\
			&+(4x^5+3x^3+4x) \\
			&+(x^4+2x^2+1) \\
			=& 2x^7+x^6+3x^5+3x^4+0x^3+3x^2+4x+1
		\end{align}
	
	
\section{4.}	
	\subsection{a)}
		\polyset{style=C, div=:,vars=x}
	    \polylongdiv{x^5+2x^4+3x^3+x^2+4x+2}{x^2+4x+3}

	\subsection{b)}
		Nebenrechnung:
		\polylongdiv{6x^5+7x^4-7x^3-22x^2+25x-15}{3x^4+2x^3-6x^2-6x-9} \\
		\polylongdiv{3x^4+2x^3-6x^2-6x-9}{3x^3-4x^2+49x-6} \\
		\polylongdiv{3x^3-4x^2+49x-6}{-47x^2-98x+3} \\
		Eukl. Algorithmus:
		\begin{align}
			6x^5+7x^4-7x^3-22x^2+25x-15 &= (2x+1)(3x^4+2x^3-6x^2-6x-9) + (3x^3-4x^2+49x-6) \\
			3x^4+2x^3-6x^2-6x-9 &= (x+2) (3x^3-4x^2+49x-6) + (-47x^2-98x+3) \\
		\end{align}
		Ich breche hier mal ab, da es mir weniger gewollt erscheint einen Eukl. Algorithmus in 
		100000er-Höhe zu berechnen.
	
\end{document}