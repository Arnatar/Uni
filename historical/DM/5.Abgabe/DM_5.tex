\documentclass[a4paper]{scrartcl}
\usepackage[ngerman]{babel}
\usepackage[utf8]{inputenc}
\usepackage[T1]{fontenc}
\usepackage{lmodern}
\usepackage{amssymb}
\usepackage{amsmath}
\usepackage{enumerate}
\usepackage{pgfplots}
\usepackage{scrpage2}\pagestyle{scrheadings}
\usepackage{tikz}
\usetikzlibrary{patterns}

\newcommand{\titleinfo}{Hausaufgaben zum 23. 11. 2012}
\title{\titleinfo}
\author{Arne Struck 6326505}
\date{\today}
\chead{\titleinfo}
\ohead{\today}
\setheadsepline{1pt}
\setcounter{secnumdepth}{0}
\setlength{\textheight}{22cm}
\newcommand{\qed}{\ \square}

\begin{document}
\maketitle
\notag
\section{1.}
	\subsection{a)}
		\makebox(0,10)
		
		\begin{center}
			\begin{tikzpicture}[overlay,
				every node/.style={fill, circle, inner sep=2pt}]
	
				\node[label=below:$a$] (A) at (-5,0) {};
				\node[label=below:$b$] (B) at (-3,0) {};
				\node[label=below:$c$] (C) at (-1,0) {};
				\node[label=below:$d$] (D) at (1,0) {};
				\node[label=below:$e$] (E) at (3,0) {};
				\node[label=below:$f$] (F) at (5,0) {};
				
				\draw[-latex] (A) to (B);
				\draw[-latex] (B) to (C);
				\draw[-latex] (C) to (D);
				\draw[-latex] (B) to[bend left=40] (E);
				\draw[-latex] (E) to (F);
				
			\end{tikzpicture}\\ 
		\end{center}
		
		\makebox(0,5)
		
		\begin{center}				
			\begin{tabular}[t]{c|cccccc}
				& a & b & c & d & e & f \\
				\hline
				a & 0 & 1 & 0 & 0 & 0 & 0 \\
				b & 0 & 0 & 1 & 0 & 1 & 0 \\
				c & 0 & 0 & 0 & 1 & 0 & 0 \\
				d & 0 & 0 & 0 & 0 & 0 & 0 \\
				e & 0 & 0 & 0 & 0 & 0 & 1 \\
				f & 0 & 0 & 0 & 0 & 0 & 0
			\end{tabular}\\
		\end{center}
		
		
	\subsection{b)}
	Hinzugefügte Kanten sind gestrichelt dargestellt. \\ \\ \\
		\begin{center}
			\begin{tikzpicture}[overlay,
				every node/.style={fill, circle, inner sep=2pt}]
	
				\node[label=below:$a$] (A) at (-5,0) {};
				\node[label=below:$b$] (B) at (-3,0) {};
				\node[label=below:$c$] (C) at (-1,0) {};
				\node[label=below:$d$] (D) at (1,0) {};
				\node[label=below:$e$] (E) at (3,0) {};
				\node[label=below:$f$] (F) at (5,0) {};
				
				\draw[-latex] (A) to (B);
				\draw[-latex] (B) to (C);
				\draw[-latex] (C) to (D);
				\draw[-latex] (B) to[bend left=40] (E);
				\draw[-latex] (E) to (F);
				
				\draw[dashed][-latex] (A) to[loop] (A);
				\draw[dashed][-latex] (B) to[loop] (B);
				\draw[dashed][-latex] (C) to[loop] (C);
				\draw[dashed][-latex] (D) to[loop] (D);
				\draw[dashed][-latex] (E) to[loop] (E);
				\draw[dashed][-latex] (F) to[loop] (F);
				
				\draw[dashed][-latex] (A) to[bend right = 30] (C);
				\draw[dashed][-latex] (A) to[bend right = 30] (D);
				\draw[dashed][-latex] (B) to[bend right = 30] (D);
				\draw[dashed][-latex] (A) to[bend left = 30] (E);
				\draw[dashed][-latex] (B) to[bend left = 30] (F);
				\draw[dashed][-latex] (A) to[bend right = 30] (F);
			\end{tikzpicture}\\ 
		\end{center}

	\newpage
	\subsection{c)}
		\begin{center}
			\begin{tikzpicture}[overlay,
				every node/.style={fill, circle, inner sep=2pt}]
				\node[label=left:$f$] (F) at (0,0) {};
				\node[label=right:$d$] (D) at (2,0) {};
				\node[label=left:$e$] (E) at (0,-1) {};
				\node[label=right:$c$] (C) at (2,-1) {};
				\node[label=left:$b$] (B) at (1,-2) {};
				\node[label=left:$a$] (A) at (1,-3) {};
				
				\draw [-] (A) to (B);
				\draw [-] (B) to (E);
				\draw [-] (B) to (C);
				\draw [-] (E) to (F);
				\draw [-] (C) to (D);
			\end{tikzpicture}
		\end{center}
		\makebox(0,80)
		

	\subsection{d)}
		\begin{align}
			S=\Big\{&R,(a,a),(b,b),(c,c),(d,d),(e,e),(f,f),(a,c),(b,d),(a,d),(b,f),(a,e),(a,f),
			(b,a),(c,b),\\&(e,b),(f,e),(c,a),(d,b),(d,a),(f,b),(e,a),(f,a)\Big\}
		\end{align}
		\(S\) umfasst eine komplette Relation \(A\rightarrow A\). \\
		\(K_1 = (a,b,c,d,e,f)\)

\section{2.}
	\subsection{a)}
		\begin{center}
			\begin{tikzpicture}[overlay, 
				every node/.style={fill, circle, inner sep=2pt}]

				\node[label=below:$a$] (A) at (-5,0) {};
				\node[label=below:$b$] (B) at (-3,0) {};
				\node[label=below:$c$] (C) at (-1,0) {};
				\node[label=below:$d$] (D) at (1,0) {};
				\node[label=below:$e$] (E) at (3,0) {};
				\node[label=below:$f$] (F) at (5,0) {};
				
				\draw[-latex] (A) to (B);
				\draw[-latex] (B) to[bend right=40] (D);
				\draw[-latex] (E) to (F);
				\end{tikzpicture}\\
		\end{center}
		
		\makebox(0,5)
		
		\begin{center}
			\begin{tabular}[t]{c|cccccc}
				& a & b & c & d & e & f\\
				\hline
				a & 0 & 1 & 0 & 0 & 0 & 0\\
				b & 0 & 0 & 0 & 1 & 0 & 0\\
				c & 0 & 0 & 0 & 0 & 0 & 0\\
				d & 0 & 0 & 0 & 0 & 0 & 0\\
				e & 0 & 0 & 0 & 0 & 0 & 1\\
				f & 0 & 0 & 0 & 0 & 0 & 0
			\end{tabular}\\			
		\end{center}
		\makebox(0,10)
	
	\subsection{b)}
		\makebox(0,10)
		
		\begin{center}
			\begin{tikzpicture}[overlay,
					every node/.style={fill, circle, inner sep=2pt}]
		
					\node[label=below:$a$] (A) at (-5,0) {};
					\node[label=below:$b$] (B) at (-3,0) {};
					\node[label=below:$c$] (C) at (-1,0) {};
					\node[label=below:$d$] (D) at (1,0) {};
					\node[label=below:$e$] (E) at (3,0) {};
					\node[label=below:$f$] (F) at (5,0) {};
					
					\draw[-latex] (A) to (B);
					\draw[-latex] (B) to[bend right=40] (D);
					\draw[-latex] (E) to (F);
					
					\draw[dashed][-latex] (A) to[bend left=40] (D);
										
					\draw[dashed][-latex] (A) to[loop] (A);
					\draw[dashed][-latex] (B) to[loop] (B);
					\draw[dashed][-latex] (C) to[loop] (C);
					\draw[dashed][-latex] (D) to[loop] (D);
					\draw[dashed][-latex] (E) to[loop] (E);
					\draw[dashed][-latex] (F) to[loop] (F);
			\end{tikzpicture}		
		\end{center}
		
	\newpage
	\subsection{c)}
		\begin{center}
			\begin{tikzpicture}[overlay, 
				every node/.style={fill, circle, inner sep=2pt}]
				\node[label=left:$d$] (D) at (0,0) {};
				\node[label=left:$b$] (B) at (0,-1) {};
				\node[label=right:$f$] (F) at (2,-1) {};
				\node[label=left:$a$] (A) at (0,-2) {};
				\node[label=right:$e$] (E) at (2,-2) {};
				\node[label=left:$c$] (C) at (1,-2) {};
				
				\draw[-] (A) to (B);
				\draw[-] (B) to (D);
				\draw[-] (E) to (F);
			\end{tikzpicture}\\
		\end{center}
		\makebox(0,80)

	\subsection{d)}
		\begin{align}
			S&=\Big\{(a,a),(b,b),(c,c),(d,d),(e,e),(f,f),(a,d),(b,a),(d,b),(f,e),(d,a)\Big\} \\
			K_1&=(a,b,d) \\
			K_2&=(c) \\
			K_3&=(e,f)
		\end{align}
		

\section{3.}
	\subsection{a)}
		\begin{center}
			\begin{tikzpicture}[overlay, 
				every node/.style={fill, circle, inner sep=2pt}]
				
				\node[label=below:$a$] (A) at (-3,0) {};
				\node[label=below:$b$] (B) at (-1,0) {};
				\node[label=below:$c$] (C) at (1,0) {};
				\node[label=below:$d$] (D) at (3,0) {};
				
				\draw[-latex] (A) to[loop] (A);
				\draw[-latex] (B) to[loop] (B);
				\draw[-latex] (C) to[loop] (C);
				\draw[-latex] (D) to[loop] (D);
				
				\draw[-latex] (A) to[bend left=30] (B);
				\draw[-latex] (B) to[bend left=30] (A);

				\draw[-latex] (B) to[bend left=30] (C);
				\draw[-latex] (C) to[bend left=30] (B);
			\end{tikzpicture}\\
		\end{center}
		\makebox(0,5)
	
	\subsection{b)}
		\begin{center}
			\begin{tikzpicture}[overlay, 
				every node/.style={fill, circle, inner sep=2pt}]
				
				\node[label=below:$a$] (A) at (-3,0) {};
				\node[label=below:$b$] (B) at (-1,0) {};
				\node[label=below:$c$] (C) at (1,0) {};
				\node[label=below:$d$] (D) at (3,0) {};
				
				\draw[-latex] (A) to[loop] (A);
				\draw[-latex] (B) to[loop] (B);
				\draw[-latex] (C) to[loop] (C);
				\draw[-latex] (D) to[loop] (D);
				
				\draw[-latex] (A) to (B);
				\draw[-latex] (B) to (C);
				\draw[-latex] (C) to[bend right = 50] (A);
			\end{tikzpicture}\\
		\end{center}	
		\makebox(0,5)

	\subsection{c)}
		\begin{center}
			\begin{tikzpicture}[overlay, 
				every node/.style={fill, circle, inner sep=2pt}]
				
				\node[label=below:$a$] (A) at (-3,0) {};
				\node[label=below:$b$] (B) at (-1,0) {};
				\node[label=below:$c$] (C) at (1,0) {};
				\node[label=below:$d$] (D) at (3,0) {};
				
				\draw[-latex] (A) to[bend right = 30] (B);
				\draw[-latex] (B) to[bend right = 30] (C);
				\draw[-latex] (C) to[bend right = 50] (A);
	
				\draw[-latex] (B) to[bend right = 30] (A);
				\draw[-latex] (C) to[bend right = 30] (B);
				\draw[-latex] (A) to[bend right = 50] (C);			
				
			\end{tikzpicture}\\
		\end{center}	
		
\newpage
\section{4.}
	\subsection{a)}
		\(R=\Big\{(1,1),(1,2),(1,3),(1,4),(1,5),(1,6),(2,2),(2,4),(2,6),(3,3),(3,6),
		(4,4),(5,5),(6,6)\Big\}\) \\	
		\makebox(0,50)
		
		\begin{center}
			\begin{tikzpicture}[overlay, 
				every node/.style={fill, circle, inner sep=2pt}]
				
				\node[label=below:$1$] (1) at (-5,0) {};
				\node[label=below:$2$] (2) at (-3,0) {};
				\node[label=below:$3$] (3) at (-1,0) {};
				\node[label=below:$4$] (4) at (1,0) {};
				\node[label=below:$5$] (5) at (3,0) {};
				\node[label=below:$6$] (6) at (5,0) {};
				
				\draw[-latex] (1) to[loop] (1);
				\draw[-latex] (1) to (2);
				\draw[-latex] (1) to[bend right = 30] (3);
				\draw[-latex] (1) to[bend right = 30] (4);
				\draw[-latex] (1) to[bend right = 30] (5);
				\draw[-latex] (1) to[bend right = 30] (6);
				
				\draw[-latex] (2) to[loop] (2);
				\draw[-latex] (2) to[bend left = 40] (4);
				\draw[-latex] (2) to[bend left = 40] (6);
				
				\draw[-latex] (3) to[loop] (3);
				\draw[-latex] (3) to[bend left = 40] (6);
				
				\draw[-latex] (4) to[loop] (4);
				
				\draw[-latex] (5) to[loop] (5);
				
				\draw[-latex] (6) to[loop] (6);		
			\end{tikzpicture}\\
		\end{center}	
		\makebox(0,40)
		
		\begin{center}
			\begin{tikzpicture}[overlay, 
				every node/.style={fill, circle, inner sep=2pt}]
				\node[label=right:$6$] (6) at (1,-1) {};
				\node[label=left:$4$] (4) at (0,-1) {};
				\node[label=left:$5$] (5) at (-1,-2) {};
				\node[label=left:$2$] (2) at (0,-2) {};
				\node[label=right:$3$] (3) at (1,-2) {};
				\node[label=below:$1$] (1) at (0,-3) {};
				
				\draw (1) -- (2);
				\draw (1) -- (5);
				\draw (1) -- (3);
				\draw (2) -- (4);
				\draw (3) -- (6);
				\draw (2) -- (6);
			\end{tikzpicture}\\		
		\end{center}
		\makebox(0,90)
	
	\subsection{b)}
		\makebox(0,10)
		
		\begin{center}
			\begin{tikzpicture}[overlay, 
				every node/.style={fill, circle, inner sep=2pt}]
				
				\node[label=below:$ \emptyset $] (0) at (-3,0) {};
				\node[label=below:$ \{1\} $] (1) at (-1,0) {};
				\node[label=below:$ \{2\} $] (2) at (1,0) {};
				\node[label=below:$ \{1{,}2\} $] (12) at (3,0) {};
				
				\draw[-latex](0) to (1);
				\draw[-latex](0) to[bend left = 30] (2);
				\draw[-latex](0) to[bend left = 30] (12);
				
				\draw[-latex](0) to[loop] (0);
				\draw[-latex](1) to[loop] (1);
				\draw[-latex](2) to[loop] (2);
				\draw[-latex](12) to[loop] (12);
			\end{tikzpicture}\\		
		\end{center}
		\makebox(0,70)
	
		\begin{center}
			\begin{tikzpicture}[overlay, 
				every node/.style={fill, circle, inner sep=2pt}]

				\node[label=below:$ \emptyset $] (0) at (0,-1) {};
				\node[label=left:$ \{1\} $] (1) at (-1,0) {};
				\node[label=right:$ \{2\} $] (2) at (1,0) {};
				\node[label=above:$ \{1{,}2\} $] (12) at (0,1) {};
				
				\draw[-](0) to (1);
				\draw[-](0) to (2);
				\draw[-](1) to (12);
				\draw[-](2) to (12);
				
			\end{tikzpicture}\\		
		\end{center}
	
	\newpage
	\subsection{c)}
		\begin{align}
			A&= \Big\{\emptyset,\{1\},\{2\},\{3\},\{1,2\},\{1,3\},\{2,3\},\{1,2,3\}\Big\}
		\end{align}
		\makebox(0,200)		
		
		\begin{center}
			\begin{tikzpicture}[overlay, 
				every node/.style={fill, circle, inner sep=2pt}]

				\node[label=below:$ \emptyset $] (0) at (0,0) {};
				\node[label=left:$ \{1\} $] (1) at (-2,2) {};
				\node[label=right:$ \{2\} $] (2) at (0,2) {};
				\node[label=right:$ \{3\} $] (3) at (2,2) {};				
				\node[label=left:$ \{1{,}2\} $] (12) at (-2,4) {};
				\node[label=right:$ \{1{,}3\} $] (13) at (0,4) {};
				\node[label=right:$ \{2{,}3\} $] (23) at (2,4) {};	
				\node[label=above:$ \{1{,}2{,}3\} $] (123) at (0,6) {};		
				
				\draw[-] (0) to (1);
				\draw[-] (0) to (2);
				\draw[-] (0) to (3);
				
				\draw[-] (1) to (12);
				\draw[-] (1) to (13);
				
				\draw[-] (2) to (12);
				\draw[-] (2) to (23);
				
				\draw[-] (3) to (13);
				\draw[-] (3) to (23);
				
				\draw[-] (12) to (123);
				\draw[-] (13) to (123);
				\draw[-] (23) to (123);
				
			\end{tikzpicture}\\		
		\end{center}
	
\end{document}