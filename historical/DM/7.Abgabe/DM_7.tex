\documentclass[a4paper]{scrartcl}
\usepackage[ngerman]{babel}
\usepackage[utf8]{inputenc}
\usepackage[T1]{fontenc}
\usepackage{lmodern}
\usepackage{amssymb}
\usepackage{amsmath}
\usepackage{enumerate}
\usepackage{pgfplots}
\usepackage{scrpage2}\pagestyle{scrheadings}
\usepackage{tikz}
\usepackage{listings}
\usetikzlibrary{patterns}

\newcommand{\titleinfo}{Hausaufgaben zum 7. 12. 2012}
\title{\titleinfo}
\author{Arne Struck 6326505}
\date{\today}
\chead{\titleinfo}
\ohead{\today}
\setheadsepline{1pt}
\setcounter{secnumdepth}{0}
\lstset{language=Java}
\newcommand{\qed}{\ \square}

\begin{document}
\maketitle
\notag
\section{1.}
	\subsection{a)}		
		Die Invertierbarkeit von 473 in \(\mathbb{Z}_{2413}\) lässt sich mit dem \(ggt(2413,473)\) 
		bestimmen:
		\begin{align}
			2413&=5\cdot 473+48 \\
			473&=9\cdot 48 +41 \\
			48&=41+7 \\
			41&=5\cdot 7+6 \\
			7&= 6 +1
		\end{align}
		\(\Rightarrow ggt(2413,473)=1\Leftrightarrow 473\) ist in \(\mathbb{Z}_{2413}\) invertierbar.
		\begin{align}
			x=1&=7-6 \\
			&=7-1(41-5\cdot 7) \\
			&=6\cdot 7 -41 \\
			&=-41+6(48-41) \\
			&=-7\cdot 41+6\cdot 48 \\
			&=6\cdot 48 -7(473-9\cdot 48) \\
			&=69\cdot 48-7\cdot 473 \\
			&=-7\cdot 473+69(2413-5\cdot 473) \\
			&=-352\cdot 473+ 69\cdot 2413
		\end{align}
		\(\Rightarrow -352\equiv \underline{2061} \mod 2413\) ist das Inverse von 473 in 
		\(\mathbb{Z}_{2413}\)
	
	\subsection{b)}
		Die Invertierbarkeit von 473 in \(\mathbb{Z}_{1672}\) lässt sich mit dem \(ggt(1672,473)\) 
		bestimmen:
		\begin{align}
			1672&=3\cdot 473+253 \\
			473&=253+220 \\
			253&=220+33 \\
			220&=6\cdot 33+22 \\
			33&=22+11 \\
			22&=2\cdot 11 \\
		\end{align}
		\(\Rightarrow ggt(1673,473)=11\Leftrightarrow 473\) ist in \(\mathbb{Z}_{1673}\) nicht 
		invertierbar.

	\subsection{c)}
		In \(\mathbb{Z}_{2413}\) entspricht 2413 der 0. Aus diesem Grund muss 2412 der -1 
		entsprechen. -1 ist zu sich selbst invers, das heißt, dass in \(\mathbb{Z}_{2413}\) 2412 das 
		Inverse von 2412 ist.

\section{2.}
	\textbf{In \(\mathbb{Z}_{19}\):}
	\begin{align}
		3^1&=3 \\
		3^2&=9 \\
		3^4&=9^2=5 \\
		3^8&=5^2=25=6 \\
		3^{16}&=6^2=36=17 \\
		3^{32}&=17^2=289=4 \\
		3^{64}&=4^2=16 \\
		3^{128}&=16^2=256=9 \\
		3^{256}&=9^2=81=5 \\
		3^{512}&=5^2=25=6		
	\end{align}
	\begin{align}
		3^{1000}&= 3^{512}\cdot 3^{256} \cdot (3\cdot 3^{64})\cdot 3^32\cdot 3^8 \\
		&=6\cdot 9\cdot 3\cdot 16\cdot 4\cdot 6 \\
		&= 62208 \equiv 2 \mod 19
	\end{align}
	
\section{3.}
	\subsection{a)}
		\[\pi = (1,7,6)\circ (2,10,8,5,11,13)\circ (3,4)\circ(9,12)\]
		
	\subsection{b)}
		\[\pi= (1,6)\circ(1,7)\circ(2,13)\circ(2,11)\circ(2,5)\circ(2,8)\circ(2,10)
		\circ(3,4)\circ(9,12)\]
	
	\subsection{c)}
		\(\pi\) hat 9  Transpositionen, also ist \(\pi\) eine ungerade Permutation, das bedeutet sign 
		\(\pi = -1\)

	
\section{4.}	
	\subsection{a)}
		Da alle Einzelelemente für \(\text{A}\times\text{B}\times\text{C}\) kombiniert werden müssen, 
		ist die Anzahl der Elemente von \(\text{A}\times\text{B}\times\text{C}\) \(3\cdot 5 \cdot 
		2=30\).

	\subsection{b)}
		Die Anzahl aller Relationen auf \(\text{A}\times\text{B}\times\text{C}\) entspricht der 
		Anzahl aller Teilmengen (wobei ich nicht sicher bin, ob man die leere Menge extra zählen 
		muss). Diese berechnet sich aus \(\sum\limits_{i=0}^{n=30}\begin{pmatrix}
		30\\i
		\end{pmatrix}= 2^n |n=30\).
	
	
\end{document}