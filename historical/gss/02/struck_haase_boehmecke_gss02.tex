% Packages & Stuff
 
\documentclass[a4paper,11pt]{scrartcl}
\usepackage[ngerman]{babel}
\usepackage[utf8]{inputenc}
\usepackage[T1]{fontenc}
\usepackage[top=1.3in, bottom=1.2in, left=0.8in, right=0.8in]{geometry}
\usepackage{amsmath}
\usepackage{lmodern}
\usepackage{enumerate}
\usepackage{fancyhdr}
\usepackage{pgfplots}
 
% ------------------------------------------------------
 
% Commands
 
\newcommand{\authorinfo}{A. Struck, S. Haase, E. Böhmecke}
\newcommand{\titleinfo}{GSS-Übungsblatt 2 zum 07.05.2014}
\newcommand{\qed}{\ \square}
\newcommand{\todo}{\textcolor{red}{\textbf{TODO}}}
\newcommand{\opt}{\textcolor{blue}{\textbf{Optional}}}
 
% ------------------------------------------------------
 
% Title & Pages
 
\title{\titleinfo}
\author{\authorinfo}
 
\pagestyle{fancy}
\fancyhf{}
\fancyhead[L]{\authorinfo}
\fancyhead[R]{\titleinfo}
\fancyfoot[C]{\thepage}
 
\begin{document}
\maketitle

\section*{Grundlagen von Betriebssystemen}
\subsection*{a)}
Auf der einen Seite muss ein Betriebssystem die Ressourcenverteilung (Zeit- und Speicherverteilung)
managen. Auf der anderen Seite dient es dazu die Systemdetails hinter einem User-Interface zu verbergen,
da ein Mensch auf Dauer nicht mit diesen umzugehen vermag.
\subsection*{b)} \todo
\subsection*{c)} \opt
\subsection*{d)} \opt

\section*{Prozesse und Threads}
\subsection*{a)} 
\subsubsection*{Programm}
Ein Programm ist eine Folge von Anweisungen, welche auf einem Computer eine bestimmte Funktionalität bereitstellen.
\subsubsection*{Prozess}
Ein Prozess ist die Instanz eines Programm in seiner Ausführung (die Abarbeitung hat begonnen und ist noch nicht terminiert worden). 
\subsubsection*{Thread}
Der Begriff Thread beschreibt einen (von mehreren möglichen) Ausführungsstrang im Ablauf eines Prozesses, dabei wird ein gemeinsamer Adressraum verwendet.
\subsection*{b)} \opt
\subsection*{c)} \opt
\subsection*{d)} \todo

\section*{n-Adressmaschine}
\subsection*{a)}
\(
\begin{array}{lll}
	\text{Befehl} & \text{Ziel, Quelle} & \text{Beschreibung} \\
	MOVE & H1, a_1 & H1 := a_1 \\
	ADD  & H1, a_2 & H1 := H1 + a_2 \\
	DIV  & H1, a_3 & H1 := H1 / a_3 \\
	MOVE & H2, b_1 & H2 := b_1 \\
	SUB  & H2, b_2 & H2 := H2 - b_2 \\
	DIV  & H2, b_3 & H2 := H2 / b_3 \\
	ADD  & H2, H1  & H2 := H2 + H1 \\
	MOVE & R, H2   & R := H2
\end{array}
\)
\subsection*{b)} \opt
\subsection*{c)} \opt

\end{document}