% Commands

\newcommand{\authorinfo}{Tim Dobert, Kai Sonnenwald, Arne Struck}
\newcommand{\titleinfo}{GDB [HA] zum 9. 1. 2014}
\newcommand{\qed}{\ \square}
\newcommand{\todo}{\textcolor{red}{\textbf{TODO}}}
\newcommand{\bra}[1]{(#1)}


% ------------------------------------------------------

% Packages & Stuff
\documentclass[a4paper,11pt,fleqn]{scrartcl}
\usepackage[german,ngerman]{babel}
\usepackage[utf8]{inputenc}
\usepackage[T1]{fontenc}
\usepackage[top=1.3in, bottom=1.2in, left=0.9in, right=0.9in]{geometry}
\usepackage{lmodern}
\usepackage{amssymb}
\usepackage[fleqn]{amsmath}
\usepackage{enumerate}
\usepackage{tikz}
\usepackage{fancyhdr}
\usepackage{pgfplots}
\usepackage{multicol}
\usepackage{vsis-gdb}
\usepackage[parfill]{parskip}
\usetikzlibrary{calc,patterns,arrows,positioning,calc,fit,shapes}


% ------------------------------------------------------

% Title & Pages

\title{\titleinfo}
\author{\authorinfo}

\pagestyle{fancy}
\fancyhf{}
\fancyhead[L]{\authorinfo}
\fancyhead[R]{\titleinfo}
\fancyfoot[C]{\thepage}

\begin{document}
    \maketitle
	\begin{enumerate}
		\item[\textbf{1.}]
		\begin{enumerate}
			\item[a)] \quad \\
			In einem bezüglich referentiellen Aktionen sicheren Schema sind Ergebnisse von SQL-Anfragen 
			nicht von der Reihenfolge der Abarbeitung durch das DBMS abhängig.

			\item[b)] \quad \\
			Referenzgraph zur Datendefinition: \\
			\begin{tikzpicture}[->,>=stealth',shorten >=1pt,auto,font=\small]
				\node[entity] (1) {Websites};
				\node[entity] (2) [below left = 4cm of 1] {Benutzer};
				\node[entity] (3) [below right = 4cm of 2] {Rubriken};
				\node[entity] (4) [below right = 4cm of 1] {Rubrikzuordnung};

				\path (1) edge [bend right] node[left] {\parbox{5cm}{\centering EingestelltVon $\rightarrow$ Benutzer.UID\\ON DELETE RESTRICT}} (2);
				\path (2) edge node[below right] {\parbox{4.5cm}{\centering Homepage $\rightarrow$ Websites.WID\\ON DELETE SET NULL}} (1);
				\path (3) edge node {\parbox{4cm}{\centering Verwalter $\rightarrow$ Benutzer.UID\\ON DELETE CASCADE}} (2);
				\path (4) edge node[above right] {\parbox{4cm}{\centering WID $\rightarrow$ Websites.WID\\ON DELETE CASCADE}} (1);
				\path (4) edge node[below] {\parbox{6cm}{\centering ZugeordnetVon $\rightarrow$ Benutzer.UID\\ON DELETE CASCADE}} (2);
				\path (4) edge node{\parbox{4cm}{\centering RID $\rightarrow$ Rubriken.RID\\ON DELETE RESTRICT}} (3);
			\end{tikzpicture}

			\item[c)] \quad \\
			Wenn ein Benutzer gelöscht wird, ergeben sich 3 unterschiedliche Pfade für die Auswirkungen 
			des Löschvorgangs, die bei der Rubrikzuordnung enden. Ebenso gibt es für den Fall, dass eine Website gelöscht wird, 
			3 verschiedene Pfade der Auswirkung, die ebenfalls bei der Rubrikzuordnung enden. Damit ist dort keine Sicherheit 
			bezüglich referentiellen Aktionen gegeben. 

			\item[d)] \quad \\
			Das \texttt{ON DELETE RESTRICT} der Rubrik Zurodnung sollte durch \texttt{ON DELETE CASCADE} 
			ersetzt werden. \\
			Das \texttt{ON DELETE CASCADE} der Rubrik Zuordnung sollte durch \texttt{ON DELETE RESTRICT},
			während das \texttt{ON DELETE CASCADE} von ZugeordnetVon \(\rightarrow\) Benutzer durch 
			\texttt{ON DELETE RESTRICT} ersetzt werden.
		\end{enumerate}
	
		\newpage
		\item[\textbf{2}.]
		\begin{enumerate}
			\item[a)] \quad \\
			Gegebenes Relationenmodell: \\
			\begin{RMSchma}
				Raumschiffe(\soliduline{RNr}, Name , Fraktion, Typ, Geschwindigkeit, Baujahr)\\
				Besatzungsmitglieder(\soliduline{BNr}, Name, Rang, \dashuline{Schiff $\rightarrow$ Raumschiffe.RNr})
			\end{RMSchma}
			
			\begin{enumerate}
				\item[i)]
				\begin{verbatim}
					CREATE VIEW EnterpriseCrew AS
					SELECT B.BNr, B.Name, B.Rang
					FROM Besatzungsmitglieder B, Raumschiffe R
					WHERE B.Schiff = R.RNr
					AND R.Name = 'Enterprise'
					WITH CASCADED CHECK OPTION;
				\end{verbatim}
				CHECK OPTION, da für die Sicht mehrere Tabellen Verknüpft wurden.

				\item[ii)]
				\begin{verbatim}
					CREATE VIEW Captains AS
					SELECT Name
					FROM Besatzungsmitglieder
					WHERE Rang = 'Captain'
					WITH CASCADED CHECK OPTION;
				\end{verbatim}
				CHECK OPTION, da der Primärschlüssel nicht in der Sicht ist.

				\item[iii)]
				\begin{verbatim}
					CREATE VIEW WarpFed AS
					SELECT RNr, Fraktion, Baujahr
					FROM Raumschiffe
					WHERE Geschwindigkeit >= 1;
				\end{verbatim}
			\end{enumerate}
		

			\item[b)] \quad \\
			Definierte Sichten:
			\begin{verbatim}
				CREATE VIEW Förderationsschiffe
				AS SELECT ? FROM Raumschiffe
				WHERE Fraktion = ?Förderation?
				WITH CASCADED CHECK OPTION;

				CREATE VIEW Forschungsschiffe
				AS SELECT ? FROM Förderationsschiffe
				WHERE Typ = ?Forschungsschiff?;

				CREATE VIEW GalaxyKlasse
				AS SELECT ? FROM Forschungsschiffe
				WHERE Geschwindigkeit = 9.8;

				CREATE VIEW NebulaKlasse
				AS SELECT ? FROM Forschungsschiffe
				WHERE Baujahr > 2365
				WITH CASCADED CHECK OPTION;
			\end{verbatim}
			
			\newpage
			\begin{enumerate}
				\item[i)]
				\begin{verbatim}
					UPDATE Förderationsschiff
					  SET Geschwindigkeit = 9.8
					  WHERE Geschwindigkeit = 9.7
						AND Typ = ?Kriegsschiff?
						AND Baujahr = 2350;
				\end{verbatim}
				Die SQL-Anweisung ist zulässig, da die Änderungen nicht mit der Definition der Sicht 'Förderationsschiffe' kollidiert.\\
				Die geänderten Daten sind in der Sicht 'Förderationschiffe' sichtbar.

				\item[ii)]
				\begin{verbatim}
					INSERT INTO GalaxyKlasse
					  VALUES (7, ?Sonnensegel?, ?Bajoraner?, ?Forschungsschiff?, 1, 1623);
				\end{verbatim}
				Die SQL-Anweisung ist zulässig, da die eingefügten Daten zwar nicht mit der Definition der Sicht 'GalaxyKlasse' konform sind, aber keine Check-Option aktiviert ist.\\
				Die eingefügten Daten sind in der Sicht 'GalaxyKlasse' sichtbar.

				\item[iii)]
				\begin{verbatim}
					UPDATE Forschungsschiffe
					  SET Baujahr = 2360
					  WHERE Name = ?Enterprise?
						AND Geschwindigkeit = 9.8;
				\end{verbatim}
				Die SQL-Anweisung ist zulässig, da in der Sicht 'Forschungschiffe' keine Check-Option aktiviert ist.\\
				Die geänderten Daten sind in der Sicht 'Forschungsschiffe' und 'GalaxyKlasse' sichtbar.

				\item[iv)]
				\begin{verbatim}
					UPDATE NebulaKlasse
					  SET Baujahr = 2360
					  WHERE Geschwindigkeit = 9.6;
				\end{verbatim}
				Die SQL-Anweisung ist unzulässig, da die Änderungen mit der Definition der Sicht 'NebulaKlasse' kollidieren und die Check-Option aktiviert ist.

				\item[v)]
				\begin{verbatim}
					INSERT INTO Ga l a x yKl a s s e
					  VALUES (88, ?DeltaFlyer?, ?Förderation?, ?Forschungsschiff?, 9.81, 2375);
				\end{verbatim}
				Die SQL-Anweisung ist zulässig, da die eingefügten Daten zwar nicht mit der Definition der Sicht 'GalaxyKlasse' konform sind, aber keine Check-Option aktiviert ist.\\
				Die eingefügten Daten sind in der Sicht 'GalaxyKlasse' sichtbar.
			\end{enumerate}
		\end{enumerate}

		\newpage
		\item[\textbf{3}] \quad \\
			Gegebene Transaktionen:\\
			$T_1$ liest B, liest A, schreibt (A+180+B) nach A.\\
			$T_2$ liest A, schreibt A nach B, schreibt (A+110) nach A.\\
			\\
			Gegebene Schedules:\\
			\(
			\begin{array}{lcccccccccc}
				S_1 &=& r_1(B) & r_1(A) & w_1(A) & r_2(A) & w_2(B) & w_2(A) \\
				S_2 &=& r_2(A) & r_1(B) & r_1(A) & w_2(B) & w_2(A) & w_1(A) \\
				S_3 &=& r_2(A) & w_2(B) & r_1(B) & w_2(A) & r_1(A) & w_1(A) \\
				S_4 &=& r_2(A) & w_2(B) & r_1(B) & r_1(A) & w_2(A) & w_1(A) \\
				S_5 &=& r_2(A) & r_1(B) & r_1(A) & w_1(A) & w_2(B) & w_2(A) \\
				S_6 &=& r_2(A) & w_2(B) & w_2(A) & r_1(B) & r_1(A) & w_1(A) \\
			\end{array}
			\)
			\\ \\
			Anfangswerte:\\
			A = 5\\
			B = 10

			\begin{enumerate}
				\item[a)]\quad \\
				Belegung von A und B nach der Ausführung von S?\\
				\\
				$S_1$: A = 305, B = 195\\
				$S_2$: A = 195, B = 5\\
				$S_3$: A = 300, B = 5\\
				$S_4$: A = 190, B = 5\\
				$S_5$: A = 115, B = 5\\
				$S_6$: A = 300, B = 5

				\item[b)]\quad \\
				Abhängigkeiten zwischen den Transaktionen während des Schedules?\\
				\\
				$S_1$: T2 arbeitet mit dem Ergebnis von T1. \\
				$S_2$: Der Effekt von T2 auf A geht verloren. \\
				$S_3$: T1 arbeitet mit den Ergebnissen von T2. \\
				$S_4$: T1 arbeitet mit einem Teil des Ergebnisses (B) von T2. \\
				$S_5$: T1 hat praktisch keinen Effekt da das Ergebnis sofort von T2 überschrieben wird. \\
				$S_6$: T1 arbeitet mit den Ergebnissen von T2. \\

				\item[c)]\quad \\
				Schedule seriell/serialisierbar/nicht serialisierbar?\\
				\\
				$S_1$: seriell\\
				$S_2$: nicht serialisierbar, da \\
				$S_3$: serialisierbar als $T_2 T_1$\\
				$S_4$: nicht serialisierbar, da \\
				$S_5$: nicht serialisierbar, da \\
				$S_6$: seriell
			\end{enumerate}

			\newpage
			\item[\textbf{4}]\quad \\
				3 Transaktionen: $T_1 T_2 T_3$ \\
				Schedule: $S = w_1(x) r_2(y) r_3(z) w_3(y) r_2(z) w_3(z) w_1(z) r_2(y) c_3 c_1 c_2$\\
				RX-Sperrverfahren mit 2PL\\
				\\
				LOCK-Table:\\
				\\
			\begin{tabular}{|p{2cm}|p{2cm}|p{2cm}|p{2cm}|p{1cm}|p{1cm}|p{1cm}|p{3cm}|}
				\hline
				Zeitschritt & $T_1$       & $T_2$       & $T_3$       & x     & y     & z     & Bemerkung\\
				\hline
				\hline
				0           &             &             &             & NL    & NL    & NL    & \\
				\hline
				1           & lock(x,X)   &             &             & $X_1$ & NL    & NL    & \\
				\hline
				2           & write(x)    & lock(y,R)   &             & $X_1$ & $R_2$ & NL    & \\
				\hline
				3           &             & read(y)     & lock(z,R)   & $X_1$ & $R_2$ & $R_3$ & \\
				\hline
				4           &             &             & read(z)     & $X_1$ & $R_2$ & $R_3$ & \\
				\hline
				5           &             &             & lock(y,X)   & $X_1$ & $R_2$ & $R_3$ & $T_3$ wartet auf Freigabe von y\\
				\hline
				6           &             & lock(z,R)   &             & $X_1$ & $R_2$ & $R_3$ $R_2$ & \\
				\hline
				7           &             & read(z)     & lock(z,X)   & $X_1$ & $R_2$ & $R_3$ $R_2$ & $T_3$ wartet auf Freigabe von z\\
				\hline
				8           & lock(z,X)   &             &             & $X_1$ & $R_2$ & $R_3$ $R_2$ & $T_1$ wartet auf Freigabe von z\\
				\hline
				9           &             & read(y)     &             & $X_1$ & $R_2$ & $R_3$ $R_2$ & \\
				\hline
				10          &             & unlock(y)   &             & $X_1$ & $X_3$ & $R_3$ $R_2$ & Benachrichtigung von $T_3$\\
				\hline
				11          &             & unlock(z)   & write(y)    & $X_1$ & $X_3$ & $X_3$ & Benachrichtigung von $T_3$\\
				\hline
				12          &             &             & write(z)    & $X_1$ & $X_3$ & $X_3$ & \\
				\hline
				13          &             &             & unlock(y)   & $X_1$ & NL    & $X_3$ & \\
				\hline
				14          &             &             & unlock(z)   & $X_1$ & NL    & $X_1$ & Benachrichtigung von $T_1$\\
				\hline
				15          & write(z)    &             & commit      & $X_1$ & NL    & $X_1$ & \\
				\hline
				16          & unlock(x)   &             &             & NL    & NL    & $X_1$ & \\
				\hline
				17          & unlock(z)   &             &             & NL    & NL    & NL    & \\
				\hline
				18          & commit      &             &             &       &       &       & \\
				\hline
				19          &             & commit      &             &       &       &       & \\
				\hline
			\end{tabular}
	\end{enumerate}
\end{document}
