% Commands

\newcommand{\authorinfo}{Tim Dobert, Kai Sonnenwald, Arne Struck}
\newcommand{\titleinfo}{GDB [HA] zum 12. 12. 2013}
\newcommand{\qed}{\ \square}
\newcommand{\todo}{\textcolor{red}{\textbf{TODO}}}
\newcommand{\bra}[1]{(#1)}
\newcommand{\projektion}[1]{\pi_{#1}}
\newcommand{\verbund}[1]{\underset{#1}{\bowtie}}
\newcommand{\natverbund}{\bowtie}
\newcommand{\selektion}[1]{\sigma_{#1}}
\newcommand{\wert}[1]{''#1''}
\newcommand{\soliduline}[1]{\underline{#1}}

% ------------------------------------------------------

% Packages & Stuff
\documentclass[a4paper,11pt,fleqn]{scrartcl}
\usepackage[german,ngerman]{babel}
\usepackage[utf8]{inputenc}
\usepackage[T1]{fontenc}
\usepackage[top=1.3in, bottom=1.2in, left=0.9in, right=0.9in]{geometry}
\usepackage{lmodern}
\usepackage{amssymb}
\usepackage{amsmath}
\usepackage{enumerate}
\usepackage{tikz}
\usepackage{fancyhdr}
\usepackage{pgfplots}
\usepackage{multicol}
\usepackage{ulem}
%\usepackage{vsis-gdb}
\usepackage{listings}
\usepackage[parfill]{parskip}
\usetikzlibrary{calc,patterns,arrows,positioning,calc,fit,shapes}


% ------------------------------------------------------

% Title & Pages

\title{\titleinfo}
\author{\authorinfo}

\pagestyle{fancy}
\fancyhf{}
\fancyhead[L]{\authorinfo}
\fancyhead[R]{\titleinfo}
\fancyfoot[C]{\thepage}

\begin{document}
    \maketitle
	\begin{enumerate}
		\item[\textbf{1.:}]
		\begin{enumerate}
			\item[a)]\quad \\
				\(\projektion{Sorte}\bra{\selektion{Vorname=''Horst''}(Person) \verbund{PNR = Entdecker} Obst} \) \\
			\item[b)]\quad \\
				\(\projektion{Vorname,Nachname}\bra{\selektion{Symptom = ''Halskratzen''}(Allergie)
				\verbund{Person = PNR}Person}\)
			\item[c)]\quad \\
				\(\projektion{Sorte,Nachname}\bra{\selektion{Symptom = ''Würgereiz''}(Allergie)
				\verbund{Person = PNR}Person \verbund{PNR = Entdecker} Obst}\)
		\end{enumerate}
		
		\item[\textbf{2.:}]
		\begin{enumerate}
			\item[a)]\quad \\
			\textbf{Erstellen der Rennort-Tabelle:}
			\begin{lstlisting}
CREATE TABLE Rennort (
	OID		int,
	Name		varchar(30) NOT NULL,
	Strecke		varchar(50) NOT NULL,
	CONSTRAINT 	pk_rennort PRIMARY KEY (OID));
			\end{lstlisting}

			\textbf{Erstellen der Rennstall-Tabelle:}
			\begin{lstlisting}
CREATE TABLE Rennstall (
	RSID		int,
	Name		varchar(15) NOT NULL,
	Teamchef	varchar(50),
	Budget		int NOT NULL UNIQUE,
	CONSTRAINT pk_rennstall PRIMARY KEY (RSID),
	CONSTRAINT ck_budget CHECK(500>Budget>0));
			\end{lstlisting}

			\textbf{Erstellen der Rennfahrer-Tabelle:}
			\begin{lstlisting}
CREATE TABLE Rennfahrer (
	RID		int,
	Vorname		varchar(20) NOT NULL,
	Nachname	varchar(30) NOT NULL,
	Geburt		date NOT NULL,
	Wohnort		varchar(50),
	Rennstall	int NOT NULL,
	CONSTRAINT pk_rennfahrer PRIMARY KEY (RID),
	CONSTRAINT fk_rennstall FOREIGN KEY (Rennstall) REFERENCES Rennstall (RSID));
			\end{lstlisting}

			\textbf{Erstellen der Platzierung-Tabelle:}
			\begin{lstlisting}
CREATE TABLE Platzierung (
	RID		int NOT NULL,
	OID		int NOT NULL,
	Platz		int NOT NULL,
	CONSTRAINT pk_platzierung PRIMARY KEY (RID, OID),
	CONSTRAINT fk_rid FOREIGN KEY (RID) REFERENCES Rennfahrer (RID),
	CONSTRAINT fk_oid FOREIGN KEY (OID) REFERENCES Rennort (OID));
			\end{lstlisting}
			\item[b)]\quad \\
				Die sofortige Prüfung der referentiellen Integrität bedeutet, dass zwingend die richtige
				Reihenfolge eingehalten werden muss um Fehler bei Verweisen auf Fremdschlüsseln zu vermeiden. Mit
				dem zusätzlichen Attribut Star refrenzieren sich die Tabellen Rennfahrer und Rennstall
				gegenseitig. Damit muss man beim Anlegen der Tabellen schrittweise vorgehen und einen der
				Fremdschlüssel nachträglich hinzufügen. Bei der Manipulation der Daten per DML kann es dazukommen,
				dass sich Zeilen aufgrund gegenseitiger Verweise nicht entfernen lassen, es sei denn man setzt 
				erst Star auf null.
			
			
			\item[c)]\quad \\
			\textbf{Füllen der Rennort-Tabelle:}
\begin{lstlisting}
INSERT INTO Rennort
VALUES (4, "Australien GP", "Albert Park Circuit");
INSERT INTO Rennort
VALUES (15, "Malaysia GP", "Sepang International Circuit");
INSERT INTO Rennort
VALUES (21, "China GP", "Shanghai International Circuit");
\end{lstlisting}

\textbf{Füllen der Rennstall-Tabelle:}
\begin{lstlisting}
INSERT INTO Rennstall
VALUES (2,"Red Bull","Christian Horner",370);
INSERT INTO Rennstall
VALUES (5,"Ferrari","Stefano Domenicali",350);
INSERT INTO Rennstall
VALUES (31,"McLaren","Martin Whitmarsh",220);
INSERT INTO Rennstall
VALUES (34,"Lotus F1","Eric Boullier",100);
\end{lstlisting}

\textbf{Füllen der Rennfahrer-Tabelle:}
\begin{lstlisting}
INSERT INTO Rennfahrer
VALUES (4,"Sebastian","Vettel","1987-07-03","Kemmental(Schweiz)",2);
INSERT INTO Rennfahrer
VALUES (6,"Fernando","Alonso","1981-07-29","Lugano(Schweiz)",5);
INSERT INTO Rennfahrer
VALUES (8,"Marc","Webber","1976-08-27","Aston Clinton(UK)",2);
INSERT INTO Rennfahrer
VALUES (9,"Lewis","Hamilton","1985-01-07","Genf(Schweiz)",31);
INSERT INTO Rennfahrer
VALUES (20,"Jenson","Button","1980-01-19","Monte Carlo(Monaco)",31);
INSERT INTO Rennfahrer
VALUES (21,"Felipe","Massa","1982-04-25","Sao Paulo(Brasilien)",5);
INSERT INTO Rennfahrer
VALUES (44,"Kimi",,"1979-10-17","Espoo(Finnland)",34);
\end{lstlisting}

\textbf{Füllen der Platzierung-Tabelle:}
\begin{lstlisting}
INSERT INTO Platzierung
VALUES (8,4,6);
INSERT INTO Platzierung
VALUES (4,15,1);
INSERT INTO Platzierung
VALUES (20,15,17);
INSERT INTO Platzierung
VALUES (4,4,3);
INSERT INTO Platzierung
VALUES (6,4,2);
INSERT INTO Platzierung
VALUES (8,15,2);
INSERT INTO Platzierung
VALUES (6,21,1);
INSERT INTO Platzierung
VALUES (9,4,5);
INSERT INTO Platzierung
VALUES (21,15,5);
INSERT INTO Platzierung
VALUES (20,4,9);
INSERT INTO Platzierung
VALUES (21,4,4);
\end{lstlisting}
			
			
			\item[d)]\quad \\
\textbf{Rennfahrer löschen die mit "F" beginnen:} \\
Erster Ansatz:
\begin{lstlisting}
DELETE FROM Rennfahrer WHERE Vorname LIKE "F%";
\end{lstlisting}
Funktioniert allerdings nicht einfach so, da es Foreign Key Verweise aus der Platzierungstabelle auf diese Rennfahrer gibt. Diese müssen zuerst gelöscht werden.

\begin{lstlisting}
DELETE FROM Platzierung
WHERE RID IN	(SELECT RID
		FROM Rennfahrer
		WHERE Vorname LIKE "F%");

DELETE FROM Rennfahrer WHERE Vorname LIKE "F%";
\end{lstlisting}

\textbf{Alle Tabellen löschen:} \\
Dabei muss besonders auf die korrekte Reihenfolge geachtet werden, da es sonst zu Verletzungen der referentiellen Integrität kommt.
\begin{lstlisting}
DROP TABLE Platzierung;
DROP TABLE Rennfahrer;
DROP TABLE Rennstall;
DROP TABLE Rennort;
\end{lstlisting}
		\end{enumerate}
		\item[\textbf{3.:}]
		Annahme, dass die Tabellen entsprechend des Relationenschemas aus Aufgabe 1 vorliegen.
		\begin{enumerate}
			\item[a)]\quad \\
			Alle Obstsorten, gegen die ein Peter Meyer allergisch ist, ohne Duplikate in absteigender Sortierung:
\begin{lstlisting}
SELECT	DISTINCT o.Sorte
FROM	Obst o, Person p, Allergie a
WHERE 	a.Obst = o.ONR
AND 	a.Person = p.PNR
AND 	p.Vorname = "Peter"
AND 	p.Nachname = "Meyer"
ORDER BY o.Sorte DESC;
\end{lstlisting}

			\item[b)]\quad \\
			Die PNR, den Nachnamen und die Anzahl der Allergien jedes Allergikers:
\begin{lstlisting}
SELECT	p.PNR, p.Nachname, COUNT(a.PNR)
FROM	Person p, Allergie a
WHERE	p.PNR = a.PNR;
\end{lstlisting}			

			\item[c)]\quad \\
			Die PNR aller Personen, die mehr als 6 Obstsorten entdeckt haben:
\begin{lstlisting}
SELECT	p.PNR
FROM	Person p, Obst o
WHERE	p.PNR = o.PNR AND (COUNT(o.Entdecker)>6);
\end{lstlisting}


			\item[d)]\quad \\
			Vorname und Nachname aller Personen, deren Lieblingsobst von einer Person gleichen Vornamens entdeckt wurde:
\begin{lstlisting}
SELECT	p.Vorname, p.Nachname
FROM	Person p, Obst o
WHERE	p.Lieblingsobst = o.ONR
AND 	p.Vorname IN	(SELECT p2.Vorname 
			 FROM Person p2, Obst o2
			 WHERE o2.Entdecker = p2.PNR
			 AND p.Lieblingsobst = o2.ONR);
\end{lstlisting}


			\item[e)]\quad \\
			Die PNR, Vorname und Nachname aller Personen, die noch keine Obstsorte entdeckt haben:
\begin{lstlisting}
SELECT	p.PNR, p.Vorname, p.Nachname
FROM	Person p, Obst o
WHERE	p.PNR NOT IN	(SELECT p2.PNR
			 FROM Person p2, Obst o2
			 WHERE p2.PNR = o.Entdecker);
\end{lstlisting}
		\end{enumerate}
		
		
		\item[\textbf{4.:}]\quad \\	
			\(\projektion{PNR,Vorname,Nachname}\bra{\selektion{Sorte=\wert{K\%}}(Obst)\verbund{ONR=Lieblingsobst}
			Personen}\) \\ \\ \\
			
			\begin{tikzpicture}[style = {level distance=2cm},
         				        level 3/.style={sibling distance=4cm, level distance=2cm},
          				        level 4/.style={level distance=2cm}]
				\node{}
           			child{node{\(\pi_{PNR,Vorname,Nachname}\)}
           				child{node{\(\underset{ONR=Lieblingsobst}{\bowtie}\)}
           					child{node{\(\sigma_{Sorte = ''K\%''}\)}
           						child{node{Obst}
	           						edge from parent node[midway, left] {25 Tupel, 3 Attribute}
           						}
           						edge from parent node[midway, above left] {5 Tupel, 3 Attribute}
           					}
           					child{node{Personen}
	           					edge from parent node[midway, above right] {2000 Tupel, 5 Attribute}
           					}
           					edge from parent node[midway, right] {400 Tupel, 8 Attribute}
           				}
           				edge from parent node[midway, right] {400 Tupel, 3 Attribute}
           			}
           		;					
			\end{tikzpicture}
		\end{enumerate}
\end{document}