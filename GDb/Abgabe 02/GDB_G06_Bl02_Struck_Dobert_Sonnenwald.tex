% Commands

\newcommand{\authorinfo}{Tim Dobert, Kai Sonnenwald, Arne Struck}
\newcommand{\titleinfo}{GDB [HA] zum 14. 11. 2013}
\newcommand{\qed}{\ \square}

% ------------------------------------------------------

% Packages & Stuff
\documentclass[a4paper,11pt,fleqn]{scrartcl}
\usepackage[german,ngerman]{babel}
\usepackage[utf8]{inputenc}
\usepackage[T1]{fontenc}
\usepackage[top=1.3in, bottom=1.2in, left=0.9in, right=0.9in]{geometry}
\usepackage{lmodern}
\usepackage{amssymb}
\usepackage{amsmath}
\usepackage{enumerate}
\usepackage{tikz}
\usepackage{fancyhdr}
\usepackage{pgfplots}
\usepackage{multicol}
\usepackage{vsis-gdb}
\usepackage[parfill]{parskip}
\usetikzlibrary{calc,patterns,arrows,positioning,calc,fit,shapes}


% ------------------------------------------------------

% Title & Pages

\title{\titleinfo}
\author{\authorinfo}

\pagestyle{fancy}
\fancyhf{}
\fancyhead[L]{\authorinfo}
\fancyhead[R]{\titleinfo}
\fancyfoot[C]{\thepage}

\begin{document}
    \maketitle
	\begin{enumerate}
		\item[\textbf{1.:}]
		\begin{enumerate}
		\item[(a)] \quad \\
			\begin{tikzpicture}
				%Person
				\node[entity] (person) {Person};
				\node[attribut] (pName) [below right =0.5cm of person] {\underline{Name}} edge(person);
				\node[attribut] (pVorName) [above =0.6cm of person] {\underline{Vorname}} edge(person);
				\node[attribut] (gebD) [above right  =0.6cm of person] {Geburtsdatum} edge(person);
				
				%leitet
				\node[relationship] (leitet) [right  =3cm of person] {leitet};
				
				%Studio
				\node[entity] (studio) [right  =2.5cm of leitet] {Studio};
				\node[attribut] (sName) [above =0.8cm of studio] {\underline{Name}} edge(studio);
				
				%produziert
				\node[relationship] (produziert) [below  =2cm of studio] {produziert};
				
				%verhandeln
				\node[relationship] (verhand) [left  =2cm of produziert] {Verhandlung};
				\node[attribut] (budget) [above left =0.6cm of verhand] {Budget} edge(verhand);
				\node[attribut] (vdatum) [above =0.6cm of verhand] {Datum} edge(verhand);
				
				%Film
				\node[entity] (film) [below  =2cm of produziert] {Film};
				\node[attribut] (fName) [above right =0.6cm of film] {\underline{Titel}} edge(film);
				\node[attribut] (fdDay) [right =0.5cm of film] {erster Drehtag} edge(film);
				\node[attribut] (ldDay) [below right  =0.8cm of film] {letzter Drehtag} edge(film);
				
				%führt Regie
				\node[relationship] (dregie) [left  =1.2cm of film] {führt Regie};
				
				%Regisseur
				\node[entity] (regisseur) [left  =1.2cm of dregie] {Regisseur};
				
				%isa
				\node[relationship] (isa) [below  =2.5cm of person] {ist ein};
				
				%Schauspieler
				\node[entity] (schauspieler)[below  =4cm of isa] {Schauspieler};
				\node[multivalentattribut] (mZeichen) [below  =0.4cm of schauspieler] {Markenzeichen} edge(schauspieler);
				
				%spielt
				\node[relationship] (spielt)[below = 1.025cm of dregie] {spielt};
				\node[attribut] (dbegin) [below  =0.7cm of spielt] {Drehbeginn} edge(spielt);
				\node[attribut] (dend) [below right =0.5cm of spielt] {Drehende} edge(spielt);
				
				%präferiert
				\node[relationship] (praef) [below = 6cm of regisseur] {präferiert};
				
				%gehört zu
				\node[relationship] (joins) [below = 6cm of film] {gehört zu};
				
				%Genre
				\node[entity] (genre) [right = 1.4cm of praef] {Genre};
				\node[attribut] (gname) [below =0.4cm of genre] {\underline{Name}} edge(genre);
				
				%Char
				\node[entity] (char) [below = 1.5cm of mZeichen] {Charakter};
				\node[attribut] (cid) [above =0.4cm of char] {\underline{CID}} edge(char);
				\node[attribut] (cname) [below =0.4cm of char] {Name} edge(char);
				
				%Paths
				%leitet
				\path (leitet) edge node[midway, above] {$1$} (person);
				\path (studio) edge node[midway, above] {$n$} (leitet);
				%Verhandlung
				\path (studio) edge node[midway, above] {$m$} (verhand);
				\path (film) edge node[midway, above right] {$1$} (verhand);
				\path (regisseur) edge node[midway, above left] {$n$} (verhand);
				%produziert
				\path (studio) edge node[midway, right] {$o$} (produziert);
				\path (film) edge node[midway, right] {$1$} (produziert);
				%führt Regie
				\path (film) edge node[midway, above] {$1$} (dregie);
				\path (regisseur) edge node[midway, above] {$m$} (dregie);
				%isa
				\path (isa) edge[erbt] (person);
				\path (regisseur) edge[erbt] (isa);
				\path (schauspieler) edge[erbt] (isa);
				%spielt
				\path (schauspieler) edge node[midway, above] {$n$} (spielt);
				\path (spielt) edge node[midway, above left] {$o$} (film);
				\path (spielt) edge node[midway, below right] {$p$} (char);
				%präferiert
				\path (praef) edge node[midway, left] {$1$} (regisseur);
				\path (praef) edge node[midway, above] {$n$} (genre);
				%gehört zu
				\path (genre) edge node[midway, above] {$m$} (joins);
				\path (joins) edge node[midway, right] {$[0,4]$} (film);
				
			\end{tikzpicture}
		    \newpage
		    
		\item[(b)]
			\begin{itemize}
				\item Ein Film kann nicht ohne Budget produziert werden.
				\item Kein Element kann vor seiner Entstehung an einer Relation partizipieren.
				\item Es kann nicht nach dem Dreh über das Budget verhandelt werden.
				\item Ein Studio kann nicht über einen Film verhandeln, welches es nicht produziert.
			\end{itemize} \quad
		\end{enumerate}

	\item[\textbf{2.:}]
	\begin{enumerate}
		\item[(a)]
			\begin{itemize}
				\item Ein Student besitzt eine eindeutige Matrikelnummer und einen Namen.
				\item Ein Studiengang besitzt einen eindeutigen Namen.
				\item Ein Student ist in genau einem Studiengang immatrikuliert.
				\item In einem Studiengang können n Studenten immatrikuliert sein.
			\end{itemize} \quad

		\item[(b)]
			\begin{itemize}
				\item Eine Universität besitzt einen eindeutigen Namen.
				\item Ein Hörsaal besitzt einen Namen und eine Platzzahl.
				\item Eine Universität hat beliebig viele, aber mindestens einen Hörsaal.
				\item Ein Hörsaal existiert nur als Teil der Universität und gehört auch nur zu einer Universität.
			\end{itemize} \quad

		\item[(c)]
			\begin{itemize}
				\item Ein Auftrag besitzt eine eindeutige ANR und ein Datum.
				\item Ein Ersatzteil ist eindeutig identifiziert über seinen Namen und das Automodell, zudem hat es einen Preis.
				\item Ein Reparaturtyp ist von eindeutiger Art und hat einen Festpreis.
				\item Eine Reparatur findet an einem bestimmten Datum zu einer bestimmten Zeit statt.
				\item Eine Reparatur kann beliebig viele Aufträge umfassen.
				\item Für eine Reparatur können beliebig viele Ersatzteile benötigt werden.
				\item Eine Reparatur kann zu beliebig vielen Reparaturtypen gehören.
			\end{itemize} \quad

		\item[(d)]
			\begin{itemize}
				\item In einem Stadion finden beliebig viele Fußballspiele statt.
				\item Ein Schiedsrichter leitet beliebig viele Fußballspiele.
				\item Eine Mannschaft spielt beliebig viele Fußballspiele gegen eine andere Mannschaft.
			\end{itemize} \quad 
	\end{enumerate}

	\newpage
	\item[\textbf{3.:}]
	\begin{enumerate}
		\item[(a)]
			\begin{itemize}
				\item Eindeutigkeit: Es darf keine zwei Datensätze mit dem gleichen Schlüssel geben.
				\item Minimalität: Der Schlüssel der Tabelle besteht aus so wenigen Attributen wie möglich. Wenn es in einer 
				Attributkombination die Möglichkeit gibt, auch mit weniger Attributen aus dieser Kombination einen eindeutigen 
				Schlüssel zu bieten, ist diese nicht minimal.
				\item Beispiele in diesem Kontext:
				\begin{enumerate}
					\item Telefonnummer
					\item (Nachname, Geburtsdatum)
				\end{enumerate}
			\end{itemize}
			Die Attributkombination (Vorname, Haus-Nr.) ist kein Schlüsselkandidat, denn sie ist nicht eindeutig. Es würde 2 
			Datensätze einem identischen Schlüssel geben. \\ 

		\item[(b)]
			Bei größeren Datenmengen wird es immer schwieriger einen eindeutigen Schlüsselkandidaten zu finden der möglichst 
			minimal ist. Je mehr Datensätze vorhanden sind, desto größer ist die Wahrscheinlichkeit, dass viele Attribute zur 
			eindeutigen Identifizierung eines einzelnen Datensatz benötigt werden. Und je mehr Attribute nötig sind, desto mehr 
			Rechenaufwand wird für die Schlüsselverwaltung gebraucht. Damit wird es ineffizient. Zur Lösung bietet sich die 
			Einführung eines ID-Feldes an, das nie null gesetzt werden darf und in dem einfach ein Integerwert inkrementiert 
			wird für jeden neuen Datensatz. 	
		\end{enumerate}	
	\end{enumerate}
\end{document}